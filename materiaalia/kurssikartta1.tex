\documentclass[finnish]{standalone}
\usepackage{forest}
\usepackage[utf8]{inputenc}

\usetikzlibrary{mindmap}
\usepackage[T1]{fontenc}
\usepackage[T2A]{fontenc}
\usepackage[finnish]{babel}
\begin{document}

\begin{forest}
  for tree={
    child anchor=west,
    parent anchor=east,
    grow'=east,
  %minimum size=1cm,%new possibility
  %text width=8cm,%
    %draw,
    anchor=west,
    edge path={
      \noexpand\path[\forestoption{edge}]
        (.child anchor) -| +(-5pt,0) -- +(-5pt,0) |-
        (!u.parent anchor)\forestoption{edge label};
    },
  }
[Morfologia kielitieteen osa-alueena, text width=3cm
    [Morfologian osa-alueita
        [Morfemiikka
            [morfien luokittelua]
        ]
        [Morfofonologia
            [esim. äännevaihtelut
                 [пеку-печёшь]
            ]
        ]
        [Morfosyntaksi
            [eim. kongruenssi-ilmiöt
                [на столе была книга / были книги]
            ]
        ]
        [Funktionaalinen morfologia
            [esim. aspektien käyttö]
        ]
    ]
    [Morfologian käsitteitä
        [sanataso
            [sanan käsite
                [lekseemi]
                [sananmuoto]
                [sane]
                [lemma]
            ]
            [Perinteiset sanaluokat
                [Itsenäiset sanaluokat
                    [substantiivit]
                    [adjektiivit]
                    [pronominit]
                    [lukusanat]
                    [verbit]
                    [adverbit
                        [tavalliset adverbit]
                        [predikatiiviadverbit]
                    ]
                ]
                [Apusanaluokat
                    [prepositiot]
                    [konjunktiot]
                    [partikkelit]
                    [interjektiot]
                ]
            ]
        ]
        [Morfeemitaso
            [morfeemi
                [morfi
                    [allomorfisuus]
                    [salkkumorfit]
                    [nollamorfit]
                ]
            ]
            [morfeemityypit
                [juurimorfit
                ]
                [affiksit
                    [prefiksit ja suffiksit
                        [käytetään sananmuodostuksessa]
                        [prefiksit: ennen juurta]
                        [suffiksit: juuren jälkeen]
                    ]
                    [taivutuspäätteet
                        [ilmaisevat kieliopillisia merkityksiä kuten sijaa, text width=3cm]
                    ]
                    [interfiksit
                        [lähinnä yhdyssanoissa]
                    ]
                    [postfiksit
                        [taivutuspäätteen jälkeen: esim. -ся]
                    ]
                ]
                [vartalot
                    [prefiksi + juuri + suffiksi - päätteet]
                ]
            ]
            [kieliopilliset kategoriat
                [{=ominaisuus/merkitys, jonka suhteen muodot oppositiossa}, text width=3cm]
                [suomessa myös: {kieliopillinen luokka/ taivutusluokka / taivutuskategoria}, text width=4cm
                    [venäjässä ei kuitenkaan aina kyse taivutuksesta (esim. konditionaali), text width=4cm]
                ]
            ]
        ]
    ]
]
\end{forest}



\end{document}
