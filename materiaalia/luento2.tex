\documentclass[]{scrartcl}
\usepackage[utf8]{inputenc}
\usepackage[T1]{fontenc}
\usepackage[T2A]{fontenc}
\usepackage[finnish]{babel}
\usepackage{linguex} 
\usepackage{amsthm}
\newtheorem{maar}{Määritelmä}
\usepackage{fixltx2e} % provides \textsubscript
\usepackage{textcomp} % provides \textsubscript
\usepackage{hyperref}
\usepackage{xcolor}

\providecommand{\tightlist}{%
  \setlength{\itemsep}{0pt}\setlength{\parskip}{0pt}}

\hypersetup{
    colorlinks,
    linkcolor={red!50!black},
    citecolor={blue!50!black},
    urlcolor={blue!80!black}
}
\author{Juho Härme}
\title{Morfologia-kurssin luentomateriaaleja}
\date{\today}
\begin{document}
\maketitle
\tableofcontents
\newpage



\section{Luento 2: morfeemityypeistä ja
sanaluokista}\label{luento-2-morfeemityypeistuxe4-ja-sanaluokista}

\begin{itemize}
\itemsep1pt\parskip0pt\parsep0pt
\item
  \href{https://mustikka.uta.fi/~juho_harme/morfologia/\#tästä-kurssista}{Takaisin
  sivun ylälaitaan}
\item
  \href{http://mustikka.uta.fi/~juho_harme/morfologia/materiaalit/luento2.pdf}{Lataa
  PDF}
\item
  \href{http://mustikka.uta.fi/~juho_harme/morfologia/presentations/luento2.html}{Tutki
  luentokalvoja}
\item
  \href{http://mustikka.uta.fi/~juho_harme/morfologia/tehtavat/luento2.pdf}{Tutki
  tuntitehtäviä}
\end{itemize}

\subsection{Morfeemien alalajeista
tarkemmin}\label{morfeemien-alalajeista-tarkemmin}

Palataan toisen luennon aluksi tarkemmin eri morfeemityyppeihin. Edellä
todettiin, että kieliopillisia morfeemeja voidaan luokitella
juurimorfeemeiksi, prefikseiksi, suffikseiksi, postfikseiksi\footnote{Jos
  ihmettelet, miksi kaikki nämä nimitykset päättyvät ``fiksi/фикс'',
  sanojen etymologia juontaa latinan \emph{figere}-verbiin, jonka
  merkitys on `liittää, kiinnittää'.}, taivutuspäätteiksi ja
interfikseiksi. Nämä voidaan jo lähtökohtaisesti jakaa kahteen ryhmään:
juurimorfeemeihin ja \emph{affikseihin} (аффиксы). Tutkitaan kumpaakin
ryhmää seuraavassa tarkemmin.

\subsubsection{Juurimorfeemit}\label{juurimorfeemit}

Juurimorfeemin käsitteeseen olemmekin jo törmänneet. Kuten Musatov
(2016: 28) huomauttaa, venäjässä ei ole sanoja, joissa ei olisi
juurimorfia. Toisin sanoen jokaikisestä venäläisestä sanasta voidaan
erottaa \emph{vähintään} juurimorfeemi. Juurimorfeemi ilmaisee sanan
lähtökohtaisen leksikaalisen merkityksen, jota muut morfeemit,
esimerkiksi prefiksit, voivat muokata. Joskus juurimorfi on ainoa morfi
koko sanassa:

\begin{enumerate}
\def\labelenumi{(\arabic{enumi})}
\itemsep1pt\parskip0pt\parsep0pt
\item
  /автор/
\end{enumerate}

On kuitenkin huomattava, että vaikka juurimorfeemi on ainoa
morfeemityyppi, joka voi jo itsessään muodostaa sanan, aina näin ei
kuitenkaan käy. Mieti esimerkiksi sanoja одеть ja надеть. Missä niissä
on juurimorfi ja voisiko se esiintyä itsenäisenä sanana?

Sanojen одеть ja надеть juurimorfeiksi voidaan erottaa /де/. Tällaisia
ei-itsenäisinä esiintyvä morfeemeja nimitetään sidotuiksi (связанные)
erotuksena vapaista (свободные) juurimorfeemeista.

\subsubsection{Affiksit}\label{affiksit}

\paragraph{Prefiksit ja suffiksit}\label{prefiksit-ja-suffiksit}

Prefiksejä (префиксы)\footnote{Venäläisperäisempi termi olisi приставка.
  Tieteellisessä diskurssissa on kuitenkin parempi käyttää
  префикс-termiä. Vrt. myös ero suomen \emph{prefiksi}- ja
  \emph{etuliite}-sanojen välillä.} ja suffikseja (суффиксы) käytetään
muodostettaessa uusia sanoja. Ne eivät siis liity taivuttamiseen, vaan
ennen kaikkea \emph{johtamiseen} ja \emph{sananmuodostukseen} (vrt.
päätteet alla). Tarkastellaan aluksi erityisen selkeää tapausta:

\begin{enumerate}
\def\labelenumi{(\arabic{enumi})}
\setcounter{enumi}{1}
\itemsep1pt\parskip0pt\parsep0pt
\item
  /со/автор/
\end{enumerate}

Esimerkissä 2 on prefiksi /со/, joka muuttaa автор-sanan merkitystä ja
muodostaa kokonaan uuden sanan, jonka merkitys (`toinen mukana ollut
kirjoittaja', `kanssakirjoittaja') tässä tapauksessa on helppo ymmärtää
ikään kuin muokatuksi juurimorfeemin merkitykseksi. Myös edellä
mainituissa надеть- ja одеть-sanoissa juurimorfia /де/ edeltävää ainesta
voidaan nimittää prefiksiksi (morfit /на/ ja /о/). Ne muokkaavat
juuresta uuden sanan, vaikka tässä tapauksessa itse juuri ei sanana
esiinnykään. Tavallisia prefiksejä ovat esimerkiksi
\emph{пере-},\emph{за-},\emph{от-},\emph{с-} jne.

Suffiksit eroavat prefikseistä siten, että siinä missä prefiksit
sijoittuvat \emph{ennen} juurimorfeemia, sijoittuvat suffiksit
juurimorfeemin \emph{jälkeen}. Myös suffiksit ovat etupäässä
sananmuodostuksellisia morfeemeja. Suffikseista voisi todeta, että muun
muassa äänteenmuutosten takia niitä on välillä vaikea erottaa, niin että
suffikseja on usein sielläkin, missä ensi katsomalta ei siltä näyttäisi.
Helposti havaittavissa on esimerkiksi adjektiivista /хитр/- juurimorfiin
liitetty ость-suffiksi esimerkissä 3:

\begin{enumerate}
\def\labelenumi{(\arabic{enumi})}
\setcounter{enumi}{2}
\itemsep1pt\parskip0pt\parsep0pt
\item
  /хитр/ость/
\end{enumerate}

Vaikeammin havaittavia ovat esimerkiksi verbinmuodostussuffiksit /и/ ja
/а/:

\begin{enumerate}
\def\labelenumi{(\arabic{enumi})}
\setcounter{enumi}{3}
\itemsep1pt\parskip0pt\parsep0pt
\item
  уч/и/ть
\item
  игр/а/ть
\end{enumerate}

Muita tyypillisiä suffikseja ovat esimerkiksi /ик/,/ец/,/тель/ jne.

Sekä prefikseistä että suffikseista on huomattava, että niitä voi
sanassa olla useita. Mitä prefiksejä ja suffikseja voit erottaa
seuraavista sanoista?

\begin{enumerate}
\def\labelenumi{(\arabic{enumi})}
\setcounter{enumi}{5}
\itemsep1pt\parskip0pt\parsep0pt
\item
  переодеваю
\item
  основательницу
\end{enumerate}

\paragraph{Taivutuspäätteet}\label{taivutuspuxe4uxe4tteet}

Taivutuspäätteillä ei muodosteta uusia sanoja vaan ilmaistaan eri
kieliopillisia kategorioita ja niiden arvoja (vrt. prefiksit ja
suffiksit yllä). Taivutuspäätteet sijaitsevat sanassa prefiksian,
juurien ja suffiksien jälkeen. Tutki esimerkiksi seuraavaa
худой-adjektiivin sananmuotoa:

\begin{enumerate}
\def\labelenumi{(\arabic{enumi})}
\setcounter{enumi}{7}
\itemsep1pt\parskip0pt\parsep0pt
\item
  /худ/ую/
\end{enumerate}

/ую/ on taivutuspääte, joka ilmaisee useampaakin kieliopillista
kategoriaa: lukua (arvona tässä yksikkö), sukua (feminiini) ja sijaa
(akkusatiivi). Osin toisenlaisista taivutuspäätteistä on kyse
seuraavissa esimerkeissä:

\begin{enumerate}
\def\labelenumi{(\arabic{enumi})}
\setcounter{enumi}{8}
\itemsep1pt\parskip0pt\parsep0pt
\item
  /гул/я/л/а/
\item
  /книг/а/
\item
  /обед/ø/
\end{enumerate}

Esimerkissä 9 on itse asiassa kaksi päätettä, /л/, joka ilmaisee
aikamuotoa (mennyt aika) ja /а/ joka ilmaisee lukua ja sukua (yksikkö,
feminiini). Esimerkki 11 on merkittävä siinä mielessä, että se esittelee
\emph{nollamorfin} (нулевой морф) käsitteen. Voidaan nimittäin katsoa,
että обед-sanassakin on -- kuten книга-sanassa -- sukua ilmaiseva pääte.
Tämä vain ei edustu jonakin sarjana foneemeja vaan nimenomaan äänteiden
puuttumisena. Nollamorfia merkitän symbolilla ø. Käsitteeseen palataan
vielä kurssin edetessä, joten ei kannata pelästyä, vaikka sen merkitys
jäisi jonkin verran avoimeksi.

Taivutuspäätteistä käytetään joskus pääte/окончание-nimitysten lisäksi
termiä fleksio (флексия).

\paragraph{Postfiksit ja interfiksit}\label{postfiksit-ja-interfiksit}

Prefiksit ja suffiksit ovat erittäin yleisiä morfeemityyppejä. Sen
sijaan sekä \emph{postfikseja} (постфиксы) että etenkin
\emph{interfiksejä} (интерфиксы) esiintyy huomattavasti harvemmin.
Postfiksit ovat periaatteessa sananmuodostuksen palveluksessa olevia
morfeemeja, jotka sijoittuvat prefiksien, juuren, suffiksien ja
taivutuspäätteiden jälkeen, siis nimensä mukaisesti viimeiselle
mahdolliselle paikalle sanassa. Venäjän tavallisimpia postfiksejä ovat
-нибудь, -то, -либо sekä ehkä tavallisimmin vastaan tuleva verbeillä
esiintyvä -cя. Interfiksit puolestaan ovat -- nekin nimensä mukaisesti
-- sanojen välissä olevia liitosmorfeemeja, joiden avulla muodostetaan
yhdyssanoja:

\begin{enumerate}
\def\labelenumi{(\arabic{enumi})}
\setcounter{enumi}{11}
\itemsep1pt\parskip0pt\parsep0pt
\item
  /чёрн/о/бел/ый/
\end{enumerate}

\subsubsection{Vartalo}\label{vartalo}

Vartalon (основа) käsitteeseen palataan tarkemmin käsiteltäessä
kieliopillisten kategorioiden konkreettisia arvoja. Jo tässä vaiheessa
on paikallaan todeta, että vartalon käsitettä ei kannata sekoittaa
juurimorfeemin käsitteeseen. Itse asiassa vartalo ei useinkaan koostu
vain yhdestä, vaan itse asiassa useammasta morfeemista. Vartaloksi
nimittäin sanotaan sitä osaa sanasta, joka jää jäljelle, kun postetaan
taivutuspäätteet, esimerkiksi (vartalo erotettu kursiivilla):

\begin{enumerate}
\def\labelenumi{(\arabic{enumi})}
\setcounter{enumi}{12}
\itemsep1pt\parskip0pt\parsep0pt
\item
  \emph{/преподава/тел}/я/
\end{enumerate}

{[}tehtävä 2.1: luokittele morfeemeja{]}

\subsection{Morfofonologisista
vaihteluista}\label{morfofonologisista-vaihteluista}

Edellisellä luennolla mainittiin, että yksi morfologian alalajeista on
nimeltään morfofonologia (морфонология). Morfofonologialla on
kielenkäyttäjän kannalta paljon käytännön merkitystä. Tämä johtuu siitä,
että morfofonologia tutkii muun muassa \emph{äänteenmuutoksia}
(чередования). Juuri äänteenmuutokset saavat usein aikaan vaikutelman
siitä, että esimerkiksi verbien persoonataivutuksessa on ``pelkkiä
poikkeuksia'' tai muotoja, joita ei voi ennustaa. Tosiassa nämä muodot
ovat usein täysin säännnmukaisia, mutta niissä vain tapahtuu yhtälailla
säännönmukaisia konsonanttien tai vokaalien vaihteluita. Katsomme tässä
yhteydessä vain lyhyesti yhtä esimerkkiä ja palaamme morfofonologisiin
vaihteluihin muun muassa substantiivien sija- ja verbien
persoonakategorian käsittelyn yhteydessä.

Ehkäpä kaikkein tavallisin äänteenmuutos on liudentuneiden ja
liudentumattomien konsonanttien vaihtelu. Vain kirjoitettua sanaa
katsoessa ei ehkä tulisi ajatelleeksikaan, että seuraavat äänneparit
(Мусатов 2016: 148-149) eivät itse asiassa edusta saman äänteen kahta
eri versiota vaan kahta eri äännettä -- kahta eri foneemia:

\begin{itemize}
\itemsep1pt\parskip0pt\parsep0pt
\item
  р -- р
\item
  м -- м'
\item
  д -- д'
\item
  т -- т'
\item
  н -- н'
\item
  с -- с'
\item
  б -- б'
\item
  л -- л'
\item
  ф -- ф'
\item
  г -- г'
\item
  х -- х'
\item
  к -- к'
\item
  в -- в'
\item
  з -- з'
\item
  п -- п'
\end{itemize}

Ajattele tässä valossa esimerkkiä 14. Lisättäessä juurimorfiin /кот/
suffiksi /ик/ morfin viimeinen foneemi vaihtuu eli äänne {[}т{]} muuttuu
äänteeksi {[}т'{]} -- tämä ero vain ei näy ortografian tasolla,
kirjoituksessa.

\begin{enumerate}
\def\labelenumi{(\arabic{enumi})}
\setcounter{enumi}{13}
\itemsep1pt\parskip0pt\parsep0pt
\item
  кот -- котик
\end{enumerate}

Tietyissä tapauksissa ero voidaan havaita ortografiankin tasolla:

\begin{enumerate}
\def\labelenumi{(\arabic{enumi})}
\setcounter{enumi}{14}
\itemsep1pt\parskip0pt\parsep0pt
\item
  путь -- путник
\end{enumerate}

esimerkissä 15 liudentunut {[}т'{]}-äänne vaihtelee liudentumattoman
{[}т{]}-äänteen kanssa, mikä näkyy pehmeän merkin puuttumisena
путник-sanan kirjoitusasusta.

Paljon näitä sananmuodostukseen liittyviä esimerkkejä konkreettisemmin
äänteenmuutokset näyttäytyvät, kuten mainittua, taivutuksessa.
Tavallisimpia tapauksia (ks. Мусатов 2016: 158) ovat esimerkiksi
{[}з'{]}/{[}ж{]}-vaihtelu (16), {[}т'{]}/{[}ч{]}-vaihtelu (17) ja
{[}с'{]}/{[}ш{]}-vaihtelu (18):

\begin{enumerate}
\def\labelenumi{(\arabic{enumi})}
\setcounter{enumi}{15}
\itemsep1pt\parskip0pt\parsep0pt
\item
  морозить -- морожу
\item
  платить -- плачу
\item
  косить -- кошу
\end{enumerate}

Tällä kurssilla morfit merkitään aina kun mahdollista tavallisen
kirjoitusasun mukaisesti. Jos huomio kuitenkin on esimerkiksi
äänteenmuutoksissa, saatetaan käyttää foneettista merkintätapaa.

\subsection{Substantiivit, adjektiivit, pronominit,
lukusanat}\label{substantiivit-adjektiivit-pronominit-lukusanat}

Tulevilla luennoilla aletaan tarkemmin perehtyä eri kieliopillisiin
kategorioihin ja niiden arvoihin. Aloitamme luku-, suku- ja
sijakategorioista, jotka ovat erityisen olennaisia substantiiveilla,
mutta myös adjektiiveillä, pronomineilla ja lukusanoilla. Ennen kuin
siirrytään varsinaisiin kieliopillisten kategorioiden arvoihin ja
toisaalta niiden käyttöön käytännössä (funktionaaliseen morfologiaan),
on syytä tutkia hivenen tarkemmin, mitä erityispiirteitä mainittuihin
sanaluokkiin liittyy. Kun myöhemmin siirrytään käsittelemään verbejä,
otetaan uudestaan käsittelyyn myös nämä kategoriat niiltä osin kuin ne
verbeillä edustuvat.

\subsubsection{Substantiivit}\label{substantiivit}

Substantiivit ovat venäjän yleisin sanaluokka: Venäjän
kansalliskorpuksen pääaineiston sanoista 28,5\% on substantiiveja.
Substantiivit ilmaisevat niin sukua, lukua, sijaa kuin elollisuuttakin.

Substantiiveja voidaan luonnollisesti jaotella edelleen esimerkiksi
semanttisin perustein (merkityksen mukaan). Voidaan erotella
\emph{yleisnimet} () ja erisnimet (имена собственные). Toisaalta voidaan
puhua abstrakteista tai konkreettisista substantiiveista. Lisäksi oma
ryhmänsä ovat ainesanat kuten каша, нефть ym. Substantiivit toimivat
erityisen usein lähtökohtina muiden sanaluokkien sanojen muodostukselle
(ks. Мусатов 2016: 312--).

\subsubsection{Adjektiivit}\label{adjektiivit}

Myös adjektiivit ilmaisevat suku- luku- ja sijakategorioita sekä ainakin
välillisesti elollisuuttakin. Niiden erikoisuus ovat lisäksi
\emph{vertailuasteet} (степени сравнения), joita niiden lisäksi
ilmaisevat vain adverbit.

Adjektiivit esiintyvät venäjässä \emph{pitkinä} ja \emph{lyhyinä}
muotoina (). Lyhyiden muotojen käyttö on syntaktisesti rajoitettua: ne
toimivat ainoastaan predikatiiveina, kuten esimerkissä 19.

\begin{enumerate}
\def\labelenumi{(\arabic{enumi})}
\setcounter{enumi}{18}
\itemsep1pt\parskip0pt\parsep0pt
\item
  Скандинавская кухня \emph{лаконична} в средствах и \emph{многообразна}
  в методах
\end{enumerate}

\subsubsection{Pronominit}\label{pronominit}

Pronominit ilmaisevat (joskin vaihtelevasti) luku-, suku- ja
sijakategorioita sekä ainoana sanaluokkana verbien lisäksi myös
persoonakategoriaa. Erotuksena verbeistä persoonan ilmaiseminen ei
tapahdu pronomineilla taivutuspäätteen vaan sanavartalon avulla
(Nikunlassi 2002, 154). Erityisesti persoonapronominit ovat taivutuksen
suhteen omalaatuinen luokka, jonka sisällä on suurta vaihtelua. Tähän
palataan tarkemmin luennolla 5. Pronominien erikoisuuksiin morfologian
kannaltaa kuuluu myös se, että niistä on usein hankala erottaa
juurimorfeemia ja taivutuspäätteitä. Ajettele vaikka sananmuotoja
\emph{мне} ja \emph{меня} ja vertaa niitä nominatiivimuotoon \emph{я}.

\subsubsection{Lukusanat}\label{lukusanat}

Lukusanat ovat, kuten edellä mainittiin, osittain hämärästi
määriteltävissä oleva luokka, johon luetaan usein morfologiselta
rakenteeltaan hyvin erilaisia sanoja. Läheskään kaikki lukusanat eivät,
nurinkurista kyllä, ilmaise luvun kieliopillista kategoriaa (niillä ei
ole erillisiä yksikkö- ja monikkomuotoja). Myöskään sukua lukusanat
eivät paria poikkeusta lukuunottamatta ilmaise. Lukusanat onkin oma
luokkansa ennen muuta sen takia, että tähän ryhmään yleensä luettavat
sanat ovat merkityksen puolesta lähellä toisiaan (tarkemmin ks.
Nikunlassi 2002: 123). Morfologiselta kannalta kannattaa kiinnittää
huomio seuraaviin luokansisäisiin jakoihin:

\paragraph{Järjestysluvut (порядковые числительные) ja peruslukusanat
(количественные
числителные).}\label{juxe4rjestysluvut-ux43fux43eux440ux44fux434ux43aux43eux432ux44bux435-ux447ux438ux441ux43bux438ux442ux435ux43bux44cux43dux44bux435-ja-peruslukusanat-ux43aux43eux43bux438ux447ux435ux441ux442ux432ux435ux43dux43dux44bux435-ux447ux438ux441ux43bux438ux442ux435ux43bux43dux44bux435.}

Järjestysluvut (esimerkiksi \emph{пятый}) toimivat monessa suhteessa
adjektiivien tavoin. Niillä ei kuitenkaan ole vertailuasteita tai
erillisiä lyhyitä ja pitkiä muotoja.

\paragraph{Perusluvut 1-4 ja muut
luvut}\label{perusluvut-1-4-ja-muut-luvut}

Peruslukusanatkaan eivät ole täysin yhtenäinen joukko. Oma
erillistapauksensa on ensinnäkin lukusana \emph{один}, joka ilmaisee
sukukategoriaa (vrt. muodot одна, одно jne), lukukategoriaa (один
vs.~одни) ja taipuu esimerkiksi sijoissa paljolti adjektiivien tavoin.
Luvuilla 2-4 on hyvin erilaiset taivutusparadigmat (konkreettiset
kieliopillisten kategorioitten, kuten sijan, arvot) kuin muilla
luvuilla. Peruslukusanat ovat siinäkin suhteessa erikoinen luokka, että
nominatiivissa ne toimivat pääsanana ja määrävät niihin liittyvän
substantiivin sijan (семь столов), mutta muuten käyttäytyvät
syntaktisessa mielessä adjektiivien tavoin (без семи столов).

\paragraph{Morfologiset substantiivit}\label{morfologiset-substantiivit}

Lukusanoiksi voidaan merkityksensä puolesta luetella myös sanat миллион,
миллиард ja тысяча, vaikka ne taipuvat kuten mitkä tahansa substantiivit
ja ilmaisevat niin suku-, luku- kuin sijakategorioita.

\paragraph{Kollektiivilukusanat}\label{kollektiivilukusanat}

Oma alauokkansa lukusanojen sisällä ovat kollektiivilukusanat
(\emph{двое}, \emph{трое} ym.) joiden käyttöympäristöt ovat melko
rajattuja. Nekään eivät ilmaise lukua eli eivät tee eroa yksikön ja
monikon välillä.

{[}tehtävä: mitä kieliopillisia kategorioita muodot ilmaisevat?{]}

Nikunlassi, Ahti 2002. \emph{Johdatus venäjän kieleen ja sen
tutkimukseen}. Helsinki: Finn Lectura.

Мусатов, Валерий 2016. \emph{Русский язык. морфемика. морфонология.
словообразование. учебное пособие}. Москва: Флинта.

\end{document}
