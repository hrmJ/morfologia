\documentclass[]{scrartcl}
\usepackage[utf8]{inputenc}
\usepackage[T1]{fontenc}
\usepackage[T2A]{fontenc}
\usepackage[finnish]{babel}
\usepackage{linguex} 
\usepackage{amsthm}
\newtheorem{maar}{Määritelmä}
\usepackage{fixltx2e} % provides \textsubscript
\usepackage{textcomp} % provides \textsubscript
\usepackage{hyperref}
\usepackage{xcolor}

\providecommand{\tightlist}{%
  \setlength{\itemsep}{0pt}\setlength{\parskip}{0pt}}

\hypersetup{
    colorlinks,
    linkcolor={red!50!black},
    citecolor={blue!50!black},
    urlcolor={blue!80!black}
}
\author{Juho Härme}
\title{Morfologia-kurssin luentomateriaaleja}
\date{\today}
\begin{document}
\maketitle
\tableofcontents
\newpage



\section{Luento 1: Mikä ihmeen
morfologia?}\label{luento-1-mikuxe4-ihmeen-morfologia}

\begin{itemize}
\tightlist
\item
  \href{http://mustikka.uta.fi/~juho_harme/morfologia/materiaalit/luento1.pdf}{Lataa
  PDF}
\item
  \href{http://mustikka.uta.fi/~juho_harme/morfologia/presentations/luento1.html}{Tutki
  luentokalvoja}
\item
  \href{http://mustikka.uta.fi/~juho_harme/morfologia/tehtavat/luento1.pdf}{Tutki
  tuntitehtäviä}
\end{itemize}

\subsection{Mitä ovat sanaluokat?}\label{mituxe4-ovat-sanaluokat}

Sanaluokan käsite esitellään meille jo ala-asteikäisenä. Sanat
muodostavat kielessä enemmän tai vähemmän selviä ryhmiä: esimerkiksi
lekseemeillä \emph{книга}, \emph{стол} ja \emph{разрушение} on jotain
yhteistä ja ne eroavat lekseemeistä \emph{играть}, \emph{слушать} ja
\emph{читать}, joilla puolestaan on keskenään jotakin yhteistä.
Sanaluokkien määrittely ei kuitenkaan ole yksiselitteistä. Mihin
sanaluokkaan mielestäsi esimerkiksi aktiivin partisiipiin preesensin
muoto ``позволяющий'' kuuluu? Entä onko sanalla \emph{тысяча} enemmän
yhteistä sanojen \emph{пять}, \emph{десять} ja \emph{семь} kanssa kuin
edellä mainittujen kolmen substantiivin kanssa? Ahti Nikunlassi(2002:
123) on käsitellyt sanaluokkajaon ongelmia laajasti. Tällä kurssilla
palautetaan mieleen perinteistä sanaluokkajakoa, mutta kehotan
muistamaan, että sanaluokat eivät ole millään tavalla ylhäältä
annettuja, objektiivisia kategorioita, vaan enemmän tai vähemmän
kompromissien tuloksena syntynyt taksonomia eli luokittelu.

Slavistisessa perinteessä on tapana erotella toisistaan itsenäiset
sanaluokat (самостоятельные части речи) ja apusanaluokat (служебные
части речи) Lisäksi luokittelua voidaan jatkaa morfologisin perustein
esimerkiksi niin, että sijoissa taipuvia sanoja (käytännössä
substantiiveja, adjektiiveja, lukusanoja, pronomineja) kutsutaan
yhteisesti \emph{nomineiksi}.

\subsubsection{Itsenäiset sanaluokat}\label{itsenuxe4iset-sanaluokat}

\begin{itemize}
\tightlist
\item
  substantiivit (существительные)
\item
  adjektiivit (прилагательные)
\item
  pronominit (местоимения)
\item
  lukusanat (числителные)
\item
  verbit (глаголы)
\item
  adverbit (наречия)
\end{itemize}

\subsubsection{Apusanaluokat}\label{apusanaluokat}

\begin{itemize}
\tightlist
\item
  prepositiot (предлоги)
\item
  konjunktiot (союзы)
\item
  partikkelit (частицы)
\item
  interjektiot (междометия)
\end{itemize}

{[}Tehtävä 1: tunnista sanaluokat{]}

\subsection{Mitä on sanan
sisällä?}\label{mituxe4-on-sanan-sisuxe4lluxe4}

Ajattele vaikka pakettia lääkepillereitä -- kuvastakoon se lausetta tai
laajemmin tekstiä. Jos sinulta kysytään, mistä lääkepaketti koostuu,
selkein vastaus olisi \emph{pillereistä}. Samaten vastaus kysymykseen
\emph{mistä lause koostuu} olisi luultavasti \emph{sanoista}. Kuitenkin
se, mikä vaikutus pillereillä on jonkin sairauden parantamisessa,
määräytyy sen mukaan, mistä ainesosista kukin pilleri on tehty. Samoin
sanojen kohdalla itse sanojen merkitys muodostuu sen perusteella, mitä
\emph{morfeja} sanassa on.

Otetaan esimerkki. Ajattele sananmuotoa \emph{выигриваю}. Jaetaan se
morfeiksi:

\begin{enumerate}
\def\labelenumi{(\arabic{enumi})}
\tightlist
\item
  /вы/игр/ыва/ю/
\end{enumerate}

Esimerkissä 1 on siis neljä morfia. Jokaisella näistä on jokin merkitys
-- jos jonkin näistä muuttaisi tai poistaisi, sanan merkitys muuttuisi.
Kokeile mielessäsi! Millä tavalla merkitys muuttuu, jos poistat
ensimmäisen morfin ja vaihdat hieman kolmatta? Entä jos vaihdat
viimeisen morfin vaikka morfiksi \emph{/ешь/}?

Kuten edellisestä käy ilmi, kielen pienintä, konkreettista yksikköä,
joka kantaa itsessään jotakin merkitystä, kutsutaan \emph{morfiksi}
(морф). Morfeja on tapana merkitä laittamalla ne kauttaviivojen väliin
(toinen vaihtoehto on erottaa morfit esimerkiksi +-merkillä).
Esimerkissä 1 viimeinen morfi /ю/ tuo verbiin sen merkityksen, että
toimintaa suorittaa yksikön ensimmäinen persoona. Sanan varsinainen
merkitys -- jota muut morfit muokkaavat -- tulee toisesta morfista,
\emph{/игр/}.

Katsotaan lisää esimerkkejä ja jaetaan ne saman tien osiksi:

\begin{enumerate}
\def\labelenumi{(\arabic{enumi})}
\setcounter{enumi}{1}
\tightlist
\item
  /пек/у/
\item
  /печ/ёшь/
\end{enumerate}

Kummassakin yllä olevista esimerkeistä on kaksi morfia. Olet varmasti
kuullut myös sanan \emph{morfeemi} (морфема). Siinä missä morfi on
konkreettinen, yksittäinen esiintymä, morfeemi on abstrakti yksikkö,
joka voi esiintyä useampana erilaisena morfina. Esimerkeissä 2 ja 3 on
kummassakin morfeemi, jota voitaisiin merkitä esimerkiksi \{пек\} tai
\{LEIPOA\}{[}\^{}morfmerk{]}. Tämän morfeemin konkreettisia esiintymiä
ovat morfit /пек/ ja /печ/. Kyseessä on siis yhden ja saman morfeemin
kaksi konkreettista esiintymää, vähän niin kuin jalkapallojoukkueessa
\{maalivahti\}-``morfeemia'' voivat edustaa ``morfit'' /Joe Hart/ ja
/Manuel Neuer/.

{[}\^{}morfmerk{]} Kuten huomasit, morfeemeja merkitään asettamalla ne
aaltosulkujen sisään. Jos morfeemiin viitataan sen merkityksellä, se on
tapana kirjoittaa isolla.

Usein on niin, että jokin tietty morfeemi edustuu aina vain yhdellä,
tietyllä tavalla. Tällöin on käytännössä melkeinpä sama, puhutaanko
morfeemista vai morfista, mutta teoriassa erottelu toki tulee edelleen
pitää mielessä. Jos mietitään esimerkin 1 /игр/-morfia, voidaan todeta
juuri näin: morfeemia \{игр\} edustaa vain yksi morfi, /игр/. Morfeista
/пек/ ja /печ/ voidaan puolestaan vielä sanoa, että ne ovat toistensa
\emph{allomorfeja}. Allomorfin käsite kannattaa painaa mieleen
nimenomaan suhteellisena käsitteenä: yksittäinen morfi voi muodostaa
allomorfiparin tai kolmikon suhteessa joihinkin muihin morfeihin.

Jäitkö miettimään, mitä morfeemeja esimerkkien 2 ja 3 toiset morfit (/у/
ja /ёшь/) edustavat? Kysymys on monimutkaisempi kuin ensi katsomalta
saattaisi olettaa. Liian informaatiotulvan välttämämiseksi siirrän
vastauksen tuonnemmaksi ja palaan siihen, kun käsittelemme tarkemmin
verbejä sanaluokkana. Lisälukemiseksi tässä esitetyistä käsitteistä ks.
esimerkiksi (Мусатов 2016: 20; Nikunlassi 2002: 104, Koivisto (2013):
66).

Pohditaan nyt vähän tarkemmin sanojen jakamista morfeihin.

{[}tehtävä 2: erottele morfeemeja{]}

Älä pelästy, vaikka morfien erottelu ei aina ole selkeää. Voi olla, että
sanasta voisi löytyä enemmän osasia kuin luulit tai voi myös olla, että
pilkot sanan niin pieniksi palasiksi, ettei niillä kaikilla
todellisuudessa ole omaa merkitystä. Kurssin kuluessa erilaisiin
morfeemeihin tutustutaan paremmin ja tätä myötä myös kyky havaita niitä
paranee.

\subsubsection{Morfien lajeja}\label{morfien-lajeja}

Edellä sanoista puhuttaessa havaittiin, että niiden voi nähdä
muodostavan erilaisia ryhmiä. Myös morfeja tutkittaessa huomataan, että
tietyillä morfeilla on jotain yhteistä, joka taas erottaa ne joistakin
muista morfeista.

Esimerkiksi edellä tarkemmin käsitellyissä morfeissa /игр/ ja /пек/ on
selkeästi jotakin yhteistä, jotakin erilaista verrattuna morfeihin /вы/,
/ыва/, /ёшь/ jne. Ensinmainitut morfit voidaan luokitella
\emph{juurimorfeiksi} (корневой морф, корень). Ne ilmaisevat sanan
varsinaisen leksikaalisen merkityksen: voisi sanoa, että ilman niitä
muut morfit ovat turhia. Muut morfit nimittäin eivät ilmaise
leksikaalista vaan \emph{kieliopillista merkitystä}. Esimerkiksi
esimerkin 3 /ёшь/-morfi ilmaisee toista persoonaa, indikatiivia ja
ei-mennyttä aikaa. Nämä ovat hyvin erityyppisiä merkityksiä kuin
juurimorfin /пек/ leksikaalinen merkitys. Kieliopilliset morfit voi
jakaa moniin alaluokkiin, joita tutkitaan tarkemmin seuraavalla
luennolla. Näitä alalajeja ovat ennen kaikkea prefiksit, suffiksit,
interefiksit, postfiksit ja taivutuspäätteet.

\subsection{Kieliopilliset kategoriat}\label{kieliopilliset-kategoriat}

Edellä puhuttiin \emph{toisesta persoonasta}, \emph{indikatiivista} ja
\emph{menneestä ajasta}. Nämä ovat asioita, joita edellä käsitellyt
morfeemit ilmaisevat. Tutkitaan lisää vähän toisenlaisia sanoja:

\begin{enumerate}
\def\labelenumi{(\arabic{enumi})}
\setcounter{enumi}{3}
\tightlist
\item
  /стол/ами/
\item
  /крив/ая/
\end{enumerate}

Esimerkin 4 /ами/-morfista voidaan sanoa, että se ilmaisee toisalta
monikkoa (sitä, että pöytiä on enemmän kuin yksi), toisaalta
instrumentaalisijaa. Esimerkin 5 /ая/-morfi ilmaisee puolestaan
yksikköä, nominatiivisijaa ja lisäksi feminiinisukua.

Huomataan siis, että vaikka kieliopilliset morfit eivät sinänsä pysty
esiintymään yksinään, nillä silti ilmaistaan monenlaisia asioita.
``Asia, jota morfeemilla ilmaistaan'' ei ole erityisen kätevä nimitys,
mutta onneksi näihin voidaan viitata kätevämmin termillä
\emph{kieliopillinen kategoria} (грамматическая категория). Venäjässä
ilmaistavia kieliopillisia kategorioita ovat ainakin

\begin{itemize}
\tightlist
\item
  suku
\item
  luku
\item
  sija
\item
  elollisuus
\item
  pääluokka
\item
  persoona
\item
  aspekti
\item
  tapaluokka
\item
  aikamuoto
\end{itemize}

Jos mietitään käsitteitä \emph{instrumentaali}, \emph{toinen persoona}
tai \emph{feminiini}, voidaan sanoa, että ne ovat \emph{kieliopillisten
kategorioiden arvoja}. Palatakseni jalkapallojoukkuemetaforaan
voitaisiin kuvitella \emph{pelipaikan} kieliopillinen kategoria ja sille
arvot \emph{maalivahti}, \emph{puolustaja}, \emph{hyökkääjä} jne. Yksi
tämän kurssin tehtävistä on tutustua näihin kategorioihin ja
konkreettisiin tapoihin, joilla kategorioiden eri arvoja eri sanoilla
ilmaistaan. Heti alkuun voidaan todeta, että läheskään kaikki sanat
eivät ilmaise tai edes voi ilmaista kaikkia kategorioita. Toiset
kategoriat ovat tyypillisiä esimerkiksi substantiiveille; joitain
kategorioita ilmaisevat vain verbit.

On huomattava, että yksi morfi voi ilmaista useaa eri kieliopillista
kategoriaa. Venäjässä tämä on enemmän sääntö kuin poikkeus (vrt. kaikkia
edellä käsiteltyjä kieliopillisia morfeja!), ja tällaisille morfeille on
oma nimityksensäkin: salkkumorfi (гибридный морф) (ks. Nikunlassi 2002,
106--7).

\subsection{Millainen kieli venäjä on morfologian
kannalta?}\label{millainen-kieli-venuxe4juxe4-on-morfologian-kannalta}

Kielten luokittelua erilaisten yhtenevien ja eriävien ominaisuuksien
perusteella kutsutaan kielitypologiaksi. Kun kieliä luokitellaan
morfologisten ominaisuuksien perusteella, on perinteisesti jaoteltu
kieliä toisaalta \emph{synteettiisiin ja analyyttisiin}.

\begin{itemize}
\tightlist
\item
  Analyyttisiksi kutsutaan kieliä, jossa kieliopillisia suhteita
  ilmaistaan kieliopillisten sanojen (prepositiot, partikkelit, ym.) ja
  sanajärjestyksen avulla
\item
  Synteettisiksi sanotaan kieliä, jossa kieliopillisia suhteita
  ilmaistaan taivutuspäätteiden avulla (vrt. Nikunlassi 2002, 119)
\end{itemize}

Kumpaan ryhmään venäjä mielestäsi kuuluu?

\subsection{Morfologian osa-alueita}\label{morfologian-osa-alueita}

\begin{itemize}
\item
  Sananmuodostusoppi (словообразование) ja taivutusoppi (словоизменение)
\item
  Morfofonologia(морфонология), Morfosyntaksi (морфосинтаксис),
  Funktionaalinen morfologia (функциональная морфология)
\end{itemize}

\hyperdef{}{ref-koivisto2013suomen}{\label{ref-koivisto2013suomen}}
Koivisto, Vesa. 2013. \emph{Suomen Sanojen Rakenne}. Suomalaisen
Kirjallisuuden Seura.

\hyperdef{}{ref-nikunl}{\label{ref-nikunl}}
Nikunlassi, Ahti. 2002. \emph{Johdatus Venäjän Kieleen Ja Sen
Tutkimukseen}. Helsinki: Finn Lectura.

\hyperdef{}{ref-musatov2016}{\label{ref-musatov2016}}
Мусатов, Валерий. 2016. \emph{Русский Язык. Морфемика. Морфонология.
Словообразование. Учебное Пособие}. Москва: Флинта.

\end{document}
