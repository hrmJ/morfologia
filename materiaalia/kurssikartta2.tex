\documentclass[finnish]{standalone}
\usepackage{forest}
\usepackage[utf8]{inputenc}

\usetikzlibrary{mindmap}
\usepackage[T1]{fontenc}
\usepackage[T2A]{fontenc}
\usepackage[finnish]{babel}
\begin{document}

\begin{forest}
  for tree={
    child anchor=west,
    parent anchor=east,
    grow'=east,
  %minimum size=1cm,%new possibility
  %text width=8cm,%
    %draw,
    anchor=west,
    edge path={
      \noexpand\path[\forestoption{edge}]
        (.child anchor) -| +(-5pt,0) -- +(-5pt,0) |-
        (!u.parent anchor)\forestoption{edge label};
    },
  }
[Venäjän morfologiaa
    [venäjä morfologisessa luokittelussa
        [{synteettinen, mutta analyyttisia piirteitä}, text width=3.5cm]
    ]
    [Kieliopilliset kategoriat
        [{=ominaisuus/merkitys, jonka suhteen muodot oppositiossa}, text width=3cm]
        [Kategorioita
            [etenkin nomineilla
                [suku, name=suku
                    [3 arvoa, name=sukuarv
                        [maskuliini, name=sukum]
                        [feminiini]
                        [neutri]
                    ]
                ]
                [luku, name=luku
                    [2 arvoa
                        [yksikkö]
                        [monikko]
                    ]
                ]
                [sija, name=sija
                    [substantiivien deklinaatiot
                        [1.]
                        [2.]
                        [3.]
                        [4.]
                        [5.]
                        [Lekseemikohtaiset poikkeukset ja sekadeklinaatiot, text width=3cm]
                    ]
                    [Tavalliset 6 arvoa
                        [nominatiivi]
                        [genetiivi]
                        [datiivi]
                        [akkusatiivi]
                        [instrumentaali]
                        [prepositionaali]
                    ]
                    [harvinaisemmat arvot
                        [partitiivinen genetiivi]
                        [lokatiivi]
                        [adnumeratiivi]
                        [vokatiivi]
                    ]
                ]
                [elollisuus
                    [etenkin akkusatiivin kannalta]
                    [2 arvoa
                        [elollinen]
                        [eloton]
                    ]
                ]
            ]
            [etenkin verbeillä, name=verbs
                [pääluokka
                    [2 arvoa
                        [aktiivi]
                        [passiivi]
                    ]
                ]
                [persoona, name=vpers
                    [verbien konjugaatiot]
                    [3 arvoa
                        [1. pers.]
                        [2. pers.]
                        [3. pers.]
                    ]
                ]
                [tapaluokka
                    [3 arvoa
                        [indikatiivi]
                        [imperatiivi]
                        [konditionaali]
                    ]
                ]
                [aikamuoto
                    [2 pääarvoa
                        [mennyt aika, name=vpast]
                        [ei-mennyt aika
                            [preesensmerkitys] 
                            [futuurimerkitys
                                [yksinkertainen futuuri]
                                [liittofutuuri]
                            ]
                        ]
                    ]
                ]
                [aspekti
                    [2 arvoa
                        [perfektiivinen
                            [erityismerkityksiä]
                        ]
                        [imperfektiivinen
                            [erityismerkityksiä]
                        ]
                    ]
                ]
                [?toiminnan luonne
                ]
            ]
        ]
    ]
    [erityisiä verbimuotoja
        [infinitiivi]
        [partisiipit, name=vpart
            [aktiivin partisiippi
                [preesens]
                [preteriti]
            ]
            [passiivin partisiippi
                [preesens]
                [preteriti]
            ]
        ]
        [gerundit
            [imperfektiivisen aspektin]
            [perfektiivisen aspektin]
        ]
        [teonlaadut]
    ]
]
\draw[->,dotted] (suku) to[out=east, in=north east] (vpast);
\draw[->,dotted] (luku) to[out=south, in=west] (vpers);
\draw[->,dotted] (sija) to[out=south, in=north] (vpart);
\end{forest}



\end{document}
