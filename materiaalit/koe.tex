\documentclass[paper=a4, fontsize=11pt]{scrartcl} 
\usepackage[a4paper,margin=3cm]{geometry}
\usepackage[utf8]{inputenc}
\usepackage[T2A]{fontenc}
\usepackage{soulutf8}

\usepackage{lmodern}

\usepackage{titlesec}
\usepackage{tabularx}
\pagenumbering{gobble}
\usepackage{arydshln}
\usepackage{multirow}
\usepackage{pdflscape}
\usepackage{longtable} 
\usepackage{booktabs}

\usepackage{bm,array}

\newcommand{\answer}[2][3cm]{\raisebox{-\baselineskip}{\shortstack{\underline{\hspace{#1}}\\#2}}}

\usepackage{sectsty} % Allows customizing section commands
\renewcommand{\thesubsection}{\alph{subsection}}
\renewcommand{\thesection}{Tehtävä \arabic{section}}
\allsectionsfont{\normalfont\sffamily\bfseries}


\usepackage{setspace}
\usepackage{enumitem}
\sectionfont{\large}
\subsectionfont{\normalsize}
\subsubsectionfont{\normalsize}

\date{}
\usepackage[normalem]{ulem}

\newcommand{\horrule}[1]{\rule{\linewidth}{#1}} % Create horizontal rule command with 1 argument of height
\title{	
\normalfont \normalsize 
\horrule{0.5pt} \\[0.4cm] 
\huge  Koe / Morfologia/ 7.12.2017  \\
\horrule{2pt} \\
\vspace{-2em}
}


\usepackage{titling}
\setlength{\droptitle}{-10em}   



\begin{document}

\maketitle
\answer[9cm]{Nimi ja opiskelijanumero}\\[1cm]

\section{}

Erottele seuraavista sanoista

\subsection{neliöllä  suffiksit ja prefiksit}
\subsection{ympyrällä taivutuspäätteet}
\subsection{alleviivaamalla postfiksit}

Huom: sanoissa voi kaikkia mainittuja aineksia olla joko nolla kappaletta tai enemmän\\[0.5cm]

Esimerkki: \framebox[1.1\width]{пере}чит\framebox[1.1\width]{ыва}\textcircled{ть}\uline{ся}\\[0.5cm]


\large
\begin{enumerate}
    \setlength\itemsep{1em}
    \item \so{задерживающегося}
    \item \so{прочитать}
    \item \so{соавтор}
    \item \so{лесах}
    \item \so{место}
    \item \so{супервысокому}
    \item \so{заслуженную}
    \item \so{проигривателями}
    \item \so{прийдя}
    \item \so{умывшись}
\end{enumerate}

\subsection{Miten \emph{vartalon} käsite eroaa \emph{juurimorfin} käsitteestä? (Anna ainakin yksi esimerkki)}

\underline{\hspace{\textwidth}} \\
\underline{\hspace{\textwidth}} \\
\underline{\hspace{\textwidth}} \\


\section{} 

Substantiiveilla voidaan erottaa kolme monikon taivutuspäätesarjaa eli
deklinaatiota. Ryhmittele seuraavat sanat niin, että samaa monikon
deklinaatiota edustavat sanat ovat samassa ryhmässä. Sillä ei ole väliä, mitä
deklinaatiota nimität ensimmäiseksi, mitä toiseksi jne.\\[.5cm]

выключатель, баня, тело, знаменитость, семья, игрок, канава, ноутбук,
носталгия, знание, пропасть, шантаж, слово, племя, книга, завод, скрипач,
дверь, тетрадь, дельфинарий, изменение, снасть, сон\\[.5cm]

\begin{tabular}[C]{p{5cm}|p{5cm}|p{5cm}}
1. deklinaatio & 2. deklinaatio & 3.deklinaatio  \\ \toprule
               &                &                \\
               &                &                \\
               &                &                \\
               &                &                \\
               &                &                \\
               &                &                \\
               &                &                \\
               &                &                \\
               &                &                \\
               &                &                \\
               &                &                \\
               &                &                \\ \bottomrule
\end{tabular}


\section{Muuta seuraavat aktiivilauseet passiiviin. Muuta subjektit passiivilauseen agenteiksi.}

Esimerkki: \emph{Милиционеры арестовали Ходорковского. $\rightarrow$ Ходорковский был арестован милиционерами.}

\begin{enumerate}
    \setlength\itemsep{1em}
    \item Ученые доказали, что так называемая темная материя на самом деле существует.\\[.3cm] \underline{\hspace{\textwidth}}
    \item Вы пропустили слишком много ошибок\\[.3cm] \underline{\hspace{\textwidth}}
    \item Студенты подняли вопрос о том, почему курсы заканчиваются уже до первой половины декабря.\\[.3cm] \underline{\hspace{\textwidth}}
    \item В качестве молотка мы использовали Тольковый словарь Ожегова.\\[.3cm] \underline{\hspace{\textwidth}}
    \item Подобные вопросы члены городского совета решают быстро и без споров. \\[.3cm] \underline{\hspace{\textwidth}}
    \item Сеть магазинов "Лента" снизила цены на молочные продукты \\[.3cm] \underline{\hspace{\textwidth}}
    \item Все маназины повысили цены на фрукты.\\[.3cm] \underline{\hspace{\textwidth}}
    \item Окно разбили местные бандиты. \\[.3cm] \underline{\hspace{\textwidth}}
    \item На следующий день полиция задержала лидера группировки. \\[.3cm] \underline{\hspace{\textwidth}}
\end{enumerate}


\section{}

Niin kutsuttu \emph{nollapääte} on monen morfologisen ilmiön selittämisessä käyttökelpoinen käsite. Selitä esimerkein, miten nollapääte liittyy

\subsection{verbien imperatiivimuotoihin}
    \underline{\hspace{\textwidth}}\\[.3cm] 
    \underline{\hspace{\textwidth}}\\[.3cm] 
    \underline{\hspace{\textwidth}}\\[.3cm] 


\subsection{substantiivien sijamuotoihin}
    \underline{\hspace{\textwidth}}\\[.3cm] 
    \underline{\hspace{\textwidth}}\\[.3cm] 
    \underline{\hspace{\textwidth}}\\[.3cm] 


\section{Lisää verbin täydennys tai subjekti oikeassa sijassa}


\section{Valitse sopivampi aspekti ympyröimällä}
\begin{enumerate}
    \item Заведи будильник, иначе ты можешь просыпать  / проспать.
    \item Почему ты еще не вставал / встал? Ведь уже поздно.
    \item Я вставал / встал, но опять лег, потому что у меня разболелась голова.
    \item Она просит не зажечь / зажигать свет.
    \item Ты когда-нибудь слышал / услышал, какие звуки издаёт фагот? 
    \item Мы посидели / сидели немного, вставали / встали и пошли дальше.
    \item Соседи весь вечер шумели / зашумели за стеной.
    \item Я несколько раз перечитывал / перечитал роман Пастернака "Доктор Живаго".
    \item В зале было тихо. Выступал  / выступил известный писатель.
\end{enumerate}

\section{Selitä vähintään kaksi edellisistä aspektitapauksista käyttämällä jotakin seuraavista perus-/erityismerkityksistä}

{\small
\begin{itemize}
    \item Konkreettis-prosessuaalinen merkitys
    \item Konkreettis-faktinen merkitys
    \item Yleisesti toteava merkitys
    \item Summatiivinen merkitys
    \item Rajoittamattoman toiston merkitys
    \item perfektin erityismerkitys
\end{itemize}
}

\underline{\hspace{\textwidth}} \\[.3cm]
\underline{\hspace{\textwidth}} \\[.3cm]
\underline{\hspace{\textwidth}} \\[.3cm]
\underline{\hspace{\textwidth}} \\[.3cm]



\section{Käännä.}

%Ryhmittele seuraavat substantiivit painotyypin perusteella
%
%Lisää verbin täydennys tai subjekti oikeassa sijassa
%
%Jotain passiivilauseista
%
%Jotain liikeverbeistä + aspektista... pieni perustelutehtävä, jossa erityismerkityksiä?
%
%
%
%Täydennä sopiva muoto:
%    - verbien taivuttamista konjugaatioissa + helppoa aspektinvalintaa
%    - joku yksittäinen gerunditapaus
%    - davai-lauseita
%
%Käännä 
%    - (elollisuuteen liittyviä?)
%    - joitain na- ja v-juttuja?
%    - konditionaaliin liittyviä

\end{document}
