
\documentclass[]{scrartcl}
\usepackage[utf8]{inputenc}
\usepackage[T1]{fontenc}
\usepackage[T2A]{fontenc}
\usepackage[finnish]{babel}
\usepackage{linguex} 
\usepackage{longtable} 
\usepackage{booktabs}
\usepackage{amsthm}
\usepackage{graphicx}
\newtheorem{maar}{Määritelmä}
\usepackage{fixltx2e} % provides \textsubscript
\usepackage{textcomp} % provides \textsubscript
\usepackage{hyperref}
\usepackage{xcolor}

\providecommand{\tightlist}{%
  \setlength{\itemsep}{0pt}\setlength{\parskip}{0pt}}

\hypersetup{
    colorlinks,
    linkcolor={red!50!black},
    citecolor={blue!50!black},
    urlcolor={blue!80!black}
}
\author{Juho Härme}
\title{Morfologia-kurssin luentomateriaaleja}
\date{\today}
\begin{document}
\maketitle
\tableofcontents
\newpage



\section{Luento 20: liikeverbit 1}\label{luento-20-liikeverbit-1}

\begin{itemize}
\tightlist
\item
  \href{https://mustikka.uta.fi/~juho_harme/morfologia/\#tästä-kurssista}{Takaisin
  sivun ylälaitaan}
\item
  \href{http://mustikka.uta.fi/~juho_harme/morfologia/materiaalit/luento20.pdf}{Lataa
  PDF}
\item
  \href{http://mustikka.uta.fi/~juho_harme/morfologia/presentations/luento20.html}{Tutki
  luentokalvoja}
\item
  \href{http://mustikka.uta.fi/~juho_harme/morfologia/tehtavat/luento20.pdf}{Tutki
  tuntitehtäviä}
\end{itemize}

Venäjän verbijärjestelmästä erottuu omaksi ryhmäkseen eräiden
liikkumista ilmaisevien verbien joukko, jota kielenopetuksessa ja
tutkimuskirjallisuudessa nimitetään \emph{liikeverbeiksi} (глаголы
движения). Liikeverbeiksi ei lueta kaikkia mahdollisia liikkumista
ilmaisevia verbejä (ei esimerkiksi `двигаться', `путешествовать',
`перемещаться'), vaan sellaiset verbit, jotka tuntuvat muodostavan
tietynlaisen \emph{parin} toisen vastaavaa liikettä tarkoittavan verbin
kanssa. Tällaisen parin muodostavat esimerkiksi verbit \emph{идти} ja
\emph{ходить}. Mikä sitten näissä verbeissä on sellaista, että ne
jossain mielessä kuuluvat yhteen?

A.V. Isachenko (2003{[}1965{]}, 310) esittää seuraavan pohdinnan
verbeihin \emph{идти} ja \emph{ходить} liittyen ja toteaa, että näistä
verbeistä tiedetään:

\begin{enumerate}
\def\labelenumi{\arabic{enumi}.}
\tightlist
\item
  Että molemmat ovat semanttisesti hyvin lähellä toisiaan (merkitsevät
  muun muassa liikettä kävellen).
\item
  Että molemmat ovat imperfektiivisen aspektin verbejä
\item
  Että kumpikaan ei edusta mitään erityistä teonlaatua
\item
  Että verbit tuntuvat jollain tavalla muodostavan parin, jota vastaavia
  voidaan löytää myös tietyillä muilla, kuitenkin tarkkaan rajatuilla
  verbeillä.
\item
  Tällaisten verbiparien ryhmä on tiukkaan rajattu ja epäproduktiivinen:
  Kieleen ei ilmesty uusia vastaavia verbipareja, vaan päinvastoin
  jotkin olemassaolevista pareista alkavat käytöltään hiipua.
\item
  Vaikka verbit muodostavat jonkinlaisen parin, ei ole mahdollista
  sanoa, että toinen verbi olisi johdettu toisesta niin kuin esimerkiksi
  perfektivaation ja imperfektivaation tapauksissa.
\end{enumerate}

Näiden verbien välinen ero ei ole siis aspektissa, mutta verbit
kuitenkin muodostavat jonkin ominaisuuden suhteen parin. Isachenko
nimittää tätä erottavaa ominaisuutta termillä \emph{характер действия}
(toiminnan luonne).

Mikä sitten on идти- ja ходить-verbien välinen ero? Voidaan esittää,
että идти-tyypin verbit \emph{ilmaisevat toiminnan yksisuuntaisena},
mutta ходить-tyypin verbit \emph{eivät ota suoraan kantaa siihen, kuinka
moneen suuntaan toimintaa suoritetaan}. Идти-tyypin verbejä voidaan
nimittää \emph{duratiivisiki} ja ходить-tyypin verbejä
\emph{iteratiivisiksi} (направленные/ненаправленные глаголы движения).

Katsotaan tähän liittyen esimerkkejä:

\begin{enumerate}
\def\labelenumi{(\arabic{enumi})}
\tightlist
\item
  Маша идет по тропинке
\end{enumerate}

Esimerkissä 1 kuvataan, kuinka Maša etenee yhteen suuntaan polkua
pitkin. Verrataan tätä esimerkkiin 2:

\begin{enumerate}
\def\labelenumi{(\arabic{enumi})}
\setcounter{enumi}{1}
\tightlist
\item
  Маша ходит по комнате
\end{enumerate}

Esimerkissä 2 kuvataan, miten Maša kävelee sinne tänne, useisiin
suuntiin, huoneessa. Toisaalta esimerkissä 3

\begin{enumerate}
\def\labelenumi{(\arabic{enumi})}
\setcounter{enumi}{2}
\tightlist
\item
  Маша \emph{ходит} в школу через парк
\end{enumerate}

kuvataan tarkalleen kaksisuuntaista liikettä (Maša menee kouluun ja
tulee koulusta tiettyä reittiä) ja esimerkissä 4 ei puhuta suunnasta
sinänsä mitään, vaan ennemminkin kyvystä:

\begin{enumerate}
\def\labelenumi{(\arabic{enumi})}
\setcounter{enumi}{3}
\tightlist
\item
  Больной уже \emph{ходит}.
\end{enumerate}

Voidaan siis todeta, että ходить-tyypin verbeillä kysymykseen toiminnan
luonteesta ei oteta samalla tavalla kantaa kuin идти-tyypin verbeillä --
ne ovat tämän oppositioparin neutraali, tunnusmerkitön osapuoli.

\subsection{Liikeverbiparit}\label{liikeverbiparit}

Edellä mainittiin, että liikkeen suunnan perusteella paritettavien
verbien määrä on rajattu, eikä ryhmään enää tule uusia verbejä. Mitä
pareja sitten ylipäätään on olemassa?

Seuraavissa taulukoissa on esitetty merkittävimmät duratiiviset ja
iteratiiviset liikeverbit. Taulukoiden kolmanteen sarakkeeseen on
merkitty, onko kyseessä transitiivinen (suoran objektin saava) vai
intransitiivinen (ilman objektia käytettävä) verbi. Liikeverbit on
jaettu kahteen taulukkoon, niin että ensimmäisessä ovat (oletettavasti)
tavallisimmin käytetyt verbit, toisessa harvinaisemmat.

\subsubsection{Tavallisimpia
liikeverbipareja}\label{tavallisimpia-liikeverbipareja}

\begin{longtable}[c]{@{}lll@{}}
\toprule
duratiivinen verbi & iteratiivinen verbi &
transitiivisuus\tabularnewline
\midrule
\endhead
идти & ходить & intransitiivinen\tabularnewline
бежать & бегать & intransitiivinen\tabularnewline
нести & носить & transitiivinen\tabularnewline
вести & водить & transitiivinen\tabularnewline
ехать & ездить & intransitiivinen\tabularnewline
лететь & летать & intransitiivinen\tabularnewline
плыть & плавать & intransitiivinen\tabularnewline
везти & возить & transitiivinen\tabularnewline
\bottomrule
\end{longtable}

идти-ходить-, бежать-бегать-, нести-носить- ja вести-водить-verbipareja
yhdistää kaikkia se, että ne ilmaisevat liikettä ilman kulkuneuvoa (tai
itse kulkuneuvon liikettä, mistä jäljempänä tarkemmin), kun taas
ехать-ездить-, лететь-летать-, плыть-плавать- ja везти-возить-verbejä
yhdistää se, että niillä tavallisesti (joskaan esimerkiksi
плыть-плавать-parilla ei toki aina) ilmaistaan liikettä kulkuneuvolla.

\subsection{Harvinaisempia
liikeverbipareja}\label{harvinaisempia-liikeverbipareja}

\begin{longtable}[c]{@{}lll@{}}
\toprule
duratiivinen verbi & iteratiivinen verbi &
transitiivisuus\tabularnewline
\midrule
\endhead
лезть & ла́зить & intransitiivinen\tabularnewline
ползти & по́лзать & intransitiivinen\tabularnewline
гнать & гоня́ть & transitiivinen\tabularnewline
катѝть & ката́ть & transitiivinen\tabularnewline
тащѝть & таска́ть & transitiivinen\tabularnewline
*брестѝ & бродѝть & intransitiivinen\tabularnewline
\bottomrule
\end{longtable}

Vaikka nämä verbit on luokiteltu harvinaisemmiksi kuin edellä mainitut,
ovat ne silti käytössä, ja koko venäjän verbikategoriaa tarkasteltaessa
kuitenkin verrattain yleisiä. Verbipari брести-бродить yleensä
ilmoitetaan yhtenä liikeverbiparina, mutta Isachenko (2003{[}1965{]},
313) ei pidä sitä varsinaisena liikeverbiparina, koska verbien
merkitykset eroavat toisistaan muutenkin kuin vain liikkeen suunnan
kuvaamisen suhteen.

\subsection{Huomioita käytöstä}\label{huomioita-kuxe4ytuxf6stuxe4}

Kieltenoppijan kannalta hankalia ovat yleensä tilanteet, joissa oma
äidinkieli kuvaa jonkin ilmiön enemmän tai vähemmän ylimalkaisesti,
mutta opiskeltava kieli tekee tarkempia erotteluja. Tätä ongelmaa voi
kuvata ongelmaksi \emph{tiheän} ja \emph{harvan} käsiteverkon välillä.

Liikeverbien suhteen suomen käsiteverkon voi hyvin todeta harvaksi
verrattuna venäjän vastaavaan: suomessa usein pärjää pelkästään sanoilla
``mennä'' tai ``tulla'', jotka eivät ota kantaa liikkeen suuntaisuuteen
tai siihen, tehdäänkö se kulkuneuvolla vai jalan. Tämän vuoksi venäjän
järjestelmä, jossa nämä asiat on välttämättä ratkaistava, aiheuttaa
suomenkieliselle venäjänopiskelijalle vaikeuksia: suomen lausetta
``Milloin sinä tulit'' voisi periaatteessa vastata venäjäksi mikä vain
esimerkiksi lauseista ``Когда ты пришел/приехал/прилетел/приплыл''.

Tässä yhteydessä kannattaa huomata, että suomalainen venäjänopiskelija
käyttää helposti \emph{ехать}-verbiä liian usein. Siinä missä suomessa
mennään ja tullaan yhtä lailla laivalla, junalla kuin lentokoneella,
venäjässä kulkuvälineen tyyppi otetaan yleensä tarkemmin huomioon.
Vertaa seuraavia lauseita:

\begin{enumerate}
\def\labelenumi{(\arabic{enumi})}
\setcounter{enumi}{4}
\tightlist
\item
  Для этого они с Дарьей заблаговременно \emph{приплыли} в Нью-Йорк, но
  не стали останавливаться в отеле, а решили остаться на борту яхты.
\item
  Президент России Владимир Путин \emph{прилетел} в Австрию с мирными
  инициативами
\item
  Два месяца назад я \emph{приехал} в Питер с мечтой о лучшей жизни.
\end{enumerate}

Jokainen esimerkkien 5 -- 7 lauseista voitaisiin suomentaa käyttämällä
tulla- tai saapua-verbejä -- suomeksi jopa olisi jokseenkin outoa, jos
lauseisiin 5 ja 6 liitettäisiin erikseen tietoa kulkuneuvosta, jolla
ollaan tultu (``He saapuivat laivalla New Yorkiin'' tai ``Venäjän
presidentti saapui lentäen itävaltaan''). Etenkin vesitse tapahtuvan
kulkemisen kuvaaminen плыть-verbillä on usein suomenkielisen kannalta
odottamatonta.

Toinen kulkuvälineisiin liittyvä tarkennus liikeverbien käytöstä koskee
kulkuvälineiden itsensä liikettä. Kun kuvataan maalla liikkuvien
kulkuvälineiden liikettä ennemmin kuin ihmisten liikkumista
kulkuvälineillä, käytetään tavallisesti идти-verbiä, kuten seuraavissa
esimerkeissä:

\begin{enumerate}
\def\labelenumi{(\arabic{enumi})}
\setcounter{enumi}{7}
\tightlist
\item
  Поезд шёл через ночной город.
\item
  Поезд идёт, останавливается, долго стоит, снова идёт
\item
  К счастью, машина шла очень медленно
\item
  Автобус шел бесконечно долго, и от остановки надо было еще бесконечно
  долго идти пешком.
\end{enumerate}

\hyperdef{}{ref-isachenko2003}{\label{ref-isachenko2003}}
Исаченко, Александр. 2003{[}1965{]}. \emph{Грамматический Строй Русского
Языка В Сопоставлении С Словацким. Морфология. Часть 1, 2}. Москва:
Языки славянской культуры.
\url{https://www.litres.ru/a-v-isachenko/grammaticheskiy-stroy-russkogo-yazyka-v-sopostavlenii-s-slovackim-morfologiya-chast-1-2/}.

\end{document}
