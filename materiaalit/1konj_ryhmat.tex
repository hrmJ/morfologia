\documentclass[]{scrartcl}
\usepackage[utf8]{inputenc}
\usepackage[T1]{fontenc}
\usepackage[T2A]{fontenc}
\usepackage[finnish]{babel}
\usepackage{linguex} 
\usepackage{longtable} 
\usepackage{booktabs}
\usepackage{amsthm}
\usepackage{graphicx}
\newtheorem{maar}{Määritelmä}
\usepackage{fixltx2e} % provides \textsubscript
\usepackage{textcomp} % provides \textsubscript
\usepackage{hyperref}
\usepackage{xcolor}

\providecommand{\tightlist}{%
  \setlength{\itemsep}{0pt}\setlength{\parskip}{0pt}}

\hypersetup{
    colorlinks,
    linkcolor={red!50!black},
    citecolor={blue!50!black},
    urlcolor={blue!80!black}
}
\author{Juho Härme}
\title{Morfologia-kurssin luentomateriaaleja}
\date{\today}
\begin{document}

\section*{Infinitiivi- ja preesensvartaloiden suhteista ensimmäisessä
deklinaatiossa}\label{infinitiivi--ja-preesensvartaloiden-suhteista-ensimmuxe4isessuxe4-deklinaatiossa}

Tarkastellaan vielä ensimmäisen deklinaation tärkeimpiä sisäisiä
sanaryhmiä, joita tässä erotellaan seitsemän.

\subsection*{1. ryhmä}\label{ryhmuxe4}

Produktiivisin ja ylivoimaisesti suurin ryhmä ovat verbit tyyppiä
\emph{читать}, joissa infinitiivivartalon lopussa on jokin vokaaleista
а, е, и/ы tai у (Шелякин 2006: 127). Yhteistä tälle ryhmälle on, että
preesensvartalo loppuu /j/-äänteeseen: читать - читаj-, покраснеть -
покраснеj, дуть - дуj. Tämän lisäksi infinitiivivartalon viimeinen
vokaali on preesensvartalossa joskus eri: открыть - откроj, мыть - моj,
брить - брej sekä и--j-vaihtelu sanoissa ltyyppiä /бить/--/бью/т
(foneettisesti: /би/т'/--/бj/ут/).

\subsection*{2. ryhmä}\label{ryhmuxe4-1}

Toisen 1. konjugaation sisäisen ryhmän muodostavat ne tutut verbit,
joiden infinitiivivartalo päättyy /ова/ (tai /ева/) ja preesensvartalo
/уj/: /интрес/ова/ -- /интерес/уj/, /жева/ -- /жуj/ (жевать, `purra').

\subsection*{3. ryhmä}\label{ryhmuxe4-2}

Hieman edellisen kaltainen ryhmä ovat verbit tyyppiä давать. Näillä
infinitiivivartalon loppu on muotoa /ава/ ja preesensvartalo /аj/.
Ryhmään kuuluvat esimerkiksi /признава/ть/ -- /призна/j/, /устава/ть/ --
/устаj/.

\subsection*{4. ryhmä}\label{ryhmuxe4-3}

Seuraavaksi ryhmäksi voidaan erottaa verbit tyyppiä \emph{писать}.
Näissä infinitiivivartalo päättyy /а/-vokaaliin, ja preesensvartalo
eroaa infinitiivivartalosta juurimorfin viimeisen konsonantin osalta:
/писа/ -- /пиш/, /плака/ -- /плач/.\footnote{Huomaa, ettei näitä verbejä
  pidä sekoittaa niihin ensimmäisen konjugaation verbeihin, joilla
  tapahtuu preesensvartalon sisällä äänteenmuutos /о/-alkuisissa
  muodoissa (ks. жить edellä).} Šeljakinin (2006: 129) mukaan tämän
tyypin verbejä on noin sata (писать, сказать, двигать,
махать,искать,брызгать ym.).

\subsection*{5. ryhmä}\label{ryhmuxe4-4}

Verbit, joiden infinitiivivartalo päättyy \emph{ну}, voidaan jakaa
kahteen ryhmään sen perusteella, miten niiden preteritimuodot
muodostetaan. Ensimmäiseen ryhmään kuuluvat \emph{momentaaniset}
(одноактный) verbit tyyppiä толкнуть, joilla preteritimuoto muodostetaan
suoraan infinitiivivartalon perusteella (/толкну/л/). Toisen ryhmän
muodostavat verbit tyyppiä \emph{остынуть} (`jäähtyä, haaleta'), jotka
ilmaisevat astettaista olotilan muutosta ja joiden preteritimuodoissa
ну-suffiksi katoaa taivutuspäätteen edeltä (остыл). Näitä jälkimmäisiä
verbejä on selvästi vähemmän, Šeljakinin (2006: 129) mukaan noin
viisikymmentä. Kummassakin tapauksessa preesensvartalo päättyy /н/
(толкн -- остын). Yhteensä näitä ryhmiä edustaa Wikisanakirjan
aineistossa 926 verbiä, esimerkiksi подвинуть, шагнуть, замёрзнуть
(preteriti замёрз).

\subsection*{6. ryhmä}\label{ryhmuxe4-5}

Šeljakin (2006: 133) laskee perustellusti omaksi ryhmäkseen verbit
tyyppiä начать/обнять (`halata') / понять/нажать (`painaa'), joissa
preesensvartalo eroaa melko selkeästi infinitiivivartalosta. Muodostus
noudattaa mallia /нача/ -- /начн/, /обня/ -- /обним/, /поня/ -- /пойм/,
/нажа/ -- /нажм/. Kaikissa tapauksissa siis preesensvartaloon ilmestyy
nasaali /м/ tai /н/ sekä tämän lisäksi joko /и/ tai /j/.

\subsection*{7. ryhmä}\label{ryhmuxe4-6}

Ehkä haastavimman ryhmän muodostavat verbit, joissa tapahtuu edellä
mainittu äänteenmuutos muissa kuin yksikön ensimmäisessä ja monikon
kolmannessa persoonassa. Lisäksi tämän ryhmän verbeillä esiintyy
esimerkiksi väistyvän vokaalin ilmiötä. Ryhmän muodostavat seuraavat
alaluokat:

\begin{itemize}
\tightlist
\item
  infinitiivissä \emph{-чь} päättyvät verbit tyyppiä печь, жечь, лечь,
  joilla infinitiivi- ja preesensvartalot ovat identtiset. Huomaa
  väistyvä vokaali жечь-verbillä (infinitiivi жечь, mutta preesensmuodot
  /жг/у, /жжёшь/ jne.) лечь-verbillä puolestaan on preesensmuodoissa
  vartalon lopussa a-vokaali: лягу, ляжешь ym.
\item
  infinitiivissä /сти/ tai /зти/ tai /сть/ päättyvät verbit kuten
  нести,спасти, везти, красть ja попасть. Näillä verbeillä äänteenmuutos
  tapahtuu liudentumattomien ja liudentuneiden konsonanttien välillä,
  kuten (selvyyden vuoksi foneettisesti esitettynä) /вез/у -- /вез'/ош
  ja /попад/у -- /попад'/ош/. Osalla сти-infinitiivin saavista verbeistä
  sekä kaikilla сть-infinitiivin saavista preteritin vartalo päättyy
  vokaaliin ja saa peräänsä normaalin л-tunnuksen (vrt. везти -- вёз ja
  брести -- брёл / попасть -- попал).
\item
  Edellisen kaltainen alaryhmänsä ovat verbit плыть ja жить, joilla
  preesensvartalo loppuu infinitiivivartalossa esiintymättömään
  в-konsonanttiin ja on siis muotoa /жив/ tai /плыв/ ja joilla myös
  toteutuu äänteenmuutos liudentuneiden ja liudentumattomien
  konsonanttien välillä. Vastaavasti verbillä учесть preesensvartalossa
  on ``ylimääräinen'' \emph{т}: учесть -- /учт/. Myös verbit tyyppiä
  умереть kuuluvat tähän ryhmään: niilläkin tapahtuu vaihtelu
  liudentuneen ja liudentumattoman konsonantin välillä. Lisäksi
  havaitaan väistyvä vokaali: infinitiivivartalo on muotoa /умере/,
  preesensvartalo /умр/.
\end{itemize}

\hyperdef{}{ref-sheljakin}{\label{ref-sheljakin}}
Шелякин, М.А. 2006. \emph{Справочник По Русской Грамматике}. drofa.

\end{document}
