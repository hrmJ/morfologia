\documentclass[]{scrartcl}
\usepackage[utf8]{inputenc}
\usepackage[T1]{fontenc}
\usepackage[T2A]{fontenc}
\usepackage[finnish]{babel}
\usepackage{linguex} 
\usepackage{longtable} 
\usepackage{booktabs}
\usepackage{amsthm}
\usepackage{graphicx}
\newtheorem{maar}{Määritelmä}
\usepackage{fixltx2e} % provides \textsubscript
\usepackage{textcomp} % provides \textsubscript
\usepackage{hyperref}
\usepackage{xcolor}

\providecommand{\tightlist}{%
  \setlength{\itemsep}{0pt}\setlength{\parskip}{0pt}}

\hypersetup{
    colorlinks,
    linkcolor={red!50!black},
    citecolor={blue!50!black},
    urlcolor={blue!80!black}
}
\author{Juho Härme}
\title{Morfologia-kurssin luentomateriaaleja}
\date{\today}
\begin{document}
\maketitle
\tableofcontents
\newpage



\section{Luento 18: Tapaluokka kieliopillisena kategoriana
1}\label{luento-18-tapaluokka-kieliopillisena-kategoriana-1}

\begin{itemize}
\tightlist
\item
  \href{https://mustikka.uta.fi/~juho_harme/morfologia/\#tästä-kurssista}{Takaisin
  sivun ylälaitaan}
\item
  \href{http://mustikka.uta.fi/~juho_harme/morfologia/materiaalit/luento18.pdf}{Lataa
  PDF}
\item
  \href{http://mustikka.uta.fi/~juho_harme/morfologia/presentations/luento18.html}{Tutki
  luentokalvoja}
\item
  \href{http://mustikka.uta.fi/~juho_harme/morfologia/tehtavat/luento18.pdf}{Tutki
  tuntitehtäviä}
\end{itemize}

Olemme keskellä kurssin toista puoliskoa (tai oikeastaan jo sen
loppupuolella), ja viime viikot käsittelyn kohteena ovat olleet verbit
ja niihin ilmentämät kieliiopilliset kategoriat. Pääluokka- ja aika- ja
aspektikategorioiden jälkeen siirrymme nyt tutkimaan \emph{tapaluokka-}
eli \emph{moduskategoriaa} (категория наклонения).

Tapaluokkakategorialla on venäjässä seuraavat arvot:

\begin{itemize}
\tightlist
\item
  Indikatiivi (изъявительное наклонение)
\item
  Konditionaali (сослогательное наклонение)
\item
  Imperatiivi(повелительное наклонение)
\end{itemize}

Koska tapaluokka on kielipillinen kategoria, kaikki verbimuodot
ilmaisevat jotakin sen kolmesta arvosta -- lukuun ottamatta
nominaalimuotoja. Tähän asti käsiteltyjen aiheiden yhteydessä lähes
kaikki tutkimamme verbitapaukset ovat olleet indikatiivitapauksia. Se
onkin tavallisin modus. Siinä missä konditionaali ilmaisee ehdollisuutta
(условность), ja imperatiivi käskyä (повелительность), on indikatiivin
tehtävänä ilmaista suoria väitteitä ilman velvoitteen tai epävarmuuden
sävyjä. Indikatiivi on myös aikamuotojen suhteen laajimpikäyttöisin
modus, sillä indikatiivimuotoja voidaan muodostaan niin menneen ajan,
nykyhetken kuin tulevankin ajan verbimuodoista. Nopea vertailu eri
moduksista voidaan tehdä seuraavan simppelin esimerkin avulla (esimerkki
1 indikatiivi, esimerkki 2 konditionaali ja esimerkki 3 imperatiivi):

\begin{enumerate}
\def\labelenumi{(\arabic{enumi})}
\tightlist
\item
  Я играю на гармошке
\item
  Я играл бы на гармошке
\item
  Играй на гармошке!
\end{enumerate}

Indikatiivin muodostus on käyty läpi, kun pohdittiin persoona-,
preteriti- ja futuurimuotojen rakentamista. Tällä luennolla keskitytään
\emph{imperatiivin} muodostamiseen, ensi luennolla puolestaan
imperatiivin käyttöön ja konditionaaliin.

\subsection{Imperatiivin muodostus
venäjässä}\label{imperatiivin-muodostus-venuxe4juxe4ssuxe4}

Kuten monesti aiemminkin, Ahti Nikunlassi (2002: 159--160) tarjoaa
koulukielioppeja syvällisemmän mallin siihen, miten imperatiivi
venäjässä muodostetaan. Muodostuksen kannalta ongelmallinen on
luonnollisesti 2. persoonan infinitiivi, joten keskitytään siihen
seuraavassa huolellisemmin.

\subsubsection{2. persoonan imperatiivin
päätteet}\label{persoonan-imperatiivin-puxe4uxe4tteet}

Ensinnäkin, voidaan erottaa kaksi päätettä 2. persoonan imperatiiville.
\textbf{Ensimmäinen näistä on /И/}, kuten muodossa

\begin{enumerate}
\def\labelenumi{(\arabic{enumi})}
\setcounter{enumi}{3}
\tightlist
\item
  /смотр/И/
\end{enumerate}

\textbf{Toinen on vanha tuttu nollapääte /ø/}, joka esiintyy esimerkiksi
muodossa

\begin{enumerate}
\def\labelenumi{(\arabic{enumi})}
\setcounter{enumi}{4}
\tightlist
\item
  \emph{/достань/ø/}.
\end{enumerate}

\textbf{Lisäksi,} vaikkei ensi katsomalta huomaisi, nollapääte on myös
muodossa

\begin{enumerate}
\def\labelenumi{(\arabic{enumi})}
\setcounter{enumi}{5}
\tightlist
\item
  \emph{/налей/} (`kaada')
\end{enumerate}

Esimerkin 6 morfologinen asu ei siis ole /нале/й/, jossa /й/ olisi
imperatiivin pääte. Ennemminkin muoto on konstruoitava näin: /налеj/ø/,
niin että imperatiivin pääte on /ø/. Miksi? Pohditaan tarkemmin.

Nikunlassin (2002: 159; vrt Шелякин 2006: 162) mukaan imperatiivin
muodostuksen pohjana on preesensvartalo. Katsotaan esimerkkien 4, 5 ja 6
preesensvartaloita, jotka ovat:

\begin{itemize}
\tightlist
\item
  смотреть - смотрят - \textbf{смотр}
\item
  достать - достанут - \textbf{достан}
\item
  налить - налеют - \textbf{налеj}
\end{itemize}

Jos yllä lueteltuja preesensvartaloita vertaa esimerkkeihin 4, 5 ja 6,
voidaan todeta, että

\begin{enumerate}
\def\labelenumi{\arabic{enumi}.}
\tightlist
\item
  esimerkissä 4 preesensvartaloon смотр on liitetty /и/
\item
  esimerkissä 5 preesensvartaloon достан on liitetty /ø/
\item
  esimerkissä 6 preesensvartaloon налеj on littetty /ø/
\end{enumerate}

Huomaa lisäksi, että esimerkeissä 5 ja 4 tapahtuu vartalon viimeisen
konsonantin liudentuminen, joten tarkkaan ottaen muodot ovat /смотр'и/
ja /достан'/ -- tämän takia esimerkissä 5 myös on kirjoitusasussa pehmeä
merkki. Ei siis niin, että ``pehmeä merkki olisi imperatiivin pääte'',
vaan niin, että imperatiivin pääte on /ø/ ja sen edellä konsonantti
liudentuu. Paras todiste nollapäätteen olemassaolosta on esimerkki 6,
jossa preesensvartalolle ei tapahdu mitään (/налеj/ on preesensvartalo
ja imperatiivi /налеj/ø.

\subsubsection{Milloin mikäkin
pääte?}\label{milloin-mikuxe4kin-puxe4uxe4te}

Päätteet ovat siis /И/ ja /ø/, mutta kielenoppijan kannalta tärkeä
kysymys on, milloin mitäkin päätettä pitäisi käyttää. Intuition valossa
vaikuttaa selvältä, että и-pääte on yleisempi -- nollapääte siis
jonkinasteinen erityistapaus.

Oleellisia asioita, joiden perusteella pääte määräytyy, ovat

\begin{enumerate}
\def\labelenumi{\arabic{enumi}.}
\tightlist
\item
  Päättyykö preesensvartalo j-äänteeseen?
\item
  Missä paino on 1. persoonan ei-menneen ajan muodossa?
\end{enumerate}

Katsotaan esimerkkiä 4 edellisten kriteerien perusteella. Kysymykseen 1
vastaus on kielteinen: preesensvartalo on /смотр'/, joten se ei pääty
j-äänteeseen. Toiseen kysymykseen voidaan vastata, että paino on
päätteellä: смотрЮ. \textbf{Lopputulos: /и/-pääte} .

Katsotaan nyt esimerkkiä 5. Kysymykseen 1 vastaus on jälleen kielteinen
(/достан/ ei pääty j:hin). Vastaus kysymykseen 2 sen sijaan on erilainen
kuin äsken. Paino ei ensimmäisen persoonan muodossa ole päätteellä vaan
vartalolla: /достАну/. \textbf{Lopputulos: /ø/-pääte}.

Siirytään vielä esimerkkiin 6. Kysymykseen 1 vastaus on myönteinen
(/налеj/), ja toiseen kysymykseen voidaan sanoa, että paino on
vartalolla (налЕю). \textbf{Lopputulos: /ø/-pääte} .

Edellä olevat esimerkit antavat perusmallin imperatiivin eri päätteiden
käytölle.

\begin{enumerate}
\def\labelenumi{\arabic{enumi}.}
\tightlist
\item
  Jos preesensvartalo päättyy j-äänteeseen, imperatiivin pääte on
  nollaäänne.
\item
  Jos paino yksikön ensimmäisen persoonan muodossa on päätteellä,
  imperatiivin pääte on /и/
\item
  Jos paino yksikön ensimmäisen persoonan muodossa on vartalolla,
  imperatiivin pääte on nollaäänne.
\end{enumerate}

Testataan esitettyä sääntöä esimerkiksi verbeillä
\href{http://ru.wiktionary.org/wiki/писать}{писать},
\href{http://ru.wiktionary.org/wiki/сидеть}{сидеть},
\href{http://ru.wiktionary.org/wiki/смотреть}{смотреть} ja
\href{http://ru.wiktionary.org/wiki/взять}{взять} (linkit Wiktionaryyn).
Kaikilla näillä:

\begin{enumerate}
\def\labelenumi{\arabic{enumi}.}
\tightlist
\item
  Preesensvartalo ei pääty j:hin
\item
  Yksikön 1. persoonassa paino on päätteellä
\item
  Imperatiivin pääte on /И/.
\end{enumerate}

Toisaalta esimerkiki петь, быть, темнеть, играть saavat nollapäätteen,
koska

\begin{enumerate}
\def\labelenumi{\arabic{enumi}.}
\tightlist
\item
  verbeillä петь, темнеть, играть on preesensvartalon lopussa j
\item
  быть-verbillä ei ole j:tä, mutta paino 1. persoonassa vartalolla
  (бУду)
\item
  Muodot ovat siis /поj/ø/, /буд'/ø/ jne.
\end{enumerate}

\subsubsection{Tarkennuksia
sääntöön}\label{tarkennuksia-suxe4uxe4ntuxf6uxf6n}

Edellä esitettyä voidaan vähän tarkentaa. Ajattele verbejä
\href{http://ru.wiktionary.org/wiki/кончить}{кончить},
\href{http://ru.wiktionary.org/wiki/прыгнуть}{прыгнуть} ja
\href{http://ru.wiktionary.org/wiki/чиcтить}{чиcтить}. Niissä

\begin{enumerate}
\def\labelenumi{\arabic{enumi}.}
\tightlist
\item
  preesensvartalo (/конч/, /прыгн/, /чист/) ei pääty j:hin
\item
  paino on yksikön 1. persoonassa vartalolla: кОнчу, прЫгну, чИщу
\end{enumerate}

Edellä esitetyn säännön perusteella tuloksena pitäisi olla
nollapäätteinen imperatiivimuoto (конч, прыгн, чист). Näin ei kuitenkaan
ole, vaan mainituilla verbeillä imperatiivin pääte on /И/. Tästä voidaan
tehdä tarkennus edellä esitettyyn sääntöön:

\begin{itemize}
\tightlist
\item
  Jos paino on yksikön 1 persoonassa vartalolla \textbf{ja
  preesensvartalo päättyy konsonanttiyhtymään} (tässä: нч, гн, ст),
  imperatiivin pääte on /И/. Huomaa, että paino myös imperatiivissa
  pysyy tällöin vartalolla, joten muodot ovat: \emph{кОнчи},
  \emph{прЫгни}, \emph{чИсти}. (vrt. kuitenkin толкнуть - толкнУ -
  толкнИ).
\end{itemize}

Toinen vastaava tarkennus on, että painollisella вы-prefiksillä
alkavilla verbeillä pääte on /И/ eikä nollapääte: вЫходить -- вЫходи,
вЫлезти - вЫлези jne.

Kolmantena tarkennuksena mainittakoon sellaiset melko harvalukuisat
verbit kuin доить ja поить, joiden imperatiivimuodot ovat /до/И/ ja
/по/И/. Näillä verbeillä preesensvartalot ovat /доj/ ja /поj/, mutta
silti imperatiivin pääte /И/: доИ ja поИ.

Neljäs tarkennus koskee verbejä tyyppiä \emph{давать}, joissa
muodostuksen pohjana onkin infinitiivivartalo: \emph{давай},
\emph{узнавай}, \emph{вставай}.

Viides tarkennus koskee väistyvää vokaalia. Verbin /бить/
preesensvartalo on /б'j/, mutta -- kuten ehkä muistat -- ennen
nollapäätettä väistyvä vokaali esiintyy ilmiäänteenä (``oikeana
vokaalina''), joten näillä verbeillä imperatiivimuodot ovat mallia
/бей/, /пей/.

Oma yksittäinen imperatiivimuotonsa on lisäksi verbillä
\href{http://ru.wiktionary.org/wiki/есть}{есть}. Šeljakin (2006: 163)
toteaa myös, että kaikista (prefiksittömistä) verbeistä imperatiivia ei
muodosteta ja mainitsee tässä yhteydessä verbit хотеть, видеть, мочь ja
слышать.

Lisäksi tietenkin on todettava, että monikkomuodot muodostetaan kaikista
edellä käsitellyistä yksikön toisen persoonan muodoista lisäämällä
lisäämällä imperatiivin päätteen perään pääte /те/: смотрите, пейте,
станьте, будьте, играйте jne.

\subsection{Muita imperatiivimuotoja}\label{muita-imperatiivimuotoja}

Edellä käsiteltiin tarkkaan yksikön toisen persoonan imperatiivin
muodostamista, koska se tapahtuu erityisten päätteiden avulla,
synteettistesti. On olemassa kuitenkin myös muita imperatiivimuotoja,
jotka muodostetaan \emph{analyyttisesti}, apusanoja käyttäen. Itse
asiassa, kun lasketaan monikkomuodot, Šeljakin (2006: 162) toteaa, että
venäjässä on peräti kuusi eri imperatiivimuotoa. Edellä niistä
käsiteltiin kahta (2. persoonan yksikkö ja monikko), mutta mitä ovat
loput neljä (tai paremminkin kaksi, joista kummastakin yksikkö- ja
monikkomuodot)?

Ensiksi voidaan mainita kolmannen persoonan käskymuodot, jotka
muodostetaan sanojen \emph{пусть/пускай} avulla:

\begin{enumerate}
\def\labelenumi{(\arabic{enumi})}
\setcounter{enumi}{6}
\tightlist
\item
  -- Пожалуйста, не кричите, на нас смотрят. -- \emph{Пускай смотрят}. Я
  -- за гласность.
\item
  -- Только не говори ему об этом, \emph{пусть остаётся} в счастливом
  неведении.
\end{enumerate}

Toiseksi omaksi käskymuodokseen voidaan erottaa käskyt, joita koskevan
joukon osa pyös puhuja on (Šeljakinilla tässä termi \emph{совместное
действие}). Nämä muodostetaan sananmuodon \emph{давай} avulla. Siihen
voidaan yhdistää joko imperfektiivisen aspektin verbi
infinitiivimuodossa tai perfektiivisen aspektin verbi
indikatiivimuodossa:

\begin{enumerate}
\def\labelenumi{(\arabic{enumi})}
\setcounter{enumi}{8}
\tightlist
\item
  -- Разве это справедливо? \emph{Давай делиться} по-честному.
\item
  -- \emph{Давай выпьем} кофе, -- предложил Мартин
\end{enumerate}

Toisaalta puhujan itsensä käsittämää joukkoa voi käskeä myös ilman
давай-sanaa:

\begin{enumerate}
\def\labelenumi{(\arabic{enumi})}
\setcounter{enumi}{10}
\tightlist
\item
  -- Ну, что ж под дождем то стоять?.. пойдем куда-нибудь, -- сказал
  Сеня.
\end{enumerate}

Myös infinitiivimuotoja voidaan käyttää käskemiseen, kuten seuraavassa
Nikunlassin (2002: 199) esimerkissä:

\begin{enumerate}
\def\labelenumi{(\arabic{enumi})}
\setcounter{enumi}{11}
\tightlist
\item
  Всем студентам собраться у главного входа!
\end{enumerate}

Näissä tapauksissa kyse on jyrkästä, kategorisesta kiellosta. Toisaalta
infinitiivi on tavallinen käskymuoto esimerkiksi resepteissä:

\begin{enumerate}
\def\labelenumi{(\arabic{enumi})}
\setcounter{enumi}{12}
\tightlist
\item
  Готовое мясо \emph{завернуть} в фольгу, чтобы \emph{сохранить} его
  горячим, а кости вместе с соком со сковородки, 40 мл воды и соком
  устриц варить в кастрюле 20 минут, после чего \emph{процедить} через
  мелкое сито, \emph{добавить} пассерованную на масле муку,
  \emph{смешать} всё венчиком и \emph{варить} 5 минут, \emph{положить}
  устрицы с шампиньонами и \emph{греть} соус ещё несколько минут.
\end{enumerate}

\hyperdef{}{ref-nikunl}{\label{ref-nikunl}}
Nikunlassi, Ahti. 2002. \emph{Johdatus Venäjän Kieleen Ja Sen
Tutkimukseen}. Helsinki: Finn Lectura.

\hyperdef{}{ref-sheljakin}{\label{ref-sheljakin}}
Шелякин, М.А. 2006. \emph{Справочник По Русской Грамматике}. drofa.

\end{document}
