\documentclass[]{scrartcl}
\usepackage[utf8]{inputenc}
\usepackage[T1]{fontenc}
\usepackage[T2A]{fontenc}
\usepackage[finnish]{babel}
\usepackage{linguex} 
\usepackage{longtable} 
\usepackage{booktabs}
\usepackage{amsthm}
\usepackage{graphicx}
\newtheorem{maar}{Määritelmä}
\usepackage{fixltx2e} % provides \textsubscript
\usepackage{textcomp} % provides \textsubscript
\usepackage{hyperref}
\usepackage{xcolor}

\providecommand{\tightlist}{%
  \setlength{\itemsep}{0pt}\setlength{\parskip}{0pt}}

\hypersetup{
    colorlinks,
    linkcolor={red!50!black},
    citecolor={blue!50!black},
    urlcolor={blue!80!black}
}
\author{Juho Härme}
\title{Morfologia-kurssin luentomateriaaleja}
\date{\today}
\begin{document}
\maketitle
\tableofcontents
\newpage



\section{Luento 11: persoona kieliopillisena
kategoriana}\label{luento-11-persoona-kieliopillisena-kategoriana}

Siirrämme nyt kurssin painopistettä substantiiveista ja muista
nomineista kohti verbejä. Aloitamme tutkimalla \emph{persoonaa}
kieliopillisena kategoriana.

Persoonakategoria (категория лица) on ennen kaikkea verbien mutta myös
persoona- ja possessiivipronominien ilmaisema kieliopillinen kategoria.
Siinä missä pronomineilla persoona ilmaistaan sanan vartalossa (vrt.
muotoja мы,я,он,наш,его), käytetään verbeillä taivutuspäätteitä.
Keskitymme tässä yhteydessä persoonaan nimenomaan verbien ominaisuutena.

Nikunlassi (2002: 154) huomauttaa, että verbien persoonapäätteet
ilmaisevat itse asiassa melko monta eri asiaa: paitsi persoonaa, myös
tapaluokkaa, aikamuotoa ja lukua. Kun siirrytään tarkastelemaan verbien
persoonataivutusta, tutkitaan siis loppujen lopuksi laajempaa ilmiötä
kuin pelkkää persoonaa.

\subsection{Persoona- ja
infinitiivivartalo}\label{persoona--ja-infinitiivivartalo}

Venäjän verbeistä puhuttaessa on tapana erottaa kaksi eri vartaloa
(perusteluita tälle ks. Исаченко 2003{[}1965{]}: 27). Muistanet
aiemmilta luennoilta, että \emph{vartalolla} (основа) tarkoitetaan sitä
osaa sanasta, joka jää jäljelle, kun otetaan pois taivutuspäätteet.

Ensimmäinen verbin vartaloista saadaan, kun poistetaan
infinitiivimuodosta infinitiivin tunnus. Näin ollen esimerkiksi verbin
смотреть infinitiivivartalo (основа инфинитива) on \emph{смотре}, verbin
играть infinitiivivartalo \emph{игра}, verbin взять taas \emph{взя} ja
niin edelleen.

Ajattele verbiä /смотр/е/ть/. Persoonamuodoissa е-suffiksia ei ole
näkyvissä, vaan esimerkiksi ensimmäisen persoonan muoto kuuluu
/смотр/ю/. Verbin toinen vartalo, \emph{preesensvartalo} (основа
настоящего времени), saadaankin poistamalla persoonapääte. Tapana on,
että muodostamiseen käytetään monikon kolmannen persoonan muotoa. Näin
ollen смотреть-verbistä preesensvartalo on /смотр/ят/ -- /смотр'/,
пугать-verbistä /пуга/ют/-- /пугаj/, вести-verbistä /вед/ут/ -- /вед/.
\textbf{Huomaa}, että вести-verbin kaltaisilla verbeillä sekä
ч-konsonanttiin infinitiivissä päättyvillä verbeillä infinitiivi- ja
preesensvartalot ovat samoja: вести -- вед \& вед, печь -- пек \& пек
jne. Nämä ovat siis infinitiivivartalon muodostamisen osalta
poikkeuksellisia (Nikunlassi 2002: 137).

Infinitiivivartaloista voidaan mainita vielä Šeljakinin (2006: 122)
huomautus siitä, että yleensä preteritimuodon vartalo on sama kuin
infinitiivivartalo: игра/ла/, смотре/ла/ jne.

\subsection{Konjugaatiot}\label{konjugaatiot}

Venäjässä verbit jaetaan persoonataivutuksen osalta kahteen
taivutustyyppiin eli \emph{konjugaatioon} (спряжение), joiden lisäksi on
joukko epäsäännöllisesti taipuvia verbejä. Taivutettaessa ся-postfiksin
sisätäviä verbejä kannattaa muistaa, että vokaalin jälkeen posfiksi on
muodossa \emph{сь} (учусь, занимаюсь jne) (Шелякин 2006: 124).

\subsubsection{1. konjugaatio}\label{konjugaatio}

Tutkitaan ensin kahta 1. konjugaatioon kuuluvaa verbiä ja tarkastellaan
taivutuspäätteitä kirjoitusasussaan:

\begin{longtable}[c]{@{}lll@{}}
\toprule
persoona & verbi 1 & verbi2\tabularnewline
\midrule
\endhead
y. 1. & игра/ю/ & жив/у/\tabularnewline
y. 2. & игра/ешь/ & жив/ёшь/\tabularnewline
y. 3. & игра/ет/ & жив/ёт/\tabularnewline
m. 1. & игра/ем/ & жив/ём/\tabularnewline
m. 2. & игра/ете/ & жив/ёте/\tabularnewline
m. 3. & игра/ют/ & жив/ут/\tabularnewline
\bottomrule
\end{longtable}

Kirjoitusasun perusteella voidaan todeta, että verbit edustavat
ensimmäisen konjugaation kahta eri sanatyyppiä. Asia kuitenkin
yksinkertaistuu, jos tutkitaan -- kuten edellä substantiivien
deklinaatioiden yhteydessä -- sanojen \emph{foneettista} asua (ks.
Nikunlassi 2002: 155).

Muodostetaan ensin kummastakin verbistä preesensvartalot.
Играть-verbillä tämä tapahtuu ottamalla играют-muodosta pois
persoonapääte. Preesensvartalo on tällöin /играj/. Samaa muotoa
käytetään kaikissa persoonissa. Жить-verbillä preesensvartalo
muodostetaan живут -- /жив/. On huomattava, että vartalo ei ole
identtinen joka muodossa, vaan siinä tapahtuu äänteenmuutos
liudentumattoman ja liudentuneen в-konsonantin välillä. Yksikön 1. ja
monikon 3. persoona saavat vartalon /жив/, mutta muut persoonat vartalon
/жив'/.

Muistatko, miten /о/-vokaali käyttäytyy painottomana ja painollisena?
Edellä olemme todenneet, että sananmuodoissa \emph{врачом},
\emph{местом} ja \emph{жителем} on foneettisesti sama /ом/-pääte, joka
painottoman tavun jälkeisessä asemassa ääntyy redusoituneessa muodossa
ja painollisena redusoimattomassa muodossa\footnote{Lisäksi
  vokaaliäänteeseen vaikuttaa jonkin verran edellisen konsonantin
  liudentuneisuus.} . Sama ilmiö tapahtuu verbitaivutuksessa. Voimme
itse asiassa tiivistää edellä esitetyt 1. konjugaation päätteet
seuraavasti:

\begin{longtable}[c]{@{}llll@{}}
\toprule
persoona & pääte & esimerkki1 & esimerkki2\tabularnewline
\midrule
\endhead
y. 1. & /у/ & /жив/у/ & /играj/у/\tabularnewline
y. 2. & /ош/ & /жив'/ош/ & /играj/ош/\tabularnewline
y. 3. & /от/ & /жив'/от/ & /играj/от/\tabularnewline
m. 1. & /ом/ & /жив'/ом/ & /играj/ом/\tabularnewline
m. 2. & /от'э/ & /жив'/от'э/ & /играj/от'э/\tabularnewline
m. 3. & /ут/ & /жив/ут/ & /играj/ут/\tabularnewline
\bottomrule
\end{longtable}

\subsubsection{2. konjugaatio}\label{konjugaatio-1}

Siirrytään sitten tarkastelemaan toista varsinaista konjugaatiotyyppiä.
Tähän konjugaatioon kuuluvat esimerkiksi sanat включить ja усвоить
(`omaksua').

\begin{longtable}[c]{@{}lll@{}}
\toprule
Persoona & verbi1 & verbi2\tabularnewline
\midrule
\endhead
y. 1. & включ/у/ & усво/ю/\tabularnewline
y. 2. & включ/ишь/ & усво/ишь/\tabularnewline
y. 3. & включ/ит/ & усво/ит/\tabularnewline
m. 1. & включ/им/ & усво/им/\tabularnewline
m. 2. & включ/ите/ & усво/ите/\tabularnewline
m. 3. & включ/ат/ & усво/ят/\tabularnewline
\bottomrule
\end{longtable}

Pohditaan jälleen, miten näistä sanoista muodostetaan preesensvartalot.
Включить-verbin monikon kolmas persoona on включат, josta
persoonapäätteen jälkeen jää jäljelle /включ/. Усвоить-verbillä
muodostus tapahtuu усвоят -- /усвоj/. Foneettisesti ilmaistuna
päätemorfeemit ovat siis seuraavat:

\begin{longtable}[c]{@{}llll@{}}
\toprule
persoona & pääte & esimerkki1 & esimerkki2\tabularnewline
\midrule
\endhead
y. 1. & /у/ & /включ/у/ & /усвоj/у/\tabularnewline
y. 2. & /иш/ & /включ/иш/ & /усвоj/иш/\tabularnewline
y. 3. & /ит/ & /включ/ит/ & /усвоj/ит/\tabularnewline
m. 1. & /им/ & /включ/им/ & /усвоj/им/\tabularnewline
m. 2. & /ит'э/ & /включ/ит'э/ & /усвоj/ит'э/\tabularnewline
m. 3. & /ат/ & /включ/ат/ & /усвоj/ат/\tabularnewline
\bottomrule
\end{longtable}

Ensimmäisen ja toisen konjugaation päätteitä tarkasteltaessa huomataan,
että yksikön ensimmäisen persoonan muoto on kummassakin konjugaatiossa
sama, muuten konjugaatiot ovat erilaisia.

\subsubsection{Äänteenmuutoksia 1.
konjugaatiolla}\label{uxe4uxe4nteenmuutoksia-1.-konjugaatiolla}

Nikunlassin (2002: 155) mukaan ensimmäisestä konjugaatiosta voidaan
äänteenmuutosten näkökulmasta erottaa omaksi ryhmäkseen kaikki
/о/-foneemilla alkavat muodot. Näissä tapahtuu äänteenmuutoksia, joista
eräs kohdatiin jo жить-verbiä tarkasteltaessa. Listataan ensin
kirjoitusasussaan kaikki жить-verbin persoonamuodot, jotka alkavat
/о/-foneemilla:

\begin{itemize}
\tightlist
\item
  живёшь
\item
  живёт
\item
  живём
\item
  живёте
\end{itemize}

Жить-verbillä äänteenmuutos tarkottaa, että liudentumaton konsonantti
vaihtuu liudentuneeseen. Preesensvartalo жить-verbillä on /жив/. Näin
ollen foneettisessa asussaan luetellut muodot ovat:

\begin{itemize}
\tightlist
\item
  /жив'/ош/
\item
  /жив'/от/
\item
  /жив'/ом/
\item
  /жив'/от'э/
\end{itemize}

Selkeämmin havaittava äänteenmuutos koskee konsonantteja /к/ ja /г/,
jotka vaihtuvat konsonanteiksi /ч/ ja /ж/. Otetaan tästä esimerkiksi
verbi \emph{лечь} (`panna maaten'). Preesensvartalo tällä verbillä
saadaan: \emph{лягут} -- \emph{ляг}. Näin ollen /о/-foneemilla alkavat
muodot ovat (konsonanttimuutoksen jälkeen):

\begin{itemize}
\tightlist
\item
  ляжешь = /ляж/ош/
\item
  ляжет = /ляж/от/
\item
  ляжем = /ляж/ом/
\item
  ляжете = /ляж/от'э/
\end{itemize}

Печь-verbillä preesensvartalo muodostetaan \emph{пекут -- /пек/}. Näin
ollen äänteenmuutoksen jälkeen saadaan:

\begin{itemize}
\tightlist
\item
  печёшь = /печ'/ош/
\item
  печёт = /печ'/от/
\item
  печём = /печ'/ом/
\item
  печёте = /печ'/от'э/
\end{itemize}

Huomaa, että verbejä tyyppiä \emph{писать} ei tässä lasketa kuuluvaksi
äänteenmuutosten piiriin, sillä preesensvartalo on kaikissa tapauksissa
/пиш/ eikä siis vaihtele muuten kuin suhteessa infinitiivivartaloon
/писа/.

\subsubsection{Äänteenmuutoksia 2.
konjugaatiolla}\label{uxe4uxe4nteenmuutoksia-2.-konjugaatiolla}

Kun mietitään toista konjugaatiota, voidaan todeta, että
äänteenmuutoksen kohteeksi joutuvia muotoja on vähemmän. Siinä, missä 1.
konjugaatiolla neljä kuudesta muodosta muuttuu, 2. konjugaatiolla
ainoastaan \textbf{yksikön ensimmäisen persoonan muoto} muuttuu
konsonanttivaihtelujen seurauksena (Nikunlassi 2002: 155). Muutos
sinänsä on kuitenkin 1. konjugaatiota monimutkaisempi. Nikunlassin
(2002: 83) ja Isachenkon (2003{[}1965{]}: 35) antamien listojen
perusteella voidaan esittää ainakin seuraavat äänteenvaihteluparit:

\begin{longtable}[c]{@{}llll@{}}
\toprule
muutos & verbin infinitiivi & verbin preesensvartalo & yks. 1.
pers.\tabularnewline
\midrule
\endhead
т'-ч & платить & /плат'/ & /плач/У/\tabularnewline
т'-щ & обратить & /обрат'/ & /обраш/У/\tabularnewline
д'-ж & садить & /сад'/ & /саж/У/\tabularnewline
з'-ж & изобразить & /изобраз'/ & /изображ/у/\tabularnewline
зд'-зж & ездить & /езд'/ & /Езж/у/\tabularnewline
с'-ш & просить & /прос'/ & /прош/у/\tabularnewline
ст'-щ & пустить & /пуст'/ & /пущ/У/\tabularnewline
п'-пл & накопить & /накоп/ & /накопл'/у\tabularnewline
б'-бл & любить & /люб'/ & /любл'/у/\tabularnewline
ф'-фл' & разграфить & /разграф'/ & /разграфл'/У/\tabularnewline
в'-вл' & ловить & /лов'/ & /ловл'/у/\tabularnewline
м'-мл & кормить & /корм'/ & /кормл'/у/\tabularnewline
\bottomrule
\end{longtable}

\hyperdef{}{ref-nikunl}{\label{ref-nikunl}}
Nikunlassi, Ahti. 2002. \emph{Johdatus Venäjän Kieleen Ja Sen
Tutkimukseen}. Helsinki: Finn Lectura.

\hyperdef{}{ref-isachenko2003}{\label{ref-isachenko2003}}
Исаченко, Александр. 2003{[}1965{]}. \emph{Грамматический Строй Русского
Языка В Сопоставлении С Словацким. Морфология. Часть 1, 2}. Москва:
Языки славянской культуры.
\url{https://www.litres.ru/a-v-isachenko/grammaticheskiy-stroy-russkogo-yazyka-v-sopostavlenii-s-slovackim-morfologiya-chast-1-2/}.

\hyperdef{}{ref-sheljakin}{\label{ref-sheljakin}}
Шелякин, М.А. 2006. \emph{Справочник По Русской Грамматике}. drofa.

\end{document}
