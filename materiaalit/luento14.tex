\documentclass[]{scrartcl}
\usepackage[utf8]{inputenc}
\usepackage[T1]{fontenc}
\usepackage[T2A]{fontenc}
\usepackage[finnish]{babel}
\usepackage{linguex} 
\usepackage{longtable} 
\usepackage{booktabs}
\usepackage{amsthm}
\usepackage{graphicx}
\newtheorem{maar}{Määritelmä}
\usepackage{fixltx2e} % provides \textsubscript
\usepackage{textcomp} % provides \textsubscript
\usepackage{hyperref}
\usepackage{xcolor}

\providecommand{\tightlist}{%
  \setlength{\itemsep}{0pt}\setlength{\parskip}{0pt}}

\hypersetup{
    colorlinks,
    linkcolor={red!50!black},
    citecolor={blue!50!black},
    urlcolor={blue!80!black}
}
\author{Juho Härme}
\title{Morfologia-kurssin luentomateriaaleja}
\date{\today}
\begin{document}
\maketitle
\tableofcontents
\newpage



\section{Luento 14 : Pääluokka kieliopillisena
kategoriana}\label{luento-14-puxe4uxe4luokka-kieliopillisena-kategoriana}

\begin{itemize}
\tightlist
\item
  \href{https://mustikka.uta.fi/~juho_harme/morfologia/\#tästä-kurssista}{Takaisin
  sivun ylälaitaan}
\item
  \href{http://mustikka.uta.fi/~juho_harme/morfologia/materiaalit/luento14.pdf}{Lataa
  PDF}
\item
  \href{http://mustikka.uta.fi/~juho_harme/morfologia/presentations/luento14.html}{Tutki
  luentokalvoja}
\item
  \href{http://mustikka.uta.fi/~juho_harme/morfologia/tehtavat/luento14.pdf}{Tutki
  tuntitehtäviä}
\end{itemize}

\emph{Pääluokka} (залог) on kieliopillinen kategoria, joka on yksinomaan
verbien ominaisuus. Kategorialla on kaksi arvoa: aktiivi (действительный
залог) ja passiivi (страдательный залог).

Pääluokkakategorian oletusarvona voi hyvin pitää aktiivia: verbit ovat
\emph{useimmiten} aktiivissa, ja passiivinen käyttö on niin suomessa
kuin venäjässäkin aina jonkinasteinen poikkeus. Tässä yhteydessä onkin
kysyttävä funktionaalisen morfologian kannalta tärkeä kysymys: Milloin
ja mihin passiivia tarvitaan?

Marjatta Vanhala-Aniszewskin (1992: 242) mukaan passiivin tärkein
funktio niin suomessa kuin venäjässä on huomion vieminen pois toiminnan
varsinaisesta suorittajasta (дефокусировка реального производителя
действия). Vertaa tähän liittyen seuraavia lauseita:

\begin{enumerate}
\def\labelenumi{(\arabic{enumi})}
\tightlist
\item
  Я написал книгу как для современников, так и для потомков
\item
  Книга написана как для современников, так и для потомков
\end{enumerate}

Yksinkertaisimmillaan aktiivin ja passiivin erossa on kyse lauseiden 1
ja 2 välisestä erosta: lauseessa 1 on selvästi ilmaistu, kuka on
vastuussa suoritetusta toiminnasta, kun taas lause 2 jättää tämän auki.
Monimutkaisemmaksi ero muuttuu, kun ajatellaan passiivilauseita, joissa
tekijä ilmaistaan agenttitäydennyksen avulla:

\begin{enumerate}
\def\labelenumi{(\arabic{enumi})}
\setcounter{enumi}{2}
\tightlist
\item
  Книга написана Достоекским.
\end{enumerate}

Kuitenkin myös lauseesta 3 voidaan todeta, ettei päähuomio ole tekijässä
vaan tekemisen kohteessa: tekijän ilmaiseminen on ainoataan täydentävää
informaatiota.

\subsection{Passiivirakenteet
venäjässä}\label{passiivirakenteet-venuxe4juxe4ssuxe4}

Venäjästä erotetaan tavallisesti kaksi erityyppistä passiivirakennetta
ja näiden lisäksi kolmas rakenne, jota käytetään passiivin funktiossa.
Varsinaisia (morfologisessa ja syntaktisessa mielessä)
passiivirakenteita ovat \emph{morfologinen} (ся-passiivi) ja
\emph{perifrastinen} passiivi (морфологический / перифрастический
пассив) (Vanhala-Aniszewski 1992: 60). Kaikkiin passiivirakenteisiin
liittyy olennaisena vaatimuksena verbin transitiivisuus.

\subsubsection{Morfologinen passiivi}\label{morfologinen-passiivi}

Morfologinen passiivi muodostetaan nimensä mukaisesti morfologisin
keinoin, lisäämällä postfiksi /ся/ \emph{imperfektiivistä aspektia
edustavaan} verbiin. Koska se muodostetaan imperfektiivisen aspektin
verbeistä, seurauksena on, että morfologisella passiivilla kuvattava
toiminta on aina \emph{prosessuaalista} (процессный), se ei koskaan
saavuta suoritetun toiminnan rajaa tai voi kertoa loppuun saatetusta
prosessista. Morfologinen passiivi voidaan muodostaa niin menneen
(esimerkit 7, 8) kuin ei-menneen ajan muodoista (esimerkit 4, 5, 6),
mukaan lukien futuurimuodot.

\begin{enumerate}
\def\labelenumi{(\arabic{enumi})}
\setcounter{enumi}{3}
\tightlist
\item
  Завод \emph{строится} на новой промышленной площадке
\item
  Тема \emph{будет обсуждаться} и на проводимой Высшей школой экономики
  Четвёртой ежегодной конференции
\item
  Узбекский хлеб \emph{печётся} в традиционных глиняных печах --
  тандырах.
\item
  Все детали будущего дома \emph{изготавливались} для вас на
  домостроительных комбинатах
\item
  Выяснилось, самолет \emph{проектировался} еще в начале пятидесятых
  специалистами фирмы «Локхид эркрафт корпорейшен»
\end{enumerate}

\subsubsection{Perifrastinen passiivi}\label{perifrastinen-passiivi}

Perifrastisella passiivilla tarkoitetaan partisiippien ja быть-verbin
avulla muodostettavaa passiivirakennetta. Tarkkaan ottaen rakenne
muodostetaan passiivin partisiipin preteritin lyhyistä muodoista. Kuten
edelliseltä luennolta muistat, passiivin partisiipin preteritimuotoja
tehdään \emph{ainoastaan perfektiivisen aspektin verbeistä}, joten
perifrastinen passiivi heijastelee perfektiivisen aspektin ominaisuuksia
siinä missä morfologinen passiivi imperfektiivisen aspektin. Myös tätä
passiivimuotoa on mahdollista käyttää niin menneen (esimerkit 9, 10)
kuin ei-menneen ajan toiminnasta (esimerkki 11).

\begin{enumerate}
\def\labelenumi{(\arabic{enumi})}
\setcounter{enumi}{8}
\tightlist
\item
  Данная технология \emph{была использована} для подготовки входных
  тестов по математике
\item
  Вертолёт \emph{был обнаружен} 6 сентября в Ингушетии
\item
  Особое внимание будет уделено наращиванию международного партнерского
  сотрудничества
\end{enumerate}

Perifrastista passiivia voidaan käyttää myös ilman быть-verbiä. Tällöin
on Vanhala-Aniszewskin (1992: 66) mukaan kyse niin kutsutusta passiivin
perfektimerkityksestä (перфект пассива). Keskeisintä tässä merkityksessä
on ajatus siitä, että toiminnan tulos on puhehetkellä voimassa. Näin
ollen esimerkissä 12 Hodorkovski on puhehetkellä pidätettynä (merkitys
on staattinen), kun taas быть-verbin sisältävässä esimerkissä 13 pidätys
kuvataan menneisyydessä tapahtuneena toimintana, jonka tuloksen ei
oleteta olevan enää voimassa:

\begin{enumerate}
\def\labelenumi{(\arabic{enumi})}
\setcounter{enumi}{11}
\tightlist
\item
  Ходорковский арестован
\item
  Ходорковский был арестован (после чего его повезли в Москву для
  допросов)
\end{enumerate}

\subsubsection{Indefiniittis-persoonainen
rakenne}\label{indefiniittis-persoonainen-rakenne}

Indefiniittis-persoonaiset (неопределенно-личные) rakenteet eivät ole
morfologian tai syntaksin kannalta varsinaisia passiivimuotoja, mutta
niitä yhtä kaikki käytetään samaan tarkoitukseen kuin passiivia: ennen
kaikkea varsinaisen tekijän häivyttämiseen tai identifioimatta
jättämiseen. Kiinnostavaa kyllä, juuri indefiniittis-persoonaiset
rakenteet vastaavat esimerkiksi Nikunlassin (2002: 162, 203) mukaan
kaikkein lähimmin suomen passiivirakenteita.

Indefiniittis-persoonaiset rakenteet muodostetaan venäjässä
tavallisimmin jättämällä lauseesta pois subjekti ja käyttämällä verbin
monikon kolmannen persoonan muotoa tai monikon preteritimuotoa. Lisäksi
on mahdollista käyttää быть-verbiä ja lyhyttä adjektiivia tai passiivin
partisiipin preteritin lyhyttä neutrimuotoa, jota usein (muttei aina)
seuraa infinitiivi. Šeljakin (2006: 261) antaa kustakin
indefiniittis-persoonaisesta lausetyypistä seuraavat esimerkit.
Esimerkissä 14 muodostuksen pohjana on monikon kolmas persoona,
esimerkissä 15 preteritin monikkomuoto, esimerkissä 16 partisiippimuoto
infinitiivin kanssa ja esimerkissä 17 partisiippimuoto ilman
infinitiiviä.

\begin{enumerate}
\def\labelenumi{(\arabic{enumi})}
\setcounter{enumi}{13}
\tightlist
\item
  По радио передают последние известия
\item
  Выставку открыли вчера
\item
  В школе были довольны его работой
\item
  Нам рекомендовано отменить поход в горы
\item
  В городе построено много домов
\end{enumerate}

Tässä yhteydessä on hyvä mainita, että indefiniittis-persoonaista
rakennetta muistuttavat myös \emph{geneeris-persoonaiset}
(обобщенно-личные) rakenteet. En tarkastele niitä kuitenkaan yhtenä
passiivin ilmaisun keinona, vaan kyseessä on ennemminkin, nimensä
mukaisesti, pyrkimys yleistää jokin ilmiö yhden esimerkkitapauksen
perusteella tarkoittamaan ketä tahansa ihmistä tai jonkin konkreettisen
ryhmän (lapset, opiskelijat ym.) edustajaa. Tässä merkityksessä
tavallinen on etenkin yksikön toinen persoona: \emph{если ты знаешь
иностранные языки, найти работу будет нетрудно}.

\subsubsection{Passiivilauseen agentti}\label{passiivilauseen-agentti}

Toisin kuin suomessa, venäjässä passiivilauseenkin tekijä voidaan
erikseen mainita. Tähän käytetään niin kutsuttua
\emph{agenttitäydennystä} (агентивное дополнение), joka ilmaistaan
instrumentaalilla. Esimerkiksi virkkeeseen 13 voitaisiin liittää tekijä,
jolloin saataisiin virke 19. Tekijä voidaan lisätä myös morfologisesti
muodostettuun passiivirakenteeseen, jolloin esimerkistä 4 saadaan 20:

\begin{enumerate}
\def\labelenumi{(\arabic{enumi})}
\setcounter{enumi}{18}
\tightlist
\item
  Ходорковский был арестован полицией
\item
  Завод строится компанией ``NCC''.
\end{enumerate}

Huomaa kuitenkin, että vaikka tekijän ilmaisu on mahdollista, se ei ole
yleistä. Vanhala-Aniszewskin (1992: 244--245) mukaan agenttimuotoja
käytetään lähinnä virallisessa asiatyylissä ja tieteellisessä tyylissä.
Nikunlassi (2002: 200) puolestaan esittää jopa tarkan luvun: vain
20\%:ssa passiivimuodoista on ilmaistu agentti.

\hyperdef{}{ref-nikunl}{\label{ref-nikunl}}
Nikunlassi, Ahti. 2002. \emph{Johdatus Venäjän Kieleen Ja Sen
Tutkimukseen}. Helsinki: Finn Lectura.

\hyperdef{}{ref-vanhala92}{\label{ref-vanhala92}}
Vanhala-Aniszewski, Marjatta. 1992. ``Функции Пассива В Русском И
Финском Языках (= Studia Philologica Jyväskyläensia 25).''
\emph{Jyväskylä: University of Jyväskylä}.

\hyperdef{}{ref-sheljakin}{\label{ref-sheljakin}}
Шелякин, М.А. 2006. \emph{Справочник По Русской Грамматике}. drofa.

\end{document}
