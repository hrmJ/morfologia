\documentclass[]{scrartcl}
\usepackage[utf8]{inputenc}
\usepackage[T1]{fontenc}
\usepackage[T2A]{fontenc}
\usepackage[finnish]{babel}
\usepackage{linguex} 
\usepackage{longtable} 
\usepackage{booktabs}
\usepackage{amsthm}
\newtheorem{maar}{Määritelmä}
\usepackage{fixltx2e} % provides \textsubscript
\usepackage{textcomp} % provides \textsubscript
\usepackage{hyperref}
\usepackage{xcolor}
\usepackage{tabu}

\providecommand{\tightlist}{%
  \setlength{\itemsep}{0pt}\setlength{\parskip}{0pt}}

\hypersetup{
    colorlinks,
    linkcolor={red!50!black},
    citecolor={blue!50!black},
    urlcolor={blue!80!black}
}
\author{Juho Härme}
\title{Morfologia-kurssin luentomateriaaleja}
\date{\today}
\begin{document}
\maketitle
\tableofcontents
\newpage



\section{Luento 4: sija kieliopillisena
kategoriana}\label{luento-4-sija-kieliopillisena-kategoriana}

\begin{itemize}
\tightlist
\item
  \href{https://mustikka.uta.fi/~juho_harme/morfologia/\#tästä-kurssista}{Takaisin
  sivun ylälaitaan}
\item
  \href{http://mustikka.uta.fi/~juho_harme/morfologia/materiaalit/luento4.pdf}{Lataa
  PDF}
\item
  \href{http://mustikka.uta.fi/~juho_harme/morfologia/presentations/luento4.html}{Tutki
  luentokalvoja}
\item
  \href{http://mustikka.uta.fi/~juho_harme/morfologia/tehtavat/luento4.pdf}{Tutki
  tuntitehtäviä}
\end{itemize}

Linkkejä online-harjoituksiin:

\begin{itemize}
\tightlist
\item
  \href{http://www.auburn.edu/~mitrege/russian/exercises/0018g.html}{Monikon
  nominatiivi}
\item
  \href{http://www.auburn.edu/~mitrege/russian/exercises/0095.html}{Genetiivi}
\item
  \href{http://www.auburn.edu/~mitrege/russian/exercises/0033.html}{Prepositionaali}
\item
  \href{http://www.auburn.edu/~mitrege/russian/exercises/0049.html}{Datiivi}
\end{itemize}

Sija (падеж) on venäjässä keskeisin nominien taivutuskategoria
(Nikunlassi 2002: 138). Sijojen ilmaisuvoima on suuri -- eri sijojen
käyttötarkoituksiin pureudutaan tarkemmin luennoilla 7--9. Tällä
luennolla tarkastellaan, mitä sijoja venäjän sijajärjestelmään kuuluu ja
miten substantiiveista eri sijat muodostetaan, toisin sanoen perehdytään
substantiivien taivutussarjoihin eli deklinaatioihin (типы склонения
существительных).

\subsection{Venäjän sijat}\label{venuxe4juxe4n-sijat}

Venäjän tavallisimmin käytetyt sijat on lueteltu seuraavassa taulukossa:

\begin{longtable}[c]{@{}lll@{}}
\toprule
nimitys (suom) & nimitys (ven.) & esimerkki
substantiivista\tabularnewline
\midrule
\endhead
nominatiivi & именительный падеж & курс\tabularnewline
genetiivi & родительный падеж & курса\tabularnewline
datiivi & дательный падеж & курсу\tabularnewline
akkusatiivi & винительный падеж & курса\tabularnewline
instrumentaali & творительный падеж & курсом\tabularnewline
prepositionaali & предложный падеж & курсе\tabularnewline
\bottomrule
\end{longtable}

Näiden yleensä mainittujen sijojen lisäksi venäjän sijajärjestelmästä
voidaan erottaa myös eräitä harvinaisempia sijoja. Näistä sijoista
(tarkemmin ks. esim. Nikunlassi 2002: 138-139) on huomattava, että ne
kaikki ovat käytössä vain yksikössä. Harvinaisempien sijojen
mainitsemisella tässä yhteydessä on se funktio, että on hyvä tiedostaa
näiden muotojen olevan selitettävissä todellisuudessa muutenkin kuin
vain ``poikkeukselliseksi genetiiviksi'', ``poikkeukselliseksi
prepositionaaliksi'' tms. Oiva lähde pohdittaessa näiden sijojen
nimittämistä \emph{sijoiksi} on
\href{http://lup.lub.lu.se/luur/download?func=downloadFile\&recordOId=3810042\&fileOId=3810043}{tämä
Lundin yliopiston opinnäyte} (Trakymaite 2004).

\textbf{Partitiivinen genetiivi}

Partitiivisen genetiivin (родительный партитивный) muoto on ainoastaan
pienellä joukolla substantiiveja. Esimerkkeinä partitiivisen genetiivin
muodoista voidaan mainita vaikka sananmuodot \emph{чаю}, \emph{народу}
ja \emph{снегу}. Partitiivisen genetiivin käyttö liittyy tilanteisiin,
joissa ilmaistan epämääräistä määrää: \emph{налить чаю}, \emph{стакан
сахару}. Nämä voidaan kuitenkin nykykielessä yleensä korvata
akkusatiivilla tai genetiivillä.

\textbf{Lokatiivi}

Lokatiivi (местный падеж) esiintyy myös rajoitetulla joukolla
substantiiveja. Sitä käytetään в- ja на-prepositioiden yhteydessä
ilmaisemaan paikkaa (\emph{на берегу}). Eron prepositionaaliin huomaa
verratessaan sanaa о-preposition yhteydessä esiintyvään muotoon: \emph{о
береге}. Huomaa myös кровь-sanan yhteydessä lokatiivimuoto \emph{в
кровИ} ja prepositionaali muoto \emph{о крОви}. Tähän ryhmään voidaan
lukea myös tyyliltään puhekieliset muodot \emph{в отпускУ}, \emph{в
чаЮ}, joiden kohdalla myös tavallinen e-päätteinen prepositionaalimuoto
on mahdollinen (vrt. Шелякин 2006: 47).

\textbf{Adnumeratiivi}

Adnumeratiivi (аднумератив) koskee käytännössä kolmea substantiivia:
шаг, ряд ja час. Näiden genetiivimuodot ovat шáга, рЯда ja чáса, mutta
lukusanojen 2-4 yhteydessä niistä käytetään muotoja шагá, рядá, часá.

\textbf{Vokatiivi}

Vokatiivin (звательный падеж) muotoja on ennen käytetty
säännönmukaisesti puhuttelun yhteydessä. Tästä jäänteenä ovat edelleen
eräät huudahdukset, kuten \emph{господи!}.

Myös nykyvenäjässä voidaan kuitenkin nähdä olevan vokatiivi
puhekielessä. Nämä muodot lankeavat yhteen monikon genetiivimuodon
kanssa, esimerkiksi: \emph{Мам!/Пап!/Саш!} (Супрун 2001: 93).

\subsection{Mitkä ihmeen
deklinaatiot?}\label{mitkuxe4-ihmeen-deklinaatiot}

\emph{Deklinaatiolla} (тип склонения) tarkoitetaan substantiivin
taivutusmuotojen sarjaa. Kielen sanoja tutkittaessa tuskin koskaan on
niin, että kaikki sanat taipuisivat juuri samalla tavalla: kaikkien
sanojen genetiivi (jos kielessä genetiiviä on) muodostettaisiin tietyllä
morfilla, kaikkien sanojen akkusatiivi olisi samoin juuri sama morfi ja
niin edelleen. Luonnollisesti asia ei ole niinkään, että esimerkiksi
sijojen muodostaminen olisi täysin sattumanvarainen prosessi, jossa
jokaisen sanan kohdalla sija muodostettaisiin aina eri tavalla.
Todellisuudessa eri lekseemien sananmuotoja tarkastelemalla havaitaan
aina, että tietyt sanat taipuvat samalla tavalla -- muodostavat ryhmiä
sen perusteella, minkälaisin morfein eri sijamuodot toteutuvat.

On kuitenkin tulkintakysymys, mikä lasketaan ryhmäksi. Kuten tällä
kurssilla olemme jo aiemmin huomanneet, ryhmien tai sääntöjen
muodostaminen on tasapainoilua parhaan kuvaustavan löytämiseksi:
toisaalta olisi hyvä, että ryhmiä olisi mahdollisimman vähän
(yleistykset olisivat mahdollisimman suuria) toisaalta ryhmään
kuuluminen ei saa olla liian väljää, niin että jouduttaisiin listaamaan
lukemattomia poikkeuksia.

Venäjän substantiiveja voidaan niiden yksikön sijamuotojen perusteella
ryhmitellä monellakin tavalla, ja mikä tapa valitaankin, jäännöksenä
syntyy aina koko joukko sanoja, jotka eivät oikein sovi mihinkään
ryhmään ja joita on pidettävä \emph{lekseemikohtaisina poikkeuksina}.

\subsection{Venäjän substantiivien jaottelu
deklinaatioihin}\label{venuxe4juxe4n-substantiivien-jaottelu-deklinaatioihin}

Koulukieliopeissa deklinaatiot typistetään usein kolmeen ja todetaan,
että kullakin suvulla on oma deklinaationsa (ks. esim. Wade 2010: 73).
Ei kuitenkaan ole mitään erityistä syytä olettaa, että suvun ja
deklinaation välillä olisi ehdottomasti syy--seuraussuhde, jossa suku
määrää taivutustyypin. Tämän vuoksi seuraavassa esitettävät viisi
deklinaatiotyyppiä (ks. Nikunlassi 2002: 140--) eivät lankea aivan
tarkalleen yksiin sukujen kanssa. Erityisesti kannattaa kiinnittää
huomiota siihen, että esimerkiksi toiseen deklinaatioon kuuluu myös
maskuliineja kuten \emph{мужчина}.

Jos suku ei määrää substantiivin taivutustyyppiä, niin mikä sitten? Jos
kyseessä on johdettu sanavartalo (toisin sanoen, jos sanan vartaloon
kuuluu juurimorfin lisäksi suffikseja), deklinaatiotyyppi määräytyy
viimeisen suffiksin mukaan. Esimerkiksi abstrakteja substantiiveja
muodostava \emph{ость}-suffiksi tuottaa kolmannen deklinaation mukaisen
taivutussarjan. Jos taas sanavartalo koostuu pelkästä juurimorfista,
määräytyy deklinaatio juurimorfin mukaan.

\subsubsection{Ensimmäinen
deklinaatio}\label{ensimmuxe4inen-deklinaatio}

Ensimmäiseen deklinaatioon kuuluvat käytännössä \textbf{kaikki
maskuliinit}. Ne on mielekästä ryhmitellä yhteen, koska sijat ilmaistaan
niissä koko lailla samoilla morfeilla.

\begin{itemize}
\tightlist
\item
  \textbf{Nominatiivin} päätteenä on nollamorfi (ø)
\item
  \textbf{Genetiivin} päätteenä on /а/. Huomaa, että jos edeltävä
  konsonantti on liudentunut, merkitään morfia kirjoituksessa
  я-kirjaimella. Itse morfi on silti sama vokaali.
\item
  \textbf{Akkusatiivin} pääte lankeaa yhteen joko nominatiivin tai
  genetiivin kanssa elollisuudesta riippuen
\item
  \textbf{Datiivin} päätteenä on morfi /у/. Jälleen kerran, jos
  edelätävä konsonantti on liudentunut, merkitään morfia kirjoituksessa
  ю-kirjaimella.
\item
  \textbf{Instrumentaalin} päätteenä on morfi /ом/
\item
  \textbf{Prepositionaalin} päätteenä on morfi /е/ tai /и/
\end{itemize}

Seuraava taulukko kuvaa ensimmäisen deklinaation kootusti esimerkkien
kautta:

\begin{longtable}[c]{@{}llll@{}}
\toprule
Sija & Päätemorfi & esim.1 & esim.2\tabularnewline
\midrule
\endhead
Nom. & ø & стол & учитель\tabularnewline
Gen. & /а/ & стола & учителя\tabularnewline
Dat. & /у/ & столу & учителю\tabularnewline
Akk. & /ø/ \textasciitilde{} /a/ & стол & учителя\tabularnewline
Instr. & /ом/ & cтолом & учителем\tabularnewline
Prep. & /е/ \textasciitilde{} /и & столе & учителе\tabularnewline
\bottomrule
\end{longtable}

\paragraph{Miten niin /ем/ ei ole oma
päätteensä?}\label{miten-niin-ux435ux43c-ei-ole-oma-puxe4uxe4tteensuxe4}

Aika todennäköisesti jokin edellisessä taulukossa ja sitä edeltävässä
listassa särähti korvaasi. Näissähän väitetään, että saneissa
\emph{учИтелем}, \emph{врачОм}, \emph{фонарЁм} ja \emph{дОмом} olisi
sama instrumentaalia ilmaiseva taivutuspääte. Kyse on fonologisista
seikoista, ja tässä kohden vaadittaneen pidempää selitystä kuin pelkkä
alaviite.

Verrataan sanoja ножОм ja мУжем. Nikunlassi (2002: 139) huomauttaa, että
syy eriäville kirjoitusasuille on oikeinkirjoitussäännössä:
konsonanttien ш, ж, ч, ц, щ jälkeen painoton pääte kirjoitetaan
\emph{-ем}, painollinen \emph{-ом}. Samoin pääte kirjoitetaan \emph{-ем}
tai \emph{-ём} liudentuneiden konsonanttien jäljessä ja /j/-foneemin
jäljessä. Tämä ei poista kuitenkaan sitä, että fonetiikan (ja
morfologian) näkökulmasta kyseessä on yhdestä ja samasta päätteestä,
morfista /ом/. Painollisen tavun jälkeisessä asemassa sijaitessaan
{[}о{]}-foneemi ääntyy joka tapauksessa redusoituneena toisen asteen
reduktion (ks. Nikunlassi 2002: 87) mukaisesti, niin että foneettisesti
katsoen /om/-päätteen vokaali sanoissa \emph{мУжем} ja \emph{дОмом} on
sama -- pieni ero on ainostaan siinä, että liudentuneen konsonantin
jälkeinen {[}о{]}-vokaalin variantti (sanassa \emph{мУжем}) eroaa hieman
liudentumattoman konsonantin jälkeisestä (sanassa \emph{дОмом}).

Kirjoitussääntöjen ei siis pidä antaa hämärtää sitä tosiasiaa, että
ortografisesti merkkijoinoilla \emph{ом} ja \emph{ем} ilmaistavat
päätteet ovat itse asiassa sama pääte, joiden sisältämä vokaalifoneemi
{[}о{]} on sama. Tällä vokaalifoneemilla on tosin päätteen osana
toimiessaan kolme eri allofonia: painollinen versio, liudentumattoman
konsonantin jälkeinen versio ja liudentuneen konsonantin jälkeinen
versio.

Fonetiikan kurssilta muistat ehkä, että {[}o{]}-foneemilla on myös niin
kutsuttu \emph{ensimmäisen asteen reduktion} mukainen allofoni, joka
esiintyy juuri painollista tavua edeltävässä tavussa -- esimerkiksi
sanoissa Россия, говорить (ensimmäinen näistä o-foneemeista edustaan
toisen asteen reduktiota, jälkimmäinen ensimmäisen asteen reduktiota).

\subsubsection{Toinen deklinaatio}\label{toinen-deklinaatio}

Toiseen deklinaatioon kuuluvat -/а/-morfiin päättyvät sanat. Niiden
sijapäätteistä voidaan todeta:

\begin{itemize}
\tightlist
\item
  Nominatiivi ilmaistaan morfilla /а/ (vrt. myös kirjoitusasu)
\item
  Genetiivi ilmaistaan morfilla /и/. Etu- ja taka-i ovat saman
  {[}и{]}-foneemin eri variantteja (allofoneja) (ks. Nikunlassi 2002:
  85, 94), joten äänneympäristöstä riippuen\footnote{Sanan alussa ja
    liudentuneen konsonantin jäljessä foneemi {[}и{]} edustuu etu-i:nä,
    liudentumattoman konsonantin jälkeen taka-i:nä. Lisäksi kun
    prepositio ja sitä seuraava sana ääntyvät yhtenä sanana, käyttöön
    tulee taka-i (с Ирой). (Nikunlassi 2002: 77, 94).} morfi voi edustua
  joko etisenä tai takaisena.
\item
  Akkusatiivi ilmaistaan morfilla /у/
\item
  Datiivissa on kaksi vaihtoehtoa: /е/ tai /и/
\item
  Instrumentaali ilmaistaan morfeilla /ой/ (tähän morfiin luetaan myös
  pääte \emph{ёй}) ja /ей/. Runokielessä on lisäksi olemassa päätteet
  /ою/, /ею/.
\item
  Prepositionaali ilmaistaan morfilla /е/ tai /и/
\end{itemize}

\begin{longtable}[c]{@{}lllll@{}}
\toprule
Sija & Päätemorfi & esim.1 & esim.2 & esim.3\tabularnewline
\midrule
\endhead
Nom. & /а/ & книга & серия & тыква\tabularnewline
Gen. & /и/ & книги & серии & тыквы\tabularnewline
Dat. & /е/ \textasciitilde{} /и/ & книге & серии & тыкве\tabularnewline
Akk. & /у/ & книгу & серию & тыкву\tabularnewline
Instr. & /ой/ \textasciitilde{} /ей/ & книгой & серией &
тыквой\tabularnewline
Prep. & /е/ \textasciitilde{} /и/ & книге & серии & тыкве\tabularnewline
\bottomrule
\end{longtable}

\subsubsection{Kolmas deklinaatio}\label{kolmas-deklinaatio}

Kolmanteen deklinaatioon kuuluvat liudentuneeseen konsonanttiin
päättyvät feminiinit. Niiden päätteet jakautuvat seuraavasti

\begin{itemize}
\tightlist
\item
  Nominatiivissa ja akkusatiivissa on nollamorfi /ø/
\item
  Genetiivissä, datiivissa ja prepositionaalissa /и/
\item
  instrumentaalissa /ju/
\end{itemize}

\begin{longtable}[c]{@{}llll@{}}
\toprule
Sija & Päätemorfi & esim.1 & esim.2\tabularnewline
\midrule
\endhead
Nom. & /ø/ & тетрадь & ночь\tabularnewline
Gen. & /и/ & тетради & ночи\tabularnewline
Dat. & /и/ & тетради & ночи\tabularnewline
Akk. & /ø/ & тетрадь & ночь\tabularnewline
Instr. & /ju/ & тетрадью & ночью\tabularnewline
Prep. & /и/ & тетради & ночи\tabularnewline
\bottomrule
\end{longtable}

\subsubsection{Neljäs deklinaatio}\label{neljuxe4s-deklinaatio}

Neljäs deklinaatio on hyvin samanlainen kuin ensimmäinen: ainoana erona
on nominatiivi, jota neljännessä deklinaatiossa ei ilmaista
nollapäätteellä vaan päätteillä /о/ ja /е/. Tämä deklinaatio voitaisiin
hyvin esittää myös osana ensimmäistä.

\begin{longtable}[c]{@{}lllll@{}}
\toprule
Sija & Päätemorfi & esim.1 & esim.2 & esim.3\tabularnewline
\midrule
\endhead
Nom. & /о/ tai {[}е{]} & место & поле & пение\tabularnewline
Gen. & /а/ & места & поля & пения\tabularnewline
Dat. & /у/ & месту & полю & пению\tabularnewline
Akk. & /ø/ & место & поле & пение\tabularnewline
Instr. & /ом/ & местом & полем & пением\tabularnewline
Prep. & /е/ \textasciitilde{} /и/ & месте & поле & пении\tabularnewline
\bottomrule
\end{longtable}

\subsubsection{Viides deklinaatio}\label{viides-deklinaatio}

Viidennen deklinaation muodostavat yksinään мя-äänneyhdistelmään
päättyvät substantiivit. Ne ovat kaikki neutreja.

\begin{longtable}[c]{@{}lll@{}}
\toprule
Sija & Päätemorfi & esim\tabularnewline
\midrule
\endhead
Nom. & /а/ & время\tabularnewline
Gen. & /и/ & времени\tabularnewline
Dat. & /и/ & времени\tabularnewline
Akk. & /ø/ & время\tabularnewline
Instr. & /ом/ & временем\tabularnewline
Prep. & /е/ \textasciitilde{} /и/ & времени\tabularnewline
\bottomrule
\end{longtable}

\subsubsection{Monikon deklinaatiot}\label{monikon-deklinaatiot}

Esittelen seuraavassa Nikunlassin (Nikunlassi 2002: 142) ehdotuksen
monikon deklinaatioiksi. Monikkomuotojen muodostuksessa helppoa on, että
datiivi-, instrumentaali- ja prepositionaalimuodot ovat käytännössä aina
samat: muodostetaan päätteillä /ам/, /ами/, ja /ах/. Taivutuksen erot
tulevat esille kaikkein yleisimmissä sijoissa eli nominatiivissa ja
genetiivissä. Usein (esim. Шелякин 2006: 44) monikolle esitetäänkin
ainoastaan yksi deklinaatio.

\subsubsection{Vaihtoehtoisia
jaotteluita}\label{vaihtoehtoisia-jaotteluita}

Kuten kurssilla on painotettu, substantiivit voidaan jakaa
deklinaatioihin useammalla eri tavalla ja kukin jaottelu on aina
kyseisen tutkijan/jaottelun tekijän teoreettinen kannanotto. Tässä
esitetty jaottelu eroaa klassisesta deklinaatiojärjestelmästä, joka
menee karkeasti ottaen seuraavasti (tarkemmin ks. Шелякин 2006: 44--50):


\begin{longtabu} to \textwidth {|X|X|X|}
\toprule
deklinaation nimi & mitkä substantiivit & esimerkki\tabularnewline
\midrule
\endhead
I deklinaatio & yksikön nominatiivi päättyy /а/-morfeemiin (suurin osa
feminiinejä) & машина,баня\tabularnewline
II deklinaatio & yksikön nominatiivi nollapääte ja suku maskuliini sekä
/о/-päätteiset neutrit & стол,учитель,море,тело, музей\tabularnewline
III deklinaatio & yksikön nominatiivi liudentuneen konsonantin jälkeinen
nollapääte ja suku feminiini & ночь, молодёжь\tabularnewline
\bottomrule
\end{longtabu}

\paragraph{Ensimmäinen monikon
deklinaatio}\label{ensimmuxe4inen-monikon-deklinaatio}

Monikon nominatiivin ja genetiivin kannalta ensimmäinen ja kolmas
deklinaatio voidaan hyvin yhdistää. Tällöin säännöksi tulee, että
nominatiivin pääte on /и/ (muista, että kyseinen morfeemi kattaa sekä
etu- että taka-i:n) ja genetiivin joko /ей/ tai /ов/. Genetiivin
päätteen osalta voidaan todeta, että:

\begin{itemize}
\tightlist
\item
  Jos sanan vartalo päättyy parilliseen liudentuneeseen konsonanttiin
  tai konsonantteihin ж, ч, ш, щ, monikon genetiivin pääte on /ей/
\item
  Muussa tapauksessa monikon genetiivin pääte on /ов/
\end{itemize}

\begin{longtable}[c]{@{}llllll@{}}
\toprule
Sija & Päätemorfi & esim. 1 & esim.2 & esim. 3 & esim. 4\tabularnewline
\midrule
\endhead
Nom. & /и/ & экраны & выключа́тели & врач & ночи\tabularnewline
Gen. & /ей/ \textasciitilde{} /ов/ & экранов & выключа́телей & врачей &
ночей\tabularnewline
Akk. & /и/ & экраны & выключатели & врачей & ночи\tabularnewline
Dat. & /ам/ & экранам & выключа́телям & врачам & ночам\tabularnewline
Instr. & /ами/ & экранами & выключа́телями & врачами &
ночами\tabularnewline
Prep. & /ах/ & экранах & выключа́телях & врачах & ночах\tabularnewline
\bottomrule
\end{longtable}

\paragraph{Toinen monikon deklinaatio}\label{toinen-monikon-deklinaatio}

Toinen monikon deklinaatio voidaan muodostaa suoraan toisen yksikön
deklinaation pohjalta, niin että monikon nominatiivin pääte on /и/ ja
monikon genetiivin pääte /ø/

\begin{longtable}[c]{@{}lll@{}}
\toprule
Sija & Päätemorfi & esim. 1\tabularnewline
\midrule
\endhead
Nom. & /и/ & книги\tabularnewline
Gen. & /ø/ & книг\tabularnewline
Akk. & /и/ & книги\tabularnewline
Dat. & /ам/ & книгам\tabularnewline
Instr. & /ами/ & книгами\tabularnewline
Prep. & /ах/ & книгах\tabularnewline
\bottomrule
\end{longtable}

\paragraph{Kolmas monikon deklinaatio}\label{kolmas-monikon-deklinaatio}

Kolmas monikon deklinaatio saadaan, kun yhdistetään yksikön neljäs ja
viides deklinaatio. Tällöin voidaan todeta, että monikon nominatiivin
pääte on /а/ ja genetiivin /ø/.

\begin{longtable}[c]{@{}llll@{}}
\toprule
Sija & Päätemorfi & esim. 1 & esim.2\tabularnewline
\midrule
\endhead
Nom. & /и/ & времена & гнёзда\tabularnewline
Gen. & /ø/ & времён & гнёзд\tabularnewline
Akk. & /и/ & времена & гнёзда\tabularnewline
Dat. & /ам/ & временам & гнездам\tabularnewline
Instr. & /ами/ & временами & гнездами\tabularnewline
Prep. & /ах/ & временах & гнездах\tabularnewline
\bottomrule
\end{longtable}

\hyperdef{}{ref-nikunl}{\label{ref-nikunl}}
Nikunlassi, Ahti. 2002. \emph{Johdatus Venäjän Kieleen Ja Sen
Tutkimukseen}. Helsinki: Finn Lectura.

\hyperdef{}{ref-trakymaite2004}{\label{ref-trakymaite2004}}
Trakymaite, Ringaile. 2004. ``Падежная Система Современного Русского
Языка.'' Lund university.
\url{https://lup.lub.lu.se/student-papers/search/publication/3810042}.

\hyperdef{}{ref-wade2010}{\label{ref-wade2010}}
Wade, Terence. 2010. \emph{A Comprehensive Russian Grammar}. Vol. 8.
John Wiley \& Sons.

\hyperdef{}{ref-supr2001}{\label{ref-supr2001}}
Супрун, ВИ. 2001. ``Антропонимы В Вокативном Употреблении.''
\emph{Известия Уральского Государственного Университета. 2001. 20}.
\url{http://elar.urfu.ru/bitstream/10995/23753/1/iurg-2001-20-17.pdf}.

\hyperdef{}{ref-sheljakin}{\label{ref-sheljakin}}
Шелякин, М.А. 2006. \emph{Справочник По Русской Грамматике}. drofa.

\end{document}
