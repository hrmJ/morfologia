\documentclass[]{scrartcl}
\usepackage[utf8]{inputenc}
\usepackage[T1]{fontenc}
\usepackage[T2A]{fontenc}
\usepackage[finnish]{babel}
\usepackage{linguex} 
\usepackage{amsthm}
\newtheorem{maar}{Määritelmä}
\usepackage{fixltx2e} % provides \textsubscript
\usepackage{textcomp} % provides \textsubscript
\usepackage{hyperref}
\usepackage{xcolor}

\providecommand{\tightlist}{%
  \setlength{\itemsep}{0pt}\setlength{\parskip}{0pt}}

\hypersetup{
    colorlinks,
    linkcolor={red!50!black},
    citecolor={blue!50!black},
    urlcolor={blue!80!black}
}
\author{Juho Härme}
\title{Morfologia-kurssin luentomateriaaleja}
\date{\today}
\begin{document}
\maketitle
\tableofcontents
\newpage



\section{Luento 3: luku, suku ja sija kieliopillisina
kategorioina}\label{luento-3-luku-suku-ja-sija-kieliopillisina-kategorioina}

\begin{itemize}
\itemsep1pt\parskip0pt\parsep0pt
\item
  \href{https://mustikka.uta.fi/~juho_harme/morfologia/\#tästä-kurssista}{Takaisin
  sivun ylälaitaan}
\item
  \href{http://mustikka.uta.fi/~juho_harme/morfologia/materiaalit/luento3.pdf}{Lataa
  PDF}
\item
  \href{http://mustikka.uta.fi/~juho_harme/morfologia/presentations/luento3.html}{Tutki
  luentokalvoja}
\item
  \href{http://mustikka.uta.fi/~juho_harme/morfologia/tehtavat/luento3.pdf}{Tutki
  tuntitehtäviä}
\end{itemize}

\subsection{Luku kieliopillisena
kategoriana}\label{luku-kieliopillisena-kategoriana}

\subsubsection{Erikoistapauksia}\label{erikoistapauksia}

{[}tehtävä 3.1{]}

Substantiiveilla lukukategorian ilmaisemiseen liittyy kaksi
erityistapausta: niin kutsutut \emph{pluralia tantum}- ja
\emph{singularia tantum} -sanat. Kummatkaan näistä sanaryhmistä eivät
käytännössä ilmaise lukukategoriaa (Nikunlassi 2002, 147).

Pluralia tantum -sanat ovat aina monikossa, singularia tantum -sanat
puolestaan aina yksikössä esiintyviä sanoja. Pluralia tantum -sanat ovat
selärajaisempi luokka (siihen kuuluu venäjässä noin 600 sanaa) siinä
mielessä, että singularia tantum -ryhmän sanoilla voi usein tietyissä
käyttötapauksissa tulkita olevan monikkomuodonkin.

Mahdollisesti tulevina kielenopettajina kannattaa kiinnittää näihin
sanaryhmiin huomiota siinä suhteessa, että vaikka mainitut ilmiöt ovat
olemassa niin suomessa kuin venäjässä, eivät niihin kuuluvien sanojen
joukot ole identtisiä. Klassinen esimerkki tästä on venäjän sana
\emph{мебель}.

Suosittelen tustumaan näiden termien hyviin Wikipedia-artikkeleihin:

\begin{itemize}
\itemsep1pt\parskip0pt\parsep0pt
\item
  \href{https://ru.wikipedia.org/wiki/Singularia_tantum}{singularia
  tantum}
\item
  \href{https://ru.wikipedia.org/wiki/Pluralia_tantum}{pluralia tantum}
\end{itemize}

\subsection{Suku kieliopillisena
kategoriana}\label{suku-kieliopillisena-kategoriana}

Semanttinen ja morfologinen määräytyminen

\subsection{Sija kieliopillisena
kategoriana}\label{sija-kieliopillisena-kategoriana}

Mitä eri sijoja venäjässä on?

Nikunlassi, Ahti 2002. \emph{Johdatus venäjän kieleen ja sen
tutkimukseen}. Helsinki: Finn Lectura.

\end{document}
