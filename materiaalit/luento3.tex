\documentclass[]{scrartcl}
\usepackage[utf8]{inputenc}
\usepackage[T1]{fontenc}
\usepackage[T2A]{fontenc}
\usepackage[finnish]{babel}
\usepackage{linguex} 
\usepackage{longtable} 
\usepackage{booktabs}
\usepackage{amsthm}
\newtheorem{maar}{Määritelmä}
\usepackage{fixltx2e} % provides \textsubscript
\usepackage{textcomp} % provides \textsubscript
\usepackage{hyperref}
\usepackage{xcolor}

\providecommand{\tightlist}{%
  \setlength{\itemsep}{0pt}\setlength{\parskip}{0pt}}

\hypersetup{
    colorlinks,
    linkcolor={red!50!black},
    citecolor={blue!50!black},
    urlcolor={blue!80!black}
}
\author{Juho Härme}
\title{Morfologia-kurssin luentomateriaaleja}
\date{\today}
\begin{document}
\maketitle
\tableofcontents
\newpage



\section{Luento 3: luku ja suku kieliopillisina
kategorioina}\label{luento-3-luku-ja-suku-kieliopillisina-kategorioina}

\begin{itemize}
\tightlist
\item
  \href{https://mustikka.uta.fi/~juho_harme/morfologia/\#tästä-kurssista}{Takaisin
  sivun ylälaitaan}
\item
  \href{http://mustikka.uta.fi/~juho_harme/morfologia/materiaalit/luento3.pdf}{Lataa
  PDF}
\item
  \href{http://mustikka.uta.fi/~juho_harme/morfologia/presentations/luento3.html}{Tutki
  luentokalvoja}
\item
  \href{http://mustikka.uta.fi/~juho_harme/morfologia/tehtavat/luento3.pdf}{Tutki
  tuntitehtäviä}
\end{itemize}

\subsection{Luku kieliopillisena
kategoriana}\label{luku-kieliopillisena-kategoriana}

Kuten edellä todettiin, lukukategoria (категория числа) koostuu
venäjässä tasan kahdesta vaihtoehdosta, joiden suhteen sananmuodot ovat
oppositiossa keskenään: jos sanat ilmaisevat lukua, ne ilmaisevat joko
\emph{yksikköä} (единственное число) tai monikkoa (множественное число).

Voidaan sanoa, että luvun \emph{oletusarvo} on yksikkö (Nikunlassi 2002:
147). Näin voidaan päätellä, koska monet sananmuodot eivät voi olla
ilmaisematta lukua: esimerkiksi verbin \emph{стать} preteritimuoto
ilmaisee aina joko yksikköä tai monikkoa (стал,стала,стало,стали). Jos
lauseessa ei ole elementtiä, joka ohjaisi (kongruenssi) käyttämään
jompaakumpaa, valitaan yksikkö, niin kuin esimerkissä 1:

\begin{enumerate}
\def\labelenumi{(\arabic{enumi})}
\tightlist
\item
  Бабушке стало лучше.
\end{enumerate}

Tarkastellaan seuraavassa muutamaa lukukategoriaan liittyvää
erityistapausta.

\subsubsection{Erikoistapauksia}\label{erikoistapauksia}

Substantiiveilla lukukategorian ilmaisemiseen liittyy kaksi
erityistapausta: niin kutsutut \emph{pluralia tantum}- ja
\emph{singularia tantum} -sanat. Kummatkaan näistä sanaryhmistä eivät
käytännössä ilmaise lukukategoriaa (Nikunlassi 2002, 147).

Pluralia tantum -sanat ovat aina monikossa, singularia tantum -sanat
puolestaan aina yksikössä esiintyviä sanoja. Pluralia tantum -sanat ovat
selvärajaisempi luokka (siihen kuuluu venäjässä noin 600 sanaa) siinä
mielessä, että singularia tantum -ryhmän sanoilla voi usein tietyissä
käyttötapauksissa tulkita olevan monikkomuodonkin.

Mahdollisesti tulevina kielenopettajina kannattaa kiinnittää näihin
sanaryhmiin huomiota siinä suhteessa, että vaikka mainitut ilmiöt ovat
olemassa niin suomessa kuin venäjässä, eivät niihin kuuluvien sanojen
joukot ole identtisiä. Klassinen esimerkki tästä on venäjän sana
\emph{мебель}.

Suosittelen tutustumaan näiden termien hyviin Wikipedia-artikkeleihin:

\begin{itemize}
\tightlist
\item
  \href{https://ru.wikipedia.org/wiki/Singularia_tantum}{singularia
  tantum}
\item
  \href{https://ru.wikipedia.org/wiki/Pluralia_tantum}{pluralia tantum}
\end{itemize}

\subsection{Suku kieliopillisena
kategoriana}\label{suku-kieliopillisena-kategoriana}

Semanttinen ja morfologinen määräytyminen

\subsubsection{Substantiivien suvun
määräytyminen}\label{substantiivien-suvun-muxe4uxe4ruxe4ytyminen}

Substantiivien suvun määrätytyminen on melko monimutkainen ilmiö.
Esitettäessä siihen liittyviä säännönmukaisuuksia voidaan päätyä
hyvinkin erilaisiin kuvausmalleihin. Esittelen tässä kahta vaihtoehtoa.

\paragraph{Nikunlassin kuvaus}\label{nikunlassin-kuvaus}

Ahti Nikunlanssi (2002: 149) esittää lähtökohtaisesti hyvin
yksinkertaisen kaavan, jota soveltamalla substantiivin suku voidaan
määritellä:

\begin{enumerate}
\def\labelenumi{\arabic{enumi}.}
\item
  mies-/urospuolinen = maskuliini. nais-/naaraspuolinen = feminiini
\item
  \begin{enumerate}
  \def\labelenumii{\arabic{enumii}.}
  \tightlist
  \item
    deklinaatio = maskuliini, 2. ja 3. deklinaatio = feminiini, 4. ja 5.
    deklinaatio = neutri
  \end{enumerate}
\item
  taipumaton sana, lyhenne tms. = erillisiä sääntöjä
\end{enumerate}

Deklinaatioiden käsitteeseen pureudutaan tarkemmin seuraavalla
luennolla, joten selvyyden vuoksi voi olla tarpeen ennen jatkamista
tutustua alla olevaan deklinaatiosta kertovaan materiaaliin.

Perusajatus Nikunlassin kaavassa on, että jokaisen sanan kohdalla
katsotaan ensin, onko sillä merkityksensä puolesta (semanttisesti) jokin
suku. Esimerkiksi \emph{папа} on merkityksen puolesta maskuliini, joten
siihen sovelletaan ensimmäistä sääntöä eikä kaavassa tarvitse edetä
pidemmälle. Sanasta \emph{город} ei puolestaan merkityksen osalta voida
vielä sanoa mitään. Tämän vuoksi on katsottava toista sääntöä. Sana
город kuuluu taivutusmuotojensa puolesta ensimmäiseen deklinaation,
joten sen suvuksi määräytyy maskuliini. Sana \emph{умность} puolestaan
kuuluu kolmanteen deklinaatiooon, joten se on feminiini. Sana
\emph{море} kuuluu neljänteen deklinaatioon ja on neutri.

Taipumattomilla sanoilla ei ole deklinaatiota -- ne eivät nimensä
mukaisesti taivu, joten taivutusmuotojen sarjaakaan ei ole olemassa.
Tämän vuoksi näiden sanojen suvun määrittelemiseen tarvitaan erillisiä
sääntöjä. Sama koskee esimerkimsi lyhenteitä.

\paragraph{Šeljakinin kuvaus}\label{ux161eljakinin-kuvaus}

Edellistä perinteisempi tapa opettaa suvun määräytymistä on todeta, että
äänteisiin X, Y ja Z päättyvät substantiivit ovat maskuliineja,
äänteisiin A, B, C päättyvät feminiinejä ja äänteisiin D, E, F taas
neutreja. Tätä tapaa noudattaa esimerkiksi Šeljakin (2006: 28--) (ks.
myös esimerkiksi Wade 2010: 54). Kuvaus on selkeästi vähemmän yleistävä
kuin Nikunlassilla, mutta toisaalta tarkempi -- se antaa edellistä
kuvausta suorempia vastauksia yhtä konkreettista sanaa koskeviin
kysymyksiin. Toisaalta kuvaus sisältää suuren määrän poikkeuksia. Esitän
seuraavassa tiivistetyn version tästä kuvaustavasta.

Šeljakin lähtee liikkeelle siitä, että sanat voidaan suvun määrätymisen
suhteen jakaa viiteen pääryhmään:

\begin{enumerate}
\def\labelenumi{\arabic{enumi}.}
\tightlist
\item
  Taipuvat substantiivit
\item
  Taipumattomat substantiivit
\item
  Useampiosaiset sanat
\item
  ище- ja ищко-suffikseihin päättyvät sanat
\item
  Erisnimet
\end{enumerate}

Tarkastelen tässä lyhyesti kahta ensimmäistä ryhmää sekä viimeistä
ryhmää. Eniten huomiota täytyy kiinnittää ensimmäiseen ryhmään, joka
onkin määrällisesti laajin. Tämän ryhmän osalta suku määräytyy
seuraavasti:

\textbf{Ensiksi}: Yksikön nominatiivin päätteen mukaan seuraavan
taulukon kuvaamalla tavalla:

\begin{longtable}[c]{@{}lll@{}}
\toprule
esimerkkisana & nominatiivin pääte & suku\tabularnewline
\midrule
\endhead
книга & -{[}а{]} & F\tabularnewline
неделя & -{[}а{]} & F\tabularnewline
море & -{[}е{]} & N\tabularnewline
тело & -{[}о{]} & N\tabularnewline
ночь & pehmeä konsonantti + ø & F\tabularnewline
обед & kova konsonantti + ø & M\tabularnewline
гений & -{[}j{]} & M\tabularnewline
\bottomrule
\end{longtable}

Tähän määräytymistapaan on kuitenkin seuraavia poikkeuksia tai
lisäyksiä.

\begin{enumerate}
\def\labelenumi{\arabic{enumi}.}
\tightlist
\item
  kaikki мя-päätteiset sanat ovat neutreja
\item
  tietyt kovaan sibilanttiäänteeseen päättyvät ovat feminiinejä --
  merkiksi tästä lisätään sanan loppuun pehmeä merkki (молодёжь, ложь,
  дрожь, мышь, роскошь)
\item
  eräät pehmeään konsonanttiin päättyvät ovat maskuliineja. Suurin
  tällainen ryhmä ovat \emph{тель}-päätteiset sanat. Toinen selkeä ryhmä
  ovat kuukausien nimet. Näiden lisäksi on noin 150 muuta sanaa, jotka
  päättyvät pehmeään konsonanttiin, mutta ovat maskuliineja.
\end{enumerate}

\textbf{Toiseksi}: sanan merkityksen mukaan

Yleissääntö on yksinkertainen:

\begin{enumerate}
\def\labelenumi{\arabic{enumi}.}
\tightlist
\item
  Kaikki miespuoliseen henkilöön tai eläimeen viittaavat ovat
  maskuliineja
\item
  Kaikki naispuoliseen henkilöön tai eläimeen viittaavat ovat
  feminiinejä
\end{enumerate}

Tilanne on todellisuudessa kuitenkin mutkikkaampi. Yksi ongelmallinen
ryhmä ovat henkilöihin viittaavat sanat, jotka päätteensä perusteella
ovat maskuliineja. Näihin tapauksiin on seuraavanlaisia ratkaisuja:

\begin{enumerate}
\def\labelenumi{\arabic{enumi}.}
\tightlist
\item
  Kun sana ei ole ammatti, toimeenkuva tai muu vastaava: esimerkiksi
  \emph{человек}, \emph{враг}, \emph{подросток} ja \emph{гений} ovat
  aina maskuliineja. (Она -- хороший человек)
\item
  Ammattia tms. tarkoittavalla sanalla voi olla feminiininen
  rinnakkaismuoto, jota tietyissä tilanteissa voidaan käyttää
  maskuliinimuodon sijaan (студентка, певица, шахматистка ym.).
\item
  Ammattia tms. tarkoittavalla sanalla ei aina ole feminiinistä
  rinnakkaismuotoa (врач, профессор, доктор, кондуктор). Nämä sanat ovat
  tavallisesti maskuliineja. Tietyissä ympäristöissä suku voi kuitenkin
  olla feminiini: \emph{в гостях была наш врач}.
\end{enumerate}

Lisäksi esimerkiksi sanoilla \emph{плакса}, \emph{убийца},
\emph{пяница}, \emph{коллега}, \emph{умница} suku määräytyy riippuen
kulloinkin puheena olevasta henkilöstä.

Toinen Šeljakinin sanaryhmä ovat taipumattomat substantiivit. Sellaiset
taipumattomat sanat, jotka eivät viittaa henkilöön, ovat yleensä
neutreja -- poikkeuksena kuitenkin esimerkiksi sanat \emph{пенальти} ja
\emph{кофе}.

Jos taipumattomat sanat viittaavat henkilöön, niiden suvun määräytyminen
muistuttaa taipuvien henkilöön viittaavien sanojen suvun määräytymistä.
Esimerkiksi sanat \emph{рефери} ja \emph{аташе} vertautuvat tällöihin
sanoihin \emph{врач}, \emph{профессор} ym. ja ovat lähtökohtaisesti
maskuliineja. On kuitenkin myös selkeästi feminiinejä taipumattomia
henkilösanoja, esimerkiksi sanat \emph{леди} ja \emph{мадам}.

Jos ei ole välttämätöntä erikseen painottaa taipumattoman eläimen sukua,
sen suku on myös maskuliini: \emph{коричневый кенгуру}, \emph{умный
шимпанзе}. Lisäksi taipumattomien sanojen sisällä on joukko sanoja,
joiden suku määräytyy niiden taustalla olevaan yleisnimeen: kielet,
kaupungit ja saaret ovat maskuliineja (vrt. язык, город, остров), monet
joet taas feminiinejä. Lyhenteiden osalta pätee yleensä lyhenteen
pääsanan suku (Московский Государственный \textbf{Университет})

Erisnimistä (viides ryhmä) Šeljakin toteaa, että etunimien suku
luonnollisesti määräytyy sen kantajan mukaan. -он, -ин ja
-cкий-päättyvistä sukunimistä on erikseen feminiiniversio, muista ei.

\hyperdef{}{ref-nikunl}{\label{ref-nikunl}}
Nikunlassi, Ahti. 2002. \emph{Johdatus Venäjän Kieleen Ja Sen
Tutkimukseen}. Helsinki: Finn Lectura.

\hyperdef{}{ref-wade2010}{\label{ref-wade2010}}
Wade, Terence. 2010. \emph{A Comprehensive Russian Grammar}. Vol. 8.
John Wiley \& Sons.

\hyperdef{}{ref-sheljakin}{\label{ref-sheljakin}}
Шелякин, М.А. 2006. \emph{Справочник По Русской Грамматике}. drofa.

\end{document}
