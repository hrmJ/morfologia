\documentclass[]{scrartcl}
\usepackage[utf8]{inputenc}
\usepackage[T1]{fontenc}
\usepackage[T2A]{fontenc}
\usepackage[finnish]{babel}
\usepackage{linguex} 
\usepackage{longtable} 
\usepackage{booktabs}
\usepackage{amsthm}
\usepackage{graphicx}
\newtheorem{maar}{Määritelmä}
\usepackage{fixltx2e} % provides \textsubscript
\usepackage{textcomp} % provides \textsubscript
\usepackage{hyperref}
\usepackage{xcolor}

\providecommand{\tightlist}{%
  \setlength{\itemsep}{0pt}\setlength{\parskip}{0pt}}

\hypersetup{
    colorlinks,
    linkcolor={red!50!black},
    citecolor={blue!50!black},
    urlcolor={blue!80!black}
}
\author{Juho Härme}
\title{Morfologia-kurssin luentomateriaaleja}
\date{\today}
\begin{document}
\maketitle
\tableofcontents
\newpage



\section{Luento 9: sijat funktionaalisen morfologian kannalta
3}\label{luento-9-sijat-funktionaalisen-morfologian-kannalta-3}

\begin{itemize}
\tightlist
\item
  \href{https://mustikka.uta.fi/~juho_harme/morfologia/\#tästä-kurssista}{Takaisin
  sivun ylälaitaan}
\item
  \href{http://mustikka.uta.fi/~juho_harme/morfologia/presentations/luento9.html}{Tutki
  luentokalvoja}
\end{itemize}

\subsection{Elollisuuskategoria}\label{elollisuuskategoria}

Elollisuus (одушевлённость) on aspektin ohella yksi niistä venäjän
kieliopillisista kategorioista, jotka suomessa eivät ole kieliopillisen
kategorian asemassa. Kuten suku, myös elollisuus on ominaisuus, joka
määräytyy substantiivin vartalon mukaan, mutta ilmenee ennen kaikkea
kongruoivien sanojen muotojen kautta (Nikunlassi 2002: 152).

\subsection{Miten elollisuuskategoria
ilmenee?}\label{miten-elollisuuskategoria-ilmenee}

Elollisuuskategoria ja akkusatiivin sija liittyvät venäjässä kiinteästi
yhteen. Substantiivin elollisuus näkyy siten, \textbf{että elollisilla
substantiivella akkusatiivi on tietyissa tapauksissa genetiivin
kaltainen}. Tarkemmin sanottuna:

\begin{enumerate}
\def\labelenumi{\arabic{enumi}.}
\tightlist
\item
  Jos substantiivi kuuluu I deklinaatioon ja on yksikössä
\item
  Riippumatta deklinaatiosta, jos substantiivi on monikossa
\end{enumerate}

Jos taas substantiivi kuuluu II deklinaatioon, akkusatiivilla on
\textbf{suvusta riippumatta} oma päätteensä:

\begin{enumerate}
\def\labelenumi{(\arabic{enumi})}
\tightlist
\item
  Иными словами, \emph{на каждого мужчину} ``от 35 и выше'' приходится
  почти две женщины соответствующего возраста.
\item
  Ты случайно \emph{Ваню Воронкова} не знаешь?
\item
  Одна молодая красивая американская девушка полюбила одного русского
  мужчину.
\end{enumerate}

Elollisuuskategorian ilmaiseminen on siis morfologisesti yksinkertaista.
Hieman mutkikkaampi on kysymys siitä, \emph{milloin} sana on elollinen.

\subsection{Milloin elollisuuskategoria
ilmenee?}\label{milloin-elollisuuskategoria-ilmenee}

Šeljakin (2006: 43) huomauttaa, että \emph{elollisuus} ja
\emph{elottomuus} eivät kieliopillisena kategoriana lankea yhteen sen
kanssa, mitä esimerkiksi biologiassa pidetään elävänä. Katso vertailun
vuoksi vaikkapa seuraavaa Wikipedian määritelmää, jonka mukaan elolliset
organismit:

\emph{maintain homeostasis, are composed of cells, undergo metabolism,
can grow, adapt to their environment, respond to stimuli, and
reproduce.}

Jos mietitään venäjän elollisuuskategoriaa, selvin esitetyn biologisen
määritelmän ulkopuolelle jäävä joukko ovat kasvien nimet: я видел
берёзы, Миша купил невесте незабытки. Hieman epämääräisempi joukko ovat
erilaiset alkeellisemmat organismit, joita ei pidetä elollisena: \emph{в
этой лаборатории изучают микробы/вирусы/эмбрионы/бактерии.} Myöskään
joukkoihin viittaavat kollektiivisubstantiivit eivät ole elollisia --
näistä Nikunlassi (2002: 152) mainitsee muun muassa sanat \emph{народ}
ja \emph{стадо}. Šeljakin puolestaan liittää vielä erikseen kiinteät,
\emph{tiettyyn joukkoon pyrkimistä} tarkoittavat ilmaisut, kuten
seuraavissa kansalliskorpuksen esimerkkilauseissa:

\begin{enumerate}
\def\labelenumi{(\arabic{enumi})}
\setcounter{enumi}{3}
\tightlist
\item
  Прежде всего: должно быть трудно «поступить в музыковеды».
\item
  Должно быть, тем же летом 1916 года \emph{я поступила в скауты.}
\end{enumerate}

\textbf{Lukusanoista:} perustilanne on, etteivät lukusanat taivu
elollisuuskategorian mukaisesti:

\begin{enumerate}
\def\labelenumi{(\arabic{enumi})}
\setcounter{enumi}{5}
\tightlist
\item
  Мы видели пять генералов.
\end{enumerate}

Tähän on kuitenkin kaksi poikkeusta: lukusanat 1--4 sekä
kollektiivilukusanat, kuten seuraavissa esimerkeissä:

\begin{enumerate}
\def\labelenumi{(\arabic{enumi})}
\setcounter{enumi}{6}
\tightlist
\item
  Едва взглянул в её глаза, как уже имел \emph{пятерых} детей,
  стиральную машину ``Индезит'' и смерть в семидесятисемилетнем возрасте
  от анемии.
\item
  Тринадцатого июля вершилась казнь. Вешали \emph{четверых} декабристов.
\end{enumerate}

\subsubsection{Erityistapauksia}\label{erityistapauksia}

Nikunlassin (2002: 153) perusteella voidaan luetella seuraavat
elollisuuskategorian ilmenemiseen liittyvät erikoistapaukset:

\begin{enumerate}
\def\labelenumi{\arabic{enumi}.}
\item
  Esimerkiksi sanat \emph{снеговик} ja \emph{кукла} voivat tulla
  kohdelluiksi elollisina, jos niihin viitataan ikään kuin ne olisivat
  eläviä henkilöitä. Samoin myös normaalisti puhtaasti esinettä
  tarkoittavat, kuten esimerkin 9 \emph{звезда}, voivat henkilöön
  viitatessaan saada elollisen tulkinnan Tähän joukkoon voidaan liittää
  myös esimerkiksi sana \emph{лицо}, joka henkilöä tarkoittessaan
  käyttäytyy elollisen substantiivin tavoin.
\item
  Kalat ja äyriäiset tulkitaan elottomiksi, jos puhutaan niiden
  syömisestä (esimerkit 10 ja 11).
\item
  Pelikorttien ja shakkinappuloiden nimet (король, дама, ym.) ovat
  elollisia.
\end{enumerate}

\begin{enumerate}
\def\labelenumi{(\arabic{enumi})}
\setcounter{enumi}{8}
\tightlist
\item
  Кстати, мне за Клевченю очень обидно: в команде он вынужден работать
  не на себя, а на \emph{голландских звёзд}.
\item
  Она обычно заказывает \emph{креветки} (`katkarapu'), свежую клубнику
  и\ldots{}
\item
  Рыбаки ловят \emph{креветок}, готовятся к завтрашней рыбалке.
\end{enumerate}

\subsection{Akkusatiivin
merkityksistä}\label{akkusatiivin-merkityksistuxe4}

Akkusatiivin perusmerkitys on selkeä: kyseessä on suoran objektin sija,
joka ilmaisee toiminnan kohdetta. Venäjässä akkusatiivilla on kuitenkin
myös muita merkityksiä. Myös tässä voidaan erottaa omaksi joukokseen
paikan ja ajan merkitykset.

Paikkaan tai tilassa tapahtuvaan liikkeeseen liittyvät merkitykset
ilmaistaan prepositiolla на, в, за, под ja о. Suurin osa näistä on
tuttuja ja koko lailla selkeitä: akkusatiivi ilmaisee liikettä
\emph{johonkin suuntaan} ennemmin kuin \emph{jossakin suunnassa}:

\begin{enumerate}
\def\labelenumi{(\arabic{enumi})}
\setcounter{enumi}{11}
\tightlist
\item
  Пришли \emph{на работу}, а потом пошли пообедать
\item
  Российская делегация прибыла \emph{в Стамбул} с намерением
  способствовать дальнейшему укреплению связей с Организацией Исламская
  конференция
\item
  Дурно пахнущие юноши из ``группы поддержки'', стоявшие возле меня,
  поспешно выскочили за дверь в коридор.
\item
  К вечеру Сергей пришёл под окно с саксофоном
\end{enumerate}

\textbf{О-preposition} tulee usein mieltäneeksi vain prepositionaaliin
liittyväksi. Kuitenkin sillä on akkusatiivin yhteydessä käytettäessä
tärkeä paikan ilmaisemiseen liittyvä merkitys, törmääminen/osuminen
johonkin.

\begin{enumerate}
\def\labelenumi{(\arabic{enumi})}
\setcounter{enumi}{15}
\tightlist
\item
  Если не помогает этот приём, попробуйте стукнуть лбом о клавиатуру.
\item
  Как будто разбиться \emph{о землю} хуже, чем сгореть в танке?
\item
  Я долго вытирал ботинки \emph{о траву} и промочил их насквозь
\end{enumerate}

Ajan ilmaisemisessa akkusatiivilla voidaan sanoa olevan venäjän sijoista
tärkein rooli. Erityisesti niin kutsutussa \emph{lokalisoivassa} tai
\emph{ajankohtaa ilmaisevassa} funktiossa (временные локализаторы) --
kun ilmaistaan \emph{milloin} tai minä ajankohtana jotakin tapahtui --
akkusatiivi on (eräiden prepositionaalitapausten ohella) tavallisin
sija. Näissä tapauksissa akkusatiivia käytetään в-preposition yhteydessä
(esimerkki 19). \emph{Kestoa} (продолжительность) eli ajallista
ekstensiota ilmaistaessa akkusatiivia käytetään ilman prepositioita
(esimerkki 20). Ajallista päämäärää ilmaistaessa prepositio on на
(esimerkki 21). Myös под+akkusatiivi on mahdollinen (esimerkki 22) ja
usein rinnastettavissa к+datiivi-tapauksiin:

\begin{enumerate}
\def\labelenumi{(\arabic{enumi})}
\setcounter{enumi}{18}
\tightlist
\item
  \emph{В вечер} моего дебюта в ложе дирекции находился Михаил Иванович
  Чулаки
\item
  Премьер Яценюк вместе с министрами \emph{всю неделю} ходил по
  парламентским фракциям.
\item
  он приглашает меня на месяц в Ялту
\item
  Зато \emph{под утро} я заснул как убитый и пробудился в половине
  двенадцатого.
\end{enumerate}

Kun mietitään на-prepositiota seuraavaa akkusatiivia, on usein
kiinnitettävä huomiota siihen, että etenkin puhekielessä paino saattaa
osua substantiivin sijasta на-prepositiolle. Usein tällöin on kyse
idiomeista, joista tavallisin lienee \emph{нА ночь гладя}. Katso pidempi
lista esimerkiksi
\href{http://fonetica.philol.msu.ru/akcent/bezud_slaboud/ud_na_pr.html}{täältä}.

Viimeisenä hajanaisena akkusatiivihuomiona kehotan muistamaan myös
seuraavat akkusatiivin käyttötilanteet: \emph{за какую цену / ни за что,
учиться на кого}.

\subsection{Instrumentaalin
merkityksiä}\label{instrumentaalin-merkityksiuxe4}

Aloitetaan myös instrumentaalin perusmerkitysten käsittely paikkaan ja
aikaan liittyvistä merkityksistä, joista suurin osa toteutuu
prepositioiden avulla. Šeljakin (Шелякин 2006: 67) ryhmittelee
instrumentaalin paikalliset perusmerkitykset merkityksiksi, jotka
kuvaavat jonkin objektin sijaintia suhteessa toiseen objektiin. Tässä
merkityksessä oleellisia ovat prepositiot над, под, перед, за, с ja
между.

\begin{enumerate}
\def\labelenumi{(\arabic{enumi})}
\setcounter{enumi}{22}
\tightlist
\item
  жарить над раскалёнными углями.
\item
  \ldots{}встанете \emph{перед картиной} и будете думать: ``Так\ldots{}
  вот она, это же она\ldots{}''
\item
  Под пиджаком у него была только рваная сетчатая майка
\end{enumerate}

Tiettyjä paikallisia merkityksiä voidaan ilmaista ilman prepositioita,
kuten ilmauksissa tyyppiä \emph{metsän läpi}, \emph{rantaa myöten} ym.

\begin{enumerate}
\def\labelenumi{(\arabic{enumi})}
\setcounter{enumi}{25}
\tightlist
\item
  Почему-то я вспомнил, как мы шли сквером (`aukio') и спорили
\item
  Кругом золотились осенние поля, мы шли дорогой, по которой\ldots{}
\item
  Он шёл лесом, и ему представлялось: вот уж нет его\ldots{}
\end{enumerate}

Myös ajalliset merkitykset ovat varmasti tuttuja, sekä prepositioiden
kanssa että ilman:

\begin{itemize}
\tightlist
\item
  перед сном / рассветом / тем, как\ldots{}
\item
  утром, вечером, днём, летом, зимой, целыми вечерами
\end{itemize}

Huomaa, että käytännössä sanat утром, вечером ym. ovat nykyvenäjässä
adverbejä, morfologisesti siis muotoa /утром/ ennemmin kuin /утр/ом/.

Ajallisesta merkityksestä on kyse myös silloin, kun kuvataan jotain
toimintaa alkavaksi suhteessa johonkin tapahtumaan: \emph{встать с
солнцем, уехать с рассветом} (Шелякин 2006: 68).

\subsubsection{Agentti}\label{agentti}

Kuten tunnettua, eräs instrumentaalin tehtävistä on ilmaista
passiivilauseiden tekijää:

\begin{enumerate}
\def\labelenumi{(\arabic{enumi})}
\setcounter{enumi}{28}
\tightlist
\item
  Мы видели канал, покрытый льдом.
\item
  Таким образом, профилактика экстремизма, проводимая государством,
  нарушает права человека и должна быть признана неконституционной
\item
  Квартиры, которые у нас строятся \emph{компанией} «Стройметресурс»,
  отличаются хорошей планировкой.
\end{enumerate}

Erikseen tämän merkityksen sisältä on syytä mainita myös lauseet, joita
Šeljakin (2006: 69) nimittää termillä \emph{значение стихийного носителя
действия}, luonnonvoiman aiheuttaman toiminnan merkitys. Tarkempi
katsaus aiheeseen on saatavilla
\href{http://www.helsinki.fi/~mustajok/pdf/Lodku_uneslo_vetrom.pdf}{tästä}
A. Mustajoen ja M. Kopotevin (2005) artikkelista.

\begin{enumerate}
\def\labelenumi{(\arabic{enumi})}
\setcounter{enumi}{31}
\tightlist
\item
  Ветром унесло лодку
\item
  Тротуар залило водой
\end{enumerate}

\subsubsection{Predikatiivinen
instrumentaali}\label{predikatiivinen-instrumentaali}

Instrumentaali on usein syntaksin kannalta erityisasemassa sen vuoksi,
että, kuten nominatiivi, se voi toimia predikatiivin roolissa
(predikatiivinen instrumentaali, творительный предикативный), eli
ilmoittaa ominaisuuden tai tilan, joka jostakin subjektista lauseessa
kerrotaan. Katso seuraavia esimerkkejä:

\begin{enumerate}
\def\labelenumi{(\arabic{enumi})}
\setcounter{enumi}{33}
\tightlist
\item
  Лицо женщины было \emph{испуганным}
\item
  Или сам, не дай Бог, будешь \emph{капитаном}, и самому придётся всех
  мучить.
\item
  Те, кто тогда были \emph{мастерами}, уже стали классиками.
\item
  И, конечно, мы очень надеемся на то, что наше взаимодействие будет
  \emph{эффективным}
\item
  Например, я когда-то была \emph{стрекозой} (`sudenkorento'), но,
  увидев Паки, влюбилась без ума и превратилась в колибри
\end{enumerate}

Etenkin menneessä ajassa ero instrumentaalin ja nominatiivin välillä on
usein siinä, että instrumentaalilla ilmaistava ominaisuus tai tila on
väliaikaisempi (Wade 2010: 126). Tämä näkyy esimerkiksi lauseissa 34 ja
38, mutta on ainoastaan tendenssi, ei itsestäänselvyys. Predikatiivin
kaltainen on myös käyttö eräiden verbien kuten стать, называть,
остаться, работать, служить:

\begin{enumerate}
\def\labelenumi{(\arabic{enumi})}
\setcounter{enumi}{38}
\tightlist
\item
  Тесто месить до тех пор, пока оно не станет блестящим
\item
  Стратегические планы Oracle в новом финансовом году останутся прежними
\item
  Эти медленно растущие и особенно ценимые кактусы иногда называют
  \emph{каменными розами}.
\end{enumerate}

\subsubsection{Prepositio + instrumentaali verbin
täydennyksenä}\label{prepositio-instrumentaali-verbin-tuxe4ydennyksenuxe4}

Monet verbit saavat täydennyksekseen prepositiolausekkeen, jonka
substantiivijäsen on instrumentaalissa. Näitä voidaan Šeljakinin (2006:
67) esityksen pohjalta jaotella esimerkiksi seuraavanlaisiin ryhmiin:

\begin{itemize}
\tightlist
\item
  Alistaminen jollekin: властвовать/начальствовать над кем-то
\item
  Mielentila suhteessa johonkin: задуматься/работать/смеяться/шутить над
  чем-либо
\item
  Alisteinen toiminta suhteessa johonkuhun: преклоняться/извиниться
  перед кем-либо
\item
  seuraaminen/välittäminen: следить за политикой, ухаживать за детьми
\item
  päämäärä: сходить за молоком, приехать за дочкой
\end{itemize}

\subsubsection{Instrumentaali verbin
täydennyksenä}\label{instrumentaali-verbin-tuxe4ydennyksenuxe4}

Waden (2010) perusteella voidaan suoran instrumentaalitäydennyksen
saavat verbit jaotella seuraaviin ryhmiin:

\begin{enumerate}
\def\labelenumi{\arabic{enumi}.}
\tightlist
\item
  Hallinta. Esimerkiksi \emph{владеть, править, управлять, ползоваться,
  распологать, распоряжаться, руководит, заведовать.}
\item
  Asenne. Esimerkiksi \emph{восхищаться, гордиться, хвастаться,
  интересоваться, увлекаться, наслаждаться, обходиться.}
\item
  Vastavuoroinen toiminta. Esimerkiksi делиться, обмениваться.
\item
  Muut, kuten (usein merkitykseltään negatiiviset) \emph{болеть,
  страдать, жертвовать, рисковать} tai positiivisemmat/neutraalit
  \emph{торговать, прославляться}.
\end{enumerate}

\begin{enumerate}
\def\labelenumi{(\arabic{enumi})}
\setcounter{enumi}{41}
\tightlist
\item
  Визенталь утверждал, что располагает \emph{длинным списком} с
  фамилиями бывших нацистов.
\item
  Профессор, академик РАЕН Евгений Николаевич Панов заведует
  \emph{лабораторией} сравнительной этологии и биокоммуникации Института
  проблем экологии и эволюции РАН.
\item
  Хвастались друг перед другом своей ловкостью, умением смухлевать,
  переторговать, махнуться не глядя.
\item
  Немецкий зоолог Рюдигер Ферхассельт делится \emph{своими
  наблюдениями}: ``У меня в аквариуме был розово-красный самец\ldots{}''
\item
  Новая столица Японии прославлялась всеми средствами массовой
  информации.
\end{enumerate}

\subsubsection{Kontrastiivisia
huomioita}\label{kontrastiivisia-huomioita}

Vertaillaan vielä suomea ja venäjää ja pohditaan, miten suomessa
ilmaistaan asioita, joita venäjässä ilmaistaan instrumentaalilla.

Instrumentaalia käytetään ensinnäkin usein suomen essiiviin:

\begin{enumerate}
\def\labelenumi{(\arabic{enumi})}
\setcounter{enumi}{46}
\tightlist
\item
  Она умерла молодой / героем
\item
  Они уехали семьей
\item
  Если приду здоровым\ldots{}
\end{enumerate}

Vertaa myös:

\begin{enumerate}
\def\labelenumi{(\arabic{enumi})}
\setcounter{enumi}{49}
\tightlist
\item
  Он сидел плечом к пленному, а спиной к дереву
\end{enumerate}

Toisaalta suomessa on usein suora objekti siinä, missä venäjässä on
instrumentaalisijainen sana. Luonnollisesti monet näistä tapauksista
liittyvät edellä esitettyivin instrumentaalitäydennyksen saaviin
verbeihin (заниматься, управлять ym.) Lisäksi kannattaa kiinnittää
huomiota esimerkiksi ruumiinosien liikuttamiseen:

\begin{enumerate}
\def\labelenumi{(\arabic{enumi})}
\setcounter{enumi}{50}
\tightlist
\item
  Я молча махнул рукой и пошёл прочь.
\item
  Принюхиваясь, очень смешно двигают носом и мигают глазами.
\end{enumerate}

Wade (2010: 122) huomauttaa, eettä näissä tapauksissa paitsi ruumiinosa
myös \emph{ruumiinosalla} (kädellä) tapahtuva liike tulkitaan samoin:

\begin{enumerate}
\def\labelenumi{(\arabic{enumi})}
\setcounter{enumi}{52}
\tightlist
\item
  На одной из них стоят два гражданина и размахивают чемоданами.
\end{enumerate}

\hyperdef{}{ref-mustajoki2005}{\label{ref-mustajoki2005}}
Mustajoki, Arto, and Михаил Вячеславович Копотев. 2005. ``Лодку Унесло
Ветром: Условия И Контексты Употребления Русской` Стихийной' Конструкции
(Лодку Унесло Ветром: The Conditions and Contexts of Use of the Russian`
Elemental' Construction).'' \emph{Russian Linguistics} 29 (1). JSTOR:
1--38.
\url{http://www.helsinki.fi/~mustajok/pdf/Lodku_uneslo_vetrom.pdf}.

\hyperdef{}{ref-nikunl}{\label{ref-nikunl}}
Nikunlassi, Ahti. 2002. \emph{Johdatus Venäjän Kieleen Ja Sen
Tutkimukseen}. Helsinki: Finn Lectura.

\hyperdef{}{ref-wade2010}{\label{ref-wade2010}}
Wade, Terence. 2010. \emph{A Comprehensive Russian Grammar}. Vol. 8.
John Wiley \& Sons.

\hyperdef{}{ref-sheljakin}{\label{ref-sheljakin}}
Шелякин, М.А. 2006. \emph{Справочник По Русской Грамматике}. drofa.

\end{document}
