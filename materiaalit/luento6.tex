\documentclass[]{scrartcl}
\usepackage[utf8]{inputenc}
\usepackage[T1]{fontenc}
\usepackage[T2A]{fontenc}
\usepackage[finnish]{babel}
\usepackage{linguex} 
\usepackage{longtable} 
\usepackage{booktabs}
\usepackage{amsthm}
\usepackage{graphicx}
\newtheorem{maar}{Määritelmä}
\usepackage{fixltx2e} % provides \textsubscript
\usepackage{textcomp} % provides \textsubscript
\usepackage{hyperref}
\usepackage{xcolor}

\providecommand{\tightlist}{%
  \setlength{\itemsep}{0pt}\setlength{\parskip}{0pt}}

\hypersetup{
    colorlinks,
    linkcolor={red!50!black},
    citecolor={blue!50!black},
    urlcolor={blue!80!black}
}
\author{Juho Härme}
\title{Morfologia-kurssin luentomateriaaleja}
\date{\today}
\begin{document}
\maketitle
\tableofcontents
\newpage



\section{Luento 6: Adjektiiveista, pronomineista ja
lukusanoista}\label{luento-6-adjektiiveista-pronomineista-ja-lukusanoista}

\begin{itemize}
\tightlist
\item
  \href{https://mustikka.uta.fi/~juho_harme/morfologia/\#tästä-kurssista}{Takaisin
  sivun ylälaitaan}
\item
  \href{http://mustikka.uta.fi/~juho_harme/morfologia/materiaalit/luento6.pdf}{Lataa
  PDF}
\item
  \href{http://mustikka.uta.fi/~juho_harme/morfologia/presentations/luento6.html}{Tutki
  luentokalvoja}
\item
  \href{http://mustikka.uta.fi/~juho_harme/morfologia/tehtavat/luento6.pdf}{Tutki
  tuntitehtäviä}
\end{itemize}

Tähän asti olemme tutustuneet sijakategoriaan vasta taivutusopin
kannalta ja vain substantiivien osalta. Tällä luennolla mietitään muiden
nominien taivutusta ja lisäksi laajemmin eräitä adjektiiveihin liittyviä
ilmiöitä. Koska kurssin painopiste on substantiivien ja verbien
kieliopillisten kategorioitten tuntemisessa, tämä luento on ehkä
kohtuuttomankin tiivis ja pikainen katsaus melko laajaan joukkoon
ilmiöitä. Pyrin kompensoimaan tätä antamalla vinkkejä lisälukemiselle ja
toisaalta kotona tehtävillä internetharjoituksilla.

\subsection{Adjektiivien ja pronominien
sijataivutus}\label{adjektiivien-ja-pronominien-sijataivutus}

Kuten substantiivien kohdalla, seuraamme myös adjektiivien osalta
Nikunlassin (2002: 143) esittämiä deklinaatiomalleja. Tosin on
todettava, että adjektiivien taivutus on joka tapauksessa simppelimpää,
eikä tulkintaeroille jää samalla tavalla sijaa kuin substantiivien
kohdalla. Adjektiivideklinaatio (адъективное склонение) Nikunlassin
esittämässä muodossa näyttää seuraavalta:

\begin{longtable}[c]{@{}lllll@{}}
\toprule
Sija & Maskuliini & Feminiini & Neutri & Monikko\tabularnewline
\midrule
\endhead
Nom. & /оj/ \textasciitilde{} /иj/ & /аja & /оje/ & /иjэ/\tabularnewline
Gen. & /ово/ & /оj/ & /ово/ & /их/\tabularnewline
Dat. & /ому/ & /оj/ & /ому/ & /им/\tabularnewline
Akk. & /ой/ \textasciitilde{} /ово/ & /уjу & /ой/ \textasciitilde{}
/ово/ & /иjэ/ \textasciitilde{} /их/\tabularnewline
Instr. & /ом/ & /оj/ & /ом/ & /им'и/\tabularnewline
Prep. & /ом/ & /оj/ & /ом/ & /их/\tabularnewline
\bottomrule
\end{longtable}

Taulukkoa katsoessa on hyvä muistaa, että muodot on, jälleen kerran,
esitetty foneettisessa asussa, mistä johtuen kirjoitusasultaan eroavat
\emph{ой} ja \emph{ей} ovat yhtä ja samaa päätettä, samoin
kirjoitusasultaan eroavat \emph{ых} ja \emph{их}. Ainoa deklinaatioon
poikkeuksen lisäävä tekijä on se, että maskuliinien nominatiivilla on
kaksi varianttia: painollisen päätteen /оj/ ja painottoman /иj/.

Tässä esitetty taivutus koskee adjektiivien pitkiä muotoja (\emph{полные
формы прилагательных}), mutta myös monia pronomineja, partisiippien
pitkiä muotoja ja suurinta osaa järjestysluvuista. Eräät muut
pronominit, sanat \emph{один} ja \emph{третьий} sekä
\emph{possessiiviadjektiivit} (притяжательные прилагательные) taipuvat
seuraavan, niin kutsutun \emph{pronominideklinaation} (местоименное
склонение) mukaan:

\begin{longtable}[c]{@{}lllll@{}}
\toprule
Sija & Maskuliini & Feminiini & Neutri & Monikko\tabularnewline
\midrule
\endhead
Nom. & /ø/ & /а/ & /о/ & /и/\tabularnewline
Gen. & /ово/ & /оj/ \textasciitilde{} /эj/ & /ово/ & /их/\tabularnewline
Dat. & /ому/ & /оj/ \textasciitilde{} /эj/ & /ому/ & /им/\tabularnewline
Akk. & /ø/ \textasciitilde{} /ово/ & /у/ & /о/ \textasciitilde{} /ово/ &
/и/ \textasciitilde{} /их/\tabularnewline
Instr. & /им/ & /оj/ \textasciitilde{} /эj/ & /им/ &
/им'и/\tabularnewline
Prep. & /ом/ & /оj/ \textasciitilde{} /эj/ & /ом/ & /их/\tabularnewline
\bottomrule
\end{longtable}

Taulukosta on huomattava, että tällä kertaa /эj/ on esitetty erillisenä
päätteenä. Tämä johtuu siitä, että kyseessä todella on eri vokaali. Tämä
käy ilmi siitä, että monilla tämän ryhmän sananmuodoista pääte on
painollinen (esim. моЕй), eikä painollisissa vokaaleissa tapahdu
reduktiota. МОй- ja моЕй-sanojen viimeiset vokaalit ovat selvästi eri
foneemeja. Se, kumpaa päätettä käytetään, määräytyy seuraavasti:
/оj/-pääte, kun sanan vartalo päättyy liudentumattomaan konsonanttiin,
/эj/-pääte kaikissa muissa tilanteissa.

Tutkitaan vielä konkreettisia esimerkkejä kummastakin
deklinaatiotyypistä.

Adjektiivideklinaatio красивый-sanalla:

\begin{longtable}[c]{@{}lllll@{}}
\toprule
Sija & Maskuliini & Feminiini & Neutri & Monikko\tabularnewline
\midrule
\endhead
Nom. & красивый & красивая & красивое & красивые\tabularnewline
Gen. & красивого & красивой & красивого & красивых\tabularnewline
Dat. & красивому & красивой & красивому & красивым\tabularnewline
Akk. & красивый / красивого & красивую & красивое / красивого & красивые
\textasciitilde{} красивых\tabularnewline
Instr. & красивом & красивой & красивом & красивыми\tabularnewline
Prep. & красивом & красивой & красивом & красивых\tabularnewline
\bottomrule
\end{longtable}

Pronominideklinaatio мой-sanalla:

\begin{longtable}[c]{@{}lllll@{}}
\toprule
Sija & Maskuliini & Feminiini & Neutri & Monikko\tabularnewline
\midrule
\endhead
Nom. & мой & моя & моё & мои\tabularnewline
Gen. & моего & моей & моего & моих\tabularnewline
Dat. & моему & моей & моему & моим\tabularnewline
Akk. & мой \textasciitilde{} моего & мою & моё \textasciitilde{} моего &
мои \textasciitilde{} моих\tabularnewline
Instr. & моим & моей & моим & моими\tabularnewline
Prep. & моём & моей & моём & моих\tabularnewline
\bottomrule
\end{longtable}

\subsection{Lukusanoista}\label{lukusanoista}

Lukusanat ovat venäjässä heterogeeninen ryhmä sanoja. Ordinaalilukusanat
(järjestysluvut, порядковые числительные) taipuvat venäjässä
adjektiivideklinaation (третьий pronominideklinaation) mukaan, joten
niitä ei tässä yhteydessä käsitellä tarkemmin. Sen sijaan
kardinaalilukusanojen (количественные числителные) taivutuksesta on
paikallaan huomauttaa eräitä erityispiirteitä (vrt. Wade 2010:
208--209):

\begin{itemize}
\tightlist
\item
  suuri osa kardinaalilukusanoista ei taivu luvussa. Tämä on itse
  asiassa hahmottavista helpottava asia: jos mietit, pitäisikö lauseessa
  \emph{minä näin kolme miestä} käyttää lukusanasta monikon vai yksikön
  genetiiviä (akkusatiivia), vastaus on, ettei sanalla ole kuin yksi
  genetiivimuoto: \emph{я видел трёх мужчин}.
\item
  lukusana \emph{один} taipuu myös suvussa ja luvussa (одна, одно, одни)
  ja noudattaa pronominideklinaatiota
\item
  lukusanat \emph{нуль} ja \emph{ноль} taipuvat kuten ensimmäisen
  deklinaation substantiivit
\item
  muut liudentuneeseen äänteeseen päättyvät lukusanat taipuvat seuraavan
  taulukon mukaisesti sijoissa mutteivät luvussa tai suvussa.
  Säännönmukaisten kymmenlukujen taivutuksessa on huomioitava myös
  yhdyssanan ensimmäinen osa. Lisäksi kannattaa muistaa väistyvä vokaali
  lukusanassa \emph{восемь}.
\end{itemize}

\begin{longtable}[c]{@{}llll@{}}
\toprule
Sija & Pääte & Esimerkki 1 & Esimerkki 2\tabularnewline
\midrule
\endhead
Nom. & /ø/ & пять & пятьдесЯть\tabularnewline
Gen. & /и/ & пятИ & пятИдесяти\tabularnewline
Dat. & /и/ & пятИ & пятИдесяти\tabularnewline
Akk. & /ø/ & пять & пятьдесЯть\tabularnewline
Instr. & /ju/ & пятьЮ & пятьЮдесяти\tabularnewline
Prep. & /и/ & пятИ & пятИдесяти\tabularnewline
\bottomrule
\end{longtable}

Lukusanojen 3--4 neljä sijataivutus on puolestaan seuraavanlainen:

\begin{longtable}[c]{@{}ll@{}}
\toprule
Sija & Esimerkki\tabularnewline
\midrule
\endhead
Nom. & три\tabularnewline
Gen. & трёх\tabularnewline
Dat. & трём\tabularnewline
Akk. & три \textasciitilde{} трёх\tabularnewline
Instr. & тремЯ\tabularnewline
Prep. & трёх\tabularnewline
\bottomrule
\end{longtable}

Lukusana kaksi on siitä poikkeuksellinen, että se ilmaisee
nominatiivi-/akkusatiivimuodoissaan myös sukua (seuraavassa taulukossa
esiintyvä \emph{две}-muoto):

\begin{longtable}[c]{@{}ll@{}}
\toprule
Sija & Esimerkki\tabularnewline
\midrule
\endhead
Nom. & два \textasciitilde{} две\tabularnewline
Gen. & двух\tabularnewline
Dat. & двум\tabularnewline
Akk. & два \textasciitilde{} две \textasciitilde{} двух\tabularnewline
Instr. & двумЯ\tabularnewline
Prep. & двух\tabularnewline
\bottomrule
\end{longtable}

Tutki vielä itsenäisesti lukusanojen
\href{https://ru.wiktionary.org/wiki/сорок}{сорок},
\href{https://ru.wiktionary.org/wiki/сто}{сто} ,
\href{https://ru.wiktionary.org/wiki/девяносто}{девяносто} ja
\href{https://ru.wiktionary.org/wiki/полтора}{полтора} taivutusta
wiktionarysta (klikkaa sanoja).

Ordinaali- ja kardinaalilukusanojen lisäksi venäjässä on erikseen
kollektiivilukusanojen (собирательные числительные) ryhmä.
Kollektiivimuoto muodostetaan ainakin teoriassa luvuista 2--10, joskin
muodot 7--10 ovat harvinaisia (Wade 2010: 221). Seuraavassa on listattu
kaikki tavallisimmat kollektiivilukusant nominatiivimuodossa (klikkaa
sanoja tutkiaksesi taivutusta):

\begin{itemize}
\tightlist
\item
  \href{https://ru.wiktionary.org/wiki/двое}{двое}
\item
  \href{https://ru.wiktionary.org/wiki/трое}{трое}
\item
  \href{https://ru.wiktionary.org/wiki/четверо}{чЕтверо}
\item
  \href{https://ru.wiktionary.org/wiki/пятеро}{пЯтеро}
\item
  \href{https://ru.wiktionary.org/wiki/шестеро}{шЕстеро}
\item
  \href{https://ru.wiktionary.org/wiki/cемеро}{cЕмеро}
\end{itemize}

Funktionaalisen morfologian kannalta voidaan todeta, että
kollektiivilukusanoja käytetään (Wade 2010: 221--222):

\begin{enumerate}
\def\labelenumi{\arabic{enumi})}
\tightlist
\item
  kollektiivisanojen yhteydessä (двое часов, четверо суток)
\item
  elollisten maskuliinisubstantiivien yhteydessä (ei aina, mutta usein):
  двое друзьей, трое мальчиков
\item
  sanan дети yhteydessä (ei myöskään aina): трое/четверо/пятеро детей
\item
  Kun lukusanasta normaalisti riippuvaa subjektina toimivaa
  substantiivia ei ole ilmaistu: \emph{Нас было двое} / \emph{Трое
  стояли на углу}.
\end{enumerate}

\subsection{Adjektiiveihin liittyviä
ilmiöitä}\label{adjektiiveihin-liittyviuxe4-ilmiuxf6ituxe4}

\subsubsection{Laatu- ja
relaatioadjektiivit}\label{laatu--ja-relaatioadjektiivit}

Adjektiivien luokan sisällä voidaan tehdä osittain merkitykseen
perustuva jako \emph{laatu-} ja \emph{relaatioadjektiiveihin} (ks.
Nikunlassi 2002: 132--133; Шелякин 2006: 72, 79). Jaottelulla on eräitä
käytännössä hyödyllisiä sovellutuksia, joten käsitellään sitä lyhyesti
myös tässä.

Laatuadjektiivit (качественные прилагательны) voivat nimittäin ilmaista
ominaisuuden asteittaisena: niihin voidaan luontevasti liittää määreeksi
sellaisia adverbeja kuin \emph{очень} tai \emph{довольно}.

\begin{enumerate}
\def\labelenumi{(\arabic{enumi})}
\tightlist
\item
  очень мягкий
\item
  довольно интересный
\end{enumerate}

Tästä seuraa, että näistä adjektiiveista voi myös muodostaa
vertailuasteita (ks. alempana) ja lyhyitä muotoja (myös tarkemmin
alempana). Nikunlassi (Nikunlassi 2002: 132) antaa relaatioadjektiivista
(относительные прилагательные) esimerkin \emph{государственный},
sheljakin (2006: 72) puolestaan mainitsee muun muassa sanat
\emph{железный} ja \emph{московский}.

\subsubsection{Lyhyet ja pitkät
muodot}\label{lyhyet-ja-pitkuxe4t-muodot}

Kuten jo peruskoulusta muistat, monilla venäjän adjektiivilekseemeillä
voidaan erottaa kaksi eri muotosarjaa, pitkät muodot (полные формы) ja
lyhyet muodot (краткие формы). Pitkistä muodoista on jo puhuttu, joten
seuraavaksi muutama sana lyhyistä.

Lyhyet muodot ilmaisevat ainoastaan \emph{suku-} ja
\emph{lukukategorioita} (joko molempia tai jompaakumpaa), eivät siis
sijaa. Lyhyitä muotoja voi muodostaa ainoastaan
\emph{laatuadjektiiveista} (ks. yllä), ei relaatioadjektiiveista tai
possessiiviadjektiiveista. Myös laatuadjektiivien suhteen Šeljakin
(2006: 79) esittää esittää seuraavat rajoitukset, niin että lyhyitä
muotoja ei muodosteta, jos:

\begin{enumerate}
\def\labelenumi{\arabic{enumi}.}
\tightlist
\item
  laatuadjektiivi on muodostettu suffiksilla \emph{-ск-} (братский ym.)
\item
  laatuadjektiivi on muodostettu \emph{-л-}suffiksilla ja muodosta
  tulisi homonyyminen verbin menneen ajan muodon kanssa (отсталый,
  бывалый ym.)
\item
  laatuadjektiivin maskuliinin nominatiivipääte on /оj/ ja vartalo
  päättyy /ов/-suffiksiin (деловой ym.)
\item
  Tiettyjen värien nimityksistä (розовый, коричневый ym.)
\end{enumerate}

Lisäksi on muistettava, että tietyillä adjektiiveilla on
\emph{pelkästään} lyhyet muodot: esimerkiksi рад, горазд, должен,
виноват (Шелякин 2006: 80; Zolotova, Onipenko, and Sidorova 2004: 90).

Niissä tapauksissa kun lyhyet muodot ovat mahdollisia, muodostaminen
tapahtuu seuraavasti niin, että sanan vartaloon lisätään sukua tai lukua
ilmaiseva pääte:

\begin{longtable}[c]{@{}llllll@{}}
\toprule
Adjektiivi & Vartalo & pääte / m & pääte / f & pääte / n & pääte /
mon\tabularnewline
\midrule
\endhead
молодой & /молод/ & /ø/ & /а/ & /о/ & /и/\tabularnewline
синий & /син'/ & /ø/ & /а/ & /о/ & /и/\tabularnewline
\bottomrule
\end{longtable}

Taulukkoa tulkittaessa kehotan muistamaan monesti jo toistetut
huomautukset päätemorfeista fonetiikan ja kirjoitusasun kannalta.
Seuraava taulukko kuvaa konkreettiset muodot kirjoitusasussaan:

\begin{longtable}[c]{@{}llll@{}}
\toprule
muoto / m & muoto / f & muoto / n & muoto / mon.\tabularnewline
\midrule
\endhead
мОлод & молодА & мОлодо & мОлоды\tabularnewline
сИнь & синЯ & сИне & сИни\tabularnewline
\bottomrule
\end{longtable}

Lyhyitä muotoja muodostettaessa tulee lisäksi muistaa, että väistyvän
vokaalin ilmiö (ks. edellä) koskee myös näitä sanoja. Kuten aiemmasta
muistetaan, perussääntö väistyvän vokaalin yhteydessä on, että ennen
nollapäätettä vokaali edustuu ilmiäänteenä (о, е, и), muulloin
nollaäänteenä. Näin ollen väistyvä vokaali lyhyiden adjektiivien
tapauksessa esiintyy konkreettisena äänteenä yksikön
maskuliinimuodoissa, esimerkiksi больной--болен, краткий--краток,
длинный--длинен.

Lyhyisiin muotoihin liittyy erittäin tärkeä \emph{funktionaalisen
morfologian} piiriin kuuluva kysymys: milloin lyhyttä muotoa käytetään?

Ensinnäkin voidaan todeta, että käyttöön liittyy yksi merkittävä
rajoite: lyhyttä adjektiivia voidaan käyttää ainoastaan predikatiivin
(именная часть сказуемого) roolissa, ilmaisemaan jotakin ominaisuutta
subjektista (дедушка болен). Tämä on kuitenkin -- filosofisesti
sanottuna -- ainoastaan välttämätön, ei riittävä ehto lyhyen adjektiivin
käytölle. Toisin sanoen ei ole myöskään niin, että predikatiivin
funktiosta aina seuraisi lyhyt muoto.

Zolotova ym. (2004: 90) huomauttavat, että joissakin tapauksissa lyhyen
adjektiivin käyttö on ``в контексте свободного авторского выбора'' --
puhujan tai kirjoittajan oma ratkaisu, joka ei ole kieliopillisesti
pakotettu -- joissain taas joko mahdotonta tai ainoa mahdollisuus.
Zolotova ym. antavat seuraavanlaisia esimerkkejä tilanteista, joissa
lyhyt adjektiivi on ainoa mahdollisuus:

\begin{enumerate}
\def\labelenumi{(\arabic{enumi})}
\setcounter{enumi}{2}
\tightlist
\item
  Ваше упорство печально
\item
  Фантастика хороша тогда, конда зритель не сразу понимает, как сделан
  трюк
\item
  Ты сам замечаешь, что молоко тебе полезно.
\end{enumerate}

Esimerkeissä 4 ja 5 on adjektiivin merkitystä rajoittava spesifi
konteksti, minkä takia pitkä muoto ei ole mahdollinen. Esimerkissä 3
puolestaan on \emph{emotionaalis-kausatiivinen} adjektiivi (vrt.
печальный, грустный, веселый). Yleisluontoisemmassa kontekstissa sekä
lyhyet että pitkät adjektiivit olisivat mahdollisia (vrt. Zolotova,
Onipenko, and Sidorova 2004: 91):

\begin{enumerate}
\def\labelenumi{(\arabic{enumi})}
\setcounter{enumi}{5}
\tightlist
\item
  Молоко полезное / полезно
\end{enumerate}

Toisaalta lyhyiden adjektiivien käyttö on rajattua, kun kyseessä ei ole
henkilösubjekti:

\begin{enumerate}
\def\labelenumi{(\arabic{enumi})}
\setcounter{enumi}{6}
\tightlist
\item
  Он был холодным / холоден. \textbf{MUTTA} Дом был холоден.
\end{enumerate}

\subsubsection{Possessiiviadjektiivit}\label{possessiiviadjektiivit}

Possessiiviadjektiivit (притяжательные прилагательные) ilmaisevat jonkin
esineen, asian tai ominaisuuden kuulumista jollekin (elolliselle)
subjektille. Šeljakinin mukaan (2006: 72) possessiiviadjektiivit
jakautuvat kahteen ryhmään: a) niihin, jotka ilmaisevat kuulumista
jollekin konkreettiselle yksilölle (папин, мамин ym.) b) niihin, jotka
ilmaisevat kuulumista jollekin ryhmälle tai joukolle. (вольчя шкура ym.)

Ensimmäisen kategorian possessiiviadjektiivit muodostetaan liittämällä
jokin suffikseista -ин, -нин ja -ов (Wade 2010: 175). Esimerkiksi:

\begin{enumerate}
\def\labelenumi{(\arabic{enumi})}
\setcounter{enumi}{7}
\tightlist
\item
  /бабушк/ -- /бабушк/ин/
\end{enumerate}

Toisen kategorian possessiiviadjektiivit puolestaan muodostetaan
liittämällä adjektiivin pääte yleisnimeen. Samalla tapahtuu
äänteenmuutoksia:

\begin{enumerate}
\def\labelenumi{(\arabic{enumi})}
\setcounter{enumi}{8}
\tightlist
\item
  /медвед'/ -- /медвеж/ий
\item
  /девиц/а/ -- /девич/ий
\end{enumerate}

Possessiiviadjektiivit taipuvat edellä esitetyn pronominideklinaation
mukaisesti.

\subsubsection{Vertailuaste kieliopillisena
kategoriana}\label{vertailuaste-kieliopillisena-kategoriana}

Vertailuaste on kieliopillinen kategoria, jota ilmaisevat ainoastaan
adjektiivit ja adverbit. Kategorialla on kolme arvoa: positiivi
(положительная степень), komparatiivi (сравнительная степень) ja
superlatiivi (превосходная степень).

Vertailuaste (komparatiivi tai superlatiivi) muodostetaan venäjässä joko
yksinkertaista muodostustapaa (простая сравнительная степень) tai
yhdistettyä muodostustapaa (сложная сравнительная степень) noudattaen.

\textbf{Yhdistetty komparatiivi} muodostetaan seuraavan simppelin kaavan
mukaisesti:

\emph{более + {[}adjektiivi/adverbi positiivimuodossa{]}}

\textbf{Yhdistetyn superlatiivin} muodostamisessa puolestaan käytetään
seuraavaa kaavaa:

\emph{самый + {[}adjektiivi/adverbi positiivimuodossa{]}}

Tämän kurssin kannalta mielekästä on keskittyä lähinnä yksinkertaisen
komparatiivin muodostukseen. Nikunlassi (2002: 145) erottelee
yksinkertaisen komparatiivin päätteiksi seuraavat kolme:

\begin{itemize}
\tightlist
\item
  /э/
\item
  /шэ/
\item
  /эjэ/ (tai vaihtoehtoisesti /эj/)
\end{itemize}

Yksinkertaisen komparatiivin muodostamisesta tekee hankalan se, että
muotoon liittyy suuri joukko äänteenmuutoksia. Selkein tapaus on
viimeksi mainittu /эjэ/-pääte, joka aiheuttaa ainoastaan vartalon
viimeisen konsonantin vaihtumisen liudentuneeseen versioon. Näin ollen
saadaan seuraavat muodot:

\begin{itemize}
\tightlist
\item
  /красив/ее/
\item
  /умн/ее/
\item
  /замечательн/ее/
\item
  /про/хлад/н/ее/
\end{itemize}

Kaksi muuta päätettä aiheuttavat monimutkaisempia muutoksia vartalon
viimeisen morfeemin ja vertailuastepäätteen rajalla. Seuraavaan on
listattu tavallisimpia muutoksia (ks. Wade 2010: 197):

\begin{longtable}[c]{@{}llll@{}}
\toprule
äänne & uusi äänne & esimerkki & huom.\tabularnewline
\midrule
\endhead
г & ж & дорогой-дороже &\tabularnewline
д & ж & молодой-моложе &\tabularnewline
д & ж & молодой-моложе &\tabularnewline
з & ж & близкий-ближе &\tabularnewline
з & ж & близкий-ближе &\tabularnewline
\bottomrule
\end{longtable}

Edellä esitettyjen muodostustapojen lisäksi on huomattava, että eräillä
tutuilla adjektiiveilla yksinkertaisten komparatiivimuotojen ei voi
sanoa sisältävän samaa juurimorfeemia kuin positiivimuotojen. Näissä
tapauksissa ei ole kyse suffiksin lisäämisestä vartaloon, vaan kokonaan
uudesta vartalosta. Tällaisia muotoja sanotaan \emph{suppletiivisiksi}
(супплетивные формы). Näitä ovat ainakin seuraavat:

\begin{itemize}
\tightlist
\item
  хороший - лучше
\item
  плохой - хуже
\item
  маленький - меньше
\end{itemize}

\paragraph{Morfosyntaksia}\label{morfosyntaksia}

Yksinkertaisen komparatiivin muodot ovat syntaksin kannalta
poikkeuksellisia muihin adjektiivimuotoihin verrattuna. Ne eivät
nimittäin ilmaise sukua, lukua tai sijaa, vaan niitä käytetään
käytännössä aina predikatiivisesti, kuten seuraavissa esimerkeissä:

\begin{enumerate}
\def\labelenumi{(\arabic{enumi})}
\setcounter{enumi}{10}
\tightlist
\item
  Они были \textbf{лучше}, они были иностранные и прекрасные.
\item
  Но те были \textbf{дешевле}, \textbf{функциональнее} и в конечном
  счёте победили
\end{enumerate}

\hyperdef{}{ref-nikunl}{\label{ref-nikunl}}
Nikunlassi, Ahti. 2002. \emph{Johdatus Venäjän Kieleen Ja Sen
Tutkimukseen}. Helsinki: Finn Lectura.

\hyperdef{}{ref-wade2010}{\label{ref-wade2010}}
Wade, Terence. 2010. \emph{A Comprehensive Russian Grammar}. Vol. 8.
John Wiley \& Sons.

\hyperdef{}{ref-zolotova}{\label{ref-zolotova}}
Zolotova, G. A., N. K. Onipenko, and M. U. Sidorova. 2004.
\emph{Kommunikativnaja Grammatika Russkogo Jazyka}. Moskva: RAN.

\hyperdef{}{ref-sheljakin}{\label{ref-sheljakin}}
Шелякин, М.А. 2006. \emph{Справочник По Русской Грамматике}. drofa.

\end{document}
