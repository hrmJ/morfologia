\documentclass[]{scrartcl}
\usepackage[utf8]{inputenc}
\usepackage[T1]{fontenc}
\usepackage[T2A]{fontenc}
\usepackage[finnish]{babel}
\usepackage{linguex} 
\usepackage{longtable} 
\usepackage{booktabs}
\usepackage{amsthm}
\usepackage{graphicx}
\newtheorem{maar}{Määritelmä}
\usepackage{fixltx2e} % provides \textsubscript
\usepackage{textcomp} % provides \textsubscript
\usepackage{hyperref}
\usepackage{xcolor}

\providecommand{\tightlist}{%
  \setlength{\itemsep}{0pt}\setlength{\parskip}{0pt}}

\hypersetup{
    colorlinks,
    linkcolor={red!50!black},
    citecolor={blue!50!black},
    urlcolor={blue!80!black}
}
\author{Juho Härme}
\title{Morfologia-kurssin luentomateriaaleja}
\date{\today}
\begin{document}
\maketitle
\tableofcontents
\newpage



\section{luento 7: Sijat funktionaalisen morfologian
kannalta}\label{luento-7-sijat-funktionaalisen-morfologian-kannalta}

Nyt siirrymme taivutusopista kohti funktionaalista morfologiaa ja
ryhdymme tarkastelemaan, miten eri merkityksiä ja käyttöyhteyksiä eri
sijoilla (sijakategorian arvoilla) on. Yksi monimerkityksisimmistä on
genetiivi.

\subsection{Genetiivin
perusmerkitykset}\label{genetiivin-perusmerkitykset}

\subsubsection{Suomen genetiiviin verrattavia
merkityksiä}\label{suomen-genetiiviin-verrattavia-merkityksiuxe4}

Genetiiville voidaan erottaa esimerkiksi seuraavia merkityksiä (vrt.
Wade 2010: 54--55 , Шелякин (2006): 62--63):

\begin{itemize}
\tightlist
\item
  Omistus (kenelle jokin kuuluu)
\item
  Rooli kokonaisuudessa
\item
  Osa kokonaisuudesta
\item
  Ominaisuus
\item
  Toiminnan tekijä
\item
  Toiminnan kohde
\item
  Syy
\end{itemize}

Omistus on luonnollisesti tyypillisin mieleen tuleva genetiivin
ominaisuus. Ajattele seuraavia esimerkkejä:

\begin{enumerate}
\def\labelenumi{(\arabic{enumi})}
\tightlist
\item
  Дом брата
\item
  Велосипед мальчика
\item
  Идея знаменитого философа
\end{enumerate}

Näissä kaikissa ilmaistaan jonkin konkreettisen tai abstraktin asian
kuulumista henkilölle. Omistussuhteen kaltainen on myös merkitys, jossa
genetiivi ilmaisee jonkin henkilön tai esineen roolia jossakin
kokonaisuudessa:

\begin{enumerate}
\def\labelenumi{(\arabic{enumi})}
\setcounter{enumi}{3}
\tightlist
\item
  член комитета
\item
  гражданин России
\end{enumerate}

Tätä lähellä on varsinainen osan merkitys:

\begin{enumerate}
\def\labelenumi{(\arabic{enumi})}
\setcounter{enumi}{5}
\tightlist
\item
  часть страны
\item
  крыша дома
\end{enumerate}

Ominaisuuden merkitys käy ilmi esimerkiksi seuraavasta:

\begin{enumerate}
\def\labelenumi{(\arabic{enumi})}
\setcounter{enumi}{7}
\tightlist
\item
  женщина высокого роста
\item
  человек твёрдых убеждений
\end{enumerate}

Genetiivi voi ilmaista sekä jonkin esineen tai henkilön olemista
toiminnan kohteena että toiminnan suorittajana:

\begin{enumerate}
\def\labelenumi{(\arabic{enumi})}
\setcounter{enumi}{9}
\tightlist
\item
  чтение газеты
\item
  изобретатель самолёта
\end{enumerate}

Jos asiaa pohtii, huomaa, että suomen ja venäjän genetiiveillä on monia
hyvin samanlaisia funktioita -- ajattele edellisten esimerkkien valossa
ilmaisuja \emph{pojan pyörä, komitean jäsen, Venäjän kansalainen, talon
katto, lehden lukeminen, lentokoneen keksijä}. Usein venäjän
genetiivillä on kuitenkin tehtäviä, joita suomessa toteuttaa partitiivi
(joskus kumpikin: maan osa / osa maata).

\subsubsection{Suomen partitiiviin verrattavia
merkityksiä}\label{suomen-partitiiviin-verrattavia-merkityksiuxe4}

Suomen partitiivia muistuttavat etenkin seuraavat käyttötilanteet:

\paragraph{Määrän ilmaisut:}\label{muxe4uxe4ruxe4n-ilmaisut}

\begin{enumerate}
\def\labelenumi{(\arabic{enumi})}
\setcounter{enumi}{11}
\tightlist
\item
  мало/много/сколько/столько интересных людей
\end{enumerate}

Tähän liittyvät myös \emph{tarpeeksi suurta määrää} kuuvaavat ilmaisut:

\begin{enumerate}
\def\labelenumi{(\arabic{enumi})}
\setcounter{enumi}{12}
\tightlist
\item
  Достачно денег
\item
  денег не хватает
\end{enumerate}

\paragraph{Vertailun kohde}\label{vertailun-kohde}

\begin{enumerate}
\def\labelenumi{(\arabic{enumi})}
\setcounter{enumi}{14}
\tightlist
\item
  Маша моложе Миши
\item
  Лев сильнее человека
\end{enumerate}

\subsubsection{Mitan ilmaukset}\label{mitan-ilmaukset}

Genetiivi voi ilmasta, että jotakin on abstraktin mittayksikön (литр
воды) tai jonkin konkreettisen esineen määrän verran (чашка молока).

\paragraph{Varsinaisia partitiivisia
merkityksiä}\label{varsinaisia-partitiivisia-merkityksiuxe4}

Kuten aiemmilta luennoilta muistat, venäjästä voidaan jopa erotella
erikseen partititivisen genetiivin sija. \emph{Partitiivisia
merkityksiä} voidaan kuitenkin ilmaista myös tavallisella genetiivillä.
Yksi erityinen käyttötilanne käy ilmi seuraavista esimerkeistä:

\begin{enumerate}
\def\labelenumi{(\arabic{enumi})}
\setcounter{enumi}{16}
\tightlist
\item
  Он взял и съел помидор.
\item
  Давайте лучше выпьем кваса и будем танцевать.
\end{enumerate}

Esimerkeissä 17 ja 18 genetiivin merkitys on nimenomaan partitiivinen:
se ilmaisee (jo ilman esimerkiksi немного-sanaa) pientä määrää jotakin
ainetta tai joitakin yksilöitä jotakin tavaraa (yleensä syötävää), jota
on saatavilla paljon / yhtenä massana tai joukkona. Wade (Wade 2010:
108--109) luettelee joukon rajoituksia, jotka säätelevät, milloin
vastaava partitiivinen merkitys on mahdollinen.

Ensinnäkin, genetiivimuotoisen sanan on oltava objektina eikä
subjektina: ei ole mahdollista sanoa \emph{на столе есть вина}. Toinen,
häilyvämpi, rajoitus on, että objektin pääverbi on useimmin
perfektiivinen. Tähän liittyy myös myöhemmin kurssilla vastaan tuleva
teonlaadun (способ действия) käsite: eräät teonlaadut ilmaisevat
luonnostaan partitiivisuutta, kuten seuraavat esimerkit:

\begin{enumerate}
\def\labelenumi{(\arabic{enumi})}
\setcounter{enumi}{18}
\tightlist
\item
  Дедушка уже нарубил дров.
\item
  Мы наелись яблок.
\end{enumerate}

Poikkeus tästä rajoituksesta ovat verbit хотеть ja просить:

\begin{enumerate}
\def\labelenumi{(\arabic{enumi})}
\setcounter{enumi}{20}
\tightlist
\item
  Если ребёнок просит конфет, надо ему сначала дать, а потом\ldots{}
\item
  Ты хочешь мороженого?
\end{enumerate}

\subsection{Genetiivi verbin
täydennyksenä}\label{genetiivi-verbin-tuxe4ydennyksenuxe4}

Syntaksin (синтаксис) eli lauseopin kannalta oleellisia käsitteitä ovat
verbin \emph{täydennykset} (\emph{дополнения} tai tarkemmin
\emph{актанты}) ja \emph{määritteet} (\emph{обстоятельства} tai
tarkemmin \emph{сиркостанты}). Täydennyksillä tarkoitetaan elementtejä,
joita ilman verbi ei voi esiintyä. Ajatele seuraavaa esimerkkiä:

\begin{enumerate}
\def\labelenumi{(\arabic{enumi})}
\setcounter{enumi}{22}
\tightlist
\item
  Завтра я починю свой компьютер\\
   `Korjaan huomenna tietokoneeni'
\end{enumerate}

Verbin починить objekti компьютер on sen täydennyksenä lauseessa 23. Sen
sijaan \emph{завтра} on saman verbin määrite, jota ilmankin lause olisi
mahdollinen.

Lauseessa 23 verbin täydennys on (nominatiivin kaltaisessa)
akkusatiivissa. Akkusatiivimuotoinen täydennys on tavallisin, muttei
ainoa vaihtoehto. Pohdi täydennystä seuraavassa esimerkissä:

\begin{enumerate}
\def\labelenumi{(\arabic{enumi})}
\setcounter{enumi}{23}
\tightlist
\item
  Нужно признать, что я вам очень завидую.
\end{enumerate}

Esimerkin 24 признать-verbin täydennys on datiivimuotoinen.

Genetiivi on täydennyksen saamana sijana venäjässä erittäin yleinen.
Ongelmallisen genetiivitäydennyksen käytöstä tekee se, että on monia
tilanteita, joissa sekä genetiivi että akkusatiivi ovat kontekstista
riippuen mahdollisia. Genetiivitäydennyksen saavat verbit voikin
karkeasti jakaa niihin, joilla täydennys on aina genetiivissä sekä
niihin, joilla täydennys on joko genetiivissä tai akkusatiivissa.

\textbf{Yleispiirteenä} häilyvissä tapauksissa (kun mietit, tulisiko
täydennyksen olla genetiivissä vai ei) voidaan todeta seuraavaa:

\begin{enumerate}
\def\labelenumi{\arabic{enumi}.}
\tightlist
\item
  Genetiivillä on taipumus olla yleisempi (geneerisempi), akkusatiivilla
  spesifimpi ja yksilöivämpi\footnote{Tästä voi johtaa muistisäännön:
    gen=gen -- genetiivi on geneerisempi.}.
\item
  Abstraktit sanat ovat oletettavammin genetiivissä kuin konkreettiset
  tai varsinkaan henkilöihin viittaavat.
\end{enumerate}

\subsubsection{Kysymistä, etsimistä ja saavuttamista tarkoittavat
verbit}\label{kysymistuxe4-etsimistuxe4-ja-saavuttamista-tarkoittavat-verbit}

Genetiivitäydennyksen saavat verbit voi ryhmitellä väljästi merkityksen
perusteella. Tähän ensimmäiseen ryhmään kuuluu sellaisia verbejä kuin
добиваться, достигать, желать ja заслуживать. Näiden täydennykset ovat
käytännössä aina genetiivissä:

\begin{enumerate}
\def\labelenumi{(\arabic{enumi})}
\setcounter{enumi}{24}
\item
  Если работа нравится, человек легко с ней справляется,
  \textbf{добивается хороших результатов}, тогда и люди вокруг кажутся
  добрыми и милыми.
\item
  Карелия вообще \textbf{заслуживает отдельного путешествия}. В Валаамо
  стоит осмотреть Нововалаамский православный монастырь.
\end{enumerate}

Akkusatiivin ja genetiivin välillä puolestaan horjuvat tähän ryhmään
kuuluvat verbit \emph{ждать/ожидать/дожидаться}, \emph{искать},
\emph{просить/хотеть} ja требовать. Näiden suhteen pätevät edellä
todetut taipumukset genetiivin yleisluontoisuudesta ja abstraktiudesta.
Tutki ensin seuraavia esimerkkejä genetiivitäydennyksistä:

\begin{enumerate}
\def\labelenumi{(\arabic{enumi})}
\setcounter{enumi}{26}
\tightlist
\item
  Такой способ требует \emph{опыта работы}.
\item
  Это \emph{требует готовности} клиентов.
\item
  Теперь заинтересованные стороны \emph{ждут решения} московской
  экологической экспертизы
\end{enumerate}

Seuraavissa esimerkeissä puolestaan on akkusatiivitäydennys:

\begin{enumerate}
\def\labelenumi{(\arabic{enumi})}
\setcounter{enumi}{29}
\tightlist
\item
  У него, естественно, \emph{требуют пропуск}.
\item
  Но немец оставался при своём и \emph{требовал ведро}
\item
  Мы с Анкой обернулись довольно быстро, но потом долго \emph{ждали
  Таню}
\end{enumerate}

\subsubsection{Pelkäämistä ja välttelyä tarkoittavat
verbit}\label{pelkuxe4uxe4mistuxe4-ja-vuxe4lttelyuxe4-tarkoittavat-verbit}

Tähän ryhmään kuuluvat verbit ovat yleensä vahvasti genetiiviä
puoltavia. Kuitenkin nykyisin (ks. Peteghem and Paykin 2013) on
ilmeisesti yleistynyt myös akkusatiivin käyttö lauseissa tyyppiä
\emph{боюсь бабушку}. Ei siis pidä täysin paikkaansa, että
ся-postfiksiin päättyvät verbit eivät voisi koskaan saada
akkusatiivitäydennystä.

Бояться-verbin lisäksi tähän ryhmään voidaan lukea muun muassa verbit
опасаться, пугаться, стесняться, стыдиться, сторониться ja избегать.

\begin{enumerate}
\def\labelenumi{(\arabic{enumi})}
\setcounter{enumi}{32}
\tightlist
\item
  Алёша, как сказал я уже выше, сначала \emph{стыдился похвал}.
\item
  Катя все время переживала из-за того, что сын нелюдим (ihmisiä
  karttava) и \emph{сторонится общества}.
\end{enumerate}

Etenkin henkilöviittauksissa akkusatiivi on mahdollinen.

\begin{enumerate}
\def\labelenumi{(\arabic{enumi})}
\setcounter{enumi}{34}
\tightlist
\item
  Мои дети не избегают меня, как я \emph{избегал папу}.
\end{enumerate}

\subsubsection{Muita
merkitysryppäitä}\label{muita-merkitysryppuxe4ituxe4}

Myös menettämistä tai poistamista tarkoittavat verbit kuten
\emph{лишиться} saavat tavallisesti genetiivitäydennyksen:

\begin{enumerate}
\def\labelenumi{(\arabic{enumi})}
\setcounter{enumi}{35}
\tightlist
\item
  Спустя два года \emph{работы} лишился еще один чиновник.
\end{enumerate}

\subsection{Genetiivi ja kieltomuodot}\label{genetiivi-ja-kieltomuodot}

Edellä käsiteltiin konkreettisia sanoja, joiden täydennykset ovat usein
tai aina genetiivissä. Kokonaan oma teemansa ovat kieltolauseet ja
genetiivin esiintyminen niissä. Venäjän niin kutsuttu
\emph{kieltogenetiivi} (genetive of negation, родительный падеж в
отрицательных конструкциях) on saanut osakseen mittavaa tutkimustyötä ja
huomiota (ks. esim. Partee and Borschev 2006).

\subsubsection{Subjekti genetiivissä}\label{subjekti-genetiivissuxe4}

Jos lähdetään liikkeelle tutusta ja yksinkertaisesta, voidaan todeta,
että venäjässä on joukko kieltorakenteita, joihin liittyy aina
genetiivi. Nämä voidaan jakaa esimerkiksi seuraaviin kategorioihin (Wade
2010: 111).

\begin{enumerate}
\def\labelenumi{\arabic{enumi}.}
\tightlist
\item
  Olemassaolon tai saatavuuden kieltäminen нет-partikkelilla: еды нет.
  Нет денег. Не было времени. Войны не будет.
\item
  Olemassaolon tai saatavuuden kieltäminen verbeillä: денег не
  осталось/не имеется. Ни одного живого \emph{существа не попадалось}.
  Такого не существует. Tässäkin kuitenkin pätee edellä esitetty huomio
  spesifistä ja geneerisestä merkityksestä. Jos kyse on tietystä
  konkreettisesta viittauskohteesta, on subjektin sija nominatiivi:
  \emph{документы, о которых мы говорили, не сохранились} mutta
  \emph{документов не сорханилось}\footnote{huomaa, että
    genetiivitapauksissa verbi taipuu oletussuvussa ja -sijassa (yksikön
    neutri).}
\item
  Aistihavainnon puuttumista ilmaisevat predikatiiviadverbit
  слышно/видно: \emph{в таком шуме даже собственных мыслей не слышно}
\item
  Ajanilmaukset: \emph{мне уже двадцать минут не отвечают}\footnote{(Huomaa
    kuitenkin, että tarkkaan ottaen nämä tapaukset voidaan laskea
    johtuvan ennemmin partitiivisuudesta kuin kieltomuodosta (King
    1995))}.
\end{enumerate}

Kannattaa myös panna merkille, että tietyissä passiivirakenteissa
kieliopillisen subjektin sijaksi tulee genetiivi:

\begin{enumerate}
\def\labelenumi{(\arabic{enumi})}
\setcounter{enumi}{36}
\tightlist
\item
  на данном этапе \emph{никаких денежных реформ} не предусмотрено.
\item
  изменения химического состава не установлено
\item
  Но пока среди продавцов дисков на улицах Москвы \emph{паники} не
  наблюдается.
\end{enumerate}

\subsubsection{Objekti genetiivissä}\label{objekti-genetiivissuxe4}

Eniten epäselvyyttä genetiivi--akkusatiivi-opposition suhteen
aiheuttavat tapaukset, joissa kieltolauseen verbillä on
objektitäydennys. Objektin sisältävissä kieltolauseissa genetiiviä voi
pitää jopa todennäköisempänä kuin akkusatiivia. Edellä mainittu
geneerisyys vs spesifisyys -tendenssi antaa joitakin suuntaviivoja,
samoin seuraavat huomiot (2010: 113). Mielenkiintoisena faktana
todettakoon, että eräissä muissa slaavilaisissa kielissä (puola,
slovakki) genetiivi on kieltomuodoissa välttämätön (King 1995: 31).

Genetiivi tulee oletuksena, jos lauseessa on kaksoiskielto tai
ни-partikkeli:

\begin{enumerate}
\def\labelenumi{(\arabic{enumi})}
\setcounter{enumi}{39}
\tightlist
\item
  Люди, которые \emph{никогда не учили иностранных языков} просто
  являются носителем нереализованной способности
\item
  Он ни на секунду не потерял \emph{достоинства}
\end{enumerate}

Samoin genetiiviä puoltaa gerundin käyttö:

\begin{enumerate}
\def\labelenumi{(\arabic{enumi})}
\setcounter{enumi}{41}
\tightlist
\item
  Не имея специального образования, защититься без адвоката совсем
  непросто
\end{enumerate}

Ymmärtämiseen ja aistihavaintoon viittaavat verbit saavat
todennäköisesti genetiivitäydennyksen kieltolauseissa:

\begin{enumerate}
\def\labelenumi{(\arabic{enumi})}
\setcounter{enumi}{42}
\tightlist
\item
  Честно говоря, я сам \emph{не знаю ответа} на поставленный вопрос,
\item
  Почему ты всё-таки дал деньги на оркестр, ты же не понимал
  \emph{классической музыки}?
\end{enumerate}

Myös abstraktit substantiivit yhdistyvät, kuten edellä todettiin,
genetiiviin. Esimerkkeinä Wade (2010: 113) mainitsee sanat
\emph{раздражание}, \emph{роль}, \emph{понятие}, \emph{внимание},
\emph{впетчатление}, \emph{время}.

Seuraavassa on joitakin esimerkkejä spesifiin objektiin tai
konkreettiseen henkilöön viittaavista, akkusatiivin valintaa puoltavista
tapauksista:

\begin{enumerate}
\def\labelenumi{(\arabic{enumi})}
\setcounter{enumi}{44}
\tightlist
\item
  Если кто-то по тем или иным причинам не получил денежную выплату за
  январь 2005 года\ldots{}
\item
  Сергей больше не увидел Таню живой.
\item
  Вы не знаете Асю. Она на такое не способна.
\end{enumerate}

Lisäksi on hyvä huomata, että akkusatiiviin tulevat esimerkin 48
kaltaiset lauseet, joissa kielto on tavalla tai toisela lievempi:

\begin{enumerate}
\def\labelenumi{(\arabic{enumi})}
\setcounter{enumi}{47}
\tightlist
\item
  Мы с ним вместе гастролировали в Японии, где чуть не убили японку.
\end{enumerate}

Samoin akkusatiivi on todennäköinen, jos kiellon kohteena ei ole
varsinaisesti verbin ilmaisema toiminta vaan esimerkiksi sen aste kuten
seuraavassa Waden (2010: 113) esimerkissä:

\begin{enumerate}
\def\labelenumi{(\arabic{enumi})}
\setcounter{enumi}{48}
\tightlist
\item
  Он не вполне усвоил урок.
\end{enumerate}

\subsection{Genetiivi ja prepositiot}\label{genetiivi-ja-prepositiot}

Genetiivi ilmaisee prepositioiden yhteydessä usein samoja asioita kuin
itsenäisestikin käytettynä. Mahdollisia merkityksiä ovat esimerkiksi

\begin{itemize}
\tightlist
\item
  syy tai päämäärä: для, от, ради, из-за, вследствие
\item
  paikka: до, из, из-за, с, от, вдоль, поперёк, вглубь, возле, около,
  среди
\item
  aika: до, после, из, с, от, около
\item
  muita: без, вместо,
\end{itemize}

\hyperdef{}{ref-king1995}{\label{ref-king1995}}
King, Tracy Holloway. 1995. \emph{Configuring Topic and Focus in
Russian}. Stanford: CSLI Publications.

\hyperdef{}{ref-partee2006genitive}{\label{ref-partee2006genitive}}
Partee, Barbara, and Vladimir Borschev. 2006. ``The Genitive of Negation
in Russian: Multiple Perspectives on a Multi-Faceted Problem.'' In
\emph{First Annual Meeting of the Slavic Linguistics Society,
Bloomington, iN, September}.
\url{http://people.umass.edu/partee/docs/SLS06_handout.pdf}.

\hyperdef{}{ref-genlahde}{\label{ref-genlahde}}
Peteghem, Marleen Van, and Katia Paykin. 2013. ``The Russian Genitive
Within the NP and the VP.'' In \emph{The Genitive}, edited by Anne
Carlier and Jean-Christophe Verstraete. Vol. 5. John Benjamins
Publishing.

\hyperdef{}{ref-wade2010}{\label{ref-wade2010}}
Wade, Terence. 2010. \emph{A Comprehensive Russian Grammar}. Vol. 8.
John Wiley \& Sons.

\hyperdef{}{ref-sheljakin}{\label{ref-sheljakin}}
Шелякин, М.А. 2006. \emph{Справочник По Русской Грамматике}. drofa.

\end{document}
