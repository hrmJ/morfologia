\documentclass[]{scrartcl}
\usepackage[utf8]{inputenc}
\usepackage[T1]{fontenc}
\usepackage[T2A]{fontenc}
\usepackage[finnish]{babel}
\usepackage{linguex} 
\usepackage{longtable} 
\usepackage{booktabs}
\usepackage{amsthm}
\usepackage{graphicx}
\newtheorem{maar}{Määritelmä}
\usepackage{fixltx2e} % provides \textsubscript
\usepackage{textcomp} % provides \textsubscript
\usepackage{hyperref}
\usepackage{xcolor}

\providecommand{\tightlist}{%
  \setlength{\itemsep}{0pt}\setlength{\parskip}{0pt}}

\hypersetup{
    colorlinks,
    linkcolor={red!50!black},
    citecolor={blue!50!black},
    urlcolor={blue!80!black}
}
\author{Juho Härme}
\title{Morfologia-kurssin luentomateriaaleja}
\date{\today}
\begin{document}
\maketitle
\tableofcontents
\newpage



\section{Luento 19: Tapaluokka kieliopillisena kategoriana
2}\label{luento-19-tapaluokka-kieliopillisena-kategoriana-2}

\begin{itemize}
\tightlist
\item
  \href{https://mustikka.uta.fi/~juho_harme/morfologia/\#tästä-kurssista}{Takaisin
  sivun ylälaitaan}
\item
  \href{http://mustikka.uta.fi/~juho_harme/morfologia/materiaalit/luento19.pdf}{Lataa
  PDF}
\item
  \href{http://mustikka.uta.fi/~juho_harme/morfologia/presentations/luento19.html}{Tutki
  luentokalvoja}
\item
  \href{http://mustikka.uta.fi/~juho_harme/morfologia/tehtavat/luento19.pdf}{Tutki
  tuntitehtäviä}
\end{itemize}

\subsection{Imperatiivi funktionaalisen morfologian
kannalta}\label{imperatiivi-funktionaalisen-morfologian-kannalta}

Edellisellä luennolla keskityttiin imperatiivin muodostamiseen.
Katsotaan tämän luennon aluksi vielä muutamia huomioita imperatiivin
käytöstä eli imperatiivista funktionaalisen morfologian kannalta.

Edellä todettiin jo, että käskyjä voidaan ilmaista monilla muillakin
muodoilla kuin varsinaisilla sananmuodostuksellisessa mielessä
imperatiiveilla (infinitiiveillä, indikatiivimuodoilla). Asia voidaan
kääntää myös toisin päin: \emph{varsinaisia sananmuodostuksellisessa
mielessä imperatiiveja voidaan käyttää myös muihin tarkoituksiin kuin
käskyn tai kehotuksen ilmaisemiseen}.

Yksi tällainen tapaus ovat konditionaalilauseen ehdot. Konditionaaliin
paneudutaan tällä tunnilla tarkemmin edempänä, mutta katsotaan jo nyt
seuraavia esimerkkiä:

\begin{enumerate}
\def\labelenumi{(\arabic{enumi})}
\tightlist
\item
  \emph{Будь} я марксистом, я бы посчитал, что эти идеи имеют классовую
  природу.
\item
  А \emph{останься} я тогда, всё свелось бы только к воспитанию
  младенцев да к ожиданию смерти.
\end{enumerate}

Suomennokset voisivat kuulua esimerkiksi:

\begin{enumerate}
\def\labelenumi{(\arabic{enumi})}
\setcounter{enumi}{2}
\tightlist
\item
  Jos olisin marxisti, olisin sitä mieltä, että nämä ajatukset ovat
  luonteeltaan luokkakysymyksiä
\item
  Jos olisin sillon jäänyt, kaikesta olisi tullut vain lasten
  kasvattamista ja kuoleman odottelua
\end{enumerate}

Kyseisissä esimerkeissä imperatiivi siis ilmaisee käskyn sijasta
\emph{ehtoa}, aivan kuten tavallinen konditionaali yleensä. Toinen
vastaava imperatiivin ei--käskevä merkitys on Nikunlassin (2002: 197)
mukaan niin kutsuttu konsessiivirakenne, kuten seuraavassa esimerkissä:

\begin{enumerate}
\def\labelenumi{(\arabic{enumi})}
\setcounter{enumi}{4}
\tightlist
\item
  Землю приходилось как-то поделить, а как ни \emph{подели}, все равно
  будет несправедливо
\item
  Этого никто не может понять, как ни \emph{объясняй}.
\end{enumerate}

Esimerkkien 5 ja 6 imperatiivit voisi suomentaa rakenteilla `Miten ikinä
jaatkin/selitätkin, joka tapauksessa/kuitenkaan\ldots{}'.

\subsubsection{Pari sanaa aspektista}\label{pari-sanaa-aspektista}

Jatketaan vielä aspektikategorian käsittelyä imperatiivien kannalta.
Ensinnäkin, kuten edellä esimerkiksi futuurin osalta todettiin,
aspektien välinen perusero säilyy myös imperatiivimuodoissa. Vertaa
esimerkiksi seuraavia esimerkkejä, joista 7 edustaa konkreettis-faktista
merkitystä ja 8 konkreettis-prosessuaalista:

\begin{enumerate}
\def\labelenumi{(\arabic{enumi})}
\setcounter{enumi}{6}
\tightlist
\item
  Поработай, а потом пойдем гулять.
\item
  Стойте здесь, никуда не уходите.
\end{enumerate}

Esimerkissä 7 kyseessä on rajattuja ja peräkkäisiä toimintoja,
esimerkissä 8 puolestaan avoimia ja samanaikaisia. Myös oppositio
toiston (rajoittamattoman toiston merkitys) ja yhden kerran
(konkreettis-faktinen merkitys) välillä säilyy, kuten seuraavassa:

\begin{enumerate}
\def\labelenumi{(\arabic{enumi})}
\setcounter{enumi}{8}
\tightlist
\item
  Перед началом работы проветрите комнату
\item
  Перед началом работы проветривайте комнату
\end{enumerate}

Esimerkki 9 ilmaisee yksittäistä konkreettista toimintaa, esimerkki 10
toistuvaa, joka kerta suoritettavaa.

Sen lisäksi, että peruserot aspektien välillä säilyvät, voidaan
kuitenkin erottaa myös eräitä erityispiirteitä aspektin käytölle
nimenomaan imperatiivissa. Perusero on, että \emph{imperfektiivinen
aspekti käskee vähemmän radikaaleja asioita}, ikään kuin
imperfektiivisellä aspektilla pyytäminen on vähemmän pyydetty. Huomaa,
ettei tämä tarkoita, että kohteliaampi pyyntö olisi aina
imperfektiivinen, vaan pikemminkin, että \emph{odottamattomampi} pyyntö
on yleensä perfektiivinen.

Vertaa seuraavia tilanteita urheilukaupassa. Ajatellaan ensin, että
lähestyt myyjää, joka ei vielä tiedä, minkälaista asiaa sinulla on. Sinä
olet tullut ostamaan tennismailaa. Aspektin kannalta olisi luontevaa
kysyä:

\begin{enumerate}
\def\labelenumi{(\arabic{enumi})}
\setcounter{enumi}{10}
\tightlist
\item
  -- Пожалуйста, покажите мне теннисные ракетки.
\end{enumerate}

Pyyntö ei ole millään lailla epäkohtelias, mutta perfektiivinen aspekti
on läsnä, koska se ei ``pelkkää retoriikkaa'', myyjää pyydetään oikeasti
tekemään jotakin. Kun myyjä vie sinut tennismailahyllylle, hän voi
puolestaan todeta:

\begin{enumerate}
\def\labelenumi{(\arabic{enumi})}
\setcounter{enumi}{11}
\tightlist
\item Вот такие у нас есть.. Выбирайте!
\end{enumerate}

Kyseessä on koko lailla retorinen, tyhjempi, käsky: sinua kehotetaan
tekemään toiminto, josta jo tiedät, että sitä sinulta odotetaan. Sama
imperfektiivisen aspektin neutraalius tulee esille, jos puhujalla ei ole
johonkin pyyntöön vahvaa kantaa tai häntä ei kiinnosta:

\begin{enumerate}
\def\labelenumi{(\arabic{enumi})}
\setcounter{enumi}{12}
\tightlist
\item
  -- Я хочу сходить сегодня к Саше. - Ну что же, {иди.}
\end{enumerate}

Imperfektiivisen aspektin ja odotetun toiminnan suhde imperatiivissa
näkyy myös kehotuksen toistamisrakenteissa. Tyypillisen tilanteen voi
kuvitella ruokapöydässä, jossa isäntä tai emäntä näkee sinun istuvan
paikallasi, etkä ole vielä ottanut mitään lautasellesi:

\begin{enumerate}
\def\labelenumi{(\arabic{enumi})}
\setcounter{enumi}{13}
\tightlist
\item
  -- Возьми, пожалуйста, булочку!
\end{enumerate}

Jos pyynnön kuultuasi alat epäröidä, etkä oikein tiedä, ottaisitko vai
etkö, emäntä/isäntä toistaa:

\begin{enumerate}
\def\labelenumi{(\arabic{enumi})}
\setcounter{enumi}{14}
\tightlist
\item
  Бери, бери, не стесняйся!
\end{enumerate}

Ensimmäisessä pyynnössä (14) käytössä on siis perfektiivisen aspektin
verbi, toisessa imperfektiivisen.

Toistettakoon vielä, ettei edellä mainittu odotettavuuden ero liity
kohteliaisuuteen, vaan päinvastoin imperfektiivinen aspekti voi usein
olla epäkohteliaampi vaihtoehto. Voisi esimerkiksi kuvitella, että
vanhus pyytää sinulta apua kadulla perfektiivisellä aspektilla:

\begin{enumerate}
\def\labelenumi{(\arabic{enumi})}
\setcounter{enumi}{15}
\tightlist
\item
  Молодой человек, помогите пожалуйста, с сумкой.
\end{enumerate}

Imperfektiivinen aspekti voisi olla kärsimätön, heti toimintaan
patistava:

\begin{enumerate}
\def\labelenumi{(\arabic{enumi})}
\setcounter{enumi}{16}
\tightlist
\item
  Ну чего же вы тут ждете, ничего не делаете, помогайте мне!
\end{enumerate}

Kieltomuodoista ja imperatiivista Šeljakin (2006: 165) esittää, että
imperfektiivisen aspektin voi katsoa usein olevan jyrkempi,
absoluuttinen: Не говори ему об этом! Не ходите туда! Perfektiivinen
aspekti kantaa ennemminkin varoituksen luonnetta: Смотри, не скажи ему
об этом! Смотри не упади!

\subsection{Konditionaali}\label{konditionaali}

Konditionaalia käytetään, kuten mainittua, ilmaisemaan jokin asiaintila
ehdollisena tai mahdollisena. Venäjän konditionaalin voi ajatella usein
tarkoittavan, että tällä hetkellä on voimassa vastakkainen asiaintila
kuin hetkellä, johon konditionaalilause viittaa (Nikunlassi 2002: 195).
Ajattele esimerkkiä 18, jossa oletetaan, että kuulija ei näe, millä
ilolla emäntä kertoi historiaa:

\begin{enumerate}
\def\labelenumi{(\arabic{enumi})}
\setcounter{enumi}{17}
\tightlist
\item
  Если бы вы видели, с каким восторгом хозяйка рассказывала его
  150-летнюю историю!
\end{enumerate}

Lause 18 esittää samalla konditionaalin muodostusperiaatteen venäjässä:
konditionaali saadaan aikaan verbin menneen ajan muodolla yhdistettynä
taipumattomaan бы-partikkeliin. Toisin kuin esimerkiksi eglannissa tai
ruotsissa, sekä ehtolauseen että päälauseen verbit ovat
konditionaalimuodossa:

\begin{enumerate}
\def\labelenumi{(\arabic{enumi})}
\setcounter{enumi}{18}
\tightlist
\item
  ``\emph{Если бы} я \emph{был} Дедом Морозом, то \emph{заставил бы}
  людей верить в чудеса''
\end{enumerate}

Lause 19 näyttää myös, että ehtolauseissa бы-konjunktion paikka on
selkeä: se seuraa heti konjunktion jälkeen. Sen sijaan päälauseissa
paikka voi jonkin verran vaihdella. Wade (2010: 334) toteaa, että paitsi
verbin jäljessä (kuten esimerkissä 19), partikkeli voi sijaita myös
verbin edessä:

\begin{enumerate}
\def\labelenumi{(\arabic{enumi})}
\setcounter{enumi}{19}
\tightlist
\item
  ``\emph{Если бы} я \emph{был} Дедом Морозом, я \emph{бы заставил}
  людей верить в чудеса''
\end{enumerate}

Šeljakin (Шелякин 2006: 167) tarkentaa, että tavallisin paikka on verbin
jäljessä ja että sijaitessaan ennen verbiä бы yleensä on koko lauseen
toinen sana.

\subsubsection{Funktionaalisen morfologian
kannalta}\label{funktionaalisen-morfologian-kannalta}

Katsotaan seuraavassa tarkemmin konkreettisia konditionaalin
käyttötilanteita.

Yksi tärkeä konditionaalin tehtävä on \emph{toiveen} ilmaiseminen (vrt.
suomen ``kunpa''/``jospa''-lauseet).

\begin{enumerate}
\def\labelenumi{(\arabic{enumi})}
\setcounter{enumi}{20}
\tightlist
\item
  Уезжали бы уже скорей, не \emph{рвали бы} душу!
\item
  \emph{Знал бы} ты, каким трудом это всё достаётся, бездельник ты такой
\end{enumerate}

Toiveen ilmaisemiseen liittyy eräs kielenoppijan kannalta kriittinen
huomio. Konditionaali nimittäin on pakollinen sivulauseessa aina, kun
toive kohdistuu johonkuhun muuhun kuin puhujaan itseensä (Wade 2010:
335; Nikunlassi 2002: 196). Tällöin бы-partikkeli yhdistyy
что-konjunktioon:

\begin{enumerate}
\def\labelenumi{(\arabic{enumi})}
\setcounter{enumi}{22}
\tightlist
\item
  Ты хочешь, чтобы я этот чёртов компьютер в окно выбросила?
\item
  Хочешь, чтобы тебя уважали и любили?
\item
  Родители часто требуют, чтобы ребёнок не вставал из-за стола до тех
  пор, пока не приготовит все уроки.
\item
  Он желал, чтобы суд над преступниками происходил с возможной степенью
  законности и гласности.
\end{enumerate}

Kuten esimerkeistä 23 -- 26 havaitaan, tavallisia tällaisissa
tilanteissa käytettäviä lievempää tai jyrkempää toivomusta ilmaisevia
verbejä ovat \emph{хотеть}, \emph{желать}, \emph{требовать}. Joukko on
kuitenkin laajempi kuin vain perusajatus ``toivomisesta'' antaisi
ymmärtää. Myös sanojen \emph{необходимо, нужно, надо} kanssa käytetään
konditionaalia:

\begin{enumerate}
\def\labelenumi{(\arabic{enumi})}
\setcounter{enumi}{26}
\tightlist
\item
  Необходимо, чтобы наши партийные организации включились в это дело
\item
  Нужно, чтобы голос бизнеса звучал громче, чем он звучал раньше.
\item
  Надо, чтобы ребёнок становился самостоятельным.
\end{enumerate}

Wade (2010: 336) listaa myös seuraavat sanat, joiden yhteydessä
käytetään vastavalla tavalla konditionaalia:

\begin{itemize}
\tightlist
\item
  важно
\item
  желательно
\item
  лучше
\item
  против
\end{itemize}

Esimerkiksi:

\begin{enumerate}
\def\labelenumi{(\arabic{enumi})}
\setcounter{enumi}{29}
\tightlist
\item
  Но они против того, чтобы иностранным гражданам продавали землю.
\item
  Но лучше, чтобы ребёнок сам увидел, что мама/папа не бросают свои
  чашки.
\end{enumerate}

Huomaa myös että сказать-verbi + konditionaali -yhdistelmä voi ilmaista
jonkin toiminnan toivottuutta tai suotavuutta / ylipäätään käskynomaista
pyyntöä:

\begin{enumerate}
\def\labelenumi{(\arabic{enumi})}
\setcounter{enumi}{31}
\tightlist
\item
  Я скажу, чтобы Джамал вам позвонил.
\item
  Но потом мама сказала, чтобы я о нём больше не спрашивала.
\item
  Помнишь, ты мне сказала, чтобы я тебя здесь подождал?
\end{enumerate}

Lisäksi konditionaali voidaan yhdistää esimerkiksi verbiin предложить
korostamaan, että kyseessä on jonkun tahdon ilmaus (Nikunlassi 2002:
196):

\begin{enumerate}
\def\labelenumi{(\arabic{enumi})}
\setcounter{enumi}{34}
\tightlist
\item
  Верно ли, что на недавней коллегии Минтранса вы предложили, чтобы
  российские чиновники летали в командировки за рубеж только рейсами
  российских авиакомпаний?
\end{enumerate}

\subsubsection{Kontrastiivisia
huomioita}\label{kontrastiivisia-huomioita}

Vaikka sekä suomessa että venäjässä, kuten useimmissa muissa kielissä,
on tapaluokka nimeltä konditionaali, se ei tarkoita, että kyseessä
olisivat täysin samanlaiset tapaluokat. Seuraavassa joitakin erityisesti
suomalaisen kielenoppijan kannalta ongelmallisia eroja.

Ensimmäinen silmiinpistävä ero on, että venäjän konditionaalissa ei ole
\emph{aikamuotoja}. Suomessa konditionaalia voidaan käyttää preesensissä
ja perfektissä (tulisi / olisi tullut), mutta venäjässä konditionaalin
aikamuoto on aina sama. Niinpä seuraava Waden (2010: 333) esimerkki
voisi saada kontekstista riippuen kaksi suomennosta: 1. Menisin, jos
minut kutsuttaisiin; 2. Olisin mennyt, jos minut olisi kutsuttu.

\begin{enumerate}
\def\labelenumi{(\arabic{enumi})}
\setcounter{enumi}{35}
\tightlist
\item
  Я пошёл бы, если бы меня пригласили.
\end{enumerate}

Toiseksi voidaan karkeasti todeta, että suomessa konditionaalia
käytetään ahkerammin kuin venäjässä. Tämä johtuu muun muassa siitä, että
suomessa niin kutsutut modaaliset predikaatit (Nikunlassi 2002: 196)
kuten tahtoa, luulla, arvella ovat usein konditionaalissa, vaikka
periaatteessa ne jo ilmankin ilmaisisivat puhujan epävarmuutta
ilmaisemastaan asiasta. Venäjässä näissä tapauksissa käytetään kuitenkin
tavallisesti indikatiivia. Vertaa seuraavia ParFin-rinnakkaiskorpuksen
esimerkkejä. Kyseessä on siis suomenkielisiä alkuperäistekstejä ja
niiden venäjännöksiä:

\begin{longtable}[c]{p{7.2cm}p{7.2cm}}
\toprule
suomenkielinen lähdeteksti & venäjänkielinen kohdeteksti\tabularnewline
\midrule
\endhead
Miten sinä pärjäät, Kaisa? \emph{Luulisin}, että sinulla on aivan
tarpeeksi paineita ilman murhajuttuihin sekaantumistakin? & Как ты,
Кайса, со всем справляешься? \emph{Думаю}, что у тебя достаточно
нагрузки и без вмешательства в дело об убийстве?\\ 
\tabularnewline
Luulisitteko kykenevänne pelaamaan sellaisella panoksella, jos tämä
tapahtuma olisi kehittynyt sellaisiin mittoihin. & Считаете ли вы себя
способным поставить на нее, если дело дойдет до таких
масштабов? \\ \tabularnewline
Mitä se sieltä haki? --- Mitä luulisit? & Что он там забыл?- Как ты
думаешь?\\ \tabularnewline
Esityksen päätyttyä mies kysyi haluaisinko juoda kahvit & Закончив
выступление, он спросил, не хочу ли я выпить кофе,\\ \tabularnewline
Minä haluaisin kernaasti ostaa sen kornetin & Я охотно куплю
трубу\\ \tabularnewline
Mies ymmärtäisi hyvin sen, että vaimo haluaisi hoitaa heidän kotiaan ja
puutarhaansa huolellisemmin. & Он хорошо понимал, что жена хочет более
тщательно ухаживать за их домом и садом.\\ \tabularnewline
\bottomrule
\end{longtable}

Jos konditionaalilla kuitenkin ilmaistaan selvästi jonkin asiaintilan
ehdollisuutta eikä vain puhujan epävarmuutta, venäjässäkin konditionaali
on tarpeen, kuten seuraavassa käännösesimerkissä:

\begin{itemize}
\tightlist
\item
  Moni tyttö haluaisi pysyä sellaisen miehen luona.\\
\item
  Многие девушки хотели бы иметь такого мужчину.
\end{itemize}

\hyperdef{}{ref-nikunl}{\label{ref-nikunl}}
Nikunlassi, Ahti. 2002. \emph{Johdatus Venäjän Kieleen Ja Sen
Tutkimukseen}. Helsinki: Finn Lectura.

\hyperdef{}{ref-wade2010}{\label{ref-wade2010}}
Wade, Terence. 2010. \emph{A Comprehensive Russian Grammar}. Vol. 8.
John Wiley \& Sons.

\hyperdef{}{ref-sheljakin}{\label{ref-sheljakin}}
Шелякин, М.А. 2006. \emph{Справочник По Русской Грамматике}. drofa.

\end{document}
