\documentclass[]{scrartcl}
\usepackage[utf8]{inputenc}
\usepackage[T1]{fontenc}
\usepackage[T2A]{fontenc}
\usepackage[finnish]{babel}
\usepackage{linguex} 
\usepackage{longtable} 
\usepackage{booktabs}
\usepackage{amsthm}
\usepackage{graphicx}
\newtheorem{maar}{Määritelmä}
\usepackage{fixltx2e} % provides \textsubscript
\usepackage{textcomp} % provides \textsubscript
\usepackage{hyperref}
\usepackage{xcolor}

\providecommand{\tightlist}{%
  \setlength{\itemsep}{0pt}\setlength{\parskip}{0pt}}

\hypersetup{
    colorlinks,
    linkcolor={red!50!black},
    citecolor={blue!50!black},
    urlcolor={blue!80!black}
}
\author{Juho Härme}
\title{Morfologia-kurssin luentomateriaaleja}
\date{\today}
\begin{document}
\maketitle
\tableofcontents
\newpage



\section{Luento 10: Sijat funktionaalisen morfologian kannalta
4}\label{luento-10-sijat-funktionaalisen-morfologian-kannalta-4}

\begin{itemize}
\tightlist
\item
  \href{https://mustikka.uta.fi/~juho_harme/morfologia/\#tästä-kurssista}{Takaisin
  sivun ylälaitaan}
\item
  \href{http://mustikka.uta.fi/~juho_harme/morfologia/materiaalit/luento10.pdf}{Lataa
  PDF}
\item
  \href{http://mustikka.uta.fi/~juho_harme/morfologia/presentations/luento10.html}{Tutki
  luentokalvoja}
\item
  \href{http://mustikka.uta.fi/~juho_harme/morfologia/tehtavat/luento10.pdf}{Tutki
  tuntitehtäviä}
\end{itemize}

\subsection{Instrumentaalin
merkityksiä}\label{instrumentaalin-merkityksiuxe4}

Aloitetaan myös instrumentaalin perusmerkitysten käsittely paikkaan ja
aikaan liittyvistä merkityksistä, joista suurin osa toteutuu
prepositioiden avulla. Šeljakin (Шелякин 2006: 67) ryhmittelee
instrumentaalin paikalliset perusmerkitykset merkityksiksi, jotka
kuvaavat jonkin objektin sijaintia suhteessa toiseen objektiin. Tässä
merkityksessä oleellisia ovat prepositiot над, под, перед, за, с ja
между.

\begin{enumerate}
\def\labelenumi{(\arabic{enumi})}
\tightlist
\item
  жарить над раскалёнными углями.
\item
  \ldots{}встанете \emph{перед картиной} и будете думать: ``Так\ldots{}
  вот она, это же она\ldots{}''
\item
  Под пиджаком у него была только рваная сетчатая майка
\end{enumerate}

Tiettyjä paikallisia merkityksiä voidaan ilmaista ilman prepositioita,
kuten ilmauksissa tyyppiä \emph{metsän läpi}, \emph{rantaa myöten} ym.

\begin{enumerate}
\def\labelenumi{(\arabic{enumi})}
\setcounter{enumi}{3}
\tightlist
\item
  Почему-то я вспомнил, как мы шли сквером (`aukio') и спорили
\item
  Кругом золотились осенние поля, мы шли дорогой, по которой\ldots{}
\item
  Он шёл лесом, и ему представлялось: вот уж нет его\ldots{}
\end{enumerate}

Myös ajalliset merkitykset ovat varmasti tuttuja, sekä prepositioiden
kanssa että ilman:

\begin{itemize}
\tightlist
\item
  перед сном / рассветом / тем, как\ldots{}
\item
  утром, вечером, днём, летом, зимой, целыми вечерами
\end{itemize}

Huomaa, että käytännössä sanat утром, вечером ym. ovat nykyvenäjässä
adverbejä, morfologisesti siis muotoa /утром/ ennemmin kuin /утр/ом/.

Ajallisesta merkityksestä on kyse myös silloin, kun kuvataan jotain
toimintaa alkavaksi suhteessa johonkin tapahtumaan: \emph{встать с
солнцем, уехать с рассветом} (Шелякин 2006: 68).

\subsubsection{Agentti}\label{agentti}

Kuten tunnettua, eräs instrumentaalin tehtävistä on ilmaista
passiivilauseiden tekijää:

\begin{enumerate}
\def\labelenumi{(\arabic{enumi})}
\setcounter{enumi}{6}
\tightlist
\item
  Мы видели канал, покрытый льдом.
\item
  Таким образом, профилактика экстремизма, проводимая государством,
  нарушает права человека и должна быть признана неконституционной
\item
  Квартиры, которые у нас строятся \emph{компанией} «Стройметресурс»,
  отличаются хорошей планировкой.
\end{enumerate}

Erikseen tämän merkityksen sisältä on syytä mainita myös lauseet, joita
Šeljakin (2006: 69) nimittää termillä \emph{значение стихийного носителя
действия}, luonnonvoiman aiheuttaman toiminnan merkitys. Tarkempi
katsaus aiheeseen on saatavilla
\href{http://www.helsinki.fi/~mustajok/pdf/Lodku_uneslo_vetrom.pdf}{tästä}
A. Mustajoen ja M. Kopotevin (2005) artikkelista.

\begin{enumerate}
\def\labelenumi{(\arabic{enumi})}
\setcounter{enumi}{9}
\tightlist
\item
  Ветром унесло лодку
\item
  Тротуар залило водой
\end{enumerate}

\subsubsection{Predikatiivinen
instrumentaali}\label{predikatiivinen-instrumentaali}

Instrumentaali on usein syntaksin kannalta erityisasemassa sen vuoksi,
että, kuten nominatiivi, se voi toimia predikatiivin roolissa
(predikatiivinen instrumentaali, творительный предикативный), eli
ilmoittaa ominaisuuden tai tilan, joka jostakin subjektista lauseessa
kerrotaan. Katso seuraavia esimerkkejä:

\begin{enumerate}
\def\labelenumi{(\arabic{enumi})}
\setcounter{enumi}{11}
\tightlist
\item
  Лицо женщины было \emph{испуганным}
\item
  Или сам, не дай Бог, будешь \emph{капитаном}, и самому придётся всех
  мучить.
\item
  Те, кто тогда были \emph{мастерами}, уже стали классиками.
\item
  И, конечно, мы очень надеемся на то, что наше взаимодействие будет
  \emph{эффективным}
\item
  Например, я когда-то была \emph{стрекозой} (`sudenkorento'), но,
  увидев Паки, влюбилась без ума и превратилась в колибри
\end{enumerate}

Etenkin menneessä ajassa ero instrumentaalin ja nominatiivin välillä on
usein siinä, että instrumentaalilla ilmaistava ominaisuus tai tila on
väliaikaisempi (Wade 2010: 126). Tämä näkyy esimerkiksi lauseissa 12 ja
16, mutta on ainoastaan tendenssi, ei itsestäänselvyys. Predikatiivin
kaltainen on myös käyttö eräiden verbien kuten стать, называть,
остаться, работать, служить kanssa:

\begin{enumerate}
\def\labelenumi{(\arabic{enumi})}
\setcounter{enumi}{16}
\tightlist
\item
  Тесто месить до тех пор, пока оно не станет блестящим
\item
  Стратегические планы Oracle в новом финансовом году останутся прежними
\item
  Эти медленно растущие и особенно ценимые кактусы иногда называют
  \emph{каменными розами}.
\end{enumerate}

\subsubsection{Prepositio + instrumentaali verbin
täydennyksenä}\label{prepositio-instrumentaali-verbin-tuxe4ydennyksenuxe4}

Monet verbit saavat täydennyksekseen prepositiolausekkeen, jonka
substantiivijäsen on instrumentaalissa. Näitä voidaan Šeljakinin (2006:
67) esityksen pohjalta jaotella esimerkiksi seuraavanlaisiin ryhmiin:

\begin{itemize}
\tightlist
\item
  Alistaminen jollekin: властвовать/начальствовать над кем-то
\item
  Mielentila suhteessa johonkin: задуматься/работать/смеяться/шутить над
  чем-либо
\item
  Alisteinen toiminta suhteessa johonkuhun: преклоняться/извиниться
  перед кем-либо
\item
  seuraaminen/välittäminen: следить за политикой, ухаживать за детьми
\item
  päämäärä: сходить за молоком, приехать за дочкой
\end{itemize}

\subsubsection{Instrumentaali verbin
täydennyksenä}\label{instrumentaali-verbin-tuxe4ydennyksenuxe4}

Waden (2010) perusteella voidaan suoran instrumentaalitäydennyksen
saavat verbit jaotella seuraaviin ryhmiin:

\begin{enumerate}
\def\labelenumi{\arabic{enumi}.}
\tightlist
\item
  Hallinta. Esimerkiksi \emph{владеть, править, управлять, ползоваться,
  распологать, распоряжаться, руководит, заведовать.}
\item
  Asenne. Esimerkiksi \emph{восхищаться, гордиться, хвастаться,
  интересоваться, увлекаться, наслаждаться, обходиться.}
\item
  Vastavuoroinen toiminta. Esimerkiksi делиться, обмениваться.
\item
  Muut, kuten (usein merkitykseltään negatiiviset) \emph{болеть,
  страдать, жертвовать, рисковать} tai positiivisemmat/neutraalit
  \emph{торговать, прославляться, заниматсья, увлекаться}.
\end{enumerate}

\begin{enumerate}
\def\labelenumi{(\arabic{enumi})}
\setcounter{enumi}{19}
\tightlist
\item
  Визенталь утверждал, что располагает \emph{длинным списком} с
  фамилиями бывших нацистов.
\item
  Профессор, академик РАЕН Евгений Николаевич Панов заведует
  \emph{лабораторией} сравнительной этологии и биокоммуникации Института
  проблем экологии и эволюции РАН.
\item
  Хвастались друг перед другом своей ловкостью, умением смухлевать,
  переторговать, махнуться не глядя.
\item
  Немецкий зоолог Рюдигер Ферхассельт делится \emph{своими
  наблюдениями}: ``У меня в аквариуме был розово-красный самец\ldots{}''
\item
  Новая столица Японии прославлялась всеми средствами массовой
  информации.
\end{enumerate}

\subsubsection{Kontrastiivisia
huomioita}\label{kontrastiivisia-huomioita}

Vertaillaan vielä suomea ja venäjää ja pohditaan, miten suomessa
ilmaistaan asioita, joita venäjässä ilmaistaan instrumentaalilla.

Instrumentaalia käytetään ensinnäkin usein suomen essiiviin:

\begin{enumerate}
\def\labelenumi{(\arabic{enumi})}
\setcounter{enumi}{24}
\tightlist
\item
  Она умерла молодой / героем
\item
  Они уехали семьей
\item
  Если приду здоровым\ldots{}
\end{enumerate}

Vertaa myös:

\begin{enumerate}
\def\labelenumi{(\arabic{enumi})}
\setcounter{enumi}{27}
\tightlist
\item
  Он сидел плечом к пленному, а спиной к дереву
\end{enumerate}

Toisaalta suomessa on usein suora objekti siinä, missä venäjässä on
instrumentaalisijainen sana. Luonnollisesti monet näistä tapauksista
liittyvät edellä esitettyivin instrumentaalitäydennyksen saaviin
verbeihin (заниматься, управлять ym.) Lisäksi kannattaa kiinnittää
huomiota esimerkiksi ruumiinosien liikuttamiseen:

\begin{enumerate}
\def\labelenumi{(\arabic{enumi})}
\setcounter{enumi}{28}
\tightlist
\item
  Я молча махнул рукой и пошёл прочь.
\item
  Принюхиваясь, очень смешно двигают носом и мигают глазами.
\end{enumerate}

Wade (2010: 122) huomauttaa, että näissä tapauksissa paitsi ruumiinosa
myös \emph{ruumiinosalla} (kädellä) tapahtuva liike tulkitaan samoin:

\begin{enumerate}
\def\labelenumi{(\arabic{enumi})}
\setcounter{enumi}{30}
\tightlist
\item
  На одной из них стоят два гражданина и размахивают чемоданами.
\end{enumerate}

\subsection{Prepositionaalin
merkityksiä}\label{prepositionaalin-merkityksiuxe4}

Nimensä mukaisesti prepositionaali on \textbf{sija, jota käytetään vain
prepositioiden yhteydessä}. Se ei siis ole suoraan esimerkiksi verbin
täydennyksenä olevan sanan suku. Prepositionaalin yhteydessä
paikallismerkityksistä tuskin tarvitsee erikseen mainita: в-prepositio +
prepositionaali merkityksessä `jossakin paikassa' on ensimmäisiä
rakenteita, jonka venäjää vieraana kielenä opiskeleva omaksuu. Tämä
rakenne vastaa suuressa määrin suomen inessiiviä, mikä tekee
omaksumisesta suomalaiselle usein helppoa, mutta toisaalta aiheuttaa
myös hankaluuksia. Tavalliset ajan ja paikan merkitykset on listattu
seuraavassa:

\begin{enumerate}
\def\labelenumi{\arabic{enumi}.}
\tightlist
\item
  Sijainti jossakin konkreettisessa tai abstraktissa tilassa: в здании,
  в зоопарке, в метро, в школе -- в обществе, в мире, в таком состоянии,
  в моем понимании, в жизни\ldots{}
\item
  Ajallisen sijainnin (kysymys \emph{milloin}) ilmaiseminen tietyillä
  substantiiveilla: в этом году, на прошлой неделе, в этом
  месяце\ldots{}
\item
  Sijainti jonkin liepeillä (при-prepositio): При школе, при дороге.
  Tämäkin merkitys toteutuu myös abstraktimmassa mielessä
  (`yhteydessä'): при жарении, при решении, при создании ym.
\item
  Harvinaisempi ajallinen merkitys on по-preposition kanssa
  muodostettava, peräkkäisyyttä ilmaiseva merkitys (ks. Шелякин 2006:
  69): по окончании университета, по приезде домой.
\end{enumerate}

Aika- ja paikkamerkitysten lisäksi on luonnollisesti muistettava
о-prepositioon liittyvä merkitys, joka kuvaa jonkin: говорит о погоде,
рассказывать о поездке, вопрос о необходимости принятых мер, мнение о
внешней политики ym.

\subsection{На vai в?}\label{ux43dux430-vai-ux432}

Šeljakin (2006: 70) toteaa в- ja на-prepositioiden peruserosta
seuraavaa:

\emph{{[}\ldots{}{]} предлог в указывает на внутренние пределы предмета,
предлог на -- на внешние пределы.}

Ilmaistaessa objektin A sijaintia suhteessa objektiin B prepositioiden
välinen työnjako onkin myös kielenoppijalle helppo. Ilmaisujen \emph{на
столе} ja \emph{в коробке} ero on helppo omaksua ja vastaa suomen sisä-
ja ulkopaikallissijojen välistä eroa. Ongelmia ilmenee kuitenkin, kun
siirrytään abstraktimpiin paikan kuvauksiin: venäjäksi sijaitaan
\emph{на работе}, suomessa taas sisäpaikallisesti \emph{töissä}.
Toisaalta suomeksi ollaan \emph{vieraisilla}, venäjässä \emph{в гостях}.
Kannattaa myös muistaa, että в- ja из-prepositioita sekä на- ja
с-prepositioita käytetään yleensä johdonmukaisesti omina ryhminään (на
работе/с работы, в школе/из школы)\footnote{tästäkin on eräitä
  poikkeuksia (ks. Wade 2010: 423).}.

Seuraaviin taulukoihin on koottu tapauksia, joissa suomen jaottelu sisä-
ja ulkopaikallissijoihin ei osu yksiin venäjän v- ja na-prepositioiden
käytön kanssa. Monissa tapauksissa sekä v- että na-prepositio ovat
mahdollisia, mutta näissäkin käyttöä rajaa usein konteksti sillä
tavalla, että prepositioilla on enemmän tai vähemmän selvä työnjako.

\subsubsection{На-prepositio
venäjässä}\label{ux43dux430-prepositio-venuxe4juxe4ssuxe4}

Ensiksi keskitytään tapauksiin, joissa suomessa on sisäpaikallissija,
mutta venäjässä ainakin tietyissä konteksteissa na.

\begin{longtable}[c]{@{}lllp{5cm}p{3cm}@{}}
\toprule
sana & в/на & suomi & huom. & ks. esimerkit\tabularnewline
\midrule
\endhead
работа & на & ssa & &\tabularnewline
земля & на & ssa & maassa (maapallolla) & 32.\tabularnewline
рука & на & ssa/lla & riippuen kontekstista: tietyissä tapauksissa
(erit. mon.) venäjässä на & 33, 34, MUTTA 35, 36\tabularnewline
практика & на & ssa & käytännössä & 37, 38\tabularnewline
предприятие & на & ssa & jonkin verran riippuvaisuuttaa kontekstista &
39, 40, 41, MUTTA 42\tabularnewline
свет (maailma) & на & ssa & mutta: maapallolla &\tabularnewline
объект & на & ssa & kohteessa & 43\tabularnewline
вода & на & ssa & tietyissä tapauksissa venäjässä на (vrt. kelluminen /
veden pinta) & 44, 45\tabularnewline
кухня & на & ssa & erit. puhekielessä joskus myös в & 46\tabularnewline
остров & на & ssa/lla & suomessakin joskus saarella, muttei aina &
47\tabularnewline
выборы & на & ssa & &\tabularnewline
юг & на & ssa & &\tabularnewline
Украина & на & ssa & Prepositio ainakin jossain määrin poliittinen
valinta &\tabularnewline
выставка & на & ssa & & 48\tabularnewline
север & на & ssa & &\tabularnewline
заседание & на & ssa & kokouksessa. ''В заседании'' mahdollinen, muttei
yleinen & 49, 50\tabularnewline
фотография & на & ssa & valokuvassa & 51\tabularnewline
этап & на & ssa & vaiheessa & 52\tabularnewline
стадия & на & ssa & vaiheessa & 53\tabularnewline
свадьба & на & ssa & häissä &\tabularnewline
природа & на & ssa & Kun merkityksessä `viettää aikaa luonnossa' & 54,
55, 56 MUTTA 57\tabularnewline
дело & на & ssa & на самом деле &\tabularnewline
встреча & на & ssa & tapaamisessa & 58\tabularnewline
почта & на & ssa & & 59\tabularnewline
праздник & на & ssa & juhlissa & 60\tabularnewline
восток & на & ssa & &\tabularnewline
Запад & на & ssa & &\tabularnewline
небо & на & ssa* & ''maan päällä niin kuin\ldots{}'' & 61\tabularnewline
родина & на & ssa & kotimaassa, synnyinmaassa, isänmaassa
&\tabularnewline
служба & на & ssa & vrt. на работе & 62\tabularnewline
семинар & на & ssa & &\tabularnewline
конференция & на & ssa & konferenssissa, symposiumissa, seminaarissa.. &
63\tabularnewline
картина & на & ssa & kuvassa, taulussa.. & 64\tabularnewline
мероприятие & на & ssa & tapahtumassa & 65\tabularnewline
война & на & ssa & & 66, 67 MUTTA 68\tabularnewline
шея & на & ssa & kaulassa (kaulalla) & 69\tabularnewline
\bottomrule
\end{longtable}

\begin{enumerate}
\def\labelenumi{(\arabic{enumi})}
\setcounter{enumi}{31}
\tightlist
\item
  Нельзя сливать на землю масло, бензин и прочую гадость
\item
  У нас на руках есть медсправки
\item
  На руке не было обручального кольца
\item
  Постоянно по дому ходит с мобилкой в руках
\item
  Люди в руках закона и сопротивления не оказывают
\item
  Однако на практике речь идёт о том, кто будет ``принимать'' приходящие
  в Россию крупные инвестиции
\item
  В практике Совета Безопасности ООН не было случаев, чтобы..
\item
  Но у нас на предприятии работают больше ста тысяч человек
\item
  В течение зимних каникул дети обычно посещали несколько ёлок (в
  детском саду или в школе, на предприятиях и учреждениях);
\item
  Забастовки на предприятиях и в организациях Российской Федерации в
  1990-- 1995 годы
\item
  Запреты не распространяются на торговлю алкоголем в предприятиях
  общественного питания.
\item
  Как всегда, в ночное время на объектах шли работы
\item
  Листья на воде
\item
  Она развивает скорость на воде до 80, а по снегу -- до 100 км/ч
\item
  Мы сидели на кухне и пили чай
\item
  Хотя на острове всё же росли деревья и кусты. У Хонма-сан на острове
  Хоккайдо десять отелей.
\item
  Прототип устройства был продемонстрирован на выставке офисного
  оборудования в Японии в 1964 году.
\item
  На заседании был заслушан вопрос о проведении Конгресса бухгалтеров и
  аудиторов России
\item
  На заседании комитета также было решено обратиться к правительству
\item
  А на фотографии нам было шесть лет
\item
  Ещё на этапе планировки сразу надо решить - сколько времени вы будете
  уделять будущему саду.
\item
  находиться в стадии: Наш институт находится в стадии
  завершения\ldots{}
\item
  Вместе мы очень любим отдыхать на природе
\item
  Находясь на природе, не нужно забывать, что после вас сюда придут
  другие люди
\item
  Отдыхать детям лучше на природе, а не в городе.
\item
  В 1783 г. Дж. Митчелл предположил, что в природе должны существовать
  тёмные звёзды; Вообще в природе встречаются и другие виды игольчатых
  кристаллов.
\item
  Российскую делегацию на встрече возглавлял заместитель министра
  иностранных дел В. И. Калюжный.
\item
  Он тут же поехал на почту и отбил Павлу Алексеевичу телеграмму.
\item
  На празднике мы были вместе с Титовым,
\item
  Как на небе, так и на земле
\item
  В квартире тихо: мама на службе, Варя и Зоя-- в школе.
\item
  В прошлом году у нас на конференции выступал замечательный финский
  экономист Пекка Сутела
\item
  Его можно увидеть на картине Саврасова ``Грачи прилетели''.
\item
  Предполагается, что на мероприятии, которое состоится в конце марта,
  будут обсуждаться все предложения
\item
  ``Я чувствую себя здесь как на войне''
\item
  Отец был военным и пропал на войне без вести
\item
  Почти все верили, что добро победит в войне
\item
  А на шее у него висел амулет
\end{enumerate}

\subsubsection{В-prepositio
venäjässä}\label{ux432-prepositio-venuxe4juxe4ssuxe4}

Siirrytään tarkastelemaan tapauksia, joissa suomessa on
ulkopaikallissija, mutta venäjässä ainakin tietyissä konteksteissa v.

\begin{longtable}[c]{@{}lllp{5cm}p{3cm}@{}}
\toprule
sana & в/на & suomi & huom. & ks. esimerkit\tabularnewline
\midrule
\endhead
область & в & ssa/lla & jollakin alueella, mutta hallinollisesti
läänissä, maakunnassa, oblastissa &\tabularnewline
сфера & в & lla & Alalla / alueella & 70\tabularnewline
регион & в & lla & Alueella & 71\tabularnewline
зона & в & lla & alueella, vyöhykkeellä &\tabularnewline
университет & в & ssa/lla & Suomessa usein ainakin arkipuheessa: töissä
ylipoistolla, opiskelee yliopistolla + ulkopaikallissija kun
`kampuksella' & 72\tabularnewline
край & в & lla & vrt. suomen seudulla & 73, MUTTA tietysti
74\tabularnewline
сезон & в & lla & kaudella & 75, 76, 77\tabularnewline
аэропорт & в & lla & lentokentällä (vrt. kuitenkin НА аеродроме, esim.
79) &\tabularnewline
Крым & в & lla & Krimillä (Krimin niemimaalla) &\tabularnewline
сектор & в & lla & sektorilla & 80, 81, 82\tabularnewline
эпоха & в & lla & aikakaudella & 83\tabularnewline
лагерь & в & lla & leirillä & 84, 85\tabularnewline
отпуск & в & lla & vrt. kuitenkin на каникулях, на отдыхе &
86\tabularnewline
\bottomrule
\end{longtable}

\begin{enumerate}
\def\labelenumi{(\arabic{enumi})}
\setcounter{enumi}{69}
\tightlist
\item
  В том числе будем говорить и о развитии нашего взаимодействия в сфере
  экономики.
\item
  Мы разделяем мнение о том, что достижение стабильности в регионе
  служит вкладом в установление мира во всём мире,
\item
  Я понял, что случай в университете ему не известен.
\item
  Дело происходило в краях довольно южных; Исключение -- только для тех,
  у кого в крае живут родственники
\item
  А он сидел на крае стула
\item
  В последнем матче с ``Сатурном'' в Раменском подопечные Никонова впервые в сезоне забили больше двух мячей
\item
  В сезоне 1994/ 95 гг., отмечая 40-летие своей творческой деятельности
\item
  Эта победа стала для пилота «Ред Булл» седьмой в сезоне
\item
  Наверное, думали, что кто-то будет встречать не в аэропорту, а у
  дверей дома.
\item
  Большой скандал разразился нынешней зимой в Магадане, когда на аэродроме были заморожены полторы сотни солдат-пограничников
\item
  В секторе обрабатывающей промышленности;
\item
  Я в секторе переработкиi
\item
  Легкоатлет в секторе для прыжков в длину.
\item
  Именно поэтому в эпохе Киевской Руси ищут сегодня истоки специфически украинских феноменов
\item
  Ольга у нас до 8 июня в лагере работает.
\item
  В лагере Анатолий Андреевич очень болел
\item
  В отпуске мы и на каникулах.
\end{enumerate}

\subsubsection{Kokoavasti}\label{kokoavasti}

Tehdään vielä joitakin kokoavia huomioita edellä olevista taulukoista.
Ensinnäkin, venäjässä tapahtumaa tai kokoontumista tarkoittavia sanoja
käytetään usein na-preposition kanssa, vaikka suomessa olisi
sisäpaikallissija: \emph{kokouksessa} mutta \emph{на заседании},
\emph{tapaamisessa} mutta \emph{на встрече}. Toiseksi, suomessa aluetta,
vyöhykettä tai kaistaletta tarkoittavat sanat taipuvat
ulkopaikallissijoissa, mutta venäjässä prepositio on tavallisesti v.
Lisäksi venäjä tuntuu erottelevan monesti tarkemmin pinnan ja sisällön:
suomeksi jotakin on taulussa, venäjäksi на картине. Tämä näkyy siinäkin,
että vettä käsitellään osittain eri tavalla: esimerkiksi lehtiä on
suomeksi vedessä, mutta venäjäsä на воде. Oma selkeä eriävä luokkansa
ovat ilmansuunnat, jotka venäjässä liitetään aina на-prepositioon,
suomessa sisäpaikallissijoihin.

Kontrastiiviselta kannalta mainittakoon myös, että kannattaa kiinnittää
huomiota prepositionaalin ja genetiivin monikkomuotojen samanlaisuuteen.
Kielenoppijana tulee usein sekoittaneeksi ilmaisut tyyppiä \emph{после
шумных мест} ja \emph{в шумных местах}.

\subsection{Prepositionaali täydennyksen
sijana}\label{prepositionaali-tuxe4ydennyksen-sijana}

Kuten mainittu, prepositionaali ei suoraan toimi verbin
nominitäydennyksen sijana. On kuitenkin koko joukko verbejä ja
adjektiiveja, joiden täydennys on muodossa v/na + nomini
prepositionaalissa. Waden (2010: 482) mukaan yksi tällainen joukko ovat
varmuutta / epäilystä / syyllisyyttä ilmaisevat sanat:
обвинять/подозревать/признаться в чем. Tässä yhteydessä kannattaa
erityisesti muistaa lyhyt adjektiivimuoto уверен/уверена, jonka
yhteyteen ei liity prepositio о vaan prepositio в:

\begin{enumerate}
\def\labelenumi{(\arabic{enumi})}
\setcounter{enumi}{86}
\tightlist
\item
  Более того, предприниматель уверен в большом будущем этой машины на
  немецких и западноевропейских дорогах
\item
  И тем не менее я был совершенно уверен в своей правоте.
\end{enumerate}

\emph{Участвовать} lienee kaikille tuttu prepositionaalitäydennyksen
saava verbi. Oman mainintansa ansaitsee kuitenkin myös verbi
\emph{разбираться в чем-либо}. Ilmaisu on etenkin puhekielessä
monikäyttöinen (`osata jotakin', `pärjätä jonkin kanssa', `ymmärtää
jonkin päälle') kuten alla olevista esimerkeistä käy ilmi. Esimerkissä
92 on lisäksi instrumentaalitäydennyksen saava versio merkityksessä
`hoidella jokin' (pois päiväjärjestyksestä).

\begin{enumerate}
\def\labelenumi{(\arabic{enumi})}
\setcounter{enumi}{88}
\tightlist
\item
  Можно сказать что он ещё не \emph{разбирается в жизни}, на неи
  табачная компания деньги делает, на его здоровье, а он не понимает..
\item
  Для этого достаточно \emph{разобраться в причинах того}, почему Россия
  в последние четыре года заметно упрочила свою репутацию ответственного
  игрока на международной арене.
\item
  И зачем \emph{разбираться в музыке}, когда можно просто петь.
\item
  Тогда Хлопонин пообещал ``разобраться'' с ними в том случае, если они
  будут вести себя ``неконструктивно''.
\end{enumerate}

\hyperdef{}{ref-mustajoki2005}{\label{ref-mustajoki2005}}
Mustajoki, Arto, and Михаил Вячеславович Копотев. 2005. ``Лодку Унесло
Ветром: Условия И Контексты Употребления Русской` Стихийной' Конструкции
(Лодку Унесло Ветром: The Conditions and Contexts of Use of the Russian`
Elemental' Construction).'' \emph{Russian Linguistics} 29 (1). JSTOR:
1--38.
\url{http://www.helsinki.fi/~mustajok/pdf/Lodku_uneslo_vetrom.pdf}.

\hyperdef{}{ref-wade2010}{\label{ref-wade2010}}
Wade, Terence. 2010. \emph{A Comprehensive Russian Grammar}. Vol. 8.
John Wiley \& Sons.

\hyperdef{}{ref-sheljakin}{\label{ref-sheljakin}}
Шелякин, М.А. 2006. \emph{Справочник По Русской Грамматике}. drofa.

\end{document}
