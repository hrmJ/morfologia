\documentclass[]{scrartcl}
\usepackage[utf8]{inputenc}
\usepackage[T1]{fontenc}
\usepackage[T2A]{fontenc}
\usepackage[finnish]{babel}
\usepackage{linguex} 
\usepackage{longtable} 
\usepackage{booktabs}
\usepackage{amsthm}
\usepackage{graphicx}
\newtheorem{maar}{Määritelmä}
\usepackage{fixltx2e} % provides \textsubscript
\usepackage{textcomp} % provides \textsubscript
\usepackage{hyperref}
\usepackage{xcolor}

\providecommand{\tightlist}{%
  \setlength{\itemsep}{0pt}\setlength{\parskip}{0pt}}

\hypersetup{
    colorlinks,
    linkcolor={red!50!black},
    citecolor={blue!50!black},
    urlcolor={blue!80!black}
}
\author{Juho Härme}
\title{Morfologia-kurssin luentomateriaaleja}
\date{\today}
\begin{document}
\maketitle
\tableofcontents
\newpage



\section{Luento 21: Liikeverbit 2}\label{luento-21-liikeverbit-2}

\begin{itemize}
\tightlist
\item
  \href{https://mustikka.uta.fi/~juho_harme/morfologia/\#tästä-kurssista}{Takaisin
  sivun ylälaitaan}
\item
  \href{http://mustikka.uta.fi/~juho_harme/morfologia/materiaalit/luento21.pdf}{Lataa
  PDF}
\item
  \href{http://mustikka.uta.fi/~juho_harme/morfologia/presentations/luento21.html}{Tutki
  luentokalvoja}
\item
  \href{http://mustikka.uta.fi/~juho_harme/morfologia/tehtavat/luento21.pdf}{Tutki
  tuntitehtäviä}
\end{itemize}

Edellisellä luennolla todettiin, että liikeverbit ovat venäjässä oma
ryhmänsä ennen kaikkia sen perusteella, että ne ne muodostavat pareja
likkeen suuntaisuuden eli toiminnan luonteen perusteella. Ilman
prefiksejä käytettäessä niin duratiiviset kuin iteratiivisetkin verbit
ovat imperfektiivisen aspektin verbejä ja kuvaavat joko yksisuuntaista
liikettä (duratiiviset) tai kaksisuuntaista, useampisuuntaista tai
suunnan suhteen määrittelemätöntä liikettä (iteratiiviset).

Tällä luennolla perehdytään siihen, mitä tapahtuu, kun liikeverbit
esiintyvät prefiksillisinä muotoina.

\subsection{Oikeinkirjoituksesta}\label{oikeinkirjoituksesta}

Liitettäessä prefiksejä iteratiivisten ja duratiivisten verbien
vartaloihin kannattaa muistaa seuraavat tapaukset:

\begin{enumerate}
\def\labelenumi{\arabic{enumi}.}
\tightlist
\item
  Kun идти-verbiin (myös preteritissä) liitetään konsonanttiin päättyvä
  prefiksi, prefiksin ja vartalon väliin lisätään о:

  \begin{itemize}
  \tightlist
  \item
    \emph{сойти, обойти, вошёл, подошёл} jne.
  \end{itemize}
\item
  Kun ехать-verbiin liitetään konsonanttiin päättyvä prefiksi, prefiksin
  ja vartalon väliin lisätään Ъ:

  \begin{itemize}
  \tightlist
  \item
    съехать, въехать jne.
  \end{itemize}
\end{enumerate}

\subsection{Muodostamisesta}\label{muodostamisesta}

Prefiksillisten liikeverbien muodostaminen on monimutkaisempi kysymys
kuin miltä ensi katsomalta vaikuttaisi. Ainoa täysin selkeä osa-alue
ovat duratiivisten (идти-tyyppi) verbien prefiksilliset johdokset
(прийти ym.). Näistä voidaan todeta, että:

\begin{enumerate}
\def\labelenumi{\arabic{enumi}.}
\tightlist
\item
  Prefiksillinen verbi muodostetaan lisäämällä prefiksi duratiivisen
  verbin vartalon edelle.
\item
  Tuloksena on perfektiivisen aspektin verbi
\end{enumerate}

Toisin sanoen:

\begin{itemize}
\tightlist
\item
  ехать-verbistä voidaan johtaa perfektiiviset muodot приехать, поехать,
  въехать, объехать, подъехать ym.
\item
  лететь-verbistä voidaan johtaa perfektiiviset muodot прилететь,
  полететь, влететь, облететь, подлететь ym.
\item
  ja niin edelleen kaikkien duratiivisten verbien osalta
\end{itemize}

Hankalammaksi tilanne muuttuu, kun ryhdytään tutkimaan iteratiivisten
verbien pohjalta (ainakin näennäisesti) johdettuja muotoja. On nimittäin
niin, että käytännössä kaikissa muissa tapauksissa prefiksin lisääminen
imperfektiivisen aspektin verbiin tuottaa perfektiivisen aspektin
verbin, mutta esimerkiksi verbi приносить onkin \emph{imperfektiivinen}.
Tämä häiritsevä huomio on johtanut eräät lingvistit, etupäässä A.V.
Isachenkon (2003{[}1965{]}), esittämään melko poikkeuksellisen teorian
prefiksillisten liikeverbien muodostuksesta. Laura Janda (2010) esittää
oman, hyvin toisenlaisen ratkaisunsa ongelmaan todeten (eittämättä
perustellusti) että perinteinen liikeverbit poikkeuksellisiksi julistava
näkemys on keinotekoinen. Tällä kurssille pyrimme olemaan puuttumatta
mainittuun muodostusongelmaan, ja otamme lähestymistavan, josta uskoisin
olevan eniten käytännön hyötyä esimerkiksi opetustyössä. Se tarkoittaa
jossain määrin pitäytymistä mainittuun ``keinotekoinen poikkeus''
-lähestymistapaan.

\textbf{Oletetaan siis,} perinteisten kielioppien tapaan, että

\begin{enumerate}
\def\labelenumi{\arabic{enumi}.}
\tightlist
\item
  Prefiksin liittäminen duratiiviseen liikeverbiin tuottaa
  perfektiivisen aspektin verbin
\item
  Prefiksin liittäminen iteratiiviseen liikeverbiin tuottaa
  imperfektiivisen aspektin verbin.
\item
  Tietyt iteratiiviset verbit käyttäytyvät poikkeuksellisella tavalla
  prefiksillistä muotoa muodostettaessa
\end{enumerate}

\subsubsection{Selkeät tapaukset}\label{selkeuxe4t-tapaukset}

Suurin osa etenkin tavallisimmista iteratiivisista liikeverbeistä
käyttäytyyy onneksi hyvin säännönmukaisesti, kun niihin liitetään
prefiksi. Tähän säännönmukaiseen ryhmään kuuluvat seuraavat verbit, eikä
näiden verbien vartalolle tapahdu mitään prefiksin liittämisen
yhteydessä:

\begin{longtable}[c]{@{}ll@{}}
\toprule
iteratiivinen verbi & от-prefiksi/i\tabularnewline
\midrule
\endhead
ходить & отходИть\tabularnewline
носить & относИть\tabularnewline
водить & отводИть\tabularnewline
летать & отлетАть\tabularnewline
возить & отвозИть\tabularnewline
гоня́ть & отгонЯть\tabularnewline
бродить & *отбродИть\tabularnewline
\bottomrule
\end{longtable}

* от-prefiksillinen muoto tästä verbistä on ainoastaan hypoteettinen

\subsubsection{Painollinen a-suffiksi}\label{painollinen-a-suffiksi}

Kolme verbiä muodostaa poikkeuksen siinä suhteessa, että paino siirtyy
prefiksillisessä muodossa kohti sanan loppua. Lisäksi ездить-verbillä
tapahtuu sama äänteenmuutos kuin persoonataivutuksessa:

\begin{longtable}[c]{@{}ll@{}}
\toprule
iteratiivinen verbi & от-prefiksi/i\tabularnewline
\midrule
\endhead
бЕгать & отбегАть\tabularnewline
пОлзать & отползАть\tabularnewline
Ездить & отъезжАть\tabularnewline
\bottomrule
\end{longtable}

\subsubsection{Muodostuksen pohjana prefiksillinen duratiivinen
verbi}\label{muodostuksen-pohjana-prefiksillinen-duratiivinen-verbi}

Lopuksi voidaan erottaa neljä tapausta, joissa ainakin
\emph{vaikuttaisi} siltä, että imperfektiivisen aspektin muoto ei itse
asiassa ole muodostettu prefiksillä iteratiivisesta prefiksittömästä
verbistä vaan ennemminkin \emph{suffiksilla duratiivisesta
prefiksillisestä} verbistä.

\begin{longtable}[c]{@{}ll@{}}
\toprule
iteratiivinen verbi & от-prefiksi/i\tabularnewline
\midrule
\endhead
плавать & отплывать\tabularnewline
ката́ть & откатывать\tabularnewline
ла́зить & отлезАть\tabularnewline
таска́ть & оттаскивать\tabularnewline
\bottomrule
\end{longtable}

Toisin sanoen, samalla tapaa kuin verbi перечитывать on muodostettu
suffiksilla verbistä перечитать, verbi отплывать ainakin vaikuttaa
muodostetun suffiksilla verbistä отплыть. Vertailun vuoksi seuraavassa
on esitetty rinnakkain prefiksilliset duratiivisesta liikeverbistä
muodostetut muodot ja tämän ryhmän imperfektiiviset muodot:

\begin{longtable}[c]{@{}ll@{}}
\toprule
iteratiivinen verbi & от-prefiksi/i\tabularnewline
\midrule
\endhead
отплыть & отплывАть\tabularnewline
откатить & откАтывать\tabularnewline
отлезть & отлезАть\tabularnewline
оттащить & оттаскивать\tabularnewline
\bottomrule
\end{longtable}

\subsubsection{Vielä yksi}\label{vieluxe4-yksi}

Tilanne ei vieläkään ole täysin tyydyttävästi selitetty. Asiaan
vaikuttavat vielä \emph{liikeverbien teonlaadut}, joita käsitellään
luennon lopuksi.

\subsection{Prefiksillisten verbien välinen ero aspektuaalisena
erona}\label{prefiksillisten-verbien-vuxe4linen-ero-aspektuaalisena-erona}

Идти- ja ходить-verbien välinen ero, kuten jo monesti todettua, on
peräisin näiden verbien eri tavasta kuvata liikkeen suuntaa. Jos
verrataan esimerkiksi проходить- ja пройти-verbejä, tilanne on kuitenkin
toinen: \emph{ero muuttuu tavallisesti aspektuaaliseksi}.

Tätä huomiota on syytä tutkia tarkemmin. Verbien пройти/проходить,
пробежать/пробегать, пролететь/пролетать välillä ei siis tämän väitteen
mukaan olekaan edellä käsiteltyä eroa liikkeen suunnassa vaan ero on
muuttunut aspektuaaliseksi eroksi imperfektiivisen (проходить) ja
perfektiivisen (пройти) verbin välillä. Tutkitaan tätä vertaamalla
seuraavia lauseita:

\begin{enumerate}
\def\labelenumi{(\arabic{enumi})}
\tightlist
\item
  Папа ходил в магазин
\item
  Папа шёл в магазин
\item
  Папа проходил мимо магазина
\item
  Папа прошёл мимо магазина
\end{enumerate}

Lauseessa 1 iteratiivinen verbi ilmaisee liikettä kahteen suuntaan,
lauseessa 2 duratiivinen verbi ilmaisee liikkeen yksisuuntaisena.
Lauseiden 3 ja 4 välinen ero kuitenkin on nimenomaan aspektuaalinen:
lauseessa 3 проходить kuvaa toiminnan sisäisesti rajaamattomana, niin
että lauseen isä kuvataan kulkemassa juuri kaupan ohitse; lauseessa 4
прошёл kuvaa toiminnan sisäisesti rajattuna, niin että kauppa on jo
ohitettu. Samoin esimerkissä 5 on verbi приезжать eikä verbiä приехать
puhtaan aspektuaalisista syistä: perfektiivinen aspekti viittaisi yhteen
konkreettiseen faktaan, joten часто-adverbin kanssa on käytettävä
imperfektiivista aspektia.

\begin{enumerate}
\def\labelenumi{(\arabic{enumi})}
\setcounter{enumi}{4}
\tightlist
\item
  Я часто приезжаю в университет на автобусе
\end{enumerate}

Ajattele tässä yhteydessä taannoista esimerkkiämme ({\textbf{???}}),
jonka toista tässä esimerkkinä 6

\begin{enumerate}
\def\labelenumi{(\arabic{enumi})}
\setcounter{enumi}{5}
\tightlist
\item
  -- Папа вчера приезжал.
\end{enumerate}

Esimerkissä 6 toki on tapahtunut liike kahteen suuntaan, mutta ennemmin
kuin liikkeen suunnasta, tässä on kyse \emph{toiminnan mitätöitymisestä}
-- samalla tapaan kuin lauseissa ``кто брал со стола фломастер'', ``вы
не открывали окно'' jne. Приехать-verbin käyttö toisi puolestaan
tulkinnan siitä, että isä on käymässä tälläkin hetkellä
perfektiivisyytensä vuoksi, ei niinkään liittyen liikkeen suuntaan.

Nikunlassi (2002: 187) kuitenkin huomauttaa, että tietyillä
liikeverbeillä on rajoituksia siinä, mitä kaikkia aspektuaalisia
merkityksiä ne voivat ilmaista. Esimerkiksi приходить-verbiä ei voi
Nikunlassin mukaan käyttää lauseessa ``*Смотри, брат приходит'', vaan on
turvauduttava prefiksittömään идти-verbiin.

\subsection{Prefiksit ja merkitykset}\label{prefiksit-ja-merkitykset}

Seuraavassa on kuvattu lyhyesti eri prefiksien tavallisesti mukanaan
tuomat merkitykset.

\subsubsection{У}\label{ux443}

\begin{itemize}
\tightlist
\item
  liike pois, pidemmäksi aikaa / kauemmaksi
\end{itemize}

\begin{enumerate}
\def\labelenumi{(\arabic{enumi})}
\setcounter{enumi}{6}
\tightlist
\item
  Он был единственным сыном в семье, были еще две старшие сестры,
  которые \emph{уехали} из России в 1918--1919 годах
\item
  Брат уходит на работу через 15 минут
\end{enumerate}

\subsubsection{Вы}\label{ux432ux44b}

\begin{itemize}
\tightlist
\item
  liikkua sisältä ulos
\item
  ylittää kahden tilan välinen raja ja jättää raja taakseen\\
\item
  ilmaisee myös lähtöpistettä, etenkin lähtöajan yhteydessä
\item
  poistua hetkeksi
\end{itemize}

\begin{enumerate}
\def\labelenumi{(\arabic{enumi})}
\setcounter{enumi}{8}
\tightlist
\item
  Затем мы \emph{вылетели} из Соединенных Штатов и прибыли в Японию
\item
  Естественно, если вы впервые \emph{выехали} за пределы Родины, то
  основной проблемой для вас наверняка является вопрос проживания на
  отдыхе
\item
  \emph{Вышел} на минуту протереть лобовое стекло и в мгновение лишился
  сумки, которая лежала на заднем сиденье
\end{enumerate}

\subsubsection{При}\label{ux43fux440ux438}

\begin{itemize}
\tightlist
\item
  Tulla jonnekin, ilmestyä jonnekin
\end{itemize}

\begin{enumerate}
\def\labelenumi{(\arabic{enumi})}
\setcounter{enumi}{11}
\tightlist
\item
  Олег приехал в Питер
\end{enumerate}

\subsubsection{За}\label{ux437ux430}

\begin{itemize}
\tightlist
\item
  Poiketa varsinaiselta reitiltä ja tulla / päätyä jonnekin vähäksi
  aikaa
\end{itemize}

Huomaa käyttö myös puhuttaessa internetissä liikkumisesta:

\begin{enumerate}
\def\labelenumi{(\arabic{enumi})}
\setcounter{enumi}{12}
\tightlist
\item
  Уважаемый посетитель, Вы \emph{зашли} на сайт как незарегистрированный
  пользователь
\end{enumerate}

\begin{itemize}
\tightlist
\item
  Päätyä jonnekin kauas, jonkin rajan taakse
\end{itemize}

\begin{enumerate}
\def\labelenumi{(\arabic{enumi})}
\setcounter{enumi}{13}
\tightlist
\item
  Предположительно, он \emph{заплыл} в подводную пещеру Ko Poo и не смог
  выбрать\ldots{}
\end{enumerate}

\begin{itemize}
\tightlist
\item
  Muu negatiivinen seuraus
\end{itemize}

\subsubsection{Под}\label{ux43fux43eux434}

\begin{itemize}
\tightlist
\item
  lähestyä jotakin
\end{itemize}

\begin{enumerate}
\def\labelenumi{(\arabic{enumi})}
\setcounter{enumi}{14}
\tightlist
\item
  Гарри \emph{подбежал} к окну, игнорируя пораженные вопли Флитвика
\end{enumerate}

\subsubsection{В}\label{ux432}

\begin{itemize}
\tightlist
\item
  liikkua/liikuttaa sisälle
\item
  ylittää kahden tilan välinen raja
\end{itemize}

\begin{enumerate}
\def\labelenumi{(\arabic{enumi})}
\setcounter{enumi}{15}
\tightlist
\item
  На жарком африканском солнышке сохло все это моментально. И земляная
  насыпь перед пирамидой, чтобы \emph{втаскивать} туда каменные блоки,
  была не нужна
\item
  Да так сильно, что бур просто \emph{влетел} в зону аномально высокого
  пластового давления.
\end{enumerate}

\subsubsection{Про}\label{ux43fux440ux43e}

\begin{itemize}
\tightlist
\item
  ohittaa / jättää taakseen
\item
  mennä sisään ja edetä peremmälle jossakin tilassa tai ensin edetä ja
  sitten astua sisälle
\item
  kulkea suoraviivainen reitti
\end{itemize}

\begin{enumerate}
\def\labelenumi{(\arabic{enumi})}
\setcounter{enumi}{17}
\tightlist
\item
  Мы \emph{прошли} в отдел новых поступлений, и девушка принесла мне
  женскую сумку Gianfranco Sisti
\item
  Они \emph{прошли} в дом через большую каменную террасу
\end{enumerate}

\subsubsection{Пере}\label{ux43fux435ux440ux435}

\begin{itemize}
\tightlist
\item
  liikkua jonkin puolelta toiselle
\item
  vaihtaa (asuin-)paikkaa
\end{itemize}

\begin{enumerate}
\def\labelenumi{(\arabic{enumi})}
\setcounter{enumi}{19}
\tightlist
\item
  \ldots{} он и есть тот самый мостик, через который русские
  антикоммунисты \emph{перебежали} из Европы в Аргентину в 1947г.
\end{enumerate}

\subsubsection{От}\label{ux43eux442}

\begin{itemize}
\tightlist
\item
  loitota pieni etäisyys / vähäksi aikaa
\item
  Toimittaa jokin sinne, minne se kuuluu / toimittaa joksikin aikaa /
  (palauttaa) /
\end{itemize}

\begin{enumerate}
\def\labelenumi{(\arabic{enumi})}
\setcounter{enumi}{20}
\tightlist
\item
  Несколько парней почти сразу отбежали в сторону
\item
  Утром мать отвела дочку в детский сад
\item
  По просьбе Генри я отвез рукопись ему
\end{enumerate}

\subsubsection{до}\label{ux434ux43e}

\begin{itemize}
\tightlist
\item
  saavuttaa jokin raja
\item
  matkan luonnehdinta
\end{itemize}

\begin{enumerate}
\def\labelenumi{(\arabic{enumi})}
\setcounter{enumi}{23}
\tightlist
\item
  За 40 вас довезут до аэропорта
\item
  Как вы доехали?
\end{enumerate}

\subsubsection{о}\label{ux43e}

\begin{itemize}
\tightlist
\item
  liikkua ympäri
\item
  käydä useissa paikoissa
\end{itemize}

\begin{enumerate}
\def\labelenumi{(\arabic{enumi})}
\setcounter{enumi}{25}
\tightlist
\item
  Он \emph{обежал} поляну и бросился вон, а за ним - вся стая
\item
  я \emph{объехал} США вдоль и поперек и такого не видел
\end{enumerate}

\subsubsection{с}\label{ux441}

\begin{itemize}
\tightlist
\item
  laskeutua
\item
  kerääntyä yhteen paikkaan
\end{itemize}

\begin{enumerate}
\def\labelenumi{(\arabic{enumi})}
\setcounter{enumi}{27}
\tightlist
\item
  Лыжник \emph{съехал} с горы
\item
  В Московской консерватории 23 октября \emph{сошлись} Восток и Запад.
\end{enumerate}

\subsubsection{Раз}\label{ux440ux430ux437}

\begin{itemize}
\tightlist
\item
  hajaantua eri suuntiin
\end{itemize}

\begin{enumerate}
\def\labelenumi{(\arabic{enumi})}
\setcounter{enumi}{29}
\tightlist
\item
  Милиция поговорила с ребятами, они \emph{разошлись} и, собственно,
  ничего не произошло.
\end{enumerate}

\subsubsection{Вз}\label{ux432ux437}

\begin{itemize}
\tightlist
\item
  nousta ylös
\end{itemize}

\begin{enumerate}
\def\labelenumi{(\arabic{enumi})}
\setcounter{enumi}{30}
\tightlist
\item
  Они уже \emph{взошли} на вершину холма, от остальной компании их
  скрывала полуразрушенная стена
\item
  Солнце уже \emph{взошло}.
\end{enumerate}

\subsection{Liikeverbien teonlaadut}\label{liikeverbien-teonlaadut}

Edellä aspektiaiheisten luentojen lopuksi puhuttiin nopeasti
\emph{teonlaatujen} (способы действия) käsitteestä. Teonlaadut ovat
selitysvoimainen termi myös eräitä prefiksillisiä liikeverbejä
tutkittaessa, tarkemmin sanottuna verbejä, jotka on muodostettu
iteratiivisista (ходить-tyyppi) liikeverbeistä.

Muistatko aspektiluennon yhteydestä verbejä tyyppiä \emph{поработать,
посидеть, покурить}? Nämähän ilmaisivat tekemistä jonkin tietyn, rajatun
ajan: После обеда мы посидели в гостиной. Samalla tapaa iteratiivisista
liikeverbeistä voidaan muodostaa tällaisia \emph{delimitaatiivista
teonlaatua} (ограничительный способ действия) edustavia perfektiivisen
aspektin verbejä:

\begin{enumerate}
\def\labelenumi{(\arabic{enumi})}
\setcounter{enumi}{32}
\tightlist
\item
  И у меня еще один вопрос: если просто \emph{поплавать} в бассейне, то
  это платно или бесплатно?
\item
  Приехали в Хельсинки, \emph{поездили} по городу на автобусе, побывали
  в соборе в скале, а потом поехали в аквапарк.
\item
  Поскольку жизнь в квартире для растущих детей ограничивает их. Им
  нужен простор, где побегать.
\end{enumerate}

On tärkeä huomata että verbit \emph{пойти} ja \emph{походить} eivät ole
aspektipareja, vaan походить on aspektipariton delimitatiivinen verbi,
joka merkitsee `kävellä vähän aikaa'.

Liikeverbeistä voidaan muodostaa myös muita teonlaatuja, kuten
perduratiivinen (продолжительный):

\begin{enumerate}
\def\labelenumi{(\arabic{enumi})}
\setcounter{enumi}{35}
\tightlist
\item
  Вы можете много лет \emph{проездить} на мощных автомобилях и ни разу в
  жизни не надавить педаль газа до пола.
\end{enumerate}

Kaikkein oleellisin liikeverbeistä muodostettava teonlaatu on kuitenkin
momentaaninen (одноактный) teonlaatu, joka liikeverbien tapauksessa
muodostetaan liittämällä с-prefiksi iteratiivisen verbin vartaloon.
Liikeverbien tapauksessa merkitykseksi muodostuu `jonnekin
poikkeaminen', usein jonkin päämäärän -- haettavan tavaran -- tähden tai
`jossain käyminen kerran'.

\begin{enumerate}
\def\labelenumi{(\arabic{enumi})}
\setcounter{enumi}{36}
\tightlist
\item
  Сегодня \emph{съездила} в треннажерку и поехала к Серёже.
\item
  Купил два конверта и написал Лёхе Матвееву, чтобы \emph{сходил} к деду
  моему и объяснил ему всё.
\item
  Ну, я \emph{сбегал} в видеосалон, быстренько просмотрел пару боевиков,
  возвращаюсь, а их нет.
\item
  Затем \emph{сходил} в сарай за топором и быстро нарубил щепы.
\item
  Однажды инженер Иванов решил \emph{сходить} за кефиром.
\item
  Вокруг лес. Есть возможность \emph{сходить} за грибами.
\end{enumerate}

Huomaa, että teonlaatujen muodostaminen iteratiivisista liikeverbeistä
tapahtuu eri tavalla kuin ``tavallisten'' perfektiivisten liikeverbien.
Teonlaatujen tapauksessa iteratiivisen verbin vartalolle ei tapahdu
mitään -- edes paino ei liiku. Ajattele esimerkin 37 verbiä
\emph{съездить}. Muodostettaessa tavallisia perfektiivisen aspektin
liikeverbiä käytetään \emph{езжать}-vartaloon parustuvia muotoja kuten
\emph{поезжать}, \emph{отъезжать} ym. Teonlaatuja muodostettaessa
lähtökohtana on kuitenkin muuttumaton iteratiivisen verbin vartalo.
Tämän takia myös esimerkissä 33 verbi on \emph{поплавать}, vaikka muuten
prefiksilliset muodot плавать-verbin tapauksessa ovat tyyppiä
\emph{подплывать}, \emph{отплывать} jne. Huomaa myös, että бЕгать-verbin
teonlaaduissa paino pysyy е-vokaalilla (дети поБегали на улице), mutta
tavallisissa prefiksillisissä muodoissa paino liikkuu viimeiselle
tavulle (спортсмены пробегАли мимо стадиона).

S-etuliitteiset momentaaniset liikeverbit ovat merkitykseltään lähellä
etuliitteettömiä iteratiivisia liikeverbejä. Äkkiseltään esimerkiksi
lauseilla 43 ja 44 ei juuri ole eroa:

\begin{enumerate}
\def\labelenumi{(\arabic{enumi})}
\setcounter{enumi}{42}
\tightlist
\item
  Я ездил в Петербург
\item
  Я съездил в Петербург
\end{enumerate}

Ero voidaan kuitenkin nähdä siinä, että perfektiiviselle aspektille
tyypillisesti esimerkki 44 painottaa tuloksen voimassa oloa, eli tässä
tapauksessa sitä, että matka onnistuttiin tekemään ja on ikään kuin
vähemmän neutraali, kun taas esimerkki 43 ainoastaan toteaa, että
tällainen matka tuli tehtyä. G.K. Skvortsova (2003) kuvaa eroa
liikeverbiharjoituskirjassaan seuraavilla selittävillä lauseilla:

\begin{enumerate}
\def\labelenumi{(\arabic{enumi})}
\setcounter{enumi}{44}
\tightlist
\item
  Я ездил в Петербург = Я был в Петербурге
\item
  Я съездил в Петербург = Мне удалось побывать в Петербурге
\end{enumerate}

Huomaa vielä, että koska с-prefiksi tuottaa myös merkityksen `liikkua
jostakin alas', syntyy edellä ensimmäiseen muodostysryhmään lueteltujen
tavallisimpien verbien (сходить, сносить ym.) kohdalla homonyymisiä
muotoja, joiden kohdalla vain konteksti kertoo, onko kyse
momentaanisesta teonlaadusta vai alaspäin vievän liikeen merkityksestä.

\hyperdef{}{ref-janda2010}{\label{ref-janda2010}}
Janda, Laura. 2010. ``Perfectives from Indeterminate Motion Verbs in
Russian.'' \emph{New Approaches to Slavic Verbs of Motion. Amsterdam:
John Benjamins}, 125--39.

\hyperdef{}{ref-nikunl}{\label{ref-nikunl}}
Nikunlassi, Ahti. 2002. \emph{Johdatus Venäjän Kieleen Ja Sen
Tutkimukseen}. Helsinki: Finn Lectura.

\hyperdef{}{ref-isachenko2003}{\label{ref-isachenko2003}}
Исаченко, Александр. 2003{[}1965{]}. \emph{Грамматический Строй Русского
Языка В Сопоставлении С Словацким. Морфология. Часть 1, 2}. Москва:
Языки славянской культуры.
\url{https://www.litres.ru/a-v-isachenko/grammaticheskiy-stroy-russkogo-yazyka-v-sopostavlenii-s-slovackim-morfologiya-chast-1-2/}.

\hyperdef{}{ref-skvortsova2003}{\label{ref-skvortsova2003}}
Скворцова, Г. Л. 2003. \emph{Глаголы Движения - Без Ошибок}. Москва:
Русский язык. Курсы.

\end{document}
