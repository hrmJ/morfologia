\documentclass[]{scrartcl}
\usepackage[utf8]{inputenc}
\usepackage[T1]{fontenc}
\usepackage[T2A]{fontenc}
\usepackage[finnish]{babel}
\usepackage{linguex} 
\usepackage{longtable} 
\usepackage{booktabs}
\usepackage{amsthm}
\usepackage{graphicx}
\newtheorem{maar}{Määritelmä}
\usepackage{fixltx2e} % provides \textsubscript
\usepackage{textcomp} % provides \textsubscript
\usepackage{hyperref}
\usepackage{xcolor}

\providecommand{\tightlist}{%
  \setlength{\itemsep}{0pt}\setlength{\parskip}{0pt}}

\hypersetup{
    colorlinks,
    linkcolor={red!50!black},
    citecolor={blue!50!black},
    urlcolor={blue!80!black}
}
\author{Juho Härme}
\title{Morfologia-kurssin luentomateriaaleja}
\date{\today}
\begin{document}
\maketitle
\tableofcontents
\newpage



\section{luento 8: Sijat funktionaalisen morfologian kannalta
2.}\label{luento-8-sijat-funktionaalisen-morfologian-kannalta-2.}

\subsection{Datiivin merkityksiä ja
käyttöä}\label{datiivin-merkityksiuxe4-ja-kuxe4yttuxf6uxe4}

Vaikka genetiivi on venäjän sijoista ehkä monikäyttöisin, myös
datiivilla on erityisesti \emph{prepositioiden yhteydessä} koko joukko
merkityksiä. Šeljakin (2006: 63--65) erottaa itse asiassa kaikkiaan
yhdeksän eri merkitysrypästä, joista seuraavassa tarkastellaan
huomionarvoisimpia ja vähemmän itsestäänselviä.

\subsubsection{Tuttuja merkityksiä}\label{tuttuja-merkityksiuxe4}

Ensinnäkin voidaan todeta, että niin kuin käytännössä kaikilla venäjän
sijoilla, myös datiivilla on koko joukko paikallismerkityksiä, jotka
suomessa ilmaistaisiin varsinaisilla paikallissijoilla (inessiivillä,
elatiivilla, adessiivilla ym.). Pohdi vaikkapa seuraavia esimerkkejä:

\begin{enumerate}
\def\labelenumi{(\arabic{enumi})}
\tightlist
\item
  Зоопарк находится \emph{к югу от} центра.
\item
  \ldots{} в ярости поднимается и идёт \emph{к окну}.
\item
  Сенька снова подошёл к кроватке, погладил мальчику по голове.
\end{enumerate}

Perusmerkitys paikassa on siis к-preposition avulla ilmaistava
lähestyminen jotakin kohti. Kuitenkin esimerkki 3 esittää myös toisen
paikkaan liittyvän merkityksen: \emph{liike pintaa pitkin}. Tämä
toteutetaan по-prepositiolla

\begin{enumerate}
\def\labelenumi{(\arabic{enumi})}
\setcounter{enumi}{3}
\tightlist
\item
  Ползут \emph{по полю} два бойца, вдруг рядом взрыв
\item
  Люди как муравьи \emph{бегают по трассе} склад-магазин-склад, вывозя
  всё, что ещё не стоит на полках
\end{enumerate}

Toinen varmasti tuttu merkitys on vastaanottajan ilmaiseminen:

\begin{enumerate}
\def\labelenumi{(\arabic{enumi})}
\setcounter{enumi}{5}
\tightlist
\item
  Возможно также, что менее образованные родители изо всех сил стремятся
  дать \emph{детям} высшее образование
\item
  Ручку всё-таки благополучно вернули хозяйке
\end{enumerate}

Myös ajan ilmaiseminen on funktio, jonka ilmaisuun suurin osa sijoista
osallistuu, eikä datiivi tee poikkeusta. Ensimmäinen ajallinen merkitys
suomentuu usein mennessä-ilmauksilla:

\begin{enumerate}
\def\labelenumi{(\arabic{enumi})}
\setcounter{enumi}{7}
\tightlist
\item
  \emph{К середине следующей недели} карточки были аккуратно
  распределены на шесть стопок.
\item
  Я по работе тоже практически весь день у компа, так \emph{к вечеру} у
  меня голова кружиться начинает.
\item
  \emph{Ближе к вечеру} мы снова отправились в лагерь
\end{enumerate}

Kannattaa kuitenkin muistaa, ettei venäjässä ole samalla tavalla
käytössä ilmauksia ``iltapäivällä'' ja ``aamupäivällä'' kuin suomessa.
К-prepositio ja datiivi ovat usein avuksi ilmaistaessa vastaavia
ajankohtia, kuten seuraavista kaunokirjallisista käännösesimerkeistä käy
ilmi\footnote{Esimerkit peräisin Tampereen yliopiston
  ParFin-korpuksesta, teoksista Konkka: Hullun taivas ja Lehtolainen:
  Kuparisydän.}:

\begin{enumerate}
\def\labelenumi{(\arabic{enumi})}
\setcounter{enumi}{10}
\tightlist
\item
  Iltapäivällä en enää muista aamun uutisia\\
   Обычно к вечеру я уже забываю утренние известия
\item
  Oli jättänyt lapun, että palaa iltapäivällä, mutta ei sitä ole
  näkynyt.\\
   Оставил записку, что вернется к вечеру, но больше не появлялся.
\end{enumerate}

Toinen ajallinen perusmerkitys on suomen instruktiivia vastaava:

\begin{enumerate}
\def\labelenumi{(\arabic{enumi})}
\setcounter{enumi}{12}
\tightlist
\item
  По понедельникам и четвергам приходила домработница убирать его дом
\end{enumerate}

\textbf{Huom} По-prepositio merkityksessä `johonkin aikarajaan asti se
mukaanlukien' saa viereensä datiivin sijasta \emph{akkusatiivin}:
магазин работает с понедельника по пятницу.

\subsubsection{Datiivi semanttisena
subjektina}\label{datiivi-semanttisena-subjektina}

Datiivi on merkittävä sija venäjässä myös siitä syystä, että sillä on
tiettyjä syntaksin kannalta tärkeitä tehtäviä: ennen muuta kyky toimia
lauseen \emph{semanttisena subjektina} (\emph{субъект} erotuksena
syntaktisesta termistä \emph{подлежащее}). Semanttinen subjekti voi
merkityksen puolesta olla esimerkiksi \emph{kokija} (ks. esim. 14) tai
\emph{agentti} (esim. 15) (kattava katsaus erilaisten lauseiden
erilaisista subjekteista suomessa ja venäjässä: Leinonen 1985: 14--15):

\begin{enumerate}
\def\labelenumi{(\arabic{enumi})}
\setcounter{enumi}{13}
\tightlist
\item
  Мне холодно/весело/жарко/хорошо/неважно.
\item
  Что нам делать?
\item
  Дедушке жаль студентов.
\item
  Мне показалось, что ситуация уже прошла.
\item
  Мне думается, что вы ошибаетесь.
\end{enumerate}

Usein näissä tapauksissa suomen kokija ilmaistaisiin adessiivilla (lla)
tai ehkä hivenen vanhahtavasti genetiivillä: Minulla/minun on
kylmä/jano/nälkä/sääli. On kuitenkin tapauksia, jotka suomenkieliselle
eivät tule yhtä intuitiivisesti, kuten присниться-verbi (`nähdä unta
jostakin'):

\begin{enumerate}
\def\labelenumi{(\arabic{enumi})}
\setcounter{enumi}{18}
\tightlist
\item
  Однажды \emph{ей приснился сон}, что Антон Иванович говорит ей
  какую-то простую обыденную фразу.
\item
  Когда, поужинав, я лёг в постель, \emph{мне приснилось, что я
  киномеханик}.
\end{enumerate}

\subsubsection{Datiivi verbin
täydennyksenä}\label{datiivi-verbin-tuxe4ydennyksenuxe4}

On joukko verbejä, joiden täydennyksen sija on datiivi. Verrattuna
genetiiviä täydennyksekseen ottaviin verbeihin datiiviverbien käyttö on
selkeämpää: vastaavaa horjuntaa akkusatiivin ja datiivin välillä ei ole
kuin akkusatiivin ja genetiivin välillä. Kieltolauseissa
datiivitäydennys \emph{ei} myöskään muutu genetiiviksi -- ei voida sanoa
``он не помогал Анны''.

Šeljakinia (2006: 64) mukaillen voidaan erottella muun muassa seuraavia
datiivin kanssa yhdistyviä verbejä:

\begin{itemize}
\tightlist
\item
  vastaavuutta tai suhdetta johonkin ilmaisevat: завидовать,
  противоречить, ответствовать, соответствовать
\item
  samanaikaista toimintaa: аккомпанировать, подпевать
\item
  kuuluminen / alistuminen / vastustaminen: сдаваться, подчиниться,
  сопротивляться
\end{itemize}

\begin{enumerate}
\def\labelenumi{(\arabic{enumi})}
\setcounter{enumi}{20}
\tightlist
\item
  В первую очередь придётся \emph{подчиниться принципу экологичности}.
\item
  Потом, когда я подросла и стала учиться в музыкальной школе, то
  \emph{аккомпанировала ей} на рояле на наших домашних концертах.
\item
  Власти этих стран долго \emph{сопротивлялись упоминанию} правозащитной
  проблематики.
\end{enumerate}

Näihin voidaan lisätä ainakin ryhmä \emph{edistämistä} tai
\emph{positiivista vaikuttamista johonkin} merkitsevät verbit:
способствовать, содействовать

\begin{enumerate}
\def\labelenumi{(\arabic{enumi})}
\setcounter{enumi}{23}
\tightlist
\item
  ``Содействовать политическому развитию и политической организации
  рабочего класса -- наша главная и основная задача''
\item
  Скорейшее преодоление данного барьера будет способствовать развитию
  бизнеса
\end{enumerate}

Kiinnitä huomiota datiivin rooliin verbeissä запрешать / разрешать
(kieltää / sallia):

\begin{enumerate}
\def\labelenumi{(\arabic{enumi})}
\setcounter{enumi}{25}
\tightlist
\item
  Иногда я и разрешаю себе подобные надежды
\item
  Впоследствии в этот закон были включены поправки, запрещающие
  американцам усыновление российских детей
\end{enumerate}

Kiellon/sallimisen kohteena oleva asia tai esine on siis akkusatiivissa
ja se, jolla oikeus asiaan on tai jolta se viedään, on datiivissa.

\subsubsection{Datiivi adjektiivien
täydennyksenä}\label{datiivi-adjektiivien-tuxe4ydennyksenuxe4}

Paitsi verbit, myös adjektiivit voivat saada datiivimuotoisen
täydennyksen. Huomaa, kuten edellä mainittiin, että tällöin käytetään
yleensä lyhyttä muotoa.

\begin{enumerate}
\def\labelenumi{(\arabic{enumi})}
\setcounter{enumi}{27}
\tightlist
\item
  Такие интеллигенты попросту \emph{вредны государству}
\item
  Твоя душа \emph{подобна Мёртвому морю}
\item
  Гёте писал в ``Фаусте'': ``Ты \emph{равен тому}, кого понимаешь''.
\end{enumerate}

\subsubsection{Lisää merkityksiä}\label{lisuxe4uxe4-merkityksiuxe4}

Seuraavaan on koottu joukko tapauksia, joissa datiivin käyttö ei
välttämättä ole yhtä tuttua kuin edellä mainituissa tapauksissa.

Yksi Šeljakinin (2006: 65) käyttämistä merkitysryhmistä voitaisiin
ilmaista termillä \emph{toiminnan kohteen omistaja}. Näissä tapauksissa
verbillä on akkusatiiviobjekti, mutta tämän lisäksi datiivilla
ilmaistaan tämän objektin haltija tai isompi kokonaisuus, johon se
kuuluu:

\begin{enumerate}
\def\labelenumi{(\arabic{enumi})}
\setcounter{enumi}{30}
\tightlist
\item
  В четвёртый, может быть, раз пожав хозяину руку, гость ступил за порог
\item
  Такое благородство восхитило меня, и я ещё сильней пожал ему руку.
\end{enumerate}

Omistettava asia voi olla abstarktikin, kuten \emph{elämä} esimerkin 33
jälkimmäisessä lauseessa.

\begin{enumerate}
\def\labelenumi{(\arabic{enumi})}
\setcounter{enumi}{32}
\tightlist
\item
  Его \emph{другу} убийца повредил левый бок, но врачи спасли
  \emph{мужчине} жизнь
\end{enumerate}

Huomaa myös konkreettisempi omistettava, ruumiinosa, esimerkin 33
ensimmäisessä lauseessa.

Katsotaan vielä eräitä po-preposition ilmaisemia merkityksiä. Yksi
tavallisimmista on ilmaista jonkun tai jonkin toiminta-alaa tai
määritettävää ominaisuutta: чемпион мира по фигурному катанию,
исследование по лингвистике, отличный по качеству. Toinen tavallinen
merkitys puolestaan ilmaisee \emph{syytä tai perustetta}: судить по
внешности, ценить по уму, уйти по болезни (näistä tarkemmin ks. Шелякин
2006: 65).

По-preposition erityisempiin merkityksiin kuuluu niin kutsuttu
\emph{distributiivinen} (распределительный) merkitys, jossa jokaiselle
antamisen kohteelle annetaan yksi kappale annettavaa esinettä:

\begin{enumerate}
\def\labelenumi{(\arabic{enumi})}
\setcounter{enumi}{33}
\tightlist
\item
  А потом в гости пришли друзья, и мама посадила всех за стол и дала
  каждому \emph{по куску} очень красивого и вкусного пирога с вишнями.
\item
  Ритуал «посвящения» первокурсников. Дипломники раздают каждому
  \emph{по гвоздике}.
\end{enumerate}

\hyperdef{}{ref-mlein}{\label{ref-mlein}}
Leinonen, Marja. 1985. \emph{Impersonal Sentences in Finnish and
Russian}. Slavica Helsingiensia 3. Hki: University of Helsinki.

\hyperdef{}{ref-sheljakin}{\label{ref-sheljakin}}
Шелякин, М.А. 2006. \emph{Справочник По Русской Грамматике}. drofa.

\end{document}
