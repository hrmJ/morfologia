\documentclass[]{scrartcl}
\usepackage[utf8]{inputenc}
\usepackage[T1]{fontenc}
\usepackage[T2A]{fontenc}
\usepackage[finnish]{babel}
\usepackage{linguex} 
\usepackage{longtable} 
\usepackage{booktabs}
\usepackage{amsthm}
\usepackage{graphicx}
\newtheorem{maar}{Määritelmä}
\usepackage{fixltx2e} % provides \textsubscript
\usepackage{textcomp} % provides \textsubscript
\usepackage{hyperref}
\usepackage{xcolor}

\providecommand{\tightlist}{%
  \setlength{\itemsep}{0pt}\setlength{\parskip}{0pt}}

\hypersetup{
    colorlinks,
    linkcolor={red!50!black},
    citecolor={blue!50!black},
    urlcolor={blue!80!black}
}
\author{Juho Härme}
\title{Morfologia-kurssin luentomateriaaleja}
\date{\today}
\begin{document}
\maketitle
\tableofcontents
\newpage



\section{Luento 17: Aspekti kieliopillisena kategoriana
3}\label{luento-17-aspekti-kieliopillisena-kategoriana-3}

Kielenoppijan kannalta kiinnostavin ja tärkein kysymys aspektiin
liittyen on, milloin kumpaakin aspektia käytetään ja mistä voisi löytää
edes jonkinlaisia ankkuripisteitä, joiden perusteella jommankumman
aspektin valintaa olisi mahdollista perustella. Yksi mahdollinen tapa
vastata tähän tarpeeseen on pyrkiä ryhmittelemään
aspektinkäyttötilanteita mahdollisimman tarkasti. Sen sijaan, että
pyrittäisiin antamaan yksi universaali vastaus -- ``käytä
imperfektiivistä aspektia aina kun haluat kuvata toiminnan
prosessuaalisena'' tms. -- voidaan yrittää eritellä, minkälaisia
aspektinvalintatilanteita liittyy esimerkiksi menneeseen/tulevaan
aikamuotoon, kieltomuotoon tai infinitiiveihin\footnote{Hyvä teos tässä
  suhteessa -- ja lähtökohta tämän luennon pohdinnoille -- on (Rassudova
  1984).}.

\subsection{Aspektin valinnasta menneessä
ajassa}\label{aspektin-valinnasta-menneessuxe4-ajassa}

Kun ajatellaan aspektin valintaa menneessä ajassa, voidaan eritellä
neljä tärkeää oppositiota, neljä merkitysparia, joiden välillä valinta
on usein suoritettava:

\begin{enumerate}
\def\labelenumi{\arabic{enumi}.}
\tightlist
\item
  Konkreettis--faktinen merkitys ja konkreettis-prosessuaalinen merkitys
\item
  Konkreettis--faktinen merkitys ja rajoittamattoman toiston merkitys
\item
  Summatiivinen merkitys ja rajoittamattoman toiston merkitys
\item
  Konkreettis--faktinen merkitys ja yleisesti toteava merkitys
\item
  Perfektin merkitys ja yleisesti toteava merkitys
\end{enumerate}

\subsubsection{Konkreettis-faktinen/-prosessuaalinen merkitys sekä
toisto}\label{konkreettis-faktinen-prosessuaalinen-merkitys-sekuxe4-toisto}

Laajin näistä merkityspareista on luonnollisesti ensimmäinen, koska kyse
on perusmerkityksistä ja siten aspektien välisestä peruserosta.
Pohditaan muutamaa yleistä huomiota tästä oppositiosta.

Ensinnäkin, KPM- ja KFM-merkitysten välinen ero saattaa näkyä hyvin
konkreettisesti siinä, mihin ajalliseen hetkeen puhujan huomio
varsinaisesti kohdistuu. Mieti vaikkapa seuraavien lauseiden
merkityseroa:

\begin{enumerate}
\def\labelenumi{(\arabic{enumi})}
\tightlist
\item
  Самолет возвращался на аэродром.
\item
  Самолет возвратился на аэродром.
\item
  Билеты уже продавали.
\item
  Билеты уже продали
\end{enumerate}

Lauseiden 1 ja 2 sekä 3 ja 4 välillä on selvä ero siinä, mihin hetkeen
puhujan huomio kohdistuu. Lauseessa 1 huomio on hetkessä, jolloin
lentokone on vielä ilmassa, mutta seuraavassa lauseessa huomio on
laskeutumisen jälkeisessä ajassa. Vastaavasti lauseessa 3 huomio on
myyntihetkessä, lauseessa 4 myynnin jälkeisessä hetkessä. KPM- ja
KFM-merkitysten välinen ero tulee erityisen hyvin ilmi, kun ne
esiintyvät aikaa ilmaisevassa sivulauseessa. Esimerkissä 5 kyse on
selkeästi yhdestä konkreettisesta prosessista, esimerkissä 6 yhdestä
rajatusta tapahtumasta:

\begin{enumerate}
\def\labelenumi{(\arabic{enumi})}
\setcounter{enumi}{4}
\tightlist
\item
  Студенты волновалиь, когда выступали c докладом
\item
  Когда студенты выступили, они neрестали волноваться
\end{enumerate}

Pidetään huomio edelleen perfektiivisen aspektin konkreettis-faktisessa
merkityksessä, mutta siirrytään imperfektiivisen aspektin osalta
tarkastelemaan rajoittamattoman toiston merkitystä. Näissä tapauksissa
kysymys siis kuuluu, onko puhe yhdestä yksittäisestä ja rajatusta
tapahtumasta vai toistuvasta toiminnasta. Rassudova (1984) esittää tähän
liittyen seuraavat esimerkit:

\begin{enumerate}
\def\labelenumi{(\arabic{enumi})}
\setcounter{enumi}{6}
\tightlist
\item
  К вечеру больному стало хуже
\item
  К вечеру больному становилось хуже
\item
  К десяти часам утра, не выдержав неслыханной нагрузки, портились все
  телефоны.
\item
  К десяти часам утра, не выдержав неслыханной нагрузки, испортились все
  телефоны.
\end{enumerate}

Perfektiivinen aspekti on vastaavissa tilanteissa syytä valita, kun ei
haluta luoda ajatusta toistuvuudesta: esimerkeissä 8 ja 9 kerrotaan
useista tapahtumista (aina iltaisin kävi niin, että sairaan vointi
huononi / joka kerta kello kymmeneen mennässä tapahtui niin, että
puhelimet menivät epäkuntoon). Vastaavasti esimerkit 7 ja 10 viittaavat
yhteen konkreettiseen faktaan.

Toistoon liittyen oma erityistapauksensa ovat tilanteet, joissa
kerrotaan, \emph{missä ajassa jokin toiminto suoritetaan}. Huomaatko
eroa seuraavissa lauseissa?

\begin{enumerate}
\def\labelenumi{(\arabic{enumi})}
\setcounter{enumi}{10}
\tightlist
\item
  Он читал за два часа 30 страниц
\item
  Он прочитал за два часа 30 страниц
\end{enumerate}

Lause 11 on varmasti harvinaisempi, mutta mahdollinen, ja ilmaisee, että
joskus -- nuoruudessaan, kouluaikoina tms. -- henkilön lukunopeus oli 30
sivua kahdessa tunnissa. Jälleen kerran, lause 12 tuottaa tulkinnan
yhdestä konkreettisesta prosessista.

Kuten varmasti muistat, myös perfektiivinen aspekti voi kuitenkin
ilmaista toistoa. Tästä seuraa, että rajoitetun toiston merkitys
(imperfektiivinen aspekti) ja summatiivinen merkitys (perfektiivinen
aspekti) kilpailevat jossain määrin keskenään. Näissä tilanteissa
auttaa, kun ajattelee, että perfektiivisen aspektin tapauksessa kyse on
luultavasti peräkkäisistä toistoista, kun taas imperfektiivinen aspekti
ei määrittele toistojen välistä aikaa vaan ennemmin toteaa vain niiden
olemassaolon. Esimerkissä 13 verbi сморгнул (`räpäyttää silmiä')
selkeästi kuvaa toinen toistaan seuraavia räpäytyksiä, esimerkissä 14
puolestaan \emph{много раз рассматривали} vain toteaa, että toistoja
todella on ollut useita eikä vain yksi.

\begin{enumerate}
\def\labelenumi{(\arabic{enumi})}
\setcounter{enumi}{12}
\tightlist
\item
  Стало жарко глазам, я несколько раз сморгнул, потом без раздумий налил
  половину фужера и тут же выпил.
\item
  Мы много раз рассматривали на заседании Президиума РАН вопросы,
  связанные с ракетной и авиационной техникой, но в первый раз в
  повестку нашего заседания поставлена тема корабельной ядерной
  энергетики
\end{enumerate}

\subsubsection{Yleisesti toteavaan erityismerkitykseen liittyvä
aspektinvalinta}\label{yleisesti-toteavaan-erityismerkitykseen-liittyvuxe4-aspektinvalinta}

Tarkastellaan vielä imperfektiivisen aspektin yleisesti toteavaa
erityismerkitystä. Sen käyttö on lähellä toisaalta perfektiivisen
aspektin konkreettis-faktista merkitystä, toisalta perfektin merkitystä.
Erona ensimmäisessä oppositiossa on, että imperfektiivisen aspektin
tapauksessa otetaan vähemmän kantaa, ollaan toteavia, kun taas
perfektiivisen aspektin tapauksessa on kertovampi, selostavampi ote.
Katsotaan taas Rassudovan (1984: 55) esimerkkejä:

\begin{enumerate}
\def\labelenumi{(\arabic{enumi})}
\setcounter{enumi}{14}
\tightlist
\item
  1-го Мая в Доме Дружбы собрались представители общественности
\item
  Собирались представители общественности 1-го Мая?
\end{enumerate}

Perfektiivinen aspekti selostaa tässä tapahtumia ja tuo mukanaan
kertovan, värikkäämmän vaikutelman. Imperfektiivinen aspekti puolestaan
keskittää huomion siihen, voidaanko todeta, että toiminta on tapahtunut
vai ei.

\begin{enumerate}
\def\labelenumi{(\arabic{enumi})}
\setcounter{enumi}{16}
\tightlist
\item
  Мне кажется, что я где-то уже видел Вас
\item
  Я Вас, кстати, увидел вчера в столовой.
\end{enumerate}

Esimerkit 17 ja 18 kuvaavat omalta osaltaan imperfektiivisen aspektin
värittömyyttä ja kantaaottamattomuutta ja toisaalta perfektiivisen
aspektin konkreettisuutta ja tarkkuutta.

Kuten mainittu, yleisesti toteava merkitys voi sekoittua myös perfektin
merkitykseen. Perfektin merkityksessähän kyse oli siitä, että
korostetaan jonkin tuloksen ajankohtaisuutta puhehetkellä. Selkeimmin
tämä oppositio tulee esille, kun puhutaan toiminnan mitätöitymisestä:

\begin{enumerate}
\def\labelenumi{(\arabic{enumi})}
\setcounter{enumi}{18}
\tightlist
\item
  Пока меня не было, кто-то включил компьютер/взял фломастер/открыл окно
\item
  Пока меня не было, кто-то явно включал компьютер/брал
  фломастер/открывал окно
\end{enumerate}

Seraava Rassudovan (1984: 65) esimerkkidialogi valaisee hyvin ajatusta
tuloksen näkymisestä puhehetkellä vs.~asian toteamista:

\begin{enumerate}
\def\labelenumi{(\arabic{enumi})}
\setcounter{enumi}{20}
\tightlist
\item
  -- Добрый день. Помните? Мы, кажется, встречались\ldots{}
\item
  -- Да, вот вспомнил. Ты изменился!
\end{enumerate}

Esimerkissä 21 on yleisesti toteava verbi встречаться, kun taas
esimerkissä 22 kyse on perfektin merkityksestä (`oletpa muuttunut').

\subsection{Aspektin valinnasta tulevassa
ajassa}\label{aspektin-valinnasta-tulevassa-ajassa}

Aspektin valintaan menneisyydessä liittyy usein eniten vaikeuksia, minkä
takia sille myös omistettiin edellä melko paljon huomiota. Pohditaan
kuitenkin myös eräitä huomioita futuurista.

Kuten edellisillä luennoilla selitettiin, venäjästä voidaan
\emph{merkityksen} perusteella erottaa kaksi futuurimuotoa,
imperfektiivinen ja perfektiivinen. Imperfektiiviselle aspektille ovat
tyypillisiä seuraavat piirteet:

\begin{itemize}
\tightlist
\item
  Prosessuaalisuus
\item
  Toistoa
\item
  Aikomusta
\item
  Epävarmuutta
\end{itemize}

Kaksi ensimmäistä lueteltua merkitystä liittyvät tietysti
imperfektiivisen aspektin yleisiin ominaisuuksiin, joista on jo paljon
puhuttu. Ajattele varbiä заниматься seuraavassa lauseessa:

\begin{enumerate}
\def\labelenumi{(\arabic{enumi})}
\setcounter{enumi}{22}
\tightlist
\item
  Я завтра не смогу поехать с вами на стадион, буду заниматься.
\end{enumerate}

On selvää, että kyseessä on konkreettis-prosessuaalinen merkitys. Sen
sijaan esimerkissä 24 luetellaan rajattuja, ei-prosessuaalisia,
yksittäisiä tapahtumia:

\begin{enumerate}
\def\labelenumi{(\arabic{enumi})}
\setcounter{enumi}{23}
\tightlist
\item
  Я поднимусь, помою посуду и пойду помогать маме.
\end{enumerate}

Voidaan siis sanoa, ettei aspektien välinen perusero häviä futuurissa
minnekään. Peruseron lisäksi on kuitenkin lisäsävyjä, jotka on hyvä
muistaa. Katsotaan ensin kahta jälkimmäistä edellä listattua
imperfektiivisen aspektin ominaisuutta. Ajattele esimerkkejä 26 ja 27
(nämäkin Rassudovalta).

\begin{enumerate}
\def\labelenumi{(\arabic{enumi})}
\setcounter{enumi}{24}
\tightlist
\item
  Ты будешь есть?
\item
  Если Дима будет уходить, позвоните мне
\item
  Если Дима уйдет, позвоните мне.
\end{enumerate}

Esimerkissä 25 aikomuksen merkitys tulee esiin kysymyksessä:
\emph{Meinaatko/onko sinulla aikomusta syödä?} -- tämä tuo esiin puhujan
epävarmuuden. Esimerkissä 26 soittaa pitää silloin, jos Dima ryhtyy
tekemään lähtöä, ilmaisee aikomuksen. Esimerkki 27 puolestaan ei kerro
aikomuksesta, vaan soittaa pitää silloin, jos Dima on jo lähtenyt.
Ylipäätään voidaan sanoa, että siinä missä imperfektiivinen aspekti
ilmaisee epävarmuutta ja aikomusta, ilmaisee perfekti varmuutta ja
päättäväisyyttä, kuten seuraavassa esimerkissä:

\begin{enumerate}
\def\labelenumi{(\arabic{enumi})}
\setcounter{enumi}{27}
\tightlist
\item
  Не волнуйся, ты справишься с этим курсом!
\end{enumerate}

Imperfektiivisen ja perfektiivisen aspektin välinen ero varmuudessa
voidaan melko hyvin tiiviistää seuraavaan esimerkkiin:

\begin{enumerate}
\def\labelenumi{(\arabic{enumi})}
\setcounter{enumi}{28}
\tightlist
\item
  Делать я, пожалуй, буду, а сделаю, не знаю.
\end{enumerate}

Perfektiiviselle aspektille tyypillisenä futuurimerkityksenä voi myös
pitää esimerkin 31 viimeistä verbiä (сейчас уберу):

\begin{enumerate}
\def\labelenumi{(\arabic{enumi})}
\setcounter{enumi}{29}
\tightlist
\item
  Я еще не убрал бумаги, уже убираю, сейчас \emph{уберу}
\end{enumerate}

Ajatus on, että perfektiivinen aspekti ilmaisee \emph{lupauksen
suorittaa toiminta loppuun}. Vertaa myös seuraava esimerkki:

\begin{enumerate}
\def\labelenumi{(\arabic{enumi})}
\setcounter{enumi}{30}
\tightlist
\item
  Что ты так долго одеваешься?\\
   -- Сейчас \emph{оденусь}
\end{enumerate}

\subsection{Aspektin valinnasta kieltomuotojen
yhteydessä}\label{aspektin-valinnasta-kieltomuotojen-yhteydessuxe4}

Aspektinvalinnan kannalta on paljon merkitystä sillä, onko kyseessä
kielto- vai myöntölause. Katsotaan aluksi joitakin esimerkkejä liittyen
aspektin käyttöön kieltomuodossa ja menneessä ajassa. Yleisesti ottaen
voidaan todeta, että \emph{kieltomuodossa imperfektiivisen aspektin
käyttöala laajenee}, ei kuitenkaan, että kieltomuoto automaattisesti
merkitsisi imperfektiivistä aspektia.

Imperfektiivisellä aspektilla on usein se merkitys, ettei toiminta ole
vielä edes alkanut:

\begin{enumerate}
\def\labelenumi{(\arabic{enumi})}
\setcounter{enumi}{31}
\tightlist
\item
  Уже стемнело, но свечи ещё не \emph{горели}.
\item
  Былеты еще не продавали
\item
  В 40-ые годы кварки не изучали.
\end{enumerate}

Toinen tyypillinen imperfektiivisen kieltomuodon merkitys on, että
toiminnassa ei tapahdu muutoksia:

\begin{enumerate}
\def\labelenumi{(\arabic{enumi})}
\setcounter{enumi}{34}
\tightlist
\item
  Дождь не переставал.
\item
  Компьютер не включался.
\item
  Ученые долго не находили ответа на вопрос о том, почему значение
  гравитационной постоянной составвляет именно 6,67408(31)·10--11
  м3·с--2·кг--1
\end{enumerate}

Esimerkissä 35 \emph{sataminen} on tila, jossa ei tapahdu muutoksia,
esimerkissä 36 puolestaan vastaava tila on tietokoneen
käynnistymättömyys (sammuksissa oleminen). Esimerkki 37 on merkittävä
siinä mielessä, että \emph{находить} on verbi, joka yleensä ei ilmaise
jatkuvaa tilaa tai kestoa, mutta joka kuitenkin kieltolauseessa näin
tekee.

Edellä käsiteltiin aspektien eroa sellaisissa lauseissa kuin ``Ты читал
книгу?'' ja ``Ты прочитал книгу?'' ja todettiin, että perfektiivinen
aspekti ottaa kantaan siihen, oliko toiminta odotettua vai ei. Jonkin
toiminnan odottamisen tai edellyttämisen merkitys näkyy myös
kieltolauseissa, kuten seuraavissa esimerkeissä:

\begin{enumerate}
\def\labelenumi{(\arabic{enumi})}
\setcounter{enumi}{37}
\tightlist
\item
  Ваня, мы по тебе скучали, что же это ты не \emph{пришел}?
\item
  Кстати, на заседание сенатского комитета представители Минпечати не
  \emph{пришли}, хотя приглашение им было направлено.
\item
  Мы не посмотрели фильм, хотя все его хвалили
\end{enumerate}

Samoin perfektin erityismerkitys voi olla syy puoltaa perfektiivisen
aspektin käyttöä. Esimerkissä 41 принести-verbillä on selvä yhteys
nykyhetkeen:

\begin{enumerate}
\def\labelenumi{(\arabic{enumi})}
\setcounter{enumi}{40}
\tightlist
\item
  -- У тебя есть со собой ноутбук?\\
   -- Нет, я его не \emph{принес}
\end{enumerate}

Etenkin verrattaessa konkreettis-faktista ja yleisesti toteavaa
erityismerkitystä aspektin valinta kieltomuodon suhteen on usein
häilyvää ja muotojen väliltä on vaikea löytää suuria merkityseroja. Näin
on esimerkiksi seuraavissa lauseissa:

\begin{enumerate}
\def\labelenumi{(\arabic{enumi})}
\setcounter{enumi}{41}
\tightlist
\item
  Меня никто не встретил/ встречал.
\item
  Автобус не останавливался / остановился?
\item
  Он не сообщил / сообщал о своем приезде.
\item
  Сегодня я не купила / покупала фруктов.
\end{enumerate}

Siirrytään nyt tarkastelemaan kieltomuotoa ja infinitiivejä. Sen
lisäksi, että kaiken kaikkiaan kieltomuoto kasvattaa imperfektiivisen
aspektin käyttöalaa, voidaan kieltomuoto + infinitiivi -rakenteiden
osalta todeta, että näissä tapauksissa imperfektiivinen aspekti on aivan
erityisen todennäköinen, joskaan ei edelleenkään millään tavalla ainoa
vaihtoehto.

Periaatteessa myös infinitiivimuotojen kohdalla on mietittävä edellä
mainittuja imperfektiivisen ja perfektiivisen aspektin peruseroja.
Esimerkiksi seuraavissa lauseissa perfektiivinen verbi ilmaisee
yksittäistä tulokertaa, imperfektiivinen toistuvia kertoja:

\begin{enumerate}
\def\labelenumi{(\arabic{enumi})}
\setcounter{enumi}{45}
\tightlist
\item
  Он не согласился \emph{прийти} в понедельник
\item
  Он не согласился \emph{приходить} в понедельник
\end{enumerate}

On kuitenkin olemassa tiettyjä kielteisiä infinitiivirakenteita, joissa
aspektin valinta on hyvin selkeää.

Ensinnäkin, jos kiellon kohteena on itse infinitiivimuoto eikä sen
pääsana, imperfektiivisen aspektin käyttö on lähes varmaa:

\begin{enumerate}
\def\labelenumi{(\arabic{enumi})}
\setcounter{enumi}{47}
\tightlist
\item
  Я старался не \emph{употреблять} слов, которые не были бы понятны
  второкласснику
\item
  Агент обязуется не \emph{заключать} аналогичные соглашения с другими
  платёжными системами
\item
  А насчёт всего здания мы предпочитаем не \emph{загадывать}
\end{enumerate}

Toiseksi, kieltomuodon ja pakkoa tai tarvetta ilmaisevien \emph{надо},
\emph{нужно}, \emph{нельзя}, \emph{должен} rakenteiden yhdistelmään
liittyy säännönmukaisesti juuri imperfektiivisen aspektin verbi:

\begin{enumerate}
\def\labelenumi{(\arabic{enumi})}
\setcounter{enumi}{50}
\tightlist
\item
  Не надо \emph{забывать}, что нефть -- невоспроизводимый ресурс.
\item
  Больше не нужно \emph{откладывать} покупку на потом!
\item
  Мы твердо убеждены, что клиент не должен \emph{переплачивать} за
  раскрученный бренд
\end{enumerate}

Vertaa tässä yhteydessä myös зачем + infinitiivi -rakenteet:

\begin{enumerate}
\def\labelenumi{(\arabic{enumi})}
\setcounter{enumi}{53}
\tightlist
\item
  Зачем \emph{писать} много о том, чьи взгляды тебе чужды?
\end{enumerate}

Huomaa kuitenkin, että нельзя-sanan yhteydessä perfektiivistä aspektia
voi käyttää, kun ilmaistaan \emph{mahdottomuutta} (vrt. potentiaalin
erityismerkitys):

\begin{enumerate}
\def\labelenumi{(\arabic{enumi})}
\setcounter{enumi}{54}
\tightlist
\item
  Нельзя \emph{сказать}, что в пьесе Ибсена только одна тема.
\item
  На муниципальном уровне нельзя \emph{решить} межгосударственные
  вопросы.
\end{enumerate}

\subsection{Aspektin valinnasta myöntömuotoisten infinitiivimuotojen
yhteydessä}\label{aspektin-valinnasta-myuxf6ntuxf6muotoisten-infinitiivimuotojen-yhteydessuxe4}

Paitsi kielto- myös myöntömuotoiseen infinitiiviin liittyy eräitä hyvin
selkeitä säännönmukaisuuksia. Yksi tavallisimmista tapauksista liittyy
niin kutsuttuihin \emph{vaihetta ilmaiseviin verbeihin}. Näitä ovat
esimerkiksi начинать, стать, кончать, продолжать. Näiden verbien
yhteydessä kohdataan jälleen kerran se imperfektiivisen aspektin
ominaisuus, että puhujan on mahdollista astua sisälle verbin ilmaisemaan
toimintaan, pilkkoa se palasiksi (Nikunlassi 2002: 176). Perfektiivisen
aspektin toiminta, joka on luonteeltaan rajattua ja totaalista, ei tätä
salli, ja siksi vaihetta ilmaisevien verbien kanssa käytetään aina
imperfektiivisen aspektin verbejä. Varsinaisten vaihetta ilmaisevian
verbien lisäksi ryhmään voi luetella monia muitakin verbejä, jotka
viittaavat esimerkiksi vaiheittaiseen oppimiseen, omaksumiseen tms.
Toisin sanoen sellaisia verbejä kuin учиться, привыкнуть, бросить
(merkityksessä `lopettaa') yms.

\begin{enumerate}
\def\labelenumi{(\arabic{enumi})}
\setcounter{enumi}{56}
\tightlist
\item
  Ситуация там \emph{продолжает оставаться} достаточно сложной\ldots{}
\item
  Бабушка сейчас ложится спать и требует, чтобы я \emph{кончала писать}
  и тоже легла
\end{enumerate}

Toisaalta tietyt verbit ovat luonteeltaan niin selkeästi tulosta
painottavia, että niiden yhteydessä käytetään käytännössä aina
perfektiivistä aspektia. Tällaisia ovat ainakin \emph{успеть} ja
\emph{удаться}:

\begin{enumerate}
\def\labelenumi{(\arabic{enumi})}
\setcounter{enumi}{58}
\tightlist
\item
  Очень рассчитываем на то, что Президенту X. Карзаю \emph{удастся
  укрепить} центральное правительство
\item
  -- Так вы уже \emph{успели прочитать} эти странички?
\end{enumerate}

Myös забыть-verbin täydennys on käytännössä aina perfektiivinen verbi:

\begin{enumerate}
\def\labelenumi{(\arabic{enumi})}
\setcounter{enumi}{60}
\tightlist
\item
  Мы забыли \emph{купить} закуски и подарки, и кинулись исправлять свой
  промах
\end{enumerate}

Mainittakoon myös, että давать-verbi + infinitiivi -rakenteissa
käytetään imperfektiivistä aspektia:

\begin{enumerate}
\def\labelenumi{(\arabic{enumi})}
\setcounter{enumi}{61}
\tightlist
\item
  Давайте придерживаться темы!
\end{enumerate}

Jos käytetään perfektiivistä aspektia, verbi esiintyy infinitiivin
sijasta persoonamuodossa:

\begin{enumerate}
\def\labelenumi{(\arabic{enumi})}
\setcounter{enumi}{62}
\tightlist
\item
  Давайте придержимся темы!
\end{enumerate}

Edellä mainittujen jossain määrin kiinteiden rakenteiden ulkopuolella
infinitiivillä ja imperfektiivisella aspektilla on taipumus useimmiten
ilmaista jotakin toistuvaa tai jollakulla tapana olevaa:

\begin{enumerate}
\def\labelenumi{(\arabic{enumi})}
\setcounter{enumi}{63}
\tightlist
\item
  Я решил \emph{делать} мемуарные записи об артистах, режиссёрах,
  писателях, с которыми мне повезло работать и дружить.
\item
  Учительница настоятельно советовала Наcте \emph{обращать} больше
  внимания на подопечного.
\item
  Везде принято \emph{давать} чаевые -- 10\% от общей суммы
\item
  И конечно, не забывайте почаще \emph{менять} пароль.
\end{enumerate}

Toisin sanoen, jos infinitiivillä ilmaistaan yksittäistä tapahtumaa,
perfektiivinen aspekti on hyvin todennäköinen:

\begin{enumerate}
\def\labelenumi{(\arabic{enumi})}
\setcounter{enumi}{67}
\tightlist
\item
  В воскресенье мы собираемся \emph{сходить} на выставку
\item
  Мне надо \emph{поговорить} с ним
\item
  Я хочу \emph{прочитать} эту книгу
\end{enumerate}

\subsection{Teonlaadun käsite}\label{teonlaadun-kuxe4site}

Yksi aspektinvalintaan vaikuttava tekijä ovat niin kutsutut
\emph{teonlaadut} (способы действия)\footnote{Isachenko
  (2003{[}1965{]}), jolla on ollut erityisen suuri vaikutus teonlaatujen
  tutkimukseen, käyttää niistä termiä \emph{совершаемость}.}, joita
tässä yhteydessä sivutaan vain hyvin ohimennen. Teonlaatu on tiettyjen
konkreettisten sanojen tai sanaryhmien ominaisuus -- vertaa tätä
aspektiin yleensä, joka on \emph{poikkeuksetta kaikkien verbien
ilmaisema kategoria}.

Jotta asia kävisi ymmärrettäväksi, katsotaan konkreettisia esimerkkejä.
Pohdi verbejä засмеяться, заплакать, зарыдать, запеть ja зашуметь. Nämä
ovat kaikki perfektiivisen aspektin verbejä, mutta niitä yhdistää jokin
tarkempi ominaisuus kuin vain ne, jotka ovat tyypillisiä
perfektiivisille verbeille. Kaikki nämä за-etuliitteiset verbit
nimittäin ilmaisevat nimenomaan toiminnan \emph{alkamista}:

\begin{enumerate}
\def\labelenumi{(\arabic{enumi})}
\setcounter{enumi}{70}
\tightlist
\item
  Когда за вдовцом хлопнула дверь, Ирина \emph{заплакала}.
\item
  Как будто я должен тут просто \emph{запеть} от радости.
\end{enumerate}

Tämä ei ole perfektiivisen aspektin ominaisuus sinänsä, eikä näillä
verbeillä ole aspektiparia. Niiden käyttö ei perustu niinkään yleisiin
syihin käyttää perfektiivistä aspektia (muuta kuin epäsuorasti), vaan
siihen, että puhuja haluaa ilmaista nimenomaan toiminnan -- yleensä
jonkin äänen tai muun aistein havaittavan ärsykkeen lähettämisen --
alkamista. Alkamista ilmaisevaa teonlaatua nimitetään
\emph{inkoatiiviseksi} (начинательный способ действия).

Useimmiten teonlaadut, kuten edellisessä tapauksessa, muodostetaan aina
tietyllä prefiksillä, ja muodostamisen tuloksena on aina perfektiivisiä
verbejä. Inkoatiivisen teonlaadun lisäksi mainittakoon tässä yhteydessä
seuraavat:

\begin{enumerate}
\def\labelenumi{\arabic{enumi}.}
\tightlist
\item
  Perduratiivinen (продолжительный) teonlaatu

  \begin{itemize}
  \tightlist
  \item
    ilmaisee tekemistä jonkin tietyn konkreettisen ajan verran
  \item
    muodostetaan про-prefiksillä
  \item
    verbejä: просидеть, промолчать, продержать, пролежать, простоять
  \item
    esimerkkilause: \emph{Как мне потом сказали, проспал я 14 часов}
  \end{itemize}
\item
  Delimitatiivinen (ограничительный) teonlaatu

  \begin{itemize}
  \tightlist
  \item
    ilmaisee myös tekemistä jonkin rajoitetun ajan, mutta yleensä vain
    vähän aikaa
  \item
    muodostetaan yleensä по-prefiksillä, usein intransitiiviverbeistä
  \item
    verbejä: посидеть, постоять, покурить, поговорить
  \item
    usein suomennettaessa lisätään esimerkiksi adverbi ``vähän'' tms.
    (istuttiin vähäsen, vähän aikaa tupakoitiin, puhutaan pikkasen)
  \item
    esimerkkilause: *Большинство спектаклей для детей повсеместно
    сделаны по железным рецептам" старым казачьим способом ``:
    \emph{поговорили}, \emph{попели}, потанцевали.*
  \end{itemize}
\item
  Momentaaninen (одноактный) teonlaatu

  \begin{itemize}
  \tightlist
  \item
    ilmaisee jotakin yksittäistä tapahtumaa (heilautusta, yskähdystä,
    hypähdystä) erotettuna mohdollisesta useiden tapahtumien sarjasta
  \item
    ei muodosteta prefiksillä vaan kyseessä ну-suffiksin sisältäviä
    verbejä
  \item
    verbejä: махнуть, прыгнуть, кашлянуть
  \item
    esimerkkilause: \emph{Он кашлянул, чертыхнулся (`kirosi'), откинул
    крючок и предстал перед ними.}
  \end{itemize}
\item
  Saturatiivinen (сатуративный) teonlaatu

  \begin{itemize}
  \tightlist
  \item
    ilmaisee jonkin asian tekemistä kyllikseen tai jopa kyllästymiseen
    asti
  \item
    muodostetaan на-prefiksillä sekä ся-postfiksilla
  \item
    verbejä: наесться, насмеяться, накупаться, нахвастаться
  \item
    esimerkkilause: Довольно \emph{нашутились} мы, пора и за дело!
  \end{itemize}
\end{enumerate}

\hyperdef{}{ref-nikunl}{\label{ref-nikunl}}
Nikunlassi, Ahti. 2002. \emph{Johdatus Venäjän Kieleen Ja Sen
Tutkimukseen}. Helsinki: Finn Lectura.

\hyperdef{}{ref-rassudova1984}{\label{ref-rassudova1984}}
Rassudova, O.P. 1984. \emph{Aspectual Usage in Modern Russian}. Moskova:
Russky Yazyk.

\hyperdef{}{ref-isachenko2003}{\label{ref-isachenko2003}}
Исаченко, Александр. 2003{[}1965{]}. \emph{Грамматический Строй Русского
Языка В Сопоставлении С Словацким. Морфология. Часть 1, 2}. Москва:
Языки славянской культуры.
\url{https://www.litres.ru/a-v-isachenko/grammaticheskiy-stroy-russkogo-yazyka-v-sopostavlenii-s-slovackim-morfologiya-chast-1-2/}.

\end{document}
