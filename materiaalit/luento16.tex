\documentclass[]{scrartcl}
\usepackage[utf8]{inputenc}
\usepackage[T1]{fontenc}
\usepackage[T2A]{fontenc}
\usepackage[finnish]{babel}
\usepackage{linguex} 
\usepackage{longtable} 
\usepackage{booktabs}
\usepackage{amsthm}
\usepackage{graphicx}
\newtheorem{maar}{Määritelmä}
\usepackage{fixltx2e} % provides \textsubscript
\usepackage{textcomp} % provides \textsubscript
\usepackage{hyperref}
\usepackage{xcolor}

\providecommand{\tightlist}{%
  \setlength{\itemsep}{0pt}\setlength{\parskip}{0pt}}

\hypersetup{
    colorlinks,
    linkcolor={red!50!black},
    citecolor={blue!50!black},
    urlcolor={blue!80!black}
}
\author{Juho Härme}
\title{Morfologia-kurssin luentomateriaaleja}
\date{\today}
\begin{document}
\maketitle
\tableofcontents
\newpage



\section{Luento 16: Aspekti kieliopillisena kategoriana
2}\label{luento-16-aspekti-kieliopillisena-kategoriana-2}

\begin{itemize}
\tightlist
\item
  \href{https://mustikka.uta.fi/~juho_harme/morfologia/\#tästä-kurssista}{Takaisin
  sivun ylälaitaan}
\item
  \href{http://mustikka.uta.fi/~juho_harme/morfologia/materiaalit/luento16.pdf}{Lataa
  PDF}
\item
  \href{http://mustikka.uta.fi/~juho_harme/morfologia/presentations/luento16.html}{Tutki
  luentokalvoja}
\item
  \href{http://mustikka.uta.fi/~juho_harme/morfologia/tehtavat/luento16.pdf}{Tutki
  tuntitehtäviä}
\end{itemize}

Edellisen luennon lopussa esiteltiin ajatus aspektuaalisten merkitysten
jakamisesta perus- ja erityismerkityksiin (общие и частные видовые
значения). Perusmerkityksillä tarkoitetaan sellaista imperfektiivisen
tai perfektiivisen aspektin käyttöä, jossa kummallekin aspektille
tyypilliset ominaisuudet tulevat kaikkein selvimmin esiin (Nikunlassi
2002: 178). Toisin sanoen perusmerkityksissä aspekti käyttäytyy
odotetulla, ikään kuin säännönmukaisimmalla mahdollisella tavalla.

\subsection{Perusmerkitykset}\label{perusmerkitykset}

Imperfektiivisen aspektin perusmerkityksenä pidetään niin kutsuttua
\emph{konkreettis-prosessuaalista} merkitystä (конкретно-процессное
значение). Konkreettis-prosessuaalinen merkitys (jatkossa lyhyemmin KPM)
viittaa nimensä mukaisesti tapauksiin, jossa imperfektiivistä aspektia
käytetään viittaamaan yhteen konkreettiseen tapahtumaan, jota kuitenkin
kuvataan ei sisäisesti rajattuna vaan avoimena, prosessuaalisena. Pohdi
esimerkin 1 jälkimmäistä lausetta:

\begin{enumerate}
\def\labelenumi{(\arabic{enumi})}
\tightlist
\item
  Когда мама пришла домой, я играл на фортепиано
\item
  Сижу дома на кухне и занимаюсь китайским.
\item
  Он думал, думал, думал, и чувствовал, что голова у него раздувается
  точно воздушный шар.
\end{enumerate}

Esimerkin 1 jälkimmäisessä lauseessa on selvästi nähtävissä, että

\begin{enumerate}
\def\labelenumi{\arabic{enumi}.}
\tightlist
\item
  Kyseessä on yksittäinen tapahtuma
\item
  Tapahtumalle ei ole asetettu rajoja, vaan se on avoin (ajattele edellä
  esitettyä vertausta usvasta)
\item
  Tapahtumaa voitaisiin siis kuvata prosessiksi, joskaan ei tässä
  tapauksessa prosessiksi siinä mielessä, että kyseessä olisi
  ``toiminto, jota ei ole saatu valmiiksi''. Oleellisempi ajatus on
  \emph{prosessuaalinen} vastakohtana \emph{rajatulle}.
\end{enumerate}

Vastaavasti myös esimerkeissä 2 ja 3 kuvataan \emph{yksittäisiä} ja
\emph{konkreettisia} tapahtumia, mutta esitetään ne ilman sisäistä rajaa
tai ajatusta toiminnan totaalisuudesta.

Jos KPM-merkityksen toimintoja esittää samassa lauseessa useita, on
tuloksena useita \emph{samanaikaisia} prosesseja:

\begin{enumerate}
\def\labelenumi{(\arabic{enumi})}
\setcounter{enumi}{3}
\tightlist
\item
  Молодая женщина \emph{сидела} у окна вагона и \emph{читала}
\item
  Она \emph{сидела} и \emph{думала} и в голову ей приходила \ldots{}
  только пустота
\item
  Я плакала и обнимала дочку.
\end{enumerate}

Perfektiivisellä aspektilla perusmerkitykseksi kutsutaan
\emph{konkreettis-faktista} merkitystä (конкретно-фактическое значение).
Samoin kuin imperfektiivisellä aspektilla, tässäkin on siis kyse yhdestä
konkreettisesta toiminnasta. Siinä missä imperfektiivinen aspekti
esittää toiminnan rajaamattomana, esiintyy se perfektiivisellä
aspektilla kuitenkin \emph{rajattuna}, kokonaisena faktana, jonka
tapahtumisesta (alkuineen ja loppuineen) kerrotaan. Tarkastellaan
esimerkkejä:

\begin{enumerate}
\def\labelenumi{(\arabic{enumi})}
\setcounter{enumi}{6}
\tightlist
\item
  Он повторил мне свой вопрос.
\end{enumerate}

Esimerkistä 7 voidaan todeta, että lause on hyvin selkeä ja helposti
tulkittavissa. Konkreettis-faktiselle merkitykselle (jatkossa lyhyemmin
KFM) onkin tyypillistä, ettei sen ilmenemiseen tarvita sen laajempaa
kontekstia, vaan se on yleensä melko yksitulkintainen. Esimerkissä 7 käy
hyvin selväksi, että kyseessä on yksi konkreettinen, rajattu tapahtuma
jonka keskelle ei hypätä, vaan jonka tapahtuminen esitetään kokonaisena
faktana.

\begin{enumerate}
\def\labelenumi{(\arabic{enumi})}
\setcounter{enumi}{7}
\tightlist
\item
  Бухгалтер сделал в учете запись.
\item
  \emph{Перешлю} ссылку на эту статью знакомым учителям.
\item
  Я \emph{заполз} под диван, \emph{лег} ничком, \emph{вынул} нож из
  кармана и \emph{прорезал} в ситце дырку, чтоб удобно было
  подсматривать.
\end{enumerate}

Erityisen havainnollinen on esimerkki 10, joka on peräisin Charles
Dickensin salapoliisitarinasta (Три рассказа о сыщиках). Verbeillä
\emph{заползти} (`ryömiä'), \emph{лечь} (ничком) -- `mennä maaten
(mahalleen)' -- , \emph{вынуть} (ottaa esille) ja прорезать (`leikata')
kuvataan kaikilla selkeästi kokonaisia, rajattuja, yksittäisiä
toimintoja -- tapahtuneita faktoja. Huomaa, että kun KFM-merkityksen
toimintoja on lauseessa monta, ne eivät tapahdu samanaikaisesti (vrt.
esim. 4) vaan \emph{peräkkäisesti}.

\subsection{Joitakin
erityismerkityksiä}\label{joitakin-erityismerkityksiuxe4}

Perusmerkitysten lisäksi voidaan erotella tutkijasta riippuen enemmän
tai vähemmän suuri joukko imperfektiivisen ja perfektiivisen aspektin
erityismerkityksiä. Siinä missä perusmerkitykset ilmenevät ilman
erityistä kontekstia ja ovat suoraviivaisia tulkittavia, tarkoitetaan
erityismerkityksillä melko spesifejä tilanteita, tiettyjä ympäristöjä,
joissa imperfektiivinen tai perfektiivinen aspekti tuo esiin tietyn
puolen toiminnasta. Erityismerkitysten voi ajatella olevan johdettuja
perusmerkityksestä: perfektiiviseen aspektiin liittyy edelleen toiminnan
rajattuus ja kokonaisuus. Näissä tilanteissa kuitenkin käyttö voi
etenkin kielenoppijan kannalta olla jossain määrin ennalta
arvaamattomampaa. Erityismerkityksistä kannattaa muistaa, että ne ovat
usein jollain tavalla värittyneitä lauseita, ja saman asian pystyisi
sanomaan neutraalimmin perusmerkityksiä käyttämällä.

\subsubsection{Toistoon liittyvät
erityismerkitykset}\label{toistoon-liittyvuxe4t-erityismerkitykset}

Aloitetaan erityismerkitysten tarkastelu joukosta merkityksiä, jotka
kaikki kuvaavat jollain tavalla toistoa. Ehkä hieman yllättäen näitä
merkityksiä on sekä imprefktiivisellä että perfektiivisellä aspektilla.

Imperfektiivisellä aspektilla voidaan erottaa kaksi toistoon liittyvää
erityismerkitystä: rajoittamattoman ja rajoitetun toiston merkitykset
(неограниченно-кратное / ограниченно-кратное значение). Tutkitaan
kumpaakin esimerkkien kautta.

Rajoittamaton toisto on hyvin tavallinen merkitys. Siinä tapahtumaa ei
sijoiteta mihinkään konkreettiseen aikaan vaan kuvataan se toistuvana,
eikä ilmaista tarkkaan, kuinka paljon toistoja on:

\begin{enumerate}
\def\labelenumi{(\arabic{enumi})}
\setcounter{enumi}{10}
\tightlist
\item
  Я часто покупаю на рынке овощи.
\item
  В Финляндии всегда играют в хоккей.
\item
  Сайт постоянно пополняется новыми материалами
\end{enumerate}

Kuten nimestä voi päätellä, rajoitetun toiston merkityksessä nimenomaan
kerrotaan, kuinka monta kertaa jokin toistuva tai toistunut toiminto
tapahtuu tai on tapahtunut. Tämä merkitys on edellistä huomattavasti
harvinaisempi ja tulee kysymykseen vain melko rajatuissa konteksteissa.
Oleellista on, että toistokertojen välinen aika jää hämärän peittoon ja
voi olla pitkäkin, kuten erityisesti esimerkissä 14:

\begin{enumerate}
\def\labelenumi{(\arabic{enumi})}
\setcounter{enumi}{13}
\tightlist
\item
  Я три раза поступал в университет
\item
  Мне это надоело. Три раза подогревала тебе суп.
\end{enumerate}

Etenkin esimerkki 15 on kuitenkin hyvin lähellä sitä toistoon liittyvää
eristyismerkitystä, joka kuuluu perfektiiviselle aspektille. Tätä
merkitystä kutsutaan \emph{summatiiviseksi} (суммарное значение) ja
aivan kuten rajoitetun toiston merkitys, se kuvaa tapahtumia, jotka ovat
toistuneet tarkkaan määritellyn määrän kertoja. Oleellinen ero kuitenkin
on siinä, että perfektiivinen aspekti -- joka, kuten muistat, on
aspekteista se \emph{kantaaottavampi osapuoli} -- ei jätä samanlaista
tulkinnanvaraa sille, voisiko tapahtumien välinen aika olla jollain
tapaa epämääräinen tai mahdollisesti pitkäkin. Pohdi seuraavia
esimerkkejä:

\begin{enumerate}
\def\labelenumi{(\arabic{enumi})}
\setcounter{enumi}{15}
\tightlist
\item
  Он несколько раз перечитал записку
\item
  Вчера я пять раз прочитал материал для экзамена.
\item
  Он несколько раз переспросил : «Точно?» --- но уже не на ухо, слышали
  и другие.
\item
  Мимо меня 15 раз прошел замечательный Вася Герелло, и в пятнадцатый
  раз сказал: ``Все, не могу больше, ухожу домой.''
\end{enumerate}

Joskus summatiivisen ja rajoitetun toiston merkityseroja on vaikea
erottaa tai niitä ei käytännössä ole, mutta vertaa esimerkkiä 16
seuraavaan imperfektiivisen aspektin esimerkkiin, ja pohdi, missä eron
voisi havaita:

\begin{enumerate}
\def\labelenumi{(\arabic{enumi})}
\setcounter{enumi}{19}
\tightlist
\item
  Я перечитывал этот роман несколько раз
\end{enumerate}

\subsubsection{Perfektiivisen aspektin muita
erityismerkityksiä}\label{perfektiivisen-aspektin-muita-erityismerkityksiuxe4}

Tutustutaan vielä kahteen perfektiivisen aspektin erityismerkitykseen:
perfektin merkitykseen (значение перфекта) ja potentiaaliseen
merkitykseen ().

Perfektin erityismerkitys liittyy perinteiseen ajatukseen ``tuloksen
voimassaolosta'' ja siitä, että juuri tätä perfektiivinen aspekti
painottaa. Jos nimittäin on tilanne, jossa puhuja kertoo jostakin
sellaisesta toiminnasta, jonka tulos on puhetilanteessa jotenkin näkyvä
tai läsnä, perfektiivinen aspekti on usein looginen valinta. Nikunlassi
(2002: 179) esittää tähän liittyen seuraavan esimerkin:

\begin{enumerate}
\def\labelenumi{(\arabic{enumi})}
\setcounter{enumi}{20}
\tightlist
\item
  Он похудел и постарел
\end{enumerate}

Esimerkissä 21 puheen kohteena olevasta henkilöstä on nähtävissä muutos,
joka on puhehetkellä voimassa. Suomessakin tällainen voimassaolo tulisi
ilmaistuksi perfektin aikamuodolla: Hän on laihtunut ja vanhentunut.
Vertaa lausetta vastaavaan imperfektiiviseen lausueeseen (он худел и
старел), jolloin kyseessä olisi konkreettis-prosessuaalinen merkitys:
katsottaisiin menneisyyteen kohti itse laihtumistapahtumaan sen sijaan,
että ihmeteltäisiin tapahtuman tulosta nykyhetkessä. Olga Rassudova
(1984) antaa lisäksi seuraavan esimerkin:

\begin{enumerate}
\def\labelenumi{(\arabic{enumi})}
\setcounter{enumi}{21}
\tightlist
\item
  Вы можете взять анкету. Я ее уже заполнил.
\end{enumerate}

Esimerkissä 22 puhuja on juuri täyttänyt lomakkeen (perfektiivinen
заполнил). Perfektiivinen aspekti ilmaisee tuloksen olevan puhehetkellä
läsnä ja relevantti.

Siirrytään nyt miettimään \emph{potentiaalista} merkitystä. Ajattele
seuraavia esimerkkilauseita:

\begin{enumerate}
\def\labelenumi{(\arabic{enumi})}
\setcounter{enumi}{22}
\tightlist
\item
  Саше можно доверять. Он всегда \emph{поможет} другу.
\item
  По лицу не \emph{поймешь}, сколько ей лет
\item
  Купил новый замок, теперь никто не \emph{пролезет} в квартиру.
\item
  Только \emph{скажешь} ему -- он немедленно приедет.
\end{enumerate}

Potentiaalisessa merkityksessä voidaan ajatella, että puhuja antaa siina
arvionsa

\begin{enumerate}
\def\labelenumi{\arabic{enumi}.}
\tightlist
\item
  Tekijän kyvystä \emph{tai kyvyttömyydestä} suorittaa jokin tehtävä
\item
  Tekijän taipumuksesta toimia tietyllä tavalla
\end{enumerate}

Esimerkissä 23 puhuja katsoo lauseen tekijällä (Саша) olevan tietty
taipumus toimia: hän on aina valmis auttamaan. Voisi ajatella, että
valitsipa minkä tahansa kerran kaikista mahdollisista kerroista, joina
ikinä pyydät Sašalta apua, niin aina voit olla varma, että hän auttaa.
Vastaavasti esimerkissä ({\textbf{???}}) voidaan kuvitella mikä tahansa
``sanomiskerta'': joka kerta pyynnön kohde varmuudella tulee.
Esimerkeissä 24 ja 25 puolestaan annetaan arvio tekijän kyvystä päätellä
henkilön ikä tai murtautua asuntoon.

\subsubsection{Imperfektiivisen aspektin
erityismerkityksiä}\label{imperfektiivisen-aspektin-erityismerkityksiuxe4}

Imperfektiivisen aspektin osalta keskitytään tarkastelemaan
\emph{yleisesti toteavaa merkitystä} (обобщённо-фактическое значение)
sekä pysyvää ominaisuutta tai tilaa korostavaa erityismerkitystä
(значение ).

Perinteinen esimerkki yleisesti toteavasta erityismerkityksestä on
seuraava:

\begin{enumerate}
\def\labelenumi{(\arabic{enumi})}
\setcounter{enumi}{26}
\tightlist
\item
  Ты читал ``Преступление и наказание''?
\end{enumerate}

Yleisesti toteava merkitys tuo esiin imperfektiivisen aspektin
neutraaliuden tai kantaaottamattomuuden. Sitä käytetään, kun halutaan
selvittää tai ilmaista, onko jokin toiminta ylipäätään tapahtunut.
Toisin kuin usein imperfektiivisen aspektin osalla, huomio viedään siis
pois toiminnasta sinäänsä ja kiinnostus kohdistuu vain siihen, oliko
toimintaa vai ei. Siinä missä perfektiivinen aspekti olisi värikkäämpi
ja esimerkiksi edellä esitetyissä Dickensin salapoliisitarinoissa
tyypillinen kerronnallinen verbimuoto, on yleisesti toteava merkitys
päinvastoin nimensä mukaisesti vain kuivan toteava. Näin ollen
esimerkissä 27 puhujaa kiinostaa, ilman mitään ennakko-odotuksia, onko
kuulija koskaan lukenut Rikosta ja rangaistusta. Vertaa tätä esimerkkiin
28:

\begin{enumerate}
\def\labelenumi{(\arabic{enumi})}
\setcounter{enumi}{27}
\tightlist
\item
  Ты прочитал ``Преступление и наказание''?
\end{enumerate}

Lause 28 ottaa edellistä enemmän kantaa toimintaan. On mahdollista, että
keskustelun osapuolet A ja B ovat aiemmin sopineet, että B:n pitää lukea
Rikos ja rangaistus. Nyt koittaa totuuden hetki ja A kysyy: Oletko
lukenut, niin kuin sovittiin? Tai toisaalta voi olla, että B kertoi
edellisenä iltana lukevansa juuri Rikosta ja Rangaistusta ja nyt A:ta
kiinnostaa, tuliko kirja luettua loppuun. Yhtä kaikki, se, onko
lukemista koskaan tapahtunut, ei ole epäilyksen alla: oletetaan, että
lukemista joko on tapahtunut tai olisi pitänyt tapahtua. Esimerkissä 27
puolestaan tähän ei oteta kantaa. Ajattele vastaavasti esimerkkiä 29:

\begin{enumerate}
\def\labelenumi{(\arabic{enumi})}
\setcounter{enumi}{28}
\tightlist
\item
  Я знаю, что Ханна поступала в магистерскую программу по
  переводоведению в Тамперском университете
\end{enumerate}

Esimerkissä 29 ollaan, samalla tavalla kuin esimerkissä 27, neutraaleja
suhteessa suoritettuun toimintaan. Esimerkissä 29 puhuja ei tiedä,
pääsikö Hanna sisään tai onko hän kenties yrittänyt pääsyä useampaan
kertaan vai ainoastaan kerran. Perfektiivisen aspektin verbi olisi aivan
eri tason kannanotto kaikkiin näihin kysymyksiin.

Yleisesti toteava merkitys voidaan jakaa moniin alamerkityksiin.
Erityisen huomionarvoinen alamerkitys on ajatus \emph{toiminnan
mitätöitymisestä} (аннулированность действия). Näissä tapauksissa
yleisesti toteava merkitys ja edellä käsitelty perfektin merkitys
muodostavat ikään kuin parin, jossa perfektiivinen aspekti merkitsee
jonkin tuloksen voimassaoloa, imperfektiivinen mitätöitymistä. Tähän
liittyvät seuraavat esimerkit:

\begin{enumerate}
\def\labelenumi{(\arabic{enumi})}
\setcounter{enumi}{29}
\tightlist
\item
  -- Почему в аудитории так холодно? Вы не открывали окно?
\item
  -- Папа вчера приезжал.
\end{enumerate}

Esimerkissä 30 kysyjä epäilee, että \emph{vaikka ikkuna on puhehetkellä
kiinni}, se on ehkä avattu hänen poissa ollessaan ja sitten taas
suljettu (avaaminen on mitätöity). Vastaavasti isä on esimerkissä 31
tullut ja edelleen lähtenyt (tuleminen mitätöity). Perfektiivisen
aspektin tapauksessa korostuisi toiminnan tuloksen voimassaolo: ikkuna
olisi auki (открыли) ja isä vielä talossa (приехал).

Yleisesti toteavan merkityksen lisäksi tarkastellaan vielä laajempaa
rypästä eri merkityksiä, jotka liittyvät imperfektiiviseen aspektiin.
Nikunlassi (2002: 182) käyttää tässä yhteydessä nimitystä \emph{tilan,
ominaisuuden tai suhteen merkitys} (значение состояния, свойства, или
отношения). Ajatus on, että jos kuvataan jonkin asian, esineen tai
ihmisen pysyvää ominaisuudetta tai tämänhetkistä tilaa, käytetään
tavallisesti imperfektiivistä aspektia. Ajattele esimerkkiä 32:

\begin{enumerate}
\def\labelenumi{(\arabic{enumi})}
\setcounter{enumi}{31}
\tightlist
\item
  Река течет через город.
\end{enumerate}

Virtaaminen on joen pysyvä ominaisuus -- kyse ei ole toistosta tai
yksittäisestä konkriittesta tapauksesta, vaan jatkuvasta tilasta. A.V.
Bondarko käyttää vastaavista tapauksista nimitystä
potentiaalis-laadullinen merkitys (потенциально-качественное значение).
Kyse voi kuitenkin olla myös pienemmän mittakaavan ominaisuuksista tai
tiloista kuten seuraavissa:

\begin{enumerate}
\def\labelenumi{(\arabic{enumi})}
\setcounter{enumi}{32}
\tightlist
\item
  Ты играешь на гитаре?
\item
  Миша болеет гриппом.
\item
  Она на самом деле считает, что солнце - это очень старая звезда
\end{enumerate}

Tilasta, ominaisuudesta ja suhteesta on siis kyse myös, kun kerrotaan
jonkun olevan sairaana tai tiettyä mieltä jostakin asiasta (mielipide
nähdään ominaisuutena).

\hyperdef{}{ref-nikunl}{\label{ref-nikunl}}
Nikunlassi, Ahti. 2002. \emph{Johdatus Venäjän Kieleen Ja Sen
Tutkimukseen}. Helsinki: Finn Lectura.

\hyperdef{}{ref-rassudova1984}{\label{ref-rassudova1984}}
Rassudova, O.P. 1984. \emph{Aspectual Usage in Modern Russian}. Moskova:
Russky Yazyk.

\end{document}
