\documentclass[paper=a4, fontsize=11pt]{scrartcl} 
\usepackage[a4paper,bindingoffset=0.2in,%
            left=0.5in,right=1in,top=1in,bottom=1in,%
            footskip=.25in]{geometry}

\usepackage{titlesec}
\usepackage{setspace}
\usepackage{enumitem}
\usepackage{sectsty} 
\usepackage{fancyhdr} 
\pagestyle{fancyplain}
\fancyfoot[C]{\thepage} 
\fancyhead[L]{Morfologia 2016} 
\fancyhead[C]{} 
\fancyhead[R]{} 
\renewcommand{\headrulewidth}{0pt}
\renewcommand{\footrulewidth}{0pt}
\setlength{\headheight}{13.6pt} 
\setlength\parindent{0pt} % Removes all indentation from paragraphs - comment this line for an assignment with lots of text

\titlespacing*{\section}
{0pt}{8.5ex plus 1ex minus .2ex}{4.3ex plus .2ex}
\titlespacing*{\subsection}
{0pt}{8.5ex plus 1ex minus .2ex}{4.3ex plus .2ex}

\newcommand{\horrule}[1]{\rule{\linewidth}{#1}} 

\title{	
\normalfont \normalsize 
\textsc{Morfologia 2016} \\ [25pt] 
\horrule{0.5pt} \\[0.4cm] 
\huge  \\ 
\horrule{2pt} \\[0.5cm] 
}

\date{}

\usepackage[utf8]{inputenc}
\usepackage[T2A]{fontenc}

\begin{document}

\onehalfspacing
%\maketitle 

\section{Suomenna parillesi} 

\begin{enumerate}
    \item Самолет возвращался на аэродром.
    \item Самолет возвратился на аэродром.
    \item Билеты уже продавали.
    \item Билеты уже продали
    \item К вечеру больному стало хуже
    \item К вечеру больному становилось хуже
    \item Он читал за два часа 30 страниц
    \item Он прочитал за два часа 30 страниц
\end{enumerate}


\section{Valitse sopivampi aspekti} 

\begin{enumerate}
    \item Сегодня ты пойдёшь в спортзал? -- Нет, я подготовлюсь / буду готовиться к тесту
    \item Завтра вечером ты свободен? -- Нет, буду заниматься / займусь дипломной работой.
    \item Я напишу / буду писать тест за полчаса.
    \item Зачем ты сказал ему это! Теперь он оскорблён и больше никогда сюда не придет / будет приходить.
    \item Ты съешь / будешь есть?
    \item Какой вкусный торт! Теперь я всегда буду завтракать/позавтракаю в этом кафе.
    \item Я не прочитаю / буду читать этот роман. Он неинтересный.
    \item Ты не решишь / будешь решать эту задачу.
    \item Что ты так долго одеваешься?  -- Сейчас буду одеваться/оденусь!
    \item Не волнуйся, ты справишься / будешь справляться с этим курсом!
    \item Я не буду помогать / помогу ему, я хочу, чтобы он научился работать самостоятельно.
\end{enumerate}


\section{Valitse sopivampi aspekti} 

\begin{enumerate}
    \item Ваня, мы по тебе скучали, что же это ты не пришел / приходил?
    \item В 40-ые годы кварки не изучали / изучили.
    \item Его ждали, а он всё не приезжал / приехал.
    \item Мы твердо убеждены, что клиент не должен  платить / заплатить за такое
    \item Больше не нужно откладывать / отложить покупку на потом! 
    \item О каком письме вы говорите? Я не получал / получил никакого письма.
    \item В пятьдесят лет он еще не решил / решал, кем стать
    \item Не надо забывать / забыть, что нефть -- невоспроизводимый ресурс.
    \item Они вчера долго не легли / ложились спать
    \item Кстати, на заседание сенатского комитета представители Минпечати не пришли / приходили, хотя приглашение им было направлено.
    \item Не понимаю, зачем он пришел сюда. Его никто не звал / позвал.
\end{enumerate}



\section{Valitse sopivampi aspekti} 

\begin{enumerate}
    \item Этот преподаватель стал лучше произносить / произнести мягкие звуки.
    \item Вчера вечером у нас было собрание, но я все-таки успел выполнять / выполнить домашнее задание.
    \item Когда ты кончишь завтракать / позавтракать, вымой посуду.
    \item В деревне она привыкла вставать / встать рано.
    \item Я рад, что мне удалось покупать / купить билет на этот матч.
    \item Чтобы работа шла успешно, собрания надо проводить / провести как можно регулярнее.
    \item Хоккеист продолжал играть/сыграть даже несмотря на острую боль.
    \item Ты опять забыл ставить / поставить машину на ручной тормоз?
    \item Он недавно бросил курить / покурить
\end{enumerate}

\end{document}
