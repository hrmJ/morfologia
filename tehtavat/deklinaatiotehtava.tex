\documentclass[paper=a4, fontsize=11pt]{scrartcl} 
\usepackage[a4paper,bindingoffset=0.2in,%
            left=0.5in,right=1in,top=1in,bottom=1in,%
            footskip=.25in]{geometry}

\usepackage{titlesec}
\usepackage{setspace}
\usepackage{enumitem}
\usepackage{sectsty} 
\usepackage{fancyhdr} 
\usepackage{hyperref}
\usepackage{longtable} 
\usepackage{booktabs}
\usepackage{amsthm}

\pagestyle{fancyplain}
\fancyfoot[C]{\thepage} 
\fancyhead[L]{Morfologia 2016} 
\fancyhead[C]{} 
\fancyhead[R]{} 
\renewcommand{\headrulewidth}{0pt}
\renewcommand{\footrulewidth}{0pt}
\setlength{\headheight}{13.6pt} 
\setlength\parindent{0pt} % Removes all indentation from paragraphs - comment this line for an assignment with lots of text

\titlespacing*{\section}
{0pt}{8.5ex plus 1ex minus .2ex}{4.3ex plus .2ex}
\titlespacing*{\subsection}
{0pt}{8.5ex plus 1ex minus .2ex}{4.3ex plus .2ex}

\newcommand{\horrule}[1]{\rule{\linewidth}{#1}} 
\usepackage[utf8]{inputenc}
\usepackage[T2A]{fontenc}

\renewcommand{\thesubsection}{\alph{subsection}}
\renewcommand{\thesection}{Tehtävä \arabic{section}}

\providecommand{\tightlist}{%
  \setlength{\itemsep}{0pt}\setlength{\parskip}{0pt}}

\begin{document}


\section{Ensimmäinen deklinaatio / Первый тип
склонения}\label{ensimmuxe4inen-deklinaatio-ux43fux435ux440ux432ux44bux439-ux442ux438ux43f-ux441ux43aux43bux43eux43dux435ux43dux438ux44f}

\begin{enumerate}
\def\labelenumi{\arabic{enumi}.}
\tightlist
\item
  Merkitkää taulukkoon kunkin sijan kohdalle morfi, jolla se ilmaistaan
  (esim. /а/)
\item
  Keksikää taulukon alle lisää tähän deklinaatioon kuuluvia sanoja --
  vähintään yksi sana per ryhmäläinen. Kirjoittakaa näistä sanoista kaikkien
  sijamuotojen mukaiset taivutusmuodot peräkkäin.
\end{enumerate}


{\setstretch{2.5}
\begin{longtable}[c]{@{}llll@{}}
\toprule
Sija & Päätemorfi(t) & esim.1 & esim.2\tabularnewline
\midrule
\endhead
Nom. & & стол & учитель\tabularnewline
Gen. & & стола & учителя\tabularnewline
Akk. & & стол & учителя\tabularnewline
Dat. & & столу & учителю\tabularnewline
Instr. & & cтолом & учителем\tabularnewline
Prep. & & столе & учителе\tabularnewline
\bottomrule
\end{longtable}
}


\begin{center}

\vspace{1.1cm} 

\underline{\hspace{12cm}} \\
\vspace{0.5cm}
\underline{\hspace{12cm}} \\
\vspace{0.5cm}
\underline{\hspace{12cm}} \\
\vspace{0.5cm}
\underline{\hspace{12cm}} \\
\vspace{0.5cm}
\underline{\hspace{12cm}} \\
\vspace{0.5cm}
\underline{\hspace{12cm}} \\
\vspace{0.5cm}
\underline{\hspace{12cm}} \\
\vspace{0.5cm}
\underline{\hspace{12cm}} \\
\vspace{0.5cm}
\underline{\hspace{12cm}} \\

\end{center}

\clearpage

\section{Toinen deklinaatio / Второй тип
склонения}\label{toinen-deklinaatio-ux432ux442ux43eux440ux43eux439-ux442ux438ux43f-ux441ux43aux43bux43eux43dux435ux43dux438ux44f}

\begin{enumerate}
\def\labelenumi{\arabic{enumi}.}
\tightlist
\item
  Merkitkää taulukkoon kunkin sijan kohdalle morfi, jolla se ilmaistaan
  (esim. /а/)
\item
  Keksikää taulukon alle lisää tähän deklinaatioon kuuluvia sanoja --
  vähintään yksi sana per ryhmäläinen. Kirjoittakaa näistä sanoista kaikkien
  sijamuotojen mukaiset taivutusmuodot peräkkäin.
\end{enumerate}


{\setstretch{2.5}
\begin{longtable}[c]{@{}lllll@{}}
\toprule
Sija & Päätemorfi & esim.1 & esim.2 & esim.3\tabularnewline
\midrule
\endhead
Nom. & & книга & серия & тыква\tabularnewline
Gen. & & книги & серии & тыквы\tabularnewline
Akk. & & книгу & серию & тыкву\tabularnewline
Dat. & & книге & серии & тыкве\tabularnewline
Instr. & & книгой & серией & тыквой\tabularnewline
Prep. & & книге & серии & тыкве\tabularnewline
\bottomrule
\end{longtable}
}

\begin{center}

\vspace{1.1cm} 

\underline{\hspace{12cm}} \\
\vspace{0.5cm}
\underline{\hspace{12cm}} \\
\vspace{0.5cm}
\underline{\hspace{12cm}} \\
\vspace{0.5cm}
\underline{\hspace{12cm}} \\
\vspace{0.5cm}
\underline{\hspace{12cm}} \\
\vspace{0.5cm}
\underline{\hspace{12cm}} \\
\vspace{0.5cm}
\underline{\hspace{12cm}} \\
\vspace{0.5cm}
\underline{\hspace{12cm}} \\
\vspace{0.5cm}
\underline{\hspace{12cm}} \\

\end{center}

\clearpage

\section{Kolmas deklinaatio / Третий тип
склонения}\label{kolmas-deklinaatio-ux442ux440ux435ux442ux438ux439-ux442ux438ux43f-ux441ux43aux43bux43eux43dux435ux43dux438ux44f}

\begin{enumerate}
\def\labelenumi{\arabic{enumi}.}
\tightlist
\item
  Merkitkää taulukkoon kunkin sijan kohdalle morfi, jolla se ilmaistaan
  (esim. /а/)
\item
  Keksikää taulukon alle lisää tähän deklinaatioon kuuluvia sanoja --
  vähintään yksi sana per ryhmäläinen. Kirjoittakaa näistä sanoista kaikkien
  sijamuotojen mukaiset taivutusmuodot peräkkäin.
\end{enumerate}

{\setstretch{2.5}
\begin{longtable}[c]{@{}llll@{}}
\toprule
Sija & Päätemorfi & esim.1 & esim.2\tabularnewline
\midrule
\endhead
Nom. & & тетрадь & ночь\tabularnewline
Gen. & & тетради & ночи\tabularnewline
Akk. & & тетрадь & ночь\tabularnewline
Dat. & & тетради & ночи\tabularnewline
Instr. & & тетрадью & ночью\tabularnewline
Prep. & & тетради & ночи\tabularnewline
\bottomrule
\end{longtable}
}


\begin{center}

\vspace{1.1cm} 

\underline{\hspace{12cm}} \\
\vspace{0.5cm}
\underline{\hspace{12cm}} \\
\vspace{0.5cm}
\underline{\hspace{12cm}} \\
\vspace{0.5cm}
\underline{\hspace{12cm}} \\
\vspace{0.5cm}
\underline{\hspace{12cm}} \\
\vspace{0.5cm}
\underline{\hspace{12cm}} \\
\vspace{0.5cm}
\underline{\hspace{12cm}} \\
\vspace{0.5cm}
\underline{\hspace{12cm}} \\
\vspace{0.5cm}
\underline{\hspace{12cm}} \\

\end{center}
\clearpage


\section{Neljäs deklinaatio / Четвёртый тип
склонения}\label{neljuxe4s-deklinaatio-ux447ux435ux442ux432ux451ux440ux442ux44bux439-ux442ux438ux43f-ux441ux43aux43bux43eux43dux435ux43dux438ux44f}

\begin{enumerate}
\def\labelenumi{\arabic{enumi}.}
\tightlist
\item
  Merkitkää taulukkoon kunkin sijan kohdalle morfi, jolla se ilmaistaan
  (esim. /а/)
\item
  Keksikää taulukon alle lisää tähän deklinaatioon kuuluvia sanoja --
  vähintään yksi sana per ryhmäläinen. Kirjoittakaa näistä sanoista kaikkien
  sijamuotojen mukaiset taivutusmuodot peräkkäin.
\end{enumerate}


{\setstretch{2.5}
\begin{longtable}[c]{@{}lllll@{}}
\toprule
Sija & Päätemorfi & esim.1 & esim.2 & esim.3\tabularnewline
\midrule
\endhead
Nom. & & место & поле & пение\tabularnewline
Gen. & & места & поля & пения\tabularnewline
Akk. & & место & поле & пение\tabularnewline
Dat. & & месту & полю & пению\tabularnewline
Instr. & & местом & полем & пением\tabularnewline
Prep. & & месте & поле & пении\tabularnewline
\bottomrule
\end{longtable}
}


\begin{center}

\vspace{1.1cm} 

\underline{\hspace{12cm}} \\
\vspace{0.5cm}
\underline{\hspace{12cm}} \\
\vspace{0.5cm}
\underline{\hspace{12cm}} \\
\vspace{0.5cm}
\underline{\hspace{12cm}} \\
\vspace{0.5cm}
\underline{\hspace{12cm}} \\
\vspace{0.5cm}
\underline{\hspace{12cm}} \\
\vspace{0.5cm}
\underline{\hspace{12cm}} \\
\vspace{0.5cm}
\underline{\hspace{12cm}} \\
\vspace{0.5cm}
\underline{\hspace{12cm}} \\

\end{center}
\clearpage


\section{Viides deklinaatio / Пятый тип
склонения}\label{viides-deklinaatio-ux43fux44fux442ux44bux439-ux442ux438ux43f-ux441ux43aux43bux43eux43dux435ux43dux438ux44f}

\begin{enumerate}
\def\labelenumi{\arabic{enumi}.}
\tightlist
\item
  Merkitkää taulukkoon kunkin sijan kohdalle morfi, jolla se ilmaistaan
  (esim. /а/)
\item
  Keksikää taulukon alle lisää tähän deklinaatioon kuuluvia sanoja --
  vähintään yksi sana per ryhmäläinen. Kirjoittakaa näistä sanoista kaikkien
  sijamuotojen mukaiset taivutusmuodot peräkkäin.
\end{enumerate}

{\setstretch{2.5}
\begin{longtable}[c]{@{}lll@{}}
\toprule
Sija & Päätemorfi & esim\tabularnewline
\midrule
\endhead
Nom. & & время\tabularnewline
Gen. & & времени\tabularnewline
Akk. & & время\tabularnewline
Dat. & & времени\tabularnewline
Instr. & & временем\tabularnewline
Prep. & & времени\tabularnewline
\bottomrule
\end{longtable}
}

\begin{center}

\vspace{1.1cm} 

\underline{\hspace{12cm}} \\
\vspace{0.5cm}
\underline{\hspace{12cm}} \\
\vspace{0.5cm}
\underline{\hspace{12cm}} \\
\vspace{0.5cm}
\underline{\hspace{12cm}} \\
\vspace{0.5cm}
\underline{\hspace{12cm}} \\
\vspace{0.5cm}
\underline{\hspace{12cm}} \\
\vspace{0.5cm}
\underline{\hspace{12cm}} \\
\vspace{0.5cm}
\underline{\hspace{12cm}} \\
\vspace{0.5cm}
\underline{\hspace{12cm}} \\

\end{center}

\end{document}
