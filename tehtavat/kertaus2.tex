\documentclass[paper=a4, fontsize=11pt]{scrartcl} 
\usepackage[a4paper,margin=0.1cm]{geometry}

\usepackage{titlesec}
\usepackage{tabularx}
\pagenumbering{gobble}
\usepackage{arydshln}
\usepackage{multirow}
\usepackage{pdflscape}
\usepackage{longtable} 
\usepackage{booktabs}

\usepackage{bm,array}

\newcommand{\answer}[2][3cm]{\raisebox{-\baselineskip}{\shortstack{\underline{\hspace{#1}}\\#2}}}



\usepackage{setspace}




%\usepackage{fourier} % Use the Adobe Utopia font for the document - comment this line to return to the LaTeX default
\usepackage{enumitem}

\usepackage{sectsty} % Allows customizing section commands

\sectionfont{\normalsize}
\subsectionfont{\normalsize}
\subsubsectionfont{\normalsize}

\setlength\parindent{0pt} % Removes all indentation from paragraphs - comment this line for an assignment with lots of text

\titlespacing*{\section}
{0pt}{8.5ex plus 1ex minus .2ex}{4.3ex plus .2ex}
\titlespacing*{\subsection}
{0pt}{8.5ex plus 1ex minus .2ex}{4.3ex plus .2ex}

%----------------------------------------------------------------------------------------
%	TITLE SECTION
%----------------------------------------------------------------------------------------

\newcommand{\horrule}[1]{\rule{\linewidth}{#1}} % Create horizontal rule command with 1 argument of height

\title{	
\normalfont \normalsize 
\textsc{Morfologia 2016} \\ [25pt] % Your university, school and/or department name(s)
\horrule{0.5pt} \\[0.4cm] % Thin top horizontal rule
\huge  \\ % The assignment title
\horrule{2pt} \\[0.5cm] % Thick bottom horizontal rule
}

%\author{} % Your name

\date{} % Today's date or a custom date


\usepackage[utf8]{inputenc}
\usepackage[T2A]{fontenc}

\renewcommand{\thesubsection}{\alph{subsection}}
\renewcommand{\thesection}{Tehtävä \arabic{section}}

\begin{document}

\section{Täydennä seuraava konjugaatiotaulukko}

\newcolumntype{C}{>{\centering\arraybackslash}p{2.5cm}}

\begin{tabular}[c]{l|l|C|C|C|C|C|C|}
Nro & infinitiivi & yks.1. & yks.2. & yks.3 & mon.1. & mon.2. & mon.3 \\[0.5cm] \toprule
1    & обещать     &        &        &       &        &        &       \\[0.5cm]
2    & держать     &        &        &       &        &        &       \\[0.5cm]
3    & смотреть    &        &        &       &        &        &       \\[0.5cm]
4    & покраснеть  &        &        &       &        &        &       \\[0.5cm]
5    & пить        &        &        &       &        &        &       \\[0.5cm]
6    & платить     &        &        &       &        &        &       \\[0.5cm]
7    & плакать     &        &        &       &        &        &       \\[0.5cm]
8    & давать      &        &        &       &        &        &       \\[0.5cm]
9    & жевать      &        &        &       &        &        &       \\[0.5cm]
10   & замёрзнуть  &        &        &       &        &        &       \\[0.5cm]
11   & нажать      &        &        &       &        &        &       \\[0.5cm]
12   & понять      &        &        &       &        &        &       \\[0.5cm]
13   & обнять      &        &        &       &        &        &       \\[0.5cm]
14   & жечь        &        &        &       &        &        &       \\[0.5cm]
15   & изобразить  &        &        &       &        &        &       \\[0.5cm]
16   & лазить      &        &        &       &        &        &       \\[0.5cm]
17   & кормить     &        &        &       &        &        &       \\[0.5cm]
18   & ездить      &        &        &       &        &        &       \\[0.5cm]
19   & пропустить  &        &        &       &        &        &       \\[0.5cm]
20   & просить     &        &        &       &        &        &       \\[0.5cm]
21   & умереть     &        &        &       &        &        &       \\[0.5cm]
\end{tabular}

% Mitä partisiippimuotoja on olemassa? (a ja p), preesens, preteriti

\section{Muodosta....}

\begin{enumerate}
    \item Aktiivin partisiipin preesensmuodot verbeistä 16 ja 7:  \underline{\hspace{8cm}}
    \item Passiivin partisiipin preesensmuodot verbeistä 1 ja 6:  \underline{\hspace{8cm}}
    \item Aktiivin partisiipin preteritimuodot verbeistä 11 ja 6:  \underline{\hspace{8cm}}
    \item Passiivin partisiipin preteritimuodot verbeistä 15 ja 19:  \underline{\hspace{8cm}}
    \item Imperatiivimuoto verbeistä 14, 7 ja 4:  \underline{\hspace{8cm}}
\end{enumerate}

\section{Käännä.}

\begin{enumerate}
    \item Isoäitini kuoli, kun olin viisi vuotta. \\ \underline{\hspace{12cm}}
    \item Jos isoäitini vielä eläisi, hän leipoisi joka päivä pullia. \\ \underline{\hspace{12cm}}
    \item Oletko joskus leiponut pullia? \\ \underline{\hspace{12cm}}
    \item Juttelimme vähän aikaa niitä näitä, minkä isoäiti alkoi kovaäänisesti nauraa.  \\ \underline{\hspace{12cm}}
    \item Maito loppui. Petja, käy kaupassa! \\ \underline{\hspace{12cm}}
    \item Helikopteri löydettiin autiomaasta. \\ \underline{\hspace{12cm}}
    \item Kun Petja tuli kaupassa lapset istuivat matolla ja nauroivat. \\ \underline{\hspace{12cm}}
    \item Kun Petja oli tullut kaupasta, lapset olivat viisi minuuttia hiljaa ja alkoivat sitten meluta. \\ \underline{\hspace{12cm}}
\end{enumerate}


\end{document}


