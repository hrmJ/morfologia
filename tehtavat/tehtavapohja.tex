\documentclass[paper=a4, fontsize=11pt]{scrartcl} 
\usepackage[a4paper,bindingoffset=0.2in,%
            left=0.5in,right=1in,top=1in,bottom=1in,%
            footskip=.25in]{geometry}

\usepackage{titlesec}
\usepackage{setspace}
\usepackage{enumitem}
\usepackage{sectsty} 
\usepackage{fancyhdr} 
\pagestyle{fancyplain}
\fancyfoot[C]{\thepage} 
\fancyhead[L]{Morfologia 2016} 
\fancyhead[C]{} 
\fancyhead[R]{} 
\renewcommand{\headrulewidth}{0pt}
\renewcommand{\footrulewidth}{0pt}
\setlength{\headheight}{13.6pt} 
\setlength\parindent{0pt} % Removes all indentation from paragraphs - comment this line for an assignment with lots of text

\titlespacing*{\section}
{0pt}{8.5ex plus 1ex minus .2ex}{4.3ex plus .2ex}
\titlespacing*{\subsection}
{0pt}{8.5ex plus 1ex minus .2ex}{4.3ex plus .2ex}

\newcommand{\horrule}[1]{\rule{\linewidth}{#1}} 

\title{	
\normalfont \normalsize 
\textsc{Morfologia 2016} \\ [25pt] 
\horrule{0.5pt} \\[0.4cm] 
\huge  \\ 
\horrule{2pt} \\[0.5cm] 
}

\date{}

\usepackage[utf8]{inputenc}
\usepackage[T2A]{fontenc}

\begin{document}

\onehalfspacing
%\maketitle 

\section{Selitä seuraavien lauseiden välinen merkitysero}\label{kuxe4uxe4nnuxe4-venuxe4juxe4ksi}

Если он будет уходить, позвоните мне \\
Если он уйдет, позвоните мне \\
\underline{\hspace{12cm}} \\
\underline{\hspace{12cm}} \\
\underline{\hspace{12cm}} 


\section{Käännä seuraavat lauseet}\label{kuxe4uxe4nnuxe4-venuxe4juxe4ksi}

\begin{enumerate}
\def\labelenumi{\arabic{enumi}.}
\item (bussissa) Aiotteko jäädä pois? \\
    \underline{\hspace{12cm}}
\item Mitä teet huomenna? \\
    -- Aamulla ennen töitä kuuntelen podcasteja. Illalla pelaan Xboxilla. \\
    \underline{\hspace{12cm}} \\
    \underline{\hspace{12cm}}
\item En voi lähteä teidän kanssanne matsiin huomenna, minä opiskelen. \\
    \underline{\hspace{12cm}}
\item (ravintolassa) -- Ahaa, olettekin jo päättäneet. Mitä te tilaatte?\\
    \underline{\hspace{12cm}}
\end{enumerate}


\section{Tutki seuraavia virkkeitä}

\begin{enumerate}
    \item \label{pappa} Наш дедушка на все руки мастер - он и замок \emph{починит}, и игрушку \emph{смастерит}
    \item \label{k1} Вы не \emph{подскажете}... второе терапевтическое где? 
    \item \label{k2} А вы не \emph{скажете}, что это за музыка, Иван Афанасич? 
    \item \label{k3} Старик тем временем открыл глаза, тяжело вдохнул и пробормотал: -- Спасибо вам большое… Мне уже лучше… Вы не \emph{поможете} мне подняться? 
    \item \label{udarit} «Мое главное «никогда» -- я никогда не \emph{ударю} девушку 
    \item \label{pravda} Но правда бывает разная, и правдой не всегда \emph{пробьешь} путь к намеченной цели
    \item \label{leicester} Я обычно, если хочу в кино, иду на Лейстер-сквер -- и всегда \emph{подберешь} себе картину по вкусу
    \item \label{murovei} Так не один раз мне приходится делать, и всегда при этом \emph{пожалеешь} муравьев, что приготовил им лишнюю работу
\end{enumerate}

\subsection{Pohdi kysymyksiä:}

\singlespacing
\begin{enumerate}
    \item Mitä perfektiivisen aspektin erityismerkitystä esimerkkien \ref{pappa}--\ref{udarit} kursivoidut verbit edustavat?
    \item Miksi virkkeet \ref{k1} - \ref{k3} ovat kohteliaita?
    \item Mitä perfektiivisen aspektin erityismerkitystä esimerkkien \ref{pravda}--\ref{murovei} kursivoidut verbit edustavat?
\end{enumerate}

\onehalfspacing


\section{Täydennä puuttuvat sanat}

\begin{enumerate}
    \item
    \raisebox{-\baselineskip}{\shortstack{\underline{\hspace{3cm}}\\sanokaa}}, пожалуйста, где выход.
    \item Возьмите, пожалуйста, ручку.. \raisebox{-\baselineskip}{\shortstack{\underline{\hspace{3cm}}\\ota}}, у меня есть еще одна 
    \item \raisebox{-\baselineskip}{\shortstack{\underline{\hspace{3cm}}\\tule sisään}}….. Ну что же ты? Не бойся, \raisebox{-\baselineskip}{\shortstack{\underline{\hspace{3cm}}\\tule sisään}}
    \item Запиши мой телефон. \raisebox{-\baselineskip}{\shortstack{\underline{\hspace{3cm}}\\kirjoita}}, пожалуйста, я очень тороплюсь
    \item \raisebox{-\baselineskip}{\shortstack{\underline{\hspace{3cm}}\\vastaa}} сначала на второй вопрос, потом вернемся к первому 
    \item Повесь объявление. \raisebox{-\baselineskip}{\shortstack{\underline{\hspace{3cm}}\\ripusta}}, осторожнее, не упади
    \item Инженер и техник готовили к испытанию новый станок. Когда все было готово, инженер сказал технику: \raisebox{-\baselineskip}{\shortstack{\underline{\hspace{3cm}}\\käynnistä}}
    \item "\raisebox{-\baselineskip}{\shortstack{\underline{\hspace{3cm}}\\käynnistä}} мотор", сказал инженер технику, решив проверить его еще раз. 
\end{enumerate}


%.
% Утром он просыпался. Обедал в столовой. После обеда делал репортаж о Cнурке. 

%Есть такой неплохой фильм Гарольда Рамиса. Главный герой – средней руки
%журналист Фил Коннорс, которого сыграл Билл Мюррей, каждое утро просыпался под
%одну и ту же песню – "I've got you, baby" в исполнении Сонни и Шер – и каждый
%день делал абсолютно не интересный репортаж о Дне сурка. И чем бы ни занимался
%наш герой, какими бы событиями ни пытался наполнить эти сутки – День сурка не
%отпускал его. 

\end{document}
