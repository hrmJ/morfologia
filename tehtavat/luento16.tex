\documentclass[paper=a4, fontsize=11pt]{scrartcl} 
\usepackage[a4paper,bindingoffset=0.2in,%
            left=0.5in,right=1in,top=1in,bottom=1in,%
            footskip=.25in]{geometry}

\usepackage{titlesec}
\usepackage{setspace}
\usepackage{amsmath}
\usepackage{enumitem}
\usepackage{sectsty} 
\usepackage{fancyhdr} 
\pagestyle{fancyplain}
\fancyfoot[C]{\thepage} 
\fancyhead[L]{Morfologia 2016} 
\fancyhead[C]{} 
\fancyhead[R]{} 
\renewcommand{\headrulewidth}{0pt}
\renewcommand{\footrulewidth}{0pt}
\setlength{\headheight}{13.6pt} 
\setlength\parindent{0pt} % Removes all indentation from paragraphs - comment this line for an assignment with lots of text

\titlespacing*{\section}
{0pt}{8.5ex plus 1ex minus .2ex}{4.3ex plus .2ex}
\titlespacing*{\subsection}
{0pt}{8.5ex plus 1ex minus .2ex}{4.3ex plus .2ex}

\newcommand{\horrule}[1]{\rule{\linewidth}{#1}} 

\title{	
\normalfont \normalsize 
\textsc{Morfologia 2016} \\ [25pt] 
\horrule{0.5pt} \\[0.4cm] 
\huge  \\ 
\horrule{2pt} \\[0.5cm] 
}

\date{}

\usepackage[utf8]{inputenc}
\usepackage[T2A]{fontenc}

\begin{document}


%\maketitle 
\section{Aspektien merkityksiä}

\begin{itemize}
    \item Konkreettis-prosessuaalinen merkitys
    \item Konkreettis-faktinen merkitys
    \item Rajoittamaton toisto
    \item Rajoitettu toisto
    \item Summatiivinen merkitys
    \item Perfektin merkitys
    \item Potentiaalinen merkitys
    \item Yleisesti toteava merkitys
    \item Tila, ominaisuus, suhde
\end{itemize}

\onehalfspacing


\subsection{Mitä edellä mainituista merkityksistä seuraavat lauseet edustavat?}

\begin{enumerate}
    \item Том и Анжела будут петь популярные песни, дарить детям лицензионные подарки для школы, рисовать, играть и дурачиться вместе со всеми желающими!
    \item Я сегодня два раза включал ноутбук.
    \item Он каждый день приходит к нам.
    \item Алла Владимировна, я вам вчера два раза звонил!
    \item Вы включали компьютер?
    \item Все любят Машу.
    \item Яна вошла в кухню, достала из шкафа початую бутылочку коньяка, налила рюмочку и выпила.
    \item Ну действительно ― с Богом разве поспоришь?
    \item Как ты загорела!
    \item Петя написал письмо и вложил его в конверт .
    \item Брат спит с открытой форточкой.
    \item Вы не читали книг в стиле «антиутопия», где все продумано до мелочей? 
    \item Вот читаю комментарии и больно на душе.
    \item Он читает на многих языках: на французском, немецком, итальянском, испанском
    \item Батюшка завтракал на балконе, погода была прекрасная.
    \item Маша прочитает лекцию на эту тему. (Она прекрасно знает материал и имеет опыт в чтении лекций студентам.)
    \item Митя несколько раз зевнул и от скуки принялся рисовать на парте гоночную машину
    \item Выигрывал ли зенит лигу чемпионов?
    \item Женщины - кто их поймет!
\end{enumerate}

\begin{enumerate}
    \item Вы когда-нибудь солили/посолили рыбу?
    \item Мой товарищ долгое время изучал/изучил английский язык.
    \item Мы сидели и внимательно слушали/прослушали. Преподаватель объяснял/объяснил значение и употребление новых слов
    \item Я пришел к концу собрания. Товарищи все ещё обсуждали/обсудили этот вопрос.
    \item На углу он попрощался/прощался с нами и пошел/шел направо, а мы пошли/шли налево.
    \item Она оделась/одевалась быстро, через пять минут мы уже вышли из дома.
    \item У меня заболел голова, поэтому я принял/принимал таблетку аспирина.
    \item Ученик написал/писал трудное слово несколько раз.
    \item Я несколько раз написал/писал ему, чтобы он прислал фотографию своих детей.
    \item Где вы повесили/вешали объявление, на первом или на втором этаже?
    \item По сосредоточенному выражению его лица было видно, что он о чем-то думал/подумал.
    \item Мне очень хотелось попасть на этот концерт, но я не доставал/достал билеты.
    \item Почему книга лежит не на месте? Вы брали/взяли ее?
    \item Учительница сидела за столом и проверила/проверяла ученические тетради.
    \item Саша не мог пойти со мной в кино, потому что готовился к экзамену.
\end{enumerate}

\end{document}
