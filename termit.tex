\section{Kurssilla esiintyviä
termejä}\label{kurssilla-esiintyviuxe4-termejuxe4}

Tähän sivulle on koottu suomen- ja venäjänkieliset variantit kurssin
olennaisimmista termeistä. Päivitän termilistaa sitä mukaa, kuin
etenemme ja jos jokin sinua askarruttava termi puuttuu, vinkkaa
ihmeessä, niin lisätään sekin. Voit hakea termejä helposti selaimen
hakutoiminnolla (ctr+f). Halutessasi voit myös ladata termilistan
excelissä avautuvana \href{data/termit.csv}{csv-tiedostona}. OpenOffice
saattaa avata tiedoston heti oikein, mutta Excelissä koodaus on
kerrottava erikseen (ks. apua
\href{https://youtu.be/GcYt1mJbwk4?t=29}{tästä videosta})

\begin{itemize}
\tightlist
\item
  \href{index.html}{Takaisin kurssin etusivulle}
\item
  Järjestä termit suomenkielisen sanan mukaan
\item
  Järjestä termit venäjänkielisen sanan mukaan
\end{itemize}

\hyperdef{}{termsux5ffi}{\label{termsux5ffi}}
\begin{longtable}[c]{@{}lll@{}}
\toprule
suomi & venäjä & määritelmä\tabularnewline
\midrule
\endhead
äänteenmuutos/-vaihtelu & чередование & morfofonologinen ilmiö, jossa
usein morfien rajalla esiintyy eri äänne sananmuodosta
riippuen\tabularnewline
adjektiivideklinaatio & адъективное склонение & (Pitkien) adjektiivien,
eräiden pronominien ja partisiippien pitkien muotojen noudattama
taivutusmuotojen sarja\tabularnewline
adjektiivit & прилагательные & laatua ilmaisevia sanoja. Ilmaisevat myös
sukua taivutuspäätteillä\tabularnewline
adverbit & наречия & taipumattomia mm. tapaa, paikkaa ja aikaa
ilmaisevia sanoja\tabularnewline
affiksi & аффикс & morfeemeja, joiden avulla johdetaan uusia sanoja.
Jakautuvat suffikseihin, prefikseihin, interfikseihin ja
postfikseihin\tabularnewline
agenttitäydennys & агентивное дополение & instrumentaalimuotoinen
nomini, joka ilmaisee passiivilauseen tekijän\tabularnewline
aikamuotokategoria & категория времени & verbien ilmaisema
kieliopillinen kategoria, jonka tärkeimmät arvot mennyt aika ja
ei-mennyt aika\tabularnewline
aktiivi & действительный залог & Pääluokkakategorian toinen
arvo\tabularnewline
allomorfi & алломорф & morfi, joka on täydennysjakaumassa toisen morfin
kanssa eli on yksi kahdesta tai useammasta konkreettisesta esiintymästä,
jona jokin morfeemi ilmenee\tabularnewline
apusanaluokat & служебные части речи & muut sanaluokat paitsi
partikkelit, prepositiot, konjunktiot ja interjektiot\tabularnewline
aspektikategoria & категория вида & koko venäjän verbijärjestelmän
läpäisevä kategoria, jolla perinteisesti nähty kaksi arvoa,
imperfektiivinen ja perfektiivinen\tabularnewline
deklinaatio & тип склонения & taivutusmuotojen sarja
nomineilla\tabularnewline
delimitatiivinen teonlaatu & ограничительный способ действия & tietyn,
yleensä vähäisen, ajan kestävää toimintaa ilmaisevia verbejä, esim
посидеть, покурить\tabularnewline
distributiivinen merkitys & распределительное значение & datiivin
merkitys, jossa jotakin esinettä annetaan kullekin yksi\tabularnewline
duratiiviset liikeverbit & однонаправленные глаголы движения & esim.
идти, ехать, ilmaisevat liikettä yhteen suuntaan\tabularnewline
elollisuuskategoria & категория одушевлённости & erityisesti
akkusatiivin osalta oleellinen kategoria, joka liittyy sanan referentin
statukseen elollisena/elottomana olentona\tabularnewline
erityismerkitys (aspektin) & частное видовое значение & kontekstista
riippuva erityinen tapa käyttää aspektia\tabularnewline
feminiini & женский род & yksi sukukategorian arvoista\tabularnewline
fleksio & флексия & sama kuin taivutuspääte\tabularnewline
funktionaalinen morfologia & функциональная морфонология & morfologian
osa-alue, joka keskittyy sanojen ja morfeemien käytön
tutkimiseen\tabularnewline
imperatiivi & повелительное наклонение & käskyä ilmaiseva
tapaluokka\tabularnewline
indikatiivi & изъявительное наклонение & tavallisin tapaluokka, ilmaisee
toiminnan ilman ehtoa, käskyä tai puhujan arviota siitä.\tabularnewline
infiniittimuoto & неспрягаемая форма глагола & ks.
nominaalimuoto\tabularnewline
infinitiivivartalo & основа инфинитива & verbin vartalo, joka
muodostetaan poistamalla infinitiivimuodosta infinitiivin
tunnus\tabularnewline
inkoatiivinen teonlaatu & начинательный способ действия & alkamista
ilmaisevia verbejä, esimerkiksi заплакать, засмеяться\tabularnewline
interfiksi & интерфикс & morfeemi, joka esiintyy sanojen välissä, esin
лес/о/заготовка\tabularnewline
interjektiot & междометия & huudahdussanoja kuten ой! ах!
ym.\tabularnewline
intransitiivinen & непереходный & verbi, joka ei saa
objektia\tabularnewline
iteratiiviset liikeverbit & ненаправленные глаголы движения & esim.
ходить, ездить, eivät ota kantaa liikkeen suuntaisuuteen\tabularnewline
itsenäiset sanaluokat & самостоятельные части речи & muut sanaluokat
paitsi partikkelit, prepositiot, konjunktiot ja
interjektiot\tabularnewline
järjestysluvut & порядковые числителные & lukusanoja, jotka ilmaisevat,
kuinka mones jokin ilmiö/asia/tms. on.\tabularnewline
johdettu sana & производное слово & sana, jonka vartaloon kuuluu
juurimorfin lisäksi affikseja\tabularnewline
johtamaton sana & непроизводное слово & sana, jonka vartaloon ei kuuluu
(ainakaan nykykielen kannalta) juurimorfin lisäksi
affikseja\tabularnewline
juurimorfi & корневой морф, корень & sanan leksikaalisen ytimen
ilmaiseva johtamaton morfi\tabularnewline
kardinaaliluvut & количественные числителные & lukusanoja, jotka
ilmaisevat määrää\tabularnewline
kieliopillinen kategoria & грамматическая категория & morfeilla
ilmaistava kielellinen piirre (esim. sija, suku, aspekti, persoona,
pääluokka) jonka suhteen kaksi eri sananmuotoa (esim. играю, играешь)
ovat oppositiossa.\tabularnewline
kollektiivilukusanat & собирательные числителные & lukusanoja, jotka
viittaavat ryhmään henkilöitä\tabularnewline
komparatiivi & сравнительная степень & toinen
vertailuaste\tabularnewline
konditionaali & сослагательное наклонение & ehtoa tai mahdollisuutta
ilmaiseva tapaluokka\tabularnewline
kongruenssi & согласование & kongruenssiksi kutsutaan ilmiötä, jossa
jokin lauseen elementti (kuten predikaattiverbi) vaikuttaa jonkin toisen
elementin muotoon. Esimerkiksi verbin menneen ajan muoto on
kongruenssissa subjektin suvun ja luvun kanssa (она играла, спортсмены
играли)\tabularnewline
konjugaatio & спряжение & verbien taivutustyyppi\tabularnewline
konjunktiot & союзы & lauseita yhdistävien apusanojen ryhmä (а,и,но,хотя
ym.)\tabularnewline
konkreettis-faktinen merkitys & конкретно-фактическое значение &
perfektiivisen aspektin perusmerkitys\tabularnewline
konkreettis-prosessuaalinen merkitys & конкретно-процессное значение &
imperfektiivisen aspektin perusmerkitys\tabularnewline
laatuadjektiivi & качественное прилагательное & adjektiivi, joka voi
ilmaista toiminnan astetta ja intensiteettiä\tabularnewline
liikeverbit & глаголы движения & joukko verbejä, jotka muodostavat parin
liikkeen suuntaisuuden suhteen\tabularnewline
liittofutuuri & сложное будущее время & ei-mennyt aikamuoto, joka
muodostetaan быть-verbistä ja imperfektiivisen aspektin
verbistä\tabularnewline
lukukategoria & категория числа & venäjässä tällä kategorialla kaksi
arvoa: yksikkö ja monikko\tabularnewline
lukusanat & числителные & numeroita ja lukumäärää ilmaisevia sanoja --
morfologisesti osin epäyhtenäinen luokka\tabularnewline
maskuliini & мужской род & yksi sukukategorian arvoista\tabularnewline
modus & наклонение & ks. tapaluokkakategoria\tabularnewline
momentaaninen teonlaatu & одноактный способ действия & verbejä, jotka
ilmaisevat yksittäisen toiminnan erotettuna vastaavien toimintojen
sarjasta, esimerkiksi махнуть, кашлянуть\tabularnewline
morfeemi & морфема & kielen pienen merkitystä kantava
yksikkö\tabularnewline
morfeemiraja & морфемный шов & morfeemien välinen raja, esim juurimorfin
ja suffiksin raja\tabularnewline
morfi & морф & morfeemin konkreettinen esiintymä\tabularnewline
morfofonologia & морфонология & morfologian osa-alue, joka tutkii muun
muassa äänteenmuutoksia\tabularnewline
morfosyntaksi & морфосинтаксис & morfologian osa-alue, joka tutkii
morfologian ja syntaksin rapaintoja, esimerkiksi
kongruenssia\tabularnewline
neutri & средний род & yksi sukukategorian arvoista\tabularnewline
nollaäänne & нуль звука & erityisesti väistyvän vokaalin yhteydessä:
äänne, joka toisissa ympäristöissä edustuu konkreettisena äänteenä,
toisissa ei\tabularnewline
nollamorfi & нулевой морф & morfeemi, joka ei ilmene minään
konkreettisena kielellisenä aineksena, mutta jonka olemassaolo on
perusteltua olettaa\tabularnewline
nollavokaali & нуль гласного & ks. nollaäänne\tabularnewline
nominaalimuoto & неспрягаемая форма глагола & verbimuotoja, jotka eivät
taivu persoonissa vaan nominien tavoin. Esimerkiksi
partisiipit.\tabularnewline
nominit & инменные части речи & sijoissa taipuvat sanaluokat (pois
lukien sijoissa taipuvat partisiipit)\tabularnewline
obliikvisija & косвенный падеж & muut sijat kuin nominatiivi ja
nominatiivin kaltainen akkusatiivi\tabularnewline
oletusarco & значение по умолчанию & kieliopillisen kategorian (esim.
luvun tai suvun) arvo, jota käytetään, jos mikään lauseen elementti ei
suoraan ohjaa käyttämään tiettyä arvoa (ks. kongruenssi), mutta arvoa ei
yhtä kaikki voi myöskään olla ilmaisematta.\tabularnewline
pääluokkakategoria & категория залога & verbien ilmaisema kategoria,
jonka arvot ovat aktiivi (действительный залог) ja passiivi
(старательный залог)\tabularnewline
painotyyppi & акцентологический тип & säännönmukainen sanapainon
ilmeneminen tiettyjen muotojen päätteellä ja tiettyjen muotojen
vartalolla\tabularnewline
partikkelit & частицы & määrällisesti suuri taipumattomien
kieliopillisten sanojen ryhmä, esim. не, бы, мол\tabularnewline
passiivi & страдательный залог & Pääluokkakategorian toinen
arvo\tabularnewline
perduratiivinen teonlaatu & продолжительный способ действия & tietyn
ajan kestäviä tapahtumia ilmaisevia verbejä, esim прожить,
прозаниматься\tabularnewline
perfektin merkitys & значение перфекта & yksi perfektiivisen aspektin
erityismerkityksistä\tabularnewline
persoonakategoria & категория лица & tavallisimmin verbeillä, mutta
joissain tapauksissa myös pronomineilla ilmaistu kieliopillinen
kategoria\tabularnewline
perusmerkitys (aspektin) & общее видовое значение & aspektin
tyyppillisin, lähtökohtainen merkitys\tabularnewline
perusmerkitys (aspektin) & общее видовое значение & ks.
perusmerkitys\tabularnewline
positiivi & положительная степень & ensimmäinen
vertailuaste\tabularnewline
possessiiviadjektiivit & притяжательные прилагательные & adjektiivin
kaltaisia sananmuotoja, joilla osoitetaan jonkin asian tai esineen
kuuluuvuus jollekin elolliselle olennolle\tabularnewline
postfiksi & постфикс & morfeemi, joka esiintyy suffiksien ja päätteiden
jäljessä, esim. ся\tabularnewline
predikatiivi & именная часть сказуемого & se, mitä jostakin entiteetistä
predikoidaan: lauseen predikaattiverbin täydennys, joka kertoo
esimerkiksi jonkin ominaisuuden subjektista (я рад / дедушка
болен)\tabularnewline
predikatiiviadverbi & предикативное наречие & adverbi, joka toimii
lauseen predikaattia, esim. скучно lauseessa мне скучно\tabularnewline
preesensvartalo & основа настоящего времени & verbin vartalo, joka
muodostetaan poistamalla persoonapääte, tavallisesti oletuksena monikon
kolmannesta persoonasta\tabularnewline
prefiksi & префикс, приставка & affiksi, joka liittyy juurimorfeemin
etupuolelle\tabularnewline
prepositiot & предлоги & yksimorfeemisia sanoja, joilla ilmaistaan
kieliopillisia kategorioita. Adpositioita, jotka edeltävät niistä
riippuvaa substantiivia.\tabularnewline
pronominideklinaatio & местоименное склонение & Eräiden pronominien ja
muun muassa possesiiviadjektiivien noudattama taivutusmuotojen
sarja\tabularnewline
pronominit & местоимения & Nominien asemesta käytettäviä
sanoja\tabularnewline
rajoitetun toiston merkitys & ограниченно-кратное значение & yksi
imperfektiivisen aspektin erityismerkityksistä\tabularnewline
rajoittamattoman toiston merkitys & неограниченно-кратное значение &
yksi imperfektiivisen aspektin erityismerkityksistä\tabularnewline
relaatioadjektiivi & относительное прилагательное & adjektiivi, joka ei
voi ilmaista toiminnan astetta ja intensiteettiä\tabularnewline
salkkumorfi & гибридный морф & morfi, joka ilmaisee samalla kertaa
useampaa kieliopillista kategoriaa, esim. красив/ая/-sanan viimeinen
morfi sijaa ja lukua\tabularnewline
sanaluokka & часть речи & sanojen ryhmittely, joka yleensä tehdään
pääosin morfologisin, osin kuitenkin semanttisin (ja syntaktisin)
perustein\tabularnewline
sananmuodostusoppi & словообразование & morfologian toinen päähaara,
joka tutkii mm. affikseja ja johtamista\tabularnewline
semelfaktiivinen teonlaatu & одноактный способ действия & sama kuin
momentaaninen teonlaatu\tabularnewline
sidottu juurimorfi & связанный корень & juurimorfi, joka ei voi
itsessään muodostaa sanaa, esim. о/де/ть\tabularnewline
sijakategoria & категория падежа & tärkein venäjän nominien
keiliopillisista kategorioista, jolla erotetaan lukuisia erilaisia
merkityksiä\tabularnewline
substantiivit & существительные & sijoissa taipuva asioita / henkilöitä
/ esineitä ym. nimeävä sanaluokka\tabularnewline
suffiksi & суффикс & affiksi, joka liittyy juurimorfeemin jälkeen ennen
taivutuspäätteitä ja postfikseja\tabularnewline
sukukategoria & категория рода & substantiivien luokittelu kolmeen
ryhmään. Sukukategoriaa ilmaisevat myös adjektiivit, eräät pronominit
sekä tietyissä muodoissa verbit\tabularnewline
summatiivinen merkitys & суммарное значение & yksi perfektiivisen
aspektin erityismerkityksistä\tabularnewline
superlatiivi & превосходная степень & kolmas vertailuaste\tabularnewline
suppletiiviset muodot & супплетивные формы & sananmuotoja, joissa
juurimorfi vaihtuu ja mahdolliset affiksit lisätään eri
morfiin.\tabularnewline
taivutusoppi & словоизменение & morfologian toinen päähäära, joka tutkii
eri sananmuotoja\tabularnewline
taivutuspääte & окончание & ei uutta sanaa vaan uuden sananmuodon
muodostava morfeemi,joka ilmaisee esim. sijaa\tabularnewline
tapaluokkakategoria & категория наклонения & verbien ilmaisema
kieliopillinen kategoria, jonka arvoja mm. indikatiivi ja
konditionaali\tabularnewline
tempus & время & ks. aikakategoria\tabularnewline
teonlaatu & способ действия & tietyt verbit voidaan ryhmitellä
kuuluvaksi merkityksen ja käyttötavan puolesta yhtenäisiksi ryhmiksi,
joita kutsutaan teonlaaduiksi. Teonlaadut tuovat usein esille
jommankumman aspektin tyypillisiä ominaisuuksia kuten toiminnan rajattua
alkupistettä.\tabularnewline
transitiivinen & переходный & verbi, joka saa objektin\tabularnewline
tunnusmerkillinen osapuoli & маркированный член оппозиции & se
opposition osapuoli, joka ilmaisee jotakin piirrettä tai
ominaisuutta\tabularnewline
tunnusmerkitön osapuoli & немаркированный член оппозиции & se opposition
osapuoli, joka jättää jonkin piirteen tai ominaisuuden ilmaisematta tai
on sen suhteen neutraali\tabularnewline
vaihetta ilmaisevat verbit & фазовые глаголы & verbit tyyppiä начинать,
стать, joiden kanssa käytetään imperfektiivistä aspektia\tabularnewline
väistyvä vokaali & беглый гласный & morfofonologinen ilmiö, jossa
vokaalit {[}o{]}, {[}e{]} ja joskus {[}и{]} vaihtelevat nollavokaalin
kanssa, kuten muodoissa сон - сне\tabularnewline
vapaa juurimorfi & свободный корень & juurimorfi, joka voi sellaisenaan
muodostaa sanan\tabularnewline
vartalo & основа & se osa,joka jää (prefiksi+juuri+suffiksit) kun
taivutuspääte otetaan pois\tabularnewline
verbit & глаголы & tekemistä, toimintaa ilmaisevia sanoja, jotka
ilmaisevat aspektia, pääluokkaa, persoonaa\tabularnewline
vertailuaste & степень сравнения & adjektiivien ja adverbien
kieliopillinen kategoria, jolla on arvot positiivi, komparatiivi,
superlatiivi\tabularnewline
yksinkertainen futuuri & простое будущее время & ei-mennyt aikamuoto,
joka muodostetaan perfektiivisen aspektin verbin
persoonamuodoista\tabularnewline
\bottomrule
\end{longtable}

\hyperdef{}{termsux5fru}{\label{termsux5fru}}
\begin{longtable}[c]{@{}lll@{}}
\toprule
venäjä & suomi & määritelmä\tabularnewline
\midrule
\endhead
агентивное дополение & agenttitäydennys & instrumentaalimuotoinen
nomini, joka ilmaisee passiivilauseen tekijän\tabularnewline
адъективное склонение & adjektiivideklinaatio & (Pitkien) adjektiivien,
eräiden pronominien ja partisiippien pitkien muotojen noudattama
taivutusmuotojen sarja\tabularnewline
акцентологический тип & painotyyppi & säännönmukainen sanapainon
ilmeneminen tiettyjen muotojen päätteellä ja tiettyjen muotojen
vartalolla\tabularnewline
алломорф & allomorfi & morfi, joka on täydennysjakaumassa toisen morfin
kanssa eli on yksi kahdesta tai useammasta konkreettisesta esiintymästä,
jona jokin morfeemi ilmenee\tabularnewline
аффикс & affiksi & morfeemeja, joiden avulla johdetaan uusia sanoja.
Jakautuvat suffikseihin, prefikseihin, interfikseihin ja
postfikseihin\tabularnewline
беглый гласный & väistyvä vokaali & morfofonologinen ilmiö, jossa
vokaalit {[}o{]}, {[}e{]} ja joskus {[}и{]} vaihtelevat nollavokaalin
kanssa, kuten muodoissa сон - сне\tabularnewline
время & tempus & ks. aikakategoria\tabularnewline
гибридный морф & salkkumorfi & morfi, joka ilmaisee samalla kertaa
useampaa kieliopillista kategoriaa, esim. красив/ая/-sanan viimeinen
morfi sijaa ja lukua\tabularnewline
глаголы & verbit & tekemistä, toimintaa ilmaisevia sanoja, jotka
ilmaisevat aspektia, pääluokkaa, persoonaa\tabularnewline
глаголы движения & liikeverbit & joukko verbejä, jotka muodostavat parin
liikkeen suuntaisuuden suhteen\tabularnewline
грамматическая категория & kieliopillinen kategoria & morfeilla
ilmaistava kielellinen piirre (esim. sija, suku, aspekti, persoona,
pääluokka) jonka suhteen kaksi eri sananmuotoa (esim. играю, играешь)
ovat oppositiossa.\tabularnewline
действительный залог & aktiivi & Pääluokkakategorian toinen
arvo\tabularnewline
женский род & feminiini & yksi sukukategorian arvoista\tabularnewline
значение перфекта & perfektin merkitys & yksi perfektiivisen aspektin
erityismerkityksistä\tabularnewline
значение по умолчанию & oletusarco & kieliopillisen kategorian (esim.
luvun tai suvun) arvo, jota käytetään, jos mikään lauseen elementti ei
suoraan ohjaa käyttämään tiettyä arvoa (ks. kongruenssi), mutta arvoa ei
yhtä kaikki voi myöskään olla ilmaisematta.\tabularnewline
изъявительное наклонение & indikatiivi & tavallisin tapaluokka, ilmaisee
toiminnan ilman ehtoa, käskyä tai puhujan arviota siitä.\tabularnewline
именная часть сказуемого & predikatiivi & se, mitä jostakin entiteetistä
predikoidaan: lauseen predikaattiverbin täydennys, joka kertoo
esimerkiksi jonkin ominaisuuden subjektista (я рад / дедушка
болен)\tabularnewline
инменные части речи & nominit & sijoissa taipuvat sanaluokat (pois
lukien sijoissa taipuvat partisiipit)\tabularnewline
интерфикс & interfiksi & morfeemi, joka esiintyy sanojen välissä, esin
лес/о/заготовка\tabularnewline
категория вида & aspektikategoria & koko venäjän verbijärjestelmän
läpäisevä kategoria, jolla perinteisesti nähty kaksi arvoa,
imperfektiivinen ja perfektiivinen\tabularnewline
категория времени & aikamuotokategoria & verbien ilmaisema
kieliopillinen kategoria, jonka tärkeimmät arvot mennyt aika ja
ei-mennyt aika\tabularnewline
категория залога & pääluokkakategoria & verbien ilmaisema kategoria,
jonka arvot ovat aktiivi (действительный залог) ja passiivi
(старательный залог)\tabularnewline
категория лица & persoonakategoria & tavallisimmin verbeillä, mutta
joissain tapauksissa myös pronomineilla ilmaistu kieliopillinen
kategoria\tabularnewline
категория наклонения & tapaluokkakategoria & verbien ilmaisema
kieliopillinen kategoria, jonka arvoja mm. indikatiivi ja
konditionaali\tabularnewline
категория одушевлённости & elollisuuskategoria & erityisesti
akkusatiivin osalta oleellinen kategoria, joka liittyy sanan referentin
statukseen elollisena/elottomana olentona\tabularnewline
категория падежа & sijakategoria & tärkein venäjän nominien
keiliopillisista kategorioista, jolla erotetaan lukuisia erilaisia
merkityksiä\tabularnewline
категория рода & sukukategoria & substantiivien luokittelu kolmeen
ryhmään. Sukukategoriaa ilmaisevat myös adjektiivit, eräät pronominit
sekä tietyissä muodoissa verbit\tabularnewline
категория числа & lukukategoria & venäjässä tällä kategorialla kaksi
arvoa: yksikkö ja monikko\tabularnewline
качественное прилагательное & laatuadjektiivi & adjektiivi, joka voi
ilmaista toiminnan astetta ja intensiteettiä\tabularnewline
количественные числителные & kardinaaliluvut & lukusanoja, jotka
ilmaisevat määrää\tabularnewline
конкретно-процессное значение & konkreettis-prosessuaalinen merkitys &
imperfektiivisen aspektin perusmerkitys\tabularnewline
конкретно-фактическое значение & konkreettis-faktinen merkitys &
perfektiivisen aspektin perusmerkitys\tabularnewline
корневой морф, корень & juurimorfi & sanan leksikaalisen ytimen
ilmaiseva johtamaton morfi\tabularnewline
косвенный падеж & obliikvisija & muut sijat kuin nominatiivi ja
nominatiivin kaltainen akkusatiivi\tabularnewline
маркированный член оппозиции & tunnusmerkillinen osapuoli & se
opposition osapuoli, joka ilmaisee jotakin piirrettä tai
ominaisuutta\tabularnewline
междометия & interjektiot & huudahdussanoja kuten ой! ах!
ym.\tabularnewline
местоимения & pronominit & Nominien asemesta käytettäviä
sanoja\tabularnewline
местоименное склонение & pronominideklinaatio & Eräiden pronominien ja
muun muassa possesiiviadjektiivien noudattama taivutusmuotojen
sarja\tabularnewline
морф & morfi & morfeemin konkreettinen esiintymä\tabularnewline
морфема & morfeemi & kielen pienen merkitystä kantava
yksikkö\tabularnewline
морфемный шов & morfeemiraja & morfeemien välinen raja, esim juurimorfin
ja suffiksin raja\tabularnewline
морфонология & morfofonologia & morfologian osa-alue, joka tutkii muun
muassa äänteenmuutoksia\tabularnewline
морфосинтаксис & morfosyntaksi & morfologian osa-alue, joka tutkii
morfologian ja syntaksin rapaintoja, esimerkiksi
kongruenssia\tabularnewline
мужской род & maskuliini & yksi sukukategorian arvoista\tabularnewline
наклонение & modus & ks. tapaluokkakategoria\tabularnewline
наречия & adverbit & taipumattomia mm. tapaa, paikkaa ja aikaa
ilmaisevia sanoja\tabularnewline
начинательный способ действия & inkoatiivinen teonlaatu & alkamista
ilmaisevia verbejä, esimerkiksi заплакать, засмеяться\tabularnewline
немаркированный член оппозиции & tunnusmerkitön osapuoli & se opposition
osapuoli, joka jättää jonkin piirteen tai ominaisuuden ilmaisematta tai
on sen suhteen neutraali\tabularnewline
ненаправленные глаголы движения & iteratiiviset liikeverbit & esim.
ходить, ездить, eivät ota kantaa liikkeen suuntaisuuteen\tabularnewline
неограниченно-кратное значение & rajoittamattoman toiston merkitys &
yksi imperfektiivisen aspektin erityismerkityksistä\tabularnewline
непереходный & intransitiivinen & verbi, joka ei saa
objektia\tabularnewline
непроизводное слово & johtamaton sana & sana, jonka vartaloon ei kuuluu
(ainakaan nykykielen kannalta) juurimorfin lisäksi
affikseja\tabularnewline
неспрягаемая форма глагола & nominaalimuoto & verbimuotoja, jotka eivät
taivu persoonissa vaan nominien tavoin. Esimerkiksi
partisiipit.\tabularnewline
неспрягаемая форма глагола & infiniittimuoto & ks.
nominaalimuoto\tabularnewline
нулевой морф & nollamorfi & morfeemi, joka ei ilmene minään
konkreettisena kielellisenä aineksena, mutta jonka olemassaolo on
perusteltua olettaa\tabularnewline
нуль гласного & nollavokaali & ks. nollaäänne\tabularnewline
нуль звука & nollaäänne & erityisesti väistyvän vokaalin yhteydessä:
äänne, joka toisissa ympäristöissä edustuu konkreettisena äänteenä,
toisissa ei\tabularnewline
общее видовое значение & perusmerkitys (aspektin) & aspektin
tyyppillisin, lähtökohtainen merkitys\tabularnewline
общее видовое значение & perusmerkitys (aspektin) & ks.
perusmerkitys\tabularnewline
ограниченно-кратное значение & rajoitetun toiston merkitys & yksi
imperfektiivisen aspektin erityismerkityksistä\tabularnewline
ограничительный способ действия & delimitatiivinen teonlaatu & tietyn,
yleensä vähäisen, ajan kestävää toimintaa ilmaisevia verbejä, esim
посидеть, покурить\tabularnewline
одноактный способ действия & momentaaninen teonlaatu & verbejä, jotka
ilmaisevat yksittäisen toiminnan erotettuna vastaavien toimintojen
sarjasta, esimerkiksi махнуть, кашлянуть\tabularnewline
одноактный способ действия & semelfaktiivinen teonlaatu & sama kuin
momentaaninen teonlaatu\tabularnewline
однонаправленные глаголы движения & duratiiviset liikeverbit & esim.
идти, ехать, ilmaisevat liikettä yhteen suuntaan\tabularnewline
окончание & taivutuspääte & ei uutta sanaa vaan uuden sananmuodon
muodostava morfeemi,joka ilmaisee esim. sijaa\tabularnewline
основа & vartalo & se osa,joka jää (prefiksi+juuri+suffiksit) kun
taivutuspääte otetaan pois\tabularnewline
основа инфинитива & infinitiivivartalo & verbin vartalo, joka
muodostetaan poistamalla infinitiivimuodosta infinitiivin
tunnus\tabularnewline
основа настоящего времени & preesensvartalo & verbin vartalo, joka
muodostetaan poistamalla persoonapääte, tavallisesti oletuksena monikon
kolmannesta persoonasta\tabularnewline
относительное прилагательное & relaatioadjektiivi & adjektiivi, joka ei
voi ilmaista toiminnan astetta ja intensiteettiä\tabularnewline
переходный & transitiivinen & verbi, joka saa objektin\tabularnewline
повелительное наклонение & imperatiivi & käskyä ilmaiseva
tapaluokka\tabularnewline
положительная степень & positiivi & ensimmäinen
vertailuaste\tabularnewline
порядковые числителные & järjestysluvut & lukusanoja, jotka ilmaisevat,
kuinka mones jokin ilmiö/asia/tms. on.\tabularnewline
постфикс & postfiksi & morfeemi, joka esiintyy suffiksien ja päätteiden
jäljessä, esim. ся\tabularnewline
превосходная степень & superlatiivi & kolmas vertailuaste\tabularnewline
предикативное наречие & predikatiiviadverbi & adverbi, joka toimii
lauseen predikaattia, esim. скучно lauseessa мне скучно\tabularnewline
предлоги & prepositiot & yksimorfeemisia sanoja, joilla ilmaistaan
kieliopillisia kategorioita. Adpositioita, jotka edeltävät niistä
riippuvaa substantiivia.\tabularnewline
префикс, приставка & prefiksi & affiksi, joka liittyy juurimorfeemin
etupuolelle\tabularnewline
прилагательные & adjektiivit & laatua ilmaisevia sanoja. Ilmaisevat myös
sukua taivutuspäätteillä\tabularnewline
притяжательные прилагательные & possessiiviadjektiivit & adjektiivin
kaltaisia sananmuotoja, joilla osoitetaan jonkin asian tai esineen
kuuluuvuus jollekin elolliselle olennolle\tabularnewline
продолжительный способ действия & perduratiivinen teonlaatu & tietyn
ajan kestäviä tapahtumia ilmaisevia verbejä, esim прожить,
прозаниматься\tabularnewline
производное слово & johdettu sana & sana, jonka vartaloon kuuluu
juurimorfin lisäksi affikseja\tabularnewline
простое будущее время & yksinkertainen futuuri & ei-mennyt aikamuoto,
joka muodostetaan perfektiivisen aspektin verbin
persoonamuodoista\tabularnewline
распределительное значение & distributiivinen merkitys & datiivin
merkitys, jossa jotakin esinettä annetaan kullekin yksi\tabularnewline
самостоятельные части речи & itsenäiset sanaluokat & muut sanaluokat
paitsi partikkelit, prepositiot, konjunktiot ja
interjektiot\tabularnewline
свободный корень & vapaa juurimorfi & juurimorfi, joka voi sellaisenaan
muodostaa sanan\tabularnewline
связанный корень & sidottu juurimorfi & juurimorfi, joka ei voi
itsessään muodostaa sanaa, esim. о/де/ть\tabularnewline
словоизменение & taivutusoppi & morfologian toinen päähäära, joka tutkii
eri sananmuotoja\tabularnewline
словообразование & sananmuodostusoppi & morfologian toinen päähaara,
joka tutkii mm. affikseja ja johtamista\tabularnewline
сложное будущее время & liittofutuuri & ei-mennyt aikamuoto, joka
muodostetaan быть-verbistä ja imperfektiivisen aspektin
verbistä\tabularnewline
служебные части речи & apusanaluokat & muut sanaluokat paitsi
partikkelit, prepositiot, konjunktiot ja interjektiot\tabularnewline
собирательные числителные & kollektiivilukusanat & lukusanoja, jotka
viittaavat ryhmään henkilöitä\tabularnewline
согласование & kongruenssi & kongruenssiksi kutsutaan ilmiötä, jossa
jokin lauseen elementti (kuten predikaattiverbi) vaikuttaa jonkin toisen
elementin muotoon. Esimerkiksi verbin menneen ajan muoto on
kongruenssissa subjektin suvun ja luvun kanssa (она играла, спортсмены
играли)\tabularnewline
сослагательное наклонение & konditionaali & ehtoa tai mahdollisuutta
ilmaiseva tapaluokka\tabularnewline
союзы & konjunktiot & lauseita yhdistävien apusanojen ryhmä (а,и,но,хотя
ym.)\tabularnewline
способ действия & teonlaatu & tietyt verbit voidaan ryhmitellä
kuuluvaksi merkityksen ja käyttötavan puolesta yhtenäisiksi ryhmiksi,
joita kutsutaan teonlaaduiksi. Teonlaadut tuovat usein esille
jommankumman aspektin tyypillisiä ominaisuuksia kuten toiminnan rajattua
alkupistettä.\tabularnewline
спряжение & konjugaatio & verbien taivutustyyppi\tabularnewline
сравнительная степень & komparatiivi & toinen
vertailuaste\tabularnewline
средний род & neutri & yksi sukukategorian arvoista\tabularnewline
степень сравнения & vertailuaste & adjektiivien ja adverbien
kieliopillinen kategoria, jolla on arvot positiivi, komparatiivi,
superlatiivi\tabularnewline
страдательный залог & passiivi & Pääluokkakategorian toinen
arvo\tabularnewline
суммарное значение & summatiivinen merkitys & yksi perfektiivisen
aspektin erityismerkityksistä\tabularnewline
супплетивные формы & suppletiiviset muodot & sananmuotoja, joissa
juurimorfi vaihtuu ja mahdolliset affiksit lisätään eri
morfiin.\tabularnewline
суффикс & suffiksi & affiksi, joka liittyy juurimorfeemin jälkeen ennen
taivutuspäätteitä ja postfikseja\tabularnewline
существительные & substantiivit & sijoissa taipuva asioita / henkilöitä
/ esineitä ym. nimeävä sanaluokka\tabularnewline
тип склонения & deklinaatio & taivutusmuotojen sarja
nomineilla\tabularnewline
фазовые глаголы & vaihetta ilmaisevat verbit & verbit tyyppiä начинать,
стать, joiden kanssa käytetään imperfektiivistä aspektia\tabularnewline
флексия & fleksio & sama kuin taivutuspääte\tabularnewline
функциональная морфонология & funktionaalinen morfologia & morfologian
osa-alue, joka keskittyy sanojen ja morfeemien käytön
tutkimiseen\tabularnewline
частицы & partikkelit & määrällisesti suuri taipumattomien
kieliopillisten sanojen ryhmä, esim. не, бы, мол\tabularnewline
частное видовое значение & erityismerkitys (aspektin) & kontekstista
riippuva erityinen tapa käyttää aspektia\tabularnewline
часть речи & sanaluokka & sanojen ryhmittely, joka yleensä tehdään
pääosin morfologisin, osin kuitenkin semanttisin (ja syntaktisin)
perustein\tabularnewline
чередование & äänteenmuutos/-vaihtelu & morfofonologinen ilmiö, jossa
usein morfien rajalla esiintyy eri äänne sananmuodosta
riippuen\tabularnewline
числителные & lukusanat & numeroita ja lukumäärää ilmaisevia sanoja --
morfologisesti osin epäyhtenäinen luokka\tabularnewline
\bottomrule
\end{longtable}

\textless{}!-- !pandoc morfologia.Rmd -f markdown -t html
--bibliography=lahteet.bib --standalone -o index.html -css
github-pandoc.css --toc --toc-depth=1 --include-after-body=scripts.js
--smart --\textgreater{}
